%"思考Python:像计算机科学家一样思考"LaTex源码
%Copyright (c) 2008 Allen B. Downey.
%中文翻译:2010 Walter Lewis (刘宇辉).

% Permission is granted to copy, distribute and/or modify this
% document under the terms of the GNU Free Documentation License,
% Version 1.1  or any later version published by the Free Software
% Foundation; with no Invariant Sections, no Front-Cover Texts,
% and no Back-Cover Texts.

% This distribution includes a file named fdl.tex that contains the text
% of the GNU Free Documentation License.  If it is missing, you can obtain
% it from www.gnu.org or by writing to the Free Software Foundation,
% Inc., 59 Temple Place - Suite 330, Boston, MA 02111-1307, USA.
%

\documentclass[10pt]{book}
\usepackage[width=5.5in,height=8.5in,
	hmarginratio=3:2,vmarginratio=1:1]{geometry} %指定了{h,v}marginratio,width和height被忽略了,可以不用指定(参考geometry.pdf, $texdoc geometry.pdf)。

%下面的这些包,你可能需要安装texlive-latex-extra(in Debian/Ubuntu)
\usepackage{pslatex}  %
\usepackage{url}
\usepackage{fancyhdr}
\usepackage{graphicx}
\usepackage{amsmath,amsthm,amssymb}
\usepackage{exercise}
\usepackage{makeidx}
\usepackage{setspace}
\usepackage{hevea}
\usepackage{upquote}
%这些是对中文的支持
\usepackage{xeCJK}
\usepackage{fontspec}
\setCJKmainfont{AR PL SungtiL GB:style=Regular}
\setmainfont{TeXGyrePagella:style=Regular}
\XeTeXlinebreaklocale "zh"
\XeTeXlinebreakskip=0pt plus 1pt minus 0.1pt

\title{思考Python}
\newcommand{\thetitle}{思考Python:像计算机科学家一样思考}
\newcommand{\theversion}{1.1.22}

%以下的这些风格被转换成html的css
\newstyle{a:link}{color:black;}
\newstyle{p+p}{margin-top:lem;margin-bottom:lem}
\newstyle{img}{border:opx}

%改变箭头(方向)
\setlinkstext
{\imgsrc[ALT="Previous"]{back.png}}
{\imgsrc[ALT="Previous"]{up.png}}
{\imgsrc[ALT="Previous"]{next.png}}

\makeindex

\begin{document}
\frontmatter
%LaTeXOnly

\input{latexonly}

\newtheorem{ex}{Exercise}[chapter]
\begin{latexonly}

\renewcommand{\blankpage}{\thispagestyle{empty} \quad \newpage}

%blankpage
%blankpage

%-half title----------------------------------------------
\thispagestyle{empty}
\begin{flushright}
\vspace*{2.0in}

\begin{spacing}{3}
{\huge 思考Python}\\
{\Large 像计算机科学家一样思考}
\end{spacing}

\vspace{0.25in}

Version \theversion

\vfill
\end{flushright}

%---------------------------
\blankpage
\blankpage

%---title page---------
\pagebreak
\thispagestyle{empty}

\begin{flushright}
\vspace*{2.0in}

\begin{spacing}{3}
{\huge 思考Python}\\
{\Large 像计算机科学家一样思考}
\end{spacing}

\vspace{0.25in}
Version \theversion

{\Large
	Allen Downey\\
}
{\Large 
	Walter Lewis\\
}



\vspace{0.5in}

{\Large Green Tea Press}
{\small Needham, Massachusetts}

\vfill
\end{flushright}

%---copyright------------------
\pagebreak
\thispagestyle{empty}

{\small
	Copyright \copyright ~2008 Allen Downey.

Printing history:

\begin{description}

\item[2002四月:] 第一版 {\em 像计算机科学家一样思考}.
\item[2007八月:] 大幅改动,把标题改为{\em 像(Python)程序员一样思考}.
\item[2008六月:] 大幅改动,把标题改为{\em 思考Python:像计算机科学家一样思考}.
\end{description}

\vspace{0.2in}

\begin{flushleft}  %左对齐,类似的有flushright,centre
Green Tea Press \\
9 Washburn Ave\\
Needham MA 02492
\end{flushleft}


Permission is granted to copy, distribute, and/or modify this document
under the terms of the GNU Free Documentation License, Version 1.1 or
any later version published by the Free Software Foundation; with no
Invariant Sections, no Front-Cover Texts, and with no Back-Cover Texts.

The GNU Free Documentation License is available from {\tt www.gnu.org}
or by writing to the Free Software Foundation, Inc., 59 Temple Place,
Suite 330, Boston, MA 02111-1307, USA.

The original form of this book is \LaTeX\ source code.  Compiling this
\LaTeX\ source has the effect of generating a device-independent
representation of a textbook, which can be converted to other formats
and printed.

The \LaTeX\ source for this book is available from
\url{http://www.thinkpython.com}
\vspace{0.2in}
}

\end{latexonly}


%htmlonly
\begin{htmlonly}

%Title page for html version

{\Large \thetitle}
\{\Large  Allen B.Downet}
\{\Large  翻译:Walter Lewis}

Version \theversion

\setcounter{chapter}{-1}
\end{htmlonly}

\chapter{前言}

\section{本书的奇怪历史}

1999年一月份的时候,我准备用Java教一门介绍性的编程课。在那之前,我已经
教了三次,而且每次我都很失望。这门课的挂课率非常之高,尽管对那些通过的学
生来说,整体的水平也是很低的。\\

我认为问题的根源之一是教科书。教科书太厚了,掺杂着大量不必要的Java细节
内容,并且没有足够高水平的引导去指导学生如何编程。学生们深陷“陷阱门“:他们起步很轻松,逐步的学习,突然,大约在第五章的某个位置,困难出现了。学生必须快速的学习大量的新内容。结果,我不得不把剩下的学期花在挑选一些片段来教学。\\

课程开始的前两周,我决定自力更生--自己编写书。我的目标是:
\begin{itemize}

\item 尽量简短.对学生来说,阅读十页比阅读无十页要好。

\item 注意词汇量。我尽量减少使用术语,而且在使用前必须先定义。

\item 逐步学习。为了避免陷阱门,我把最难的部分分解成一系列的小步骤。

\item 把重心房子编程,而不是编程语言。我采用最少的有用的Java语言的语法,
忽略其他的。
\end{itemize}























\end{document}
