%"思考Python:像计算机科学家一样思考"LaTex源码
%Copyright (c) 2008 Allen B. Downey.
%中文翻译:2010 Walter Lewis (刘宇辉).

% Permission is granted to copy, distribute and/or modify this
% document under the terms of the GNU Free Documentation License,
% Version 1.1  or any later version published by the Free Software
% Foundation; with no Invariant Sections, no Front-Cover Texts,
% and no Back-Cover Texts.

% This distribution includes a file named fdl.tex that contains the text
% of the GNU Free Documentation License.  If it is missing, you can obtain
% it from www.gnu.org or by writing to the Free Software Foundation,
% Inc., 59 Temple Place - Suite 330, Boston, MA 02111-1307, USA.
%

\documentclass[10pt]{book}
\usepackage[width=5.5in,height=8.5in,hmarginratio=3:2,vmarginratio=1:1]{geometry} %指定了{h,v}marginratio,width和height被忽略了,可以不用指定(参考geometry.pdf, $texdoc geometry.pdf)。

%下面的这些包,你可能需要安装texlive-latex-extra(in Debian/Ubuntu)
\usepackage{pslatex}  %
\usepackage{url}
\usepackage{fancyhdr}
\usepackage{graphicx}
\usepackage{amsmath,amsthm,amssymb}
\usepackage{exercise}
\usepackage{makeidx}
\usepackage{setspace}
\usepackage{hevea}
\usepackage{upquote}
%这些是对中文的支持
\usepackage{xeCJK}
\usepackage{fontspec}
\setCJKmainfont{FangSong_GB2312:style=Regular}
\setmainfont{FangSong_GB2312:style=Regular}
\XeTeXlinebreaklocale "zh"
\XeTeXlinebreakskip=0pt plus 1pt minus 0.1pt

\title{思考Python}
\newcommand{\thetitle}{思考Python:像计算机科学家一样思考}
\newcommand{\theversion}{1.1.22}

%以下的这些风格被转换成html的css
\newstyle{a:link}{color:black;}
\newstyle{p+p}{margin-top:lem;margin-bottom:lem}
\newstyle{img}{border:opx}

%改变箭头(方向)
\setlinkstext
{\imgsrc[ALT="Previous"]{back.png}}
{\imgsrc[ALT="Previous"]{up.png}}
{\imgsrc[ALT="Previous"]{next.png}}

\makeindex

\begin{document}

\frontmatter



%LaTeXOnly

\input{latexonly}

\newtheorem{ex}{Exercise}[chapter]
\begin{latexonly}

\renewcommand{\blankpage}{\thispagestyle{empty} \quad \newpage}

%blankpage
%blankpage

%-half title----------------------------------------------
\thispagestyle{empty}
\begin{flushright}
\vspace*{2.0in}

\begin{spacing}{3}
{\huge 思考Python}\\
{\Large 像计算机科学家一样思考}
\end{spacing}

\vspace{0.25in}

Version \theversion

\vfill
\end{flushright}

%---------------------------
\blankpage
\blankpage

%---title page---------
\pagebreak
\thispagestyle{empty}

\begin{flushright}
\vspace*{2.0in}

\begin{spacing}{3}
{\huge 思考Python}\\
{\Large 像计算机科学家一样思考}
\end{spacing}

\vspace{0.25in}
Version \theversion

{\Large
	Allen Downey\\
}




\vspace{0.5in}

{\Large Green Tea Press}\\
{\small Needham, Massachusetts}

\vfill
\end{flushright}

%---copyright------------------
\pagebreak
\thispagestyle{empty}

{\small
	Copyright \copyright ~2008 Allen Downey.

Printing history:

\begin{description}

\item[2002四月:] 第一版 {\em 像计算机科学家一样思考}.
\item[2007八月:] 大幅改动,把标题改为{\em 像(Python)程序员一样思考}.
\item[2008六月:] 大幅改动,把标题改为{\em 思考Python:像计算机科学家一样思考}.
\end{description}

\vspace{0.2in}

\begin{flushleft}  %左对齐,类似的有flushright,centre
Green Tea Press \\
9 Washburn Ave\\
Needham MA 02492
\end{flushleft}


Permission is granted to copy, distribute, and/or modify this document
under the terms of the GNU Free Documentation License, Version 1.1 or
any later version published by the Free Software Foundation; with no
Invariant Sections, no Front-Cover Texts, and with no Back-Cover Texts.

The GNU Free Documentation License is available from {\tt www.gnu.org}
or by writing to the Free Software Foundation, Inc., 59 Temple Place,
Suite 330, Boston, MA 02111-1307, USA.

The original form of this book is \LaTeX\ source code.  Compiling this
\LaTeX\ source has the effect of generating a device-independent
representation of a textbook, which can be converted to other formats
and printed.

The \LaTeX\ source for this book is available from
\url{http://www.thinkpython.com}
\vspace{0.2in}
}

\end{latexonly}


%htmlonly
\begin{htmlonly}

%Title page for html version

{\Large \thetitle}
\{\Large  Allen B.Downet}
\{\Large  翻译:Walter Lewis}

Version \theversion

\setcounter{chapter}{-1}
\end{htmlonly}

\chapter{前言}

\section{本书的奇怪历史}

1999年一月份的时候,我准备用Java教一门介绍性的编程课。在那之前,我已经
教了三次,而且每次我都很失望。这门课的挂课率非常之高,尽管对那些通过的学
生来说,整体的水平也是很低的。\\

我认为问题的根源之一是教科书。教科书太厚了,掺杂着大量不必要的Java细节
内容,并且没有足够高水平的引导去指导学生如何编程。学生们深陷“陷阱门“:他们起步很轻松,逐步的学习,突然,大约在第五章的某个位置,困难出现了。学生必须快速的学习大量的新内容。结果,我不得不把剩下的学期花在挑选一些片段来教学。\\

课程开始的前两周,我决定自力更生---自己编写书。我的目标是:
\begin{itemize}

\item 尽量简短.对学生来说,阅读十页比阅读五十页要好。

\item 注意词汇量。我尽量减少使用术语,而且在使用前必须先定义。

\item 逐步学习。为了避免陷阱门,我把最难的部分分解成一系列的小步骤。

\item 把重心放在编程,而不是编程语言。我采用最少的有用的Java语言的语法,
忽略其他的。
\end{itemize}

我需要一个书名,所以我就临时地把它叫做《像计算机科学家一样思考》\\

第一版很粗糙,但是很成功。学生们很乐意看它,并且能很好理解我在课堂上讲的难点,趣点和让他们实践的内容(这个最重要).\\

我用GNU自由文档许可证发布了这本书,读者们可以自由的复制,修改,发布这本书。 \\

\index{GNU Free Documentation License}
\index{Free Documentation License,GNU}

 接下来发生的事儿极其的有趣。Jeff Elkner,居住在弗尼亚的高中老师,改变了我的书,把它翻译成了Python。他给我寄了份他翻译的副本,于是乎我就有了一段不寻常的学习Python的经历--通过阅读我自己的书。\\

 Jeff和我随后修订了这本书,加入了Chris Meyers提供的一个案例学习。在2001年,我们共同发布了《像计算机科学家一样思考:Python编程》,当然同样是用GNU自由文档许可证。通过Gree Tea Press,我出版了这本书,并且开始在亚马逊和大学书店卖纸质书。Gree Tea Press出版的书可以从这儿获得\url{greenteapress.com}\\

 2003年,我开始在Olin College教书。第一次,我开始教Python。和教授Java的情况相反,学生们不再陷入泥潭,学到了更多,参与了很多有趣的项目,越学越带劲。\\

 在过去的五年里,我一直继续完善这本书,改正错误,提过某些例子的质量,加入一些其他的材料,特别是练习。在2008年,我开始重写这本书---同时,剑桥大学出版社的编辑联系到了我,他想出版本书的下一板。美妙的时刻!\\

 结构就出现了现在的这本书,不过有了一个简洁的名字《思考Python》。变化有:
 \begin{itemize}

 \item 在每一章末尾加了点调试的部分。这些部分提供了发现和避免bug的通用技巧,也对Python的陷阱提出了警告。

 \item 删除了最后几章关于列表和树实现的内容。虽然,我万分不舍,但是考虑到和本书余下的部分不协调,只能忍痛割爱。

 \item 增加了一些案例学习---提供了练习,答案和相关讨论的大例子。一些东西是基于Swampy,这是我为了教学而设计的Python程序。
 Swampy,代码实例和部分答案可以从这儿获得\url{thinkpython.com}.

 \item 扩展了关于程序构建计划和基本的设计模式的讨论。

 \item Python运用的更加地道。虽然这本书仍然是讨论编程的,而不是Python本身,但是现在我不得不承认这本书深受Python浸染。
 
 \end{itemize}

  我希望读者们可以享受这本书,也希望帮助你学习程序设计和像计算机科学家一样思考,哪怕是一丁丁点儿。\\

Allen B. Downey\\
Needham MA\\

Allen Downey 是Olin College 大学计算机科学与技术系的副教授。



\section*{致谢}
首先也是最重要的,我要感谢Jeff Elkner将我的Java教材译为Python版本,为此项目奠基并且向我介绍了Python语言,它现在已经是我最的首选编程语言.

感谢Chris Meyers,向{\em 如何像一个计算机科学家思考}一书也就是本书的前身贡献了若干内容.

感谢自由软件基金会开发了GNU Free Document License这一授权协议,使我与Jeff,Chris的共事得以成为现实.

\index{GNU Free Document License}
\index{Free Document License, GNU}

感谢Lulu的编辑们为本书前身{\em 如何像一个计算机科学家思考}付出的努力.

感谢所有为此书早期版本做出努力的学生与所有向我寄送更正与建议的贡献者(在后面列出).

更感谢我妻子Lisa为此书,还有绿茶出版,以及其他所有事情做出的工作.

\section*{贡献者列表}

\index{贡献者}

100多位目光锐利,头脑灵活的读者在过去的几年中寄给我许多建议与更正。他们的贡献和热忱对项目建功颇多.

如果你有建议或者更正,也请寄送email至{\tt feedback@thinkpython.com}.如果我基於你的反馈做出了修改,你将被列於此贡献者列表(除非你自己申明愿意被忽略).

如果能够附上出现错误的句子,哪怕是一部分,将有助我进行搜索。页码与章节序号亦可,不过句子更方便。谢谢。

\small

\begin{itemize}

\item Lloyd Hugh Allen 提交了第8.4节的一个更正.

\item Yvon Boulianne 寄给我第5章的一个更正.

\item Fred Bremmer 提交了第2.1节的一个更正.

\item Jonah Cohen 撰写了将LaTeX源代码转换为漂亮的HTML文件的Perl脚本.

\item Michael Conlon 提交了第2章的一个语法更正与第1章的一个文风改善意见,还发起了关於解释器的技术层面的讨论。

\item Benoit Girard 提交了第5.6节的一个比较滑稽错误的更正。

\item Courtney Gleason 与 Katherine Smith 撰写了{\tt horsebet.py},用作此书早期版本的一个参考案例。此程序可以在网上找到。

\item Lee Harr 提交了很多更正,因为空间有限无法一一列出。因为贡献较大,其实他可以称作此书的重要编辑之一。

\item James Kaylin 是一个使用此教材的学生,提交了大量的更正。

\item David Kershaw 修正了第3.10节的{\tt catTwice}函数。

\item Eddie Lam 提交了第1,2,3章的大量更正。他也修正了Makefile,使得在第一次运行时可以生成一个索引。本书的版本计画也由他发起。

\item Man-Yong Lee 提交了第2.4节的一个例题代码的一个更正。

\item David Mayo 指出第1章的一个用词不当。

\item Chris McAloon 提交了第3.9节与第3.10节的若干更正。

\item Matthew J. Moelter 曾长期关注此书,提交了大量的更正与建议。

\item Simon Dicon Montford 报告了第3章的一个缺失函数定义错误与若干拼写错误。亦指出第13章中{\tt increment}函数的错误。

\item John Ouzts 更正了第3章中的关於返回值的定义。

\item Kevin Parks 针对如何提高此书的发行质量提交了宝贵的评论与建议.

\item David Pool 提交了第1章的术语表中一个拼写错误,而且给我们打气。

\item Michael Schmitt 提交了关於文件於异常章节中的一个更正.

\item Robin Shaw 指出了13.1中的一个错误(一个范例中使用printTime函数但是没有事先定义)。

\item Paul Sleigh 找出了第7章的一个错误,且发现了Jonah Cohen用来从LaTeX生成HTML页面的Perl 脚本中的一个bug.

\item Craig T. Snydal於Drew大学实践此教材。他贡献了数处有用的建议与更正. 

\item Ian Thomas 与他的学生在一个编程课程中使用此教材。他们是此书后面一半章节最早的实践者,也提出了大量的更正与建议.

\item Keith Verheyden提交了第3章的一个更正.

\item Peter Winstanley给我们指出了第3章中一个长期存在的错误。

\item Chris Wrobel 贡献了若干关於文件I/O与异常章节中的更正. 

\item Moshe Zadka对此项目做出了难能可贵的贡献.除了为此书开头部分撰写了辞典范例相关内容,在整个此书的早期阶段他一直提供指导。

\item Christoph Zwerschke 提交了数处更正与教学考虑上的建议, 并向我们解释了{\em gleich}与{\em selbe}之间的差别.

\item James Mayer 向我们提交了一大堆的拼写与录入错误,包含这个贡献者列表中的两个.

\item Hayden McAfee 指出了两个范例中可能存在的令人困惑的不一致.

\item Angel Arnal 是本书的西班牙文版本翻译组成员之一.他亦找出了英文版本的数处错误.

\item Tauhidul Hoque 与 Lex Berezhny 创建了第1章插图并提高了其他数处插图的质量.

\item Dr. Michele Alzetta 找出了第8章的一个错误并且针对Fibonacci与Old Maid这两个范例提交了一些从教学方面的评论与建议.

\item Andy Mitchell 找出了第1章的一个拼写错误与第2章的某范例的阙漏之处.

\item Kalin Harvey 建议对第7章的一处概念澄清并且提交了一些拼写错误.

\item Christopher P. Smith 找出了数处拼写错误并且协助我们将此书针对Python 2.2做了更新。

\item David Hutchins 找出了序中的一个拼写错误.

\item Gregor Lingl於奥地利维也纳的某高中教授Python,他正在从事此书的德文翻译并且找出了第5章的若干重要错误。

\item Julie Peters 找出了前言中的一个拼写错误.

\item Florin Oprina 提交了{\tt makeTime}中的一个改善,一个{\tt printTime}中的更正, 与一个拼写错误.

\item D.~J.~Webre 建议了第3章的一个概念澄清.

\item Ken 找出了第8, 9,11章中的大量错误.

\item Ivo Wever 找出了第5章中的一个拼写错误并且建议了第3章中的一个概念澄清.

\item Curtis Yanko 建议了第2章中的一个概念澄清.

\item Ben Logan 提交了大量的拼写错误与若干将此书转换为HTML格式的错误。

\item Jason Armstrong 指出了第2章一处语法阙漏.

\item Louis Cordier指出第16章某处代码与文字不相符合.

\item Brian Cain 建议了数处第2章与第3章的若干概念澄清.

\item Rob Black 提交了大量更正,包含某些针对Python 2.2的修改.

\item Jean-Philippe Rey来自於巴黎的Ecole Centrale,提交了大量的补丁,包含某些针对Python 2.2的修改与一些有想法的改善.

\item Jason Mader来自於George Washington 大学提交了大量的有用的建议与更正.

\item Jan Gundtofte-Bruun指出一个语法错误.

\item Abel David 与 Alexis Dinno 提醒我们``matrix''的复数形式为``matrices'',而非``matrixes''.此错误存在於本书中有数年之久,但是两位读者在同一天向我们指出了,不可不谓奇。

\item Charles Thayer指出了数处语法错误。

\item Roger Sperberg 指出了第3章某处逻辑有误.

\item Sam Bull 指出了第2章某处令人困惑的段落 .

\item Andrew Cheung 指出了两处使用之后才定义的错误。

\item C. Corey Capel 指出了一个语法错误与一个拼写错误.

\item Alessandra帮助我们澄清了某些概念.

\item Wim Champagne在辞典范例中指出一处错误.

\item Douglas Wright 指出了{\tt arc}中的一个除法相关错误.

\item Jared Spindor指出一些语法错误.

\item Lin Peiheng 提交了大量的非常有益的建议.

\item Ray Hagtvedt提交了两处错误与一处似是而非的地方.

\item Torsten H\"{u}bsch 指出了Swampy范例中某处不一致.

\item Inga Petuhhov 更正了第14章中的某个一个范例.

\item Arne Babenhauserheide 提交了数处有益的更正.

\item Mark E. Casida非常擅长於给我们指出重复用词的地方.

\item Scott Tyler 提交了大量更正.

\item Gordon Shephard 给我们发送了多封信件,提交了数处更正.

\item Andrew Turner指出了第8章的一个错误.

\item Adam Hobart 修正了{\tt arc}中的一个除法相关错误.

\item Daryl Hammond 与 Sarah Zimmerman 指出我过早地提到了{\tt math.pi} .  Zim也指出了一个拼写错误.

\item George Sass 在调试章节指出了一个错误.

\item Brian Bingham 建议了练习题~\ref{exrotatepairs}.

\item Leah Engelbert-Fenton 指出了我使用{\tt tuple}作为变量名,与之前的我自己的建议背道而驰.  之后又发现了大量拼写错误与一处``定义之前使用''。

\item Joe Funke指出了一个拼写错误.

\item Chao-chao Chen 发现了Fibonacci范例中的某处不一致.

\item Jeff Paine指出了一处语法错误.

\item Lubos Pintes 提交了一个拼写错误.

\item Gregg Lind 与 Abigail Heithoff 建议了练习题~\ref{checksum}.

\item Max Hailperin 提交了大量的更正与建议.  Max 是著名的{\em
    Concrete Abstractions}一书的作者之一,你读完此书再把那本书找来阅读学习效果可能会更好。.

\item Chotipat Pornavalai 指出了一个错误信息中的错误.

\item Stanislaw Antol 提交了大量的非常有益的建议.

\item Eric Pashman 提交了第4--11章的大量的更正.

\item Miguel Azevedo指出若干拼写错误.

\item Jianhua Liu 提交了大量的更正.

\item Nick King指出一处语法阙漏.

\item Martin Zuther 提交了大量的建议.

\item Adam Zimmerman指出数处错误.

\item Ratnakar Tiwari 建议了一个解释退化三角形的脚注.

\item Anurag Goel 建议了若干更正.

\item Kelli Kratzer指出一处拼写错误.

\item Mark Griffiths 指出了第3章的范例中某令人困惑之处.

\item Roydan Ongie指出关於Newton法的一个错误.

\item Patryk Wolowiec 协助我修正了关於HTML版本的一个错误.

\item Mark Chonofsky 告诉我Python 3.0的一个新的关键词.

\item Russell Coleman在几何学方面协助过我.

\item Wei Huang指出数处录入错误.

\item Karen Barber指出书中最古老的拼写错误之一.

\item Nam Nguyen指出一个拼写错误并且指出了我使用了Decorator模式但是没有提到它的名字。

\item St\'{e}phane Morin提交了数处更正与建议.

\item Paul Stoop更正了一个拼写错误.

\item Eric Bronner指出了操作符优先级别讨论中的一处困惑之处。

\item Alexandros Gezerlis指出了大量的高质量的建议,我们相当感激!

\item Gray Thomas指出了一处错误.

% ENDCONTRIB

\end{itemize}


\normalsize
\clearemptydoublepage

\begin{latexonly}

\tableofcontents

\clearemptydoublepage

\end{latexonly}


\mainmatter

<<<<<<< HEAD
\chapter{编程的方式}
\include{chapter1}

\chapter{变量、表达式和语句}
\include{chapter2}

\chapter{函数}
\include{chapter3}

\chapter{实例学习:接口设计}
\include{chapter4}

\chapter{条件语句和递归}
\include{chapter5}

\chapter{卓有成效的函数}
\include{chapter6}

\chapter{迭代}
\include{chapter7}

\chapter{字符串}
\include{chapter8}

\chapter{实例学习:文字处理}
\include{chapter9}

\chapter{列表}
\chapter{列表}

\index{列表}
\index{类型!列表}


\section{列表是一个序列}

类似字符串,{\bf 列表}一个序列的值。在字符串中,每个值是字符;在一个列表中可以是任何数据类型。列表中的数值称为{\bf 元素},有时也称为{\bf 项目}。

\index{元素}
\index{序列}
\index{项目}

有多种方法可以创建一个新的列表;最简单的方法是用方括号(\verb"["和\verb"]")将元素包括起来:

\beforeverb
\begin{verbatim}
[10, 20, 30, 40]
['crunchy frog', 'ram bladder', 'lark vomit']
\end{verbatim}
\afterverb
%
第一个例子是包含4个整数的列表。第二个例子是包含3个字符串的列表。一个列表中的元素不需要是相同数据类型。下面的列表包含一个字符串、一个浮点数、一个整数和另一个列表:

\beforeverb
\begin{verbatim}
['spam', 2.0, 5, [10, 20]]
\end{verbatim}
\afterverb
%
在一个列表中的列表称为{\bf 嵌套}。

\index{嵌套列表}
\index{列表!嵌套}

没有任何元素的列表称为空列表。你可以使用空的括号\verb"[]"创建一个空列表。

\index{空列表}
\index{列表!空}

正如你可能期望的,列表的值可以被赋值给变量:

\beforeverb
\begin{verbatim}
>>> cheeses = ['Cheddar', 'Edam', 'Gouda']
>>> numbers = [17, 123]
>>> empty = []
>>> print cheeses, numbers, empty
['Cheddar', 'Edam', 'Gouda'] [17, 123] []
\end{verbatim}
\afterverb
%

\index{赋值}

% From Jeff: write sum for a nested list?


\section{列表是可改变的}

\index{列表!元素}
\index{访问}
\index{下标}
\index{括号运算符}
\index{运算符!括号}

访问列表中元素的语法和访问字符串中字符的语法相同,都是通过括号运算符实现的。括号中的表达式指定了下标。记住下标从0开始:

\beforeverb
\begin{verbatim}
>>> print cheeses[0]
Cheddar
\end{verbatim}
\afterverb
%
与字符串不同,列表是可以改变的。当括号运算符出现在赋值语句的左边,它指向列表中将被赋值的元素。

\index{可改变的}

\beforeverb
\begin{verbatim}
>>> numbers = [17, 123]
>>> numbers[1] = 5
>>> print numbers
[17, 5]
\end{verbatim}
\afterverb
%
{\tt numbers}中的第一个元素,原来是123,现在是5。
used to be 123, is now 5.

\index{下标!从0开始}
\index{0,下标开始于}

你可以将列表看成下标和元素的对应关系。这种关系成为{\bf 映射}。每个下标“对应”一个元素。这里给出{\tt cheeses},{\tt numbers}和{\tt empty}的状态图:

\index{状态图}
\index{图!状态}
\index{映射}

\beforefig
\centerline{\includegraphics{figs/list_state.eps}}
\afterfig

列表用外部标有“list”的盒子表示,内部是列表中的元素。{\tt cheeses}是一个有3个元素的列表,下标分别是0,1和2。{\tt numbers}包含2个元素。状态图显示了第二个元素原来是123,被重新赋值为5。{\tt empty}对应一个没有元素的列表。

\index{项目赋值}
\index{赋值!项目}

列表下标的工作原理和字符串的相同:

\begin{itemize}

\item 任何整数表达式可以作为下标。

\item 试图读写一个不存在的元素将得到{\tt 下标错误}。

\index{异常!下标错误}
\index{下标错误}

\item 下标可以取负数,它将从后往前访问列表。

\end{itemize}

\index{列表!下标}


\index{列表!成员}
\index{成员!列表}
\index{in运算符}
\index{运算符!in}

{\tt in}运算符同样使用列表。

\beforeverb
\begin{verbatim}
>>> cheeses = ['Cheddar', 'Edam', 'Gouda']
>>> 'Edam' in cheeses
True
>>> 'Brie' in cheeses
False
\end{verbatim}
\afterverb


\section{遍历列表}
\index{列表!遍历}
\index{遍历!列表}
\index{for循环}
\index{循环!for}
\index{语句!for}

最常用的遍历列表的方式是使用{\tt for}循环。语法类似字符串:

\beforeverb
\begin{verbatim}
for cheese in cheeses:
    print cheese
\end{verbatim}
\afterverb
%
这种写法适用于只读列表中的元素。如果你需要写或者更新元素,你需要通过下标访问。一个常用的做法是结合{\tt range}和{\tt len}函数:

\index{循环!下标}
\index{下标!循环}

\beforeverb
\begin{verbatim}
for i in range(len(numbers)):
    numbers[i] = numbers[i] * 2
\end{verbatim}
\afterverb
%
这个循环遍历列表并更新每个元素。{\tt len}函数返回列表中元素个数。{\tt range}函数返回一个从0到$n-1$的下标的列表,其中$n$是列表的长度。每次循环中,{\tt i}得到下一个元素的下标。循环主体中的赋值语句使用{\tt i}读取老的值并赋值新的值。

\index{项目更新}
\index{更新!项目}

对于一个空列表的{\tt for}循环将不会执行循环的主体:

\beforeverb
\begin{verbatim}
for x in empty:
    print 'This never happens.'
\end{verbatim}
\afterverb
%
虽然一个列表可以包含另一个列表,被嵌套的列表作为单独的一个元素。以下列表的长度为4:

\index{嵌套列表}
\index{列表!嵌套}

\beforeverb
\begin{verbatim}
['spam', 1, ['Brie', 'Roquefort', 'Pol le Veq'], [1, 2, 3]]
\end{verbatim}
\afterverb



\section{列表操作}
\index{列表!操作}

运算符{\tt +}连接列表:

\index{连接!列表}
\index{列表!连接}

\beforeverb
\begin{verbatim}
>>> a = [1, 2, 3]
>>> b = [4, 5, 6]
>>> c = a + b
>>> print c
[1, 2, 3, 4, 5, 6]
\end{verbatim}
\afterverb
%
类似的,运算符{\tt *}给定次数地重复列表:

\index{重复!列表}
\index{列表!重复}

\beforeverb
\begin{verbatim}
>>> [0] * 4
[0, 0, 0, 0]
>>> [1, 2, 3] * 3
[1, 2, 3, 1, 2, 3, 1, 2, 3]
\end{verbatim}
\afterverb
%
第一个例子重复{\tt [0]}4次。第二个例子重复列表{\tt [1, 2, 3]}3次。


\section{列表切片}

\index{切片运算符}
\index{运算符!切片}
\index{下标!切片}
\index{列表!切片}
\index{切片!列表}

切片运算符同样适用于列表:

\beforeverb
\begin{verbatim}
>>> t = ['a', 'b', 'c', 'd', 'e', 'f']
>>> t[1:3]
['b', 'c']
>>> t[:4]
['a', 'b', 'c', 'd']
>>> t[3:]
['d', 'e', 'f']
\end{verbatim}
\afterverb
%
如果你忽略第一个下标,切片从列表头开始。如果你忽略第二个,切片到列表尾部结束。因此如果你忽略两个,切片为整个列表的拷贝。

\index{列表!复制}
\index{切片!复制}
\index{复制!切片}

\beforeverb
\begin{verbatim}
>>> t[:]
['a', 'b', 'c', 'd', 'e', 'f']
\end{verbatim}
\afterverb
%
由于列表是可改变的,有必要在折叠、旋转或切断操作前复制列表。

\index{可改变}

赋值语句左边的切片运算符可以更新多个元素:

\index{slice!update}
\index{update!slice}

\beforeverb
\begin{verbatim}
>>> t = ['a', 'b', 'c', 'd', 'e', 'f']
>>> t[1:3] = ['x', 'y']
>>> print t
['a', 'x', 'y', 'd', 'e', 'f']
\end{verbatim}
\afterverb
%

% You can add elements to a list by squeezing them into an empty
% slice:

% \beforeverb
% \begin{verbatim}
% >>> t = ['a', 'd', 'e', 'f']
% >>> t[1:1] = ['b', 'c']
% >>> print t
% ['a', 'b', 'c', 'd', 'e', 'f']
% \end{verbatim}
% \afterverb
%
% And you can remove elements from a list by assigning the empty list to
% them:

% \beforeverb
% \begin{verbatim}
% >>> t = ['a', 'b', 'c', 'd', 'e', 'f']
% >>> t[1:3] = []
% >>> print t
% ['a', 'd', 'e', 'f']
% \end{verbatim}
% \afterverb
%
% But both of those operations can be expressed more clearly
% with list methods.


\section{列表方法}

\index{列表!方法}
\index{方法,列表}

Python提供了一个列表的方法。例如,{\tt append}方法将新的元素添加到列表尾部:

\index{append方法}
\index{方法!append}

\beforeverb
\begin{verbatim}
>>> t = ['a', 'b', 'c']
>>> t.append('d')
>>> print t
['a', 'b', 'c', 'd']
\end{verbatim}
\afterverb
%
{\tt extend}方法读取一个列表作为参数,并附加其中所有的元素:

\index{extend方法}
\index{方法!extend}

\beforeverb
\begin{verbatim}
>>> t1 = ['a', 'b', 'c']
>>> t2 = ['d', 'e']
>>> t1.extend(t2)
>>> print t1
['a', 'b', 'c', 'd', 'e']
\end{verbatim}
\afterverb
%
这个例子中{\tt t2}没有改变。

{\tt sort}方法从小到大对列表中的元素进行排序:

\index{sort方法}
\index{方法!sort}

\beforeverb
\begin{verbatim}
>>> t = ['d', 'c', 'e', 'b', 'a']
>>> t.sort()
>>> print t
['a', 'b', 'c', 'd', 'e']
\end{verbatim}
\afterverb
%
列表的方法都是空的,它们对列表进行修改并返回{\tt None}。如果你写了{\tt t = t.sort()},你不会得到你想要的结果。

\index{空方法}
\index{方法!空}
\index{特殊值None}
\index{特殊值!None}


\section{映射,筛选和归并}

对列表中所有元素求和,你可以这么使用循环:

% see add.py

\beforeverb
\begin{verbatim}
def add_all(t):
    total = 0
    for x in t:
        total += x
    return total
\end{verbatim}
\afterverb
%
{\tt total}被初始化为0。每次经过循环,{\tt x}从列表中读取一个元素。运算符{\tt +=}提供可一个快捷的更新变量的方法。这是{\bf 增量赋值语句}:

\index{更新运算符}
\index{运算符!更新}

\index{赋值!增量}
\index{增量赋值}

\beforeverb
\begin{verbatim}
    total += x
\end{verbatim}
\afterverb
%
相当于:

\beforeverb
\begin{verbatim}
    total = total + x
\end{verbatim}
\afterverb
%
当循环执行时,{\tt total}记录了元素的和。这样的变量称为{\bf 累加器}。

\index{累加器!和}

对列表中元素求和是一个普通的操作,Python提供了内建函数{\tt sum}:

\beforeverb
\begin{verbatim}
>>> t = [1, 2, 3]
>>> sum(t)
6
\end{verbatim}
\afterverb
%
类似将一个序列的元素合并到一个单一的数值的操作称为{\bf 归并}。

\index{归并模式}
\index{模式!归并}
\index{遍历}


有时你需要遍历一个列表来构建另一个列表。例如,下面的函数读取一个字符串列表作为参数,返回大写后的新列表:

\beforeverb
\begin{verbatim}
def capitalize_all(t):
    res = []
    for s in t:
        res.append(s.capitalize())
    return res
\end{verbatim}
\afterverb
%
{\tt res}被初始化为一个空的列表。每次循环中,我们附加下一个元素。因此{\tt res}是另一种累加器。

\index{累加器!列表}
类似\verb"capitalize_all"有时被称为{\bf 映射},因为它对序列中的每个元素“映射”某个函数(在本例中是方法{\tt capitalize})。

\index{映射模式}
\index{模式!映射}
\index{筛选模式}
\index{模式!筛选}

另一个常见的操作是从列表中选择一些元素,并返回一个子列表。例如,下面的函数读取一个字符串列表,并返回仅包含大写字符串的列表:

\beforeverb
\begin{verbatim}
def only_upper(t):
    res = []
    for s in t:
        if s.isupper():
            res.append(s)
    return res
\end{verbatim}
\afterverb
%
{\tt isupper}是一个字符串的方法,如果字符串仅含有大写字母,则返回{\tt True}。

类似\verb"only_upper"的操作被称为{\bf 筛选},它选出一部分元素,而过滤其他元素。

绝大多数列表操作可以表示为映射、筛选和归并的组合。由于这些操作很常用,Python提供了语言特点的支持,包括内建函数{\tt map}和一个称为“列表理解”的运算符。

\index{列表!理解}

\begin{ex}
\label{累积}
\index{累积求和}

编写函数,读取一个数字列表作为参数,返回累积求和值。即第$i$个元素是原列表中前$i+1$个元素的和。例如,{\tt [1, 2, 3]}的累积求和为{\tt [1, 3, 6]}。
\end{ex}


\section{删除元素}

\index{元素删除}
\index{删除,列表中的元素}

有多种方法从列表中删除一个元素。如果你知道元素的下标,你可以使用{\tt pop}:

\index{pop方法}
\index{方法!pop}

\beforeverb
\begin{verbatim}
>>> t = ['a', 'b', 'c']
>>> x = t.pop(1)
>>> print t
['a', 'c']
>>> print x
b
\end{verbatim}
\afterverb
%
{\tt pop}修改列表,并返回被删除的元素。如果你不提供下标,它将删除最后一个元素。

如果你不需要被删除的值,你可以使用{\tt del}运算符:

\index{del运算符}
\index{运算符!del}

\beforeverb
\begin{verbatim}
>>> t = ['a', 'b', 'c']
>>> del t[1]
>>> print t
['a', 'c']
\end{verbatim}
\afterverb
%
如果你知道你要删除的元素(但不知道下标),你可以使用{\tt remove}:

\index{remove方法}
\index{方法!remove}

\beforeverb
\begin{verbatim}
>>> t = ['a', 'b', 'c']
>>> t.remove('b')
>>> print t
['a', 'c']
\end{verbatim}
\afterverb
%
{\tt remove}的返回值是{\tt None}。

\index{特殊值None}
\index{特殊值!None}

要删除多于一个元素,你可以对{\tt del}使用切片下标:

\beforeverb
\begin{verbatim}
>>> t = ['a', 'b', 'c', 'd', 'e', 'f']
>>> del t[1:5]
>>> print t
['a', 'f']
\end{verbatim}
\afterverb
%
照常,切片选择从第一个下标到第二个下标(不包括第二个下标)中的所有元素。


\section{列表和字符串}

\index{列表}
\index{字符串}
\index{序列}

字符串是字符的序列,列表是值的序列,但是字符的列表不同于字符串。你可以使用{\tt list}将字符串转化为字符的列表:

\index{list!函数}
\index{函数!list}

\beforeverb
\begin{verbatim}
>>> s = 'spam'
>>> t = list(s)
>>> print t
['s', 'p', 'a', 'm']
\end{verbatim}
\afterverb
%
由于{\tt list}是内建函数名,你应该避免将它作为变量名。我同样避免使用{\tt l}因为它看起来像{\tt 1}。于是我用了{\tt t}。

{\tt list}函数将字符串分割成单独的字符。如果你要将字符串分割成单词,你可以使用{\tt split}方法:

\index{split方法}
\index{方法!split}

\beforeverb
\begin{verbatim}
>>> s = 'pining for the fjords'
>>> t = s.split()
>>> print t
['pining', 'for', 'the', 'fjords']
\end{verbatim}
\afterverb
%
一个可选的参数称为{\bf 分割符},它指定了什么字符作为分界线。下面的例子使用连字符作为分割符:

\index{可选参数}
\index{参数!可选}
\index{分割服}

\beforeverb
\begin{verbatim}
>>> s = 'spam-spam-spam'
>>> delimiter = '-'
>>> s.split(delimiter)
['spam', 'spam', 'spam']
\end{verbatim}
\afterverb
%
{\tt join}功能和{\tt split}相反。它将一个列表字符串连接起来。{\tt join}是一个字符串方法,因此你需要在一个分割符上调用它,并传递一个列表作为参数:

\index{join方法}
\index{方法!join}
\index{连接}

\beforeverb
\begin{verbatim}
>>> t = ['pining', 'for', 'the', 'fjords']
>>> delimiter = ' '
>>> delimiter.join(t)
'pining for the fjords'
\end{verbatim}
\afterverb
%
在这个例子中分割符是一个空格,{\tt join}在每个单词中间添加一个空格。如果不需要使用空格连接,你可以使用一个空的字符串\verb"''"作为分割符。

\index{空字符串}
\index{字符串!空}


\section{对象和值}

\index{对象}
\index{值}

如果我们执行以下的赋值语句:

\beforeverb
\begin{verbatim}
a = 'banana'
b = 'banana'
\end{verbatim}
\afterverb
%
我们知道{\tt a}和{\tt b}都指向一个字符串,但我们不知道他们是否指向一个{\em 相同的}字符串。有些两种可能:

\index{别名}

\beforefig
\centerline{\includegraphics{figs/list1.eps}}
\afterfig

在第一种情况中,{\tt a}和{\tt b}指向两个有相同值的不同对象。在第二种情况中,它们指向同一个对象。

\index{is运算符}
\index{运算符!is}

你可以使用{\tt is}运算符检查两个变量是否指向同一个对象:

\beforeverb
\begin{verbatim}
>>> a = 'banana'
>>> b = 'banana'
>>> a is b
True
\end{verbatim}
\afterverb
%
在这个例子中,Python只产生了一个字符串对象,{\tt a}和{\tt b}都指向它。

但是如果你创建两个列表,你得到两个对象:

\beforeverb
\begin{verbatim}
>>> a = [1, 2, 3]
>>> b = [1, 2, 3]
>>> a is b
False
\end{verbatim}
\afterverb
%
状态图看起来是这样的:

\index{状态图}
\index{图!状态}

\beforefig
\centerline{\includegraphics{figs/list2.eps}}
\afterfig

在本例中,我们称这两个列表是{\bf 相等的},因为它们有相同的元素,但不是{\bf 相同的},因为它们不是同一个对象。如果两个对象是相同的,它们也是相等的,但是如果它们是相等的,它们不一定相同。

\index{相等}
\index{相同}

目前为止,我们可以交换地使用“对象”和“值”,但更精确地说是对象包含一个值。如果你执行{\tt [1,2,3]},你会得到一个整数序列的对象。如果另一个列表有相同的元素,我们称它有相同的值,但它不是相同的对象。

\index{对象}
\index{值}


\section{别名}

\index{别名}
\index{引用!别名}

如果{\tt a}指向一个对象,然后你进行赋值{\tt b = a},那么两个变量都指向同一个对象:

\beforeverb
\begin{verbatim}
>>> a = [1, 2, 3]
>>> b = a
>>> b is a
True
\end{verbatim}
\afterverb
%
状态图如图所示:

\index{状态图}
\index{图!状态}

\beforefig
\centerline{\includegraphics{figs/list3.eps}}
\afterfig

一个变量和一个对象的关联称为{\bf 引用}。在这个例子中,同一个对象有两个引用。

\index{引用}

如果一个对象有多于一个引用,我们称这个对象是有{\bf 别名}的。

\index{可改变}
有别名的对象是可改变的,对一个别名的改动会影响另一个:

\beforeverb
\begin{verbatim}
>>> b[0] = 17
>>> print a
[17, 2, 3]
\end{verbatim}
\afterverb
%
这个行为虽然很有用,但容易造成错误。通常,对于可改变的对象避免使用别名相对更安全。

\index{不可改变}

对于不可改变的对象,如字符串,使用别名没有任何问题。例如:

\beforeverb
\begin{verbatim}
a = 'banana'
b = 'banana'
\end{verbatim}
\afterverb
%
使用{\tt a}或者{\tt b}指向同一个字符串基本上没有任何区别。


\section{列表参数}

\index{列表!作为参数}
\index{参数}
\index{参数!列表}
\index{引用}
\index{参数}

当你将一个列表作为参数传给一个函数,函数将得到这个列表的一个引用。如果函数对这个列表参数进行了修改,调用者会看见变动。例如,\verb"delete_head"删除列表的第一个元素:

\beforeverb
\begin{verbatim}
def delete_head(t):
    del t[0]
\end{verbatim}
\afterverb
%
它是这么使用的:

\beforeverb
\begin{verbatim}
>>> letters = ['a', 'b', 'c']
>>> delete_head(letters)
>>> print letters
['b', 'c']
\end{verbatim}
\afterverb
%
参数{\tt t}和变量{\tt letters}是同一个对象的别名。栈图如下:

\index{栈图}
\index{图!栈}

\beforefig
\centerline{\includegraphics{figs/stack5.eps}}
\afterfig

由于列表被两个帧共享,我把它画在它们中间。

需要注意的是修改列表操作和创建列表操作间的区别,例如,{\tt append}方法是修改一个列表,而{\tt +}运算符是创建一个新的列表:

\index{append方法}
\index{方法!append}
\index{列表!连接}
\index{连接!列表}

\beforeverb
\begin{verbatim}
>>> t1 = [1, 2]
>>> t2 = t1.append(3)
>>> print t1
[1, 2, 3]
>>> print t2
None

>>> t3 = t1 + [3]
>>> print t3
[1, 2, 3]
>>> t2 is t3
False
\end{verbatim}
\afterverb

如果你要编写函数修改列表,这个区别就很重要。例如,下面函数{\em 没有}删除列表的第一个元素:

\beforeverb
\begin{verbatim}
def bad_delete_head(t):
    t = t[1:]              # WRONG!
\end{verbatim}
\afterverb
切片操作创建一个新的列表,并使{\tt t}指向它。但这些操作对作为参数的列表都没有影响。

\index{切片运算符}
\index{运算符!切片}

一个替代的写法是创建并返回一个新的列表。例如,{\tt tail}返回不包含第一个元素的列表:

\beforeverb
\begin{verbatim}
def tail(t):
    return t[1:]
\end{verbatim}
\afterverb
%
这个函数并不修改原来的列表。下面给出如何使用这个函数:

\beforeverb
\begin{verbatim}
>>> letters = ['a', 'b', 'c']
>>> rest = tail(letters)
>>> print rest
['b', 'c']
\end{verbatim}
\afterverb


\begin{ex}

编写函数{\tt chop},读取一个列表并进行修改,删除第一个和最后一个元素,并返回{\tt None}。

再编写函数{\tt middle},读取一个列表作为参数,返回一个包含除了第一个和最后一个元素的新列表。

\end{ex}


\section{调试}
\index{调试}

粗心的使用列表(以及其他可改变的对象)会导致长时间的调试。下面给出一些常见的陷阱以及避免它们的方法:

\begin{enumerate}

\item 记住大多数的列表的方法对参数进行修改,然后返回{\tt None}。字符串的方法则相反,它们保留原始的字符串并返回一个新的字符串。

如果你习惯编写处理字符串的代码,如:

\beforeverb
\begin{verbatim}
word = word.strip()
\end{verbatim}
\afterverb

你可能会写出下面的代码:

\beforeverb
\begin{verbatim}
t = t.sort()           # WRONG!
\end{verbatim}
\afterverb

\index{sort方法}
\index{方法!sort}

由于{\tt sort}返回{\tt None},你对{\tt t}的下一个操作可能会失败。

在使用列表的方法和运算符前,你应该仔细阅读文档,并在交互模式下测试。列表和其他序列(如字符串)共有的方法和运算符的文档在\url{docs.python.org/lib/typesseq.html}。可改变的序列独有的方法和运算符的文档在\url{docs.python.org/lib/typesseq-mutable.html}。


\item 养成自己的编码习惯。

列表中的一个问题是有太多的途径做相同的事。例如,要删除列表中的一个元素,你可以使用{\tt pop},{\tt remove},{\tt del}甚至切片赋值。

要添加一个元素,你可以使用{\tt append}方法或{\tt +}运算符。但记住什么是正确的:

\beforeverb
\begin{verbatim}
t.append(x)
t = t + [x]
\end{verbatim}
\afterverb

以下是错误的:

\beforeverb
\begin{verbatim}
t.append([x])          # WRONG!
t = t.append(x)        # WRONG!
t + [x]                # WRONG!
t = t + x              # WRONG!
\end{verbatim}
\afterverb

在交互模式下测试每一个例子,保证你明白它们做了什么。注意只有最后一个会导致运行时错误,其他的都是合法的,但做了错误的事情。


\item 复制拷贝,避免别名。

\index{别名!复制以避免}
\index{赋值!以避免别名}

如果你要使用类似{\tt sort}的方法来修改参数,但同时有要保留原列表,你可以复制一个拷贝。

\beforeverb
\begin{verbatim}
orig = t[:]
t.sort()
\end{verbatim}
\afterverb

在这个例子中你还可以使用内建函数{\tt sorted},它将返回一个新的已排序的列表,原列表将保持不变。注意你需要避免使用{\tt sorted}作为变量名!

\end{enumerate}



\section{术语}

\begin{description}

\item[列表:] 一个序列的值。
\index{列表}

\item[元素:] 列表(或序列)中的一个值,也称为项目。
\index{元素}

\item[下标:] 对应列表中的元素的整数。
\index{下标}

\item[嵌套列表:] 列表中的元素是另一个列表。
\index{嵌套列表}

\item[列表遍历:] 对列表中的元素按顺序访问。
\index{列表!遍历}

\item[映射:] 一个集合中的元素和另一个集合中的元素的对应关系。例如,列表是下标到元素的映射。
\index{映射}

\item[累加器:] 循环中用于相加或累积的变量。
\index{累加器}

\item[增量赋值:] 更新变量的语句,使用类似\verb"+="的运算符。
\index{赋值!增量}
\index{增量赋值}

\index{遍历}

\item[归并:] 遍历序列,将所有元素求和为一个值的处理模式。
\index{归并模式}
\index{模式!归并}

\item[映射:] 遍历序列,对每个元素执行操作的处理模式。
\index{映射模式}
\index{模式!映射}

\item[筛选:] 遍历序列,选出满足一定标准的处理模式。
\index{筛选模式i}
\index{模式!筛选}

\item[对象:] 变量可以指向的东西。一个对象有其数据类型和值。
\index{对象}

\item[相等:] 有相同的值。
\index{相等}

\item[相同:] 是用一个对象(隐含相等)。
\index{相同}

\item[引用:] 变量和值间的关联。
\index{引用}

\item[别名:] 两个或两个以上的变量指向同一个对象。
\index{别名}

\item[分割符:] 用于指示字符串分割位置的字符或者字符串。
\index{分割符}

\end{description}


\section{练习}

\begin{ex}
编写函数\verb"is_sorted",读取一个列表作为参数,如果列表是升序排序的,则返回{\tt True},否则返回{\tt False}。你可以假设(作为先决条件)列表中的元素可以用关系运算符如{\tt <},{\tt >}等比较。

\index{先决条件}

例如,\verb"is_sorted([1,2,2])"将返回{\tt True},\verb"is_sorted(['b','a'])"将返回{\tt False}。
\end{ex}


\begin{ex}
\label{回文}

\index{回文}

两个单词是回文的,如果你可以重新排列一个的字符后可以拼写出另一个。编写函数\verb"is_anagram",读取两个字符串,如果它们是回文的则返回{\tt True}。
\end{ex}


\begin{ex}
\label{复制}

这也被称为生日悖论:

\begin{enumerate}

\index{生日悖论}
\index{赋值}

\item 编写函数\verb"has_duplicates",读取一个列表作为参数,如果任何元素出现超过一次,则返回{\tt True}。函数不能改变原列表。

\item 如果你班级上有23个学生,2个学生生日相同的概率是多少?你可以通过随即产生23个生日并检查匹配来估计概率。提示:你可以使用{\tt random}模块中的{\tt randint}函数来生成随即生日。

\index{random模块}
\index{模块!random}
\index{randint函数}
\index{函数!randint}

\end{enumerate}

你可以在\url{wikipedia.org/wiki/Birthday_paradox}了解这个问题,你可以在\url{thinkpython.com/code/birthday.py}找到我的程序。

\end{ex}


\begin{ex}

\index{赋值}
\index{唯一}

编写函数\verb"remove_duplicates",参数为一个列表,返回一个新的列表,其中只包含原列表中唯一的元素。提示:列表中的元素不一定按照原来的顺序。
\end{ex}


\begin{ex}
\index{append方法}
\index{方法append}
\index{列表!连接}
\index{连接!列表}

编写函数,读取文件{\tt words.txt},建立一个列表,每个单词为一个元素。编写两个版本函数,一个使用{\tt append}方法,另一个使用{\tt t = t + [x]}。那个版本运行得慢?为什么?

你可以在\url{thinkpython.com/code/wordlist.py}中找到我的程序。
\end{ex}


\begin{ex}
\label{单词表1}
\label{两分法}

\index{成员!两分法搜索}
\index{两分法搜索}
\index{搜索,两分法}

\index{成员!二进制搜索}
\index{二进制搜索}
\index{搜索,二进制}

检查一个单词是否在单词表中,你可以使用{\tt in}运算符,但这很慢,因为它按顺序查找单词。

由于单词是按照字母顺序排序的,我们可以使用两分法(也称二进制搜索)来加快速度,类似你在字典中查找单词的方法。你从中间开始,如果你要找的单词在中间的单词之前,你查找前半部分,否则你查找后半部分。

每次查找,你将搜索范围减小一半。如果单词表有113,809个单词,你只需要17步来找到这个单词,或着知道单词不存在。

编写函数{\tt bisect},参数为一个已排序的列表和一个目标值,返回该值在列表中的位置,如果不存在则返回{\tt None}。

\index{bisect模块}
\index{模块!bisect}

或者你可以阅读{\tt bisect}模块的文档并使用!
\end{ex}

\begin{ex}
\index{反转词对}

两个单词被称为是“反转词对”,如果一个是另一个的反转。编写函数,找出单词表中所有的反转词对。
\end{ex}

\begin{ex}
\index{连锁词}

两个单词被称为是“连锁词”,如果交替的从两个单词中取出字符将组成一个新的单词\footnote{这个练习来自\url{puzzlers.org}中的一个例子。}。例如,“shoe”和“cold”连锁后成为“schooled”。

\begin{enumerate}

\item 编写程序,找出所有的连锁词。提示:不用列举所有的单词对。

\item 你能够找到三重连锁的单词吗?即每个字母依次从3个单词得到。

\end{enumerate}
\end{ex}


\chapter{字典}
\chapter{字典}
\index{dictionary 字典}

\index{dictionary 字典}
\index{type!dict}
\index{key 关键字}
\index{key-value pair 关键字-值对}
\index{index 索引}

字典像列表一样,但是更一般。列表的索引必须是整数,但字典的索引(键)几乎可以是任何类型。\\

可以把字典当作是索引集合(关键字)和值集合之间的映射。每一个关键字对应
一个值。关键字和对应的值称为键-值对,或者项。

我们将构造一个英语单词及其对应西班牙语单词的字典,关键字和关键字值都
是字符串。\\

{\tt dict}函数创建一个空字典。由于{\tt dict}是内建函数名,所以,我们
应该避免使用它作为变量名。\\

\index{dict function dict函数}
\index{function!dict}

\beforeverb
\begin{verbatim}
>>> eng2sp = dict()
>>> print eng2sp
{}
\end{verbatim}
\afterverb

大括号\verb"{}",代表空字典。向字典中添加一个项,可以使用方括号:

\index{squiggly bracket 大括号}
\index{bracket!squiggly}

\beforeverb
\begin{verbatim}
>>> eng2sp['one'] = 'uno'
\end{verbatim}
\afterverb

这行代码创建了一个从关键字{\tt 'one'}到值\verb"'uno'"的映射。如果
再次输出字典,可以看到关键字-值对,和他们之间的冒号:

\beforeverb
\begin{verbatim}
>>> print eng2sp
{'one': 'uno'}
\end{verbatim}
\afterverb

上面输出的格式,也是输入格式。比如,可以创建一个拥有三项的字典:

\beforeverb
\begin{verbatim}
>>> eng2sp = {'one': 'uno', 'two': 'dos', 'three': 'tres'}
\end{verbatim}
\afterverb
%

但是如果输出{\tt eng2sp},你可能会感到惊讶:

\beforeverb
\begin{verbatim}
>>> print eng2sp
{'one': 'uno', 'three': 'tres', 'two': 'dos'}
\end{verbatim}
\afterverb

关键字-值对的顺序和输入的不一样。事实上,如果在读者的机器上尝试这个例子,
也可能得到不同的结果。一般来说,字典项的顺序是随机的。\\

但这个也不会有什么问题,因为字典的元素是不是通过索引来获取的。可以使用
关键字来查询对应的值:

\beforeverb
\begin{verbatim}
>>> print eng2sp['two']
'dos'
\end{verbatim}
\afterverb

关键字{\tt 'two'}总是对应值\verb"'dos'",所以项的顺序没有什么关系。

如果关键字不在字典中,就会抛出异常:

\index{exception!KeyError}
\index{KeyError}

\beforeverb
\begin{verbatim}
>>> print eng2sp['four']
KeyError: 'four'
\end{verbatim}
\afterverb

{\tt len}函数对于字典也是适用的。它返回键-值对的数目:

\index{len function len函数}
\index{function!len}

\beforeverb
\begin{verbatim}
>>> len(eng2sp)
3
\end{verbatim}
\afterverb

{\tt in}运算符对字典也同样使用。它显示某个键是否在字典中作为关键字(.

\index{membership!dictionary}
\index{in operator in运算符}
\index{operator!in}


\beforeverb
\begin{verbatim}
>>> 'one' in eng2sp
True
>>> 'uno' in eng2sp
False
\end{verbatim}
\afterverb

如果想查看某个值是否在字典中,可以使用方法{\tt values},返回包含关键字值
的列表,然后使用{\tt in}运算符:

\index{values method values方法}
\index{method!values}

\beforeverb
\begin{verbatim}
>>> vals = eng2sp.values()
>>> 'uno' in vals
True
\end{verbatim}
\afterverb

{\tt in}运算符操作列表和字典时使用不同的算法。对于列表,使用搜索算法,,
参考\ref{find}部分。随着列表变长,搜索时间成比例增加;最于字典,Python
使用{\bf 散列}算法,带来一个很显著的效果:无论字典有多少项,{\tt in}
运算符花费近乎同样的时间。在这里,我不对此做出更多的解释,详情参考
\url{wikipedia.org/wiki/Hash_table} 。

\index{hashtable 散列}

\begin{ex}
\label{wordlist2}

\index{set membership}
\index{membership!set}

编写函数,读取{\tt words.txt}文件里的单词,把他们作为字典键存储在字典里,
关键字值随便是什么。然后,使用{\tt in}运算符查看某个字符串是否在字典里。\\


如果做了练习\ref{wordlist1},可以比较一下这个实现和列表{\tt
	in}运算符与二分搜索的速度。

\end{ex}

\section{把字典作为计数器}
\label{histogram}

\index{counter 计数器}


假设给定一个字符串,统计每个字母出现的次数。可以有好几种:

\begin{enumerate}

\item
创建26个变量,每个代表一个字母。然后遍历字符串,对每一个字母,增加对应对应的计数器,可以使用链条件语句。

\item 船舰一个26元素的列表。然后把每个字符变换为数字(使用内建{\tt ord}函数),
把数字作为列表的索引,增加相应的计数器。

\item 创建一个字典,字母作为关键字,计数器作为对应的值。第一次遇到衣蛾
字符,把它加入字典。然后可以增加相应项的值。

\end{enumerate}


以上的每个方法实现同样的计算,但是实现的方法不同。

\index{implementation 实现}

实现是实施计算的一种方法,存在某些实现比其他的要好。比如, 用字典实现
的好处是我们不必要实现知道哪个字母会出现在字符串里,我们只需为出现的
字母分配空间。

下面是字典实现的代码:

\beforeverb
\begin{verbatim}
def histogram(s):
    d = dict()
    for c in s:
        if c not in d:
            d[c] = 1
        else:
            d[c] += 1
    return d
\end{verbatim}
\afterverb

函数名是{\bf histogram},是统计学术语,用来直观表示频率。

\index{histogram 直方图}
\index{frequency 频率}
\index{traversal 遍历}

函数的第一行,创建了一个空字典。{\tt for}循环遍历字符串。每次循环,如果字符{\tt
	c}
	不在字典里,我们创建一个键为{\tt c},初值为1的项。如果{\tt c}
	已经在字典里,我们增加{\tt d[c]}的值。

\index{histogram 直方图}

看看它是如何工作的:

\beforeverb
\begin{verbatim}
>>> h = histogram('brontosaurus')
>>> print h
{'a': 1, 'b': 1, 'o': 2, 'n': 1, 's': 2, 'r': 2, 'u': 2, 't': 1}
\end{verbatim}
\afterverb

直方图({\tt histogram})显示,字母{\tt
	'a'}和\verb"'b'"出现一次,\verb"'o'"出现两次,等等。


\begin{ex}

\index{get method get方法}
\index{method!get}

字典有一个{\tt get}函数,接受一个键和一个缺省值。如果键在字典里,{\tt
	get}返回对应的值;否则返回缺省值。比如:

\beforeverb
\begin{verbatim}
>>> h = histogram('a')
>>> print h
{'a': 1}
>>> h.get('a', 0)
1
>>> h.get('b', 0)
0
\end{verbatim}
\afterverb

使用{\tt get}编写一个精巧的{\tt histogram}函数。应该不使用{\tt if}语句就可
以实现。
\end{ex}


\section{循环和字典}

\index{dictionary!looping with}
\index{looping!with dictionaries}
\index{traversal 遍历}

如果在{\tt
	for}语句中使用字典,程序遍历字典的关键字。比如,\verb"print_hist"输出每个关键字和对应的值:


\beforeverb
\begin{verbatim}
def print_hist(h):
    for c in h:
        print c, h[c]
\end{verbatim}
\afterverb

下面是输出结果:

\beforeverb
\begin{verbatim}
>>> h = histogram('parrot')
>>> print_hist(h)
a 1
p 1
r 2
t 1
o 1
\end{verbatim}
\afterverb

可以再一次看到,关键字是无序的。


\begin{ex}

\index{keys method keys方法}
\index{method!keys}

字典有一个方法{\tt keys},以列表形式返回字典的关键字。

修改\verb"print_hist"以字典顺序\footnote{译注:此处的字典顺序指的是字母顺序,因为英文的字典是按照字母的顺序排列的。}打印关键字和对应的值。

\end{ex}



\section{颠倒查询}

\index{dictionary!lookup}
\index{dictionary!reverse lookup}
\index{lookup,dictionary 查询,字典}
\index{reverse lookup,dictionary 颠倒查询,字典}

给定一个字典{\tt d}和关键字{\tt k},可以很容易的查询对应的值{\tt v = d[k]}。
这个操作叫做查询。

但是如果,给定一个{\tt v},想查找对应的{\tt k}该怎么办?有两个问题:
第一,可能有多个关键字映射到同一个值{\tt
	v}。根据实际应用,可能需要选择其中一个,也可能创建一个列表来容纳所有的关键字。第二,没有一个简单的语法来实施颠倒查询,必须使用搜索。


下面是一个函数,接受一个值,返回第一个对应于该值的关键字:

\beforeverb
\begin{verbatim}
def reverse_lookup(d, v):
    for k in d:
        if d[k] == v:
            return k
    raise ValueError
\end{verbatim}
\afterverb
%

这个函数也是搜索的一个例子,但是使用了一个我们没有见过的特性,{\tt raise}.
{\tt raise}语句引发一个异常;此处,引发一个{\tt ValueError}异常,一般意味着,
参数的值出现了问题。

\index{search}
\index{pattern!search}
\index{raise statement 引发异常}
\index{statement!raise}
\index{exception!ValueError}
\index{ValueError}


如果我们到达了循环的末尾,意味着{\tt v}没有出现在字典关键字值里,所以我们
引发一个异常。

下面是一个成功颠倒查询的例子:

\beforeverb
\begin{verbatim}
>>> h = histogram('parrot')
>>> k = reverse_lookup(h, 2)
>>> print k
r
\end{verbatim}
\afterverb

一个查询失败的例子:

\beforeverb
\begin{verbatim}
>>> k = reverse_lookup(h, 3)
Traceback (most recent call last):
  File "<stdin>", line 1, in ?
  File "<stdin>", line 5, in reverse_lookup
ValueError
\end{verbatim}
\afterverb

%

手动引发一个异常和Python引发异常是相同的:输出回溯路径和错误信息。

\index{traceback 回溯}
\index{optional argument}
\index{argument!optional}

{\tt raise}语句接受一个详细的错误信息作为可选参数。比如:

\beforeverb
\begin{verbatim}
>>> raise ValueError, 'value does not appear in the dictionary'
Traceback (most recent call last):
  File "<stdin>", line 1, in ?
ValueError: value does not appear in the dictionary
\end{verbatim}
\afterverb

颠倒查询比正常查询慢很多。如国必须经常颠倒查询,或者当字典变的很大时,程序
的表现会很糟糕。

\begin{ex}
修改\verb"reverse_lookup",让它一列表形式返回一个对应于值{\tt v}的所有关键字。
, 如果没有,返回一个空的列表。
\end{ex}



\section{字典和列表}

 列表可以作为字典的关键字值。比如,给定一个从字母到频率映射的字典,可能
 想反转它,也就是说,创建一个从频率到字母映射的字典。因为可能有好几个字母
 的频率是一样的,所以,反转字典的关键字值应该表示成由字母构成的列表。

 \index{invert dictionary 反转字典}
 \index{dictionary!invert}

 下面是一个反转字典的函数:

 \beforeverb
\begin{verbatim}
def invert_dict(d):
    inv = dict()
    for key in d:
        val = d[key]
        if val not in inv:
            inv[val] = [key]
        else:
            inv[val].append(key)
    return inv
\end{verbatim}
\afterverb


每次循环,{\tt key}从{\tt d}得到一个关键字,{\tt val}得到对应的值。如果{\tt val}
不在{\tt inv}里,意味着,我们之前还没有见过它,所以我们创建一个
新的项,用包含一个值的列表初始化它。否则,我们把对应的关键字追加到列表里。

\index{singleton }

下面是一个例子:

\beforeverb
\begin{verbatim}
>>> hist = histogram('parrot')
>>> print hist
{'a': 1, 'p': 1, 'r': 2, 't': 1, 'o': 1}
>>> inv = invert_dict(hist)
>>> print inv
{1: ['a', 'p', 't', 'o'], 2: ['r']}
\end{verbatim}
\afterverb

这里有个图表,显示了{\tt hist}和{\tt inv}的变化:

\index{state diagram 状态图}
\index{diagram!state}

\beforefig
\centerline{\includegraphics{figs/dict1.eps}}
\afterfig

字典用盒子来表示,类型{\tt dict}在盒子上方,键-值对在盒子里面。
如果值为整型,浮点型或者字符串,通常画在盒子里面,如果是列表,则画在
盒子外面,这样做仅仅是为了保持图的简洁。

列表可以作为字典的关键字值,正如上面的例子演示的。但是,不能作为字典的关键字。请看下面:

\index{TypeError}
\index{exception!TypeError}


\beforeverb
\begin{verbatim}
>>> t = [1, 2, 3]
>>> d = dict()
>>> d[t] = 'oops'
Traceback (most recent call last):
  File "<stdin>", line 1, in ?
TypeError: list objects are unhashable
\end{verbatim}
\afterverb
%

我在之前提到过,字典是用散列方法实现的,这就意味着,关键字必须是可散列的。

\index{hash function 散列函数}
\index{hashable 可散列的}

散列函数,接受一个任何类型的值,返回一个整数。字典使用这些
整数(散列值)存储,查询键-值对。

\index{immutability 不可变性}


如果关键字是不可变的,则一切正常。但是,如果关键字是可变的,比如
列表,麻烦来了。比如,当创建一个键-值对,Python散列关键字,并且
把它存放在相应的位置。如果修改关键字,然后再一次散列,就会散列到
另外一个位置。这种情况下,将会得到两个有着相同键的项,或者可能
无法获取关键字。无论那种情况,字典都不能正常工作。

这就是为什么关键字必须是可散列的,可变数据类型像列表不能作为关键字的原因。最
最简单的解决方式是使用元组,我们在下一章将会遇到。

虽然字典是可变的,不能作为关键字,但是可以作为关键字值。

\begin{ex}
查阅字典方法{\tt setdefault}的文档,用它编写一个更简洁的\verb"invert_dict"。

\index{setdefault method setdefault方法}
\index{method!setdefault}

\end{ex}




\section{备忘录}

如果尝试过\ref{另外一个例子}部分的{\tt fibonacci}函数,可能会
注意到,提供的参数越大,函数运行花费的时间越长。确切地说,运行时间增长
长的很快。

\index{fibonacci function fibonacci函数}
\index{function!fibonacci}

为了一探究竟,看看这个调用图,其中{\tt n = 4}:

\beforefig
\centerline{\includegraphics[height=2in]{figs/fibonacci.eps}}
\afterfig


调用图包含了一系列函数框图,每个框图及其调用的函数框图用直线连接。
在调用图的顶部,{\tt n=4}的{\tt fibonacci}调用{\tt n=3} 的
{\tt fibonacci}和{\tt n=2}的{\tt fibonacci}。依次地,{\tt n=3}的{\tt fibonacci}
调用{\tt n=2}和{\tt n=1}的{\tt fibonacci}函数......

\index{function frame 函数框图}
\index{frame}
\index{call graph}

计算一下{\tt fibonacci(0)}和{\tt fibonaci(1)}分别被调用了多少次。可以看出,
这不是解决这个问题的高效方式,并且随着参数变大,效率会更低。

\index{memo 备忘录}

另外一种方法就是跟踪已经被计算的值---把它存储在字典里。存储已经计算的留作后用叫做备忘录\footnote{参考\url{wikipedia.org/wiki/Memoization}。}。下面是使用备忘录实现
的{\tt fibonacci}:

\beforeverb
\begin{verbatim}
known = {0:0, 1:1}

def fibonacci(n):
    if n in known:
        return known[n]

    res = fibonacci(n-1) + fibonacci(n-2)
    known[n] = res
    return res
\end{verbatim}
\afterverb

{\tt known}是一个字典,跟踪我们已经计算出的Fibonacci数字。起始项为:
0映射到0,1映射到1。

无论何时调用{\tt fibonacci},函数检查{\tt known}字典。如果字典包含需要的结果,
函数立刻返回,否则函数计算新值,并把它加入字典,然后返回。

\begin{ex}
运行此版本的{\tt fibonacci}函数和原先的版本,多传递几个参数,然后比较运行的时间。
\end{ex}


\section{全局变量}

\index{global variable 全局变量}
\index{variabl!global 全局}

在以往的例子中,我们在函数外面创建{\tt
	known},所以它属于特殊的框图\verb"__mian__"。
\verb"__main__"内的变量有时称为全局变量,因为任何函数都可以访问它们。不像局部、
变量,在函数返回时销毁,全局变量在函数调用过程中,都是存在的。

\index{flag 标记}

把全局变量作为标记来使用是很常见的,也就是说,布尔变量(“标记”)表明条件是否为真。
比如,有些程序使用{\tt verbose}标记控制输出的层次:

\beforeverb
\begin{verbatim}
verbose = True

def example1():
    if verbose:
        print 'Running example1'
\end{verbatim}
\afterverb

如果试着给一个全局变量重新赋值\footnote{译注:这个地方使用的术语是不准确的。
我们知道Python中使用的是引用,所以不存在赋值,我们可以说重新绑定(rebind)},我们可能会很惊讶。下面的例子跟踪函数是否被
调用:

\index{mutiple assignment 多重赋值}
\index{assignment!multiple 多次}

\beforeverb
\begin{verbatim}
been_called = False

def example2():
    been_called = True         # WRONG
\end{verbatim}
\afterverb

如果运行程序,将会看到\verb"been_called"的值并没有改变。问题在于
{\tt example2}创建了一个新的局部变量\verb"been_called"。局部变量随着函数的终结
而销毁,而对全局变量没有任何影响。

\index{global statement 全局语句}
\index{statement!global 全局}
\index{declaration 声明}

如果要在函数体里为全局变量重新赋值,我们必须在使用前声明全局变量:

\beforeverb
\begin{verbatim}
been_called = False

def example2():
    global been_called 
    been_called = True
\end{verbatim}
\afterverb

{\tt
	global}语句告诉解释器“在这个函数里,当我说\verb"been_called",它就是全局变量,不要创建一个局部变量了。“

\index{update!global variable 全局变量}
\index{global variable!update 更新}

下面是一个更新全局变量值的例子:

\beforeverb
\begin{verbatim}
count = 0

def example3():
    count = count + 1          # WRONG
\end{verbatim}
\afterverb
如果运行它,会得到:

\index{UnboundLocalError}
\index{exception!UnboundLocalError}

\beforeverb
\begin{verbatim}
UnboundLocalError: local variable 'count' referenced before assignment
\end{verbatim}
\afterverb
Python认为{\tt count}是一个局部变量,这也意味着在写之前必须读取它。
解决的方法是,声明{\tt count}为全局变量。

\index{counter 计数器}

beforeverb
\begin{verbatim}
def example3():
    global count
    count += 1
\end{verbatim}
\afterverb
%

如果全局变量是可变的,就可以不声明而修改它:

\index{mutability 可变}

\beforeverb
\begin{verbatim}
known = {0:0, 1:1}

def example4():
    known[2] = 1
\end{verbatim}
\afterverb
%

所以,我们可以添加,删除,替换全局列表和全局字典的元素,但是如果想要
给变量重新赋值,必须声明为全局变量:

\beforeverb
\begin{verbatim}
def example5():
    global known
    known = dict()
\end{verbatim}
\afterverb
%

\section{长整数}

\index{long integer 长整数}
\index{integer!long 长}
\index{type!long 长}

如果计算{\tt fibonacci(50)},我们得到:
\beforeverb
\begin{verbatim}
>>> fibonacci(50)
12586269025L
\end{verbatim}
\afterverb
%

结果尾部的{\tt L}表明该数是一个长整型\footnote{在Python 3.0中,类型{\tt
	long}不存在了;所有整数,即使非常大,也是{\tt int}类型,或者
{\tt long}类型}。

\index{Python 3.0}

{\tt
	int}类型的值是有范围的;长整型值可以是任意大,但是值越大,消耗的空间和时间就越大。

基本数学运算符对长整型适用,{\tt math}模块的函数也如此,所以,一般来说,
{\tt int}的代码也同样适用于{\tt long}。

任何时候,计算结果太大,而整型无法表示的时候,Python自动把结果转换为
长整型:

\beforeverb
\begin{verbatim}
>>> 1000 * 1000
1000000
>>> 100000 * 100000
10000000000L
\end{verbatim}
\afterverb

第一种情况下,结果是{\tt int}型;第二种情况,是{\tt long}型。

\begin{ex}

\index{encryption 加密}
\index{RSA algorithm RSA算法}
\index{algorithm!RSA}

大整数幂是一般公钥加密算法的基础。阅读维基百科中RSA算法页面\footnote{
	\url{wikipedia.org/wiki/RSA}.},然后编写一个函数加密解密信息。
\end{ex}

\section{调试}
\index{debugging 调试}

随着程序数据的增加,通过手动输出并检查数据变得越来越笨拙。下面是关于此种情况
的一些建议:

\begin{description}

\item [缩小输入数据量:]如果可以,尽量减少数据量。比如,如果程序想要读取一个
文本文档,就可以仅仅读取开始的10行,或者其他的你认为比较小的部分。可以修改
文件本身,或者(最好这样)修改程序,让程序只读取文件的前{\tt n}行。

如果出现错误,可以缩小{\tt n}的值,到出现错误的地方,然后逐渐增加它,直到找放到
并更正错误。

\item [检查概要和类型:]不必要输出检查全部数据,考虑输出数据的概要:比如,
字典项的数目或者列表中数字的个数。

一个常见的运行时错误是数据类型错误。调试这种粗无,通常输出值的类型就足够了。

\item [编写自我测试:]有时,可以编写代码自动检查错误。比如,计算列表中数字的
平均数,可以检查结果应该不大于列表的最大元素,不大于最小元素。这个叫做“健康测试”,因为测试发现不健康的结果。


\index{sanity check 健康测试}
\index{consistency check 连贯性测试}

另外一种测试比较两个不同的计算结果,看看他们是否连贯。这叫做“连贯性测试”。

\item
[精巧的打印输出:]格式化调试输出可以更容易发现粗无。参看\ref{factdebug}部分。
{\tt pprint}模块提供{\tt pprint}函数,以人类可读格式显示内置类型。

\index{pretty print 精巧的输出}
\index{pprint module pprint模块}
\index{module!pprint}

\end{description}

再次提醒:花费在搭建“脚手架”的时间越多,用来调试的时间就越少。

\index{scaffolding 脚手架}




\section{术语表}

\begin{description}

\item [dictionary 字典:]键集合到对应值的映射。
\index{dictionary 字典}

\item [key-value pair 键值对:]键--值映射的表示。
\index{key-value pair 键值对}

\item [item 项:]键值对的别名。
\index{item!dictionary 字典}

\item [key 键:] 字典中,键值对的第一个部分。
\index{key 键}

\item [value 关键字值:]字典中,键值对的第二部分。比我们以前使用的单词“值”,
更特殊。
\index{value 关键字值}

\item [implementation 实现:]计算的方式。
\index{implementation 实现}

\item [hashtable 散列表:]实现Python字典的一种算法。
\index{hashtable 散列表}

\item [hash function 散列函数:]计算键位置的函数。
\index{hash function 散列函数}

\item [hashable 散列体:]拥有散列函数的数据类型。不可变类型,像整型,浮点型,
字符串都是散列体;可变类型,比如列表和字典不是。
\index{hashable 散列体}

\item [lookup 查询:]通过关键字查找关键字值的字典操作。
\index{lookup 查询}

\item [reverse lookup 颠倒查询:]通过关键字值查找对应关键字的字典操作。
\index{reverse lookup ,dictionary 颠倒查询,字典}

\item[singleton 独子体]只有一个元素的线性数据结构.
\index{singleton 独子体}

\item [call graph 调用框图:]显示在程序执行时,从调用函数到被调用函数框图的箭
头的图表。
\index{call graph 调用框图}
\index{diagram!call graph 调用框图}

\item [histogram 直方图:]计数器的集合。
index{histogram 直方图}

\item [memo 备忘录:]存储已经计算出的值,留作后用的方法。
\index{memo 备忘录}

\item [global variable 全局变量:]函数体外定义的变量。全局变量可以从任何
函数中访问。
\index{global variable 全局变量}

\item [flag 标记:]表明条件是否成立的布尔变量。
\index{flag 标记}

\item [declaration 声明:] 像{\tt global}语句一样,告诉解释器关于变量的一些
信息。
\index{declaration 声明}

\end{description}



section{练习}

\begin{ex}
\index{duplicate}

如果做了练习\ref{duplicate},已经编写了一个\verb"has_duplicates"函数,它接受
一个列表作为参数,如果有一个对象在列表中出现多次,就返回{\tt True}。

使用字典编写一个更快,更简洁的\verb"has_duplicates"函数。
\end{ex}



\begin{ex}
\label{exroptatepairs}

\index{letter rotation }
\index{rotation!letters}

两个单词,如果“旋转”其中一个可以得到第二个就称为“旋转对”(参看练习\ref{exrotate} \verb"rotate_word")

编写一个程序,读取单词列表,发现所有的旋转对。
\end{ex}



\begin{ex}
\index{Car Talk}
\index{Puzzler 难题}

下面又是一个难题,来自{\em Car
	Talk}\footnote{\url{www.cartalk.com/content/puzzler/transcripts/200717}.}:


\begin{quote}
这个难题来自于Dan O'Leary。他遇到一个常见的由一个音节,五个字母构成的
单词。这个单词具有如下的性质:当移除第一个字母,剩下的字母组成了一个
与原来单词发音形同的单词。如果去掉第二个字母,其他字母不变,剩下的又是
一个同音词。问这个单词是什么?

现在给个不成功的例子。我们随手举个五个字母的单词为例,“wrack“。W-R-A-C-K,之
,比如`wrack with pain' 。如果去掉第一个字母,剩下'R-A-C-K'。就像,`Holy
cow, did you see the rack on that buck!It must have been a nine-pointer!'.
它确实是一个很完美的同音词。如果去掉'r',剩下`wack',它确实是一个单词,但是不
不是同音词了。

但是确实至少有一个符合这个条件的单词---去掉前两个字母中的一个都可以构成
原来单词的同音词。问题是,这个单词是什么?
\end{quote}

\index{homophone 同音词}
\index{reducible world 可化简单词}
\index{word,reducible 单词,可化简}

可以使用练习\ref{wordlist2}的字典,检查字符串是否在字典列表里。

如果要检查两个单词是否是同音词,可以使用卡内基.梅隆大学发音辞典。可以
从这里下载:
\url{www.speech.cs.cmu.edu/cgi-bin/cmudict}或者从这里\url{thinkpython.com/code/c06d} 
也可以下载\url{thinkpython.com/code/pronounce.py},这个模块提供了
函数\verb"read_dictionary",读取发音辞典,返回Python字典,提供了从单词到
对应发音(以字符串表示)的映射。

编写程序,列举能够解决这个难题的所有单词。参看我的解答\url{thinkpython.com/code/homophone.py}。


\end{ex}





































































































































































\chapter{Tuples}
\chapter{元组}
\label{元组}

\section{元组是不可变的}

\index{元组}
\index{类型!元组}
\index{序列}

元组是一组序列的值。元组中的值可以是任何数据类型,使用整数作为下标,在这个方面元组很想列表。但是一个主要的区别是元组是不可改变的。

\index{可改变的}
\index{不可改变的}

从语法构成上来看,元组是用逗号隔开的值的序列:

\beforeverb
\begin{verbatim}
>>> t = 'a', 'b', 'c', 'd', 'e'
\end{verbatim}
\afterverb
%
通常用括号包含元组,虽然这不是必要的:

\index{括号!元组位于}

\beforeverb
\begin{verbatim}
>>> t = ('a', 'b', 'c', 'd', 'e')
\end{verbatim}
\afterverb
%
要创建只含一个元素的元组,你需要包含最后的逗号:

\index{单一}
\index{元组!单一}

\beforeverb
\begin{verbatim}
>>> t1 = 'a',
>>> type(t1)
<type 'tuple'>
\end{verbatim}
\afterverb
%
在括号中的值不是元组:

\beforeverb
\begin{verbatim}
>>> t2 = ('a')
>>> type(t2)
<type 'str'>
\end{verbatim}
\afterverb
%
创建元组的另一个方式是使用内减函数{\tt tuple}。当没有参数时,函数创建一个空的元组:

\index{tuple函数}
\index{函数!tuple}

\beforeverb
\begin{verbatim}
>>> t = tuple()
>>> print t
()
\end{verbatim}
\afterverb
%
如果参数是一个序列(字符串、列表或元组),返回结果是使用序列中的元素构成的元组:

\beforeverb
\begin{verbatim}
>>> t = tuple('lupins')
>>> print t
('l', 'u', 'p', 'i', 'n', 's')
\end{verbatim}
\afterverb
%
由于{\tt tuple}是一个内建函数的名字,你应该避免使用它作为变量名。

大多数的列表运算符适用于元组。括号运算符可以索引一个元素:

\index{括号运算符}
\index{运算符!括号}

\beforeverb
\begin{verbatim}
>>> t = ('a', 'b', 'c', 'd', 'e')
>>> print t[0]
'a'
\end{verbatim}
\afterverb
%
切片运算符选择一个范围内的元素。

\index{切片运算符}
\index{运算符!切片}
\index{元组!切片}
\index{切片!元组}

\beforeverb
\begin{verbatim}
>>> print t[1:3]
('b', 'c')
\end{verbatim}
\afterverb
%
但是如果你试图修改元组中的元素,你会得到错误:

\index{异常!类型错误}
\index{类型错误}
\index{项目赋值}
\index{赋值!项目}

\beforeverb
\begin{verbatim}
>>> t[0] = 'A'
TypeError: object doesn't support item assignment
\end{verbatim}
\afterverb
%
你不能修改一个元组的元素,但是你可以使用另一个元组代替原来的元组:

\beforeverb
\begin{verbatim}
>>> t = ('A',) + t[1:]
>>> print t
('A', 'b', 'c', 'd', 'e')
\end{verbatim}
\afterverb
%

\section{元组赋值}
\label{元组赋值}

\index{元组!赋值}
\index{赋值!元组}
\index{交换模式}
\index{模式!交换}

我们经常会用到两个变量间的值的交换。对于传统的赋值,你需要使用一个临时变量。例如,要交换{\tt a}和{\tt b}:

\beforeverb
\begin{verbatim}
>>> temp = a
>>> a = b
>>> b = temp
\end{verbatim}
\afterverb
%
这个方法显得笨拙,{\bf 元组赋值}就优雅许多:

\beforeverb
\begin{verbatim}
>>> a, b = b, a
\end{verbatim}
\afterverb
%
左侧是变量组成的元组,右侧是表达式组成的元组。每个值被赋值给对应的变量。右侧所有的表达式被计算后再赋值。

左侧的变量个数必须等于右侧值的个数:

\index{异常!值错误}
\index{值错误}

\beforeverb
\begin{verbatim}
>>> a, b = 1, 2, 3
ValueError: too many values to unpack
\end{verbatim}
\afterverb
%
更一般的,右侧可以是任何序列(字符串、列表或元组)。例如,要分离一个email地址的用户名和域,你可以这么写:

\index{split方法}
\index{方法!split}
\index{email地址}

\beforeverb
\begin{verbatim}
>>> addr = 'monty@python.org'
>>> uname, domain = addr.split('@')
\end{verbatim}
\afterverb
%
{\tt split}的返回值是两个元素的列表。第一个元素被赋值给{\tt uname},第二个被赋值给{\tt domain}。

\beforeverb
\begin{verbatim}
>>> print uname
monty
>>> print domain
python.org
\end{verbatim}
\afterverb
%

\section{元组作为返回值}

\index{元组}
\index{值!元组}
\index{返回值!元组}
\index{函数,元组作为返回值}

严格地说,一个函数只能返回一个值,但是如果这个值是一个元组,等效于返回多个值。例如,如果你要计算两个整数的除法并得到商和余数,分别计算{\tt x/y}和{\tt x\%y}的效率太低,一个好的方法是同时计算这两个值。

\index{divmod}

内建函数{\tt divmod}读取两个参数,返回有两个值的元组,分别是商和余数。你可以将结果保存在一个元组中:

\beforeverb
\begin{verbatim}
>>> t = divmod(7, 3)
>>> print t
(2, 1)
\end{verbatim}
\afterverb
%
或者使用元组赋值将元素分开保存:

\index{元组赋值}
\index{赋值!元组}

\beforeverb
\begin{verbatim}
>>> quot, rem = divmod(7, 3)
>>> print quot
2
>>> print rem
1
\end{verbatim}
\afterverb
%
下面给出一个返回元组的函数的例子:

\beforeverb
\begin{verbatim}
def min_max(t):
    return min(t), max(t)
\end{verbatim}
\afterverb
%
{\tt max}和{\tt min}是内建函数,分别查找序列中最大和最小的元素, \verb"min_max"计算两者并返回包含两个值的元组。

\index{max函数}
\index{函数!max}
\index{min函数}
\index{函数!min}


\section{变长参数元组}

\index{变长参数元组}
\index{参数!变长元组}
\index{聚集}
\index{参数!聚集}
\index{参数!聚集}

函数可以读取一个变长的参数。一个以{\tt *}开头的参数将参数{\bf 聚集}为一个元组。例如,{\tt printall}可以接收任意个数的参数,并打印它们:

\beforeverb
\begin{verbatim}
def printall(*args):
    print args
\end{verbatim}
\afterverb
%
聚集的参数可以取任何你喜欢的名字,但是习惯上使用{\tt args}。下面给出函数是如何工作的:

\beforeverb
\begin{verbatim}
>>> printall(1, 2.0, '3')
(1, 2.0, '3')
\end{verbatim}
\afterverb
%
和聚集相对应的是{\bf 散布}。如果你有一个值的序列要作为多个参数传给一个函数,你可以使用{\tt *}运算符。例如,{\tt divmod}需要两个参数,而不是一个元组:

\index{散布}
\index{参数散布}

\index{类型错误}
\index{异常!类型错误}

\beforeverb
\begin{verbatim}
>>> t = (7, 3)
>>> divmod(t)
TypeError: divmod expected 2 arguments, got 1
\end{verbatim}
\afterverb
%
但是如果你散布元组,它将工作正常:

\beforeverb
\begin{verbatim}
>>> divmod(*t)
(2, 1)
\end{verbatim}
\afterverb
%
\begin{ex}
许多内建函数使用变长参数元组。例如,{\tt max}和{\tt min}可以读取任意个数的参数:

\index{max函数}
\index{函数!max}
\index{min函数}
\index{函数!min}

\beforeverb
\begin{verbatim}
>>> max(1,2,3)
3
\end{verbatim}
\afterverb
%
但是{\tt sum}函数不是这样。

\index{sum函数}
\index{函数!sum}

\beforeverb
\begin{verbatim}
>>> sum(1,2,3)
TypeError: sum expected at most 2 arguments, got 3
\end{verbatim}
\afterverb
%
编写函数{\tt sumall},可以接收任意多个参数,并返回它们的和。

\end{ex}


\section{列表和元组}

\index{zip函数}
\index{函数!zip}

{\tt zip}是一个内建函数,参数为两个或两个以上的序列,并将它们“拉链”成一个元组的列表,每个元组包含每个序列中的一个元素\footnote{在Python 3.0中,{\tt zip}返回一个元组的迭代器,但是对于大多数情况,迭代器表现的像一个列表。}。

\index{Python 3.0}

下面是一个字符串和一个列表的拉链的例子:

\beforeverb
\begin{verbatim}
>>> s = 'abc'
>>> t = [0, 1, 2]
>>> zip(s, t)
[('a', 0), ('b', 1), ('c', 2)]
\end{verbatim}
\afterverb
%
结果是一个元组的列表,每个元组包含字符串中的一个字符和列表中对应的元素。

\index{列表!元组的}

如果序列的长度不同,结果的长度和较短的序列相同。

\beforeverb
\begin{verbatim}
>>> zip('Anne', 'Elk')
[('A', 'E'), ('n', 'l'), ('n', 'k')]
\end{verbatim}
\afterverb
%
你可以在{\tt for}循环中使用元组赋值来遍历一个元组的列表:

\index{遍历}
\index{元组赋值}
\index{赋值!元组}

\beforeverb
\begin{verbatim}
t = [('a', 0), ('b', 1), ('c', 2)]
for letter, number in t:
    print number, letter
\end{verbatim}
\afterverb
%
对于每次循环,Python选择列表中下一个元组并将元素赋值给{\tt letter}和{\tt number}。循环的输出是:

\index{循环}

\beforeverb
\begin{verbatim}
0 a
1 b
2 c
\end{verbatim}
\afterverb
%
如果你结合{\tt zip},{\tt for}和元组赋值,你得到一个同时遍历两个(或多个)序列的常用写法。例如,\verb"has_match"读取两个序列,{\tt t1}和{\tt t2},如果有下标{\tt i}使得{\tt t1[i] == t2[i]},则返回{\tt True} :

\index{for循环}

\beforeverb
\begin{verbatim}
def has_match(t1, t2):
    for x, y in zip(t1, t2):
        if x == y:
            return True
    return False
\end{verbatim}
\afterverb
%
如果你要遍历一个序列中的元素和它们的下标,你可以使用内建函数{\tt enumerate}:

\index{遍历}
\index{enumerate函数}
\index{函数!enumerate}

\beforeverb
\begin{verbatim}
for index, element in enumerate('abc'):
    print index, element
\end{verbatim}
\afterverb
%
循环的输出为:

\beforeverb
\begin{verbatim}
0 a
1 b
2 c
\end{verbatim}
\afterverb
%


\section{字典和元组}

\index{字典}
\index{items方法}
\index{方法!items}
\index{键-值对}

字典有个方法称为{\tt items},返回一个元组的列表,每个元组是键-值对\footnote{在Python 3.0中稍有不同。}。

\beforeverb
\begin{verbatim}
>>> d = {'a':0, 'b':1, 'c':2}
>>> t = d.items()
>>> print t
[('a', 0), ('c', 2), ('b', 1)]
\end{verbatim}
\afterverb
%
正如你对字典所期望的,列表中的项目没有固定的顺序。

\index{字典!初始化}

相反的,你可以使用一个元组列表来初始化一个字典:

\beforeverb
\begin{verbatim}
>>> t = [('a', 0), ('c', 2), ('b', 1)]
>>> d = dict(t)
>>> print d
{'a': 0, 'c': 2, 'b': 1}
\end{verbatim}
\afterverb

结合{\tt dict}和{\tt zip}给出了一个简洁的创建字典的方法: 

\index{zip函数!和dict一同使用}

\beforeverb
\begin{verbatim}
>>> d = dict(zip('abc', range(3)))
>>> print d
{'a': 0, 'c': 2, 'b': 1}
\end{verbatim}
\afterverb
%
字典的另一个方法{\tt update}读取一个元组列表,将它作为键-值对加入现有的字典。

\index{update方法}
\index{方法!update}

\index{遍历!字典}
\index{字典!遍历}

结合{\tt items},元组赋值和{\tt for}循环,你可以得到遍历字典的键-值对的常用写法:

\beforeverb
\begin{verbatim}
for key, val in d.items():
    print val, key
\end{verbatim}
\afterverb
%
循环的输出为:

\beforeverb
\begin{verbatim}
0 a
2 c
1 b
\end{verbatim}
\afterverb
%

\index{元组!作为字典中的键}
\index{哈希表}

通常使用元组作为字典的键(主要因为你不能使用列表)。例如,电话簿是姓-名对到电话号码的映射。假设我们已经定义了{\tt last},{\tt first}和{\tt number},我们可以这么写:

\beforeverb
\begin{verbatim}
directory[last,first] = number
\end{verbatim}
\afterverb
%
括号中的表达式是一个元组。我们可以使用元组赋值来遍历字典。

\index{元组!在括号中}

\beforeverb
\begin{verbatim}
for last, first in directory:
    print first, last, directory[last,first]
\end{verbatim}
\afterverb
%
这个循环遍历{\tt directory}中作为键的元组。将每个元组中的元素赋值给{\tt last}和{\tt first},然后打印姓名和对应的电话号码。

有两种在状态图中标示元组的方式。下面的更详细的版本给出了类似列表的下标和元素。例如,元组\verb"('Cleese', 'John')"将会标示成这样:

\index{状态图}
\index{图!状态}

\beforefig
\centerline{\includegraphics{figs/tuple1.eps}}
\afterfig
但是在一个更大的图中你也许希望忽略细节。例如,一个电话簿的状态图会是这样:

\beforefig
\centerline{\includegraphics{figs/dict2.eps}}
\afterfig

这里元组根据Python的语法以图形速记的形式给出。

图中的电话号码是BBC的投诉电话,请不要拨打。



\section{元组比较}

\index{比较!元组}
\index{元组!比较}
\index{sort方法}
\index{方法!sort}

关系运算符使用于元组和其他序列。Python从每个序列的第一个元素开始比较。如果它们相同,则继续比较下一个元素,依次类推,直到找到有区别的元素,以后的元素将不被考虑(即使它们很大)。

\beforeverb
\begin{verbatim}
>>> (0, 1, 2) < (0, 3, 4)
True
>>> (0, 1, 2000000) < (0, 3, 4)
True
\end{verbatim}
\afterverb
%
{\tt sort}函数的工作原理类似,它根据第一个元素排序,如果第一个元素相同,则对第二个元素进行排序,依次类推。

这个特点对应{\bf DSU}模式: 

\begin{description}

\item[Decorate] 装饰序列,生成元组列表,将一个或多个排序关键字放在元素的最前面。

\item[Sort] 对元组列表排序

\item[Undecorate] 通过从已排序的序列中抽出元素来还原。

\end{description}

\label{DSU}
\index{DSU模式}
\index{模式!DSU}
\index{装饰-排序-还原模式}
\index{模式!装饰-排序-还原}

例如,假设你有列单词,你需要从最长到最短将它们排序:

\beforeverb
\begin{verbatim}
def sort_by_length(words):
    t = []
    for word in words:
       t.append((len(word), word))

    t.sort(reverse=True)

    res = []
    for length, word in t:
        res.append(word)
    return res
\end{verbatim}
\afterverb
%
第一个循环构造了一个元组列表,每个元组有单词长度和单词组成。

{\tt sort}函数首先比较第一个元素,即单词长度,仅当单词长度相同时才考虑第二个元素。关键字参数{\tt reverse=True}告诉{\tt sort}使用降序排序。

\index{关键字参数}
\index{参数!关键字}
\index{遍历}

第二个循环遍历元组列表,构建一个按长度递减排列的单词列表。

\begin{ex}
在这个例子中,长度相同的单词是按照字母顺序排序的。对于一些其他的应用,你也许需要它们是随机排序的。修改例子是长度相同的单词随机排序。提示:参考{\tt random}模块中的{\tt random}函数。


\index{random模块}
\index{模块!random}
\index{random函数}
\index{函数!random}

\end{ex}


\section{序列的序列}
\index{序列}

我主要使用元组的列表,事实上几乎本章所有的例子都还可以使用列表的列表,元组的元组和列表的元组来实现。为了避免枚举可能的组合,有时说成序列的序列更为方便。

在很多情况下,不同的序列(字符串,列表和元组)可以交换的使用。那么该如何选择,为什么这么选择呢?

\index{字符串}
\index{列表}
\index{元组}
\index{可改变}
\index{不可改变}

显而易见的是,字符串相比其他序列受到更多的限制,因为元素必须是字符。同时它们是不可改变的。如果你需要能够修改字符串中的字符(而非创建一个新的字符串),你需要使用字符的列表。

列表比元组更为常用,主要因为它们是可改变的。但有一些情况你或许会更倾向于元组:

\begin{enumerate}

\item 在某些情况下,如{\tt return}语句,语法上创建一个元组比创建一个列表更方便。在其他情况下,你也许更倾向列表。

\item 如果你要使用一个类似字典关键字的序列,你必须使用类似元组或字符串的不可改变的数据类型。

\item 如果你将序列作为函数参数,使用元组会减小潜在的因为别名而造成的意外的行为。

\end{enumerate}

由于元组是不可改变的,它们不提供类似{\tt sort}或{\tt reverse}等修改列表的方法。但是Python提供内建函数{\tt sorted}和{\tt reversed},它们读取任何序列作为参数,并返回一个新的排序后的列表。

\index{sorted函数}
\index{函数!sorted}
\index{reversed函数}
\index{函数!reversed}


\section{调试}

\index{调试}
\index{数据结构}
\index{形状错误}
\index{错误!形状}
列表、字典和元组通常被成为{\bf 数据结构}。本章中我们开始接触复合数据结构,如元组的列表、以元组作为键列表作为值的字典。复合数据结构十分有用,但它们容易造成我习惯称呼的{\bf 形状错误},即由于数据结构含有错误的类型、大小或符合而造成的错误。例如,你期望得到一个只含一个整数的列表,而我给你的是一个整数(不是一个列表),这将导致错误。

\index{structshape模块}
\index{模块!structshape}

为了帮助调试此类错误,我编写了一个叫做{\tt structshape}的模块,提供一个{\tt structshape}函数,它可以读取任何数据结构作为参数,并返回一个描述该结构的字符串。你可以在\url{thinkpython.com/code/structshape.py}下载。

下面给出一个简单列表的结果:

\beforeverb
\begin{verbatim}
>>> from structshape import structshape
>>> t = [1,2,3]
>>> print structshape(t)
list of 3 int
\end{verbatim}
\afterverb
%
更好的程序也许会输出``list of 3 int{\em s},'',但不考虑复数相对简单。下面是一个列表的列表:

\beforeverb
\begin{verbatim}
>>> t2 = [[1,2], [3,4], [5,6]]
>>> print structshape(t2)
list of 3 list of 2 int
\end{verbatim}
\afterverb
%
如果列表中的元素不是同一数据类型,{\tt structshape}按类型聚合它们:

\beforeverb
\begin{verbatim}
>>> t3 = [1, 2, 3, 4.0, '5', '6', [7], [8], 9]
>>> print structshape(t3)
list of (3 int, float, 2 str, 2 list of int, int)
\end{verbatim}
\afterverb
%
下面是一个元素的列表:

\beforeverb
\begin{verbatim}
>>> s = 'abc'
>>> lt = zip(t, s)
>>> print structshape(lt)
list of 3 tuple of (int, str)
\end{verbatim}
\afterverb
%
下面是有3项从整数到字符串映射的字典。

\beforeverb
\begin{verbatim}
>>> d = dict(lt) 
>>> print structshape(d)
dict of 3 int->str
\end{verbatim}
\afterverb
%
如果你在跟踪数据结构时遇到了问题,{\tt structshape}可以帮助你。


\section{术语}

\begin{description}

\item[元组:] 不可改变的元素的序列。
\index{元组}

\item[元组赋值:] 一个序列在右、一个元组变量在左的赋值。右边首先计算值,然后将其元素赋值给左边的变量。
\index{元组赋值}
\index{赋值!元组}

\item[聚集:] 对变长参数元组的集合操作。
\index{聚集}

\item[散布:] 将序列作为参数列表的操作。
\index{散布}

\item[DSU:] ``decorate-sort-undecorate''的简称,一个构建元组列表、排序、提取结果的模式。
\index{DSU模式}

\item[数据结构:] 相关数据的集合,通常组织为列表、字典、元组等。
\index{数据结构}

\item[形状(数据结构的):] 对数据结构类型、大小和组成的概述。
\index{形状}

\end{description}


\section{练习}

\begin{ex}
编写函数\verb"most_frequent",参数为一个字符串,以频率降序打印字符出现的次数。从不同语言的测试文本中寻找频率的不同。将你的结果和\url{wikipedia.org/wiki/Letter_frequencies}中的表格作比较。

\index{字母频率}
\index{频率!字母}

\end{ex}


\begin{ex}
\label{回文}

\index{回文集合}
\index{集合!回文}

更多关于回文的练习!

\begin{enumerate}

\item 编写函数,从文件中读取一个单词表(参考章节~\ref{单词表}),打印所有回文的单词。

下面的例子给出可能的输出结果:

\beforeverb
\begin{verbatim}
['deltas', 'desalt', 'lasted', 'salted', 'slated', 'staled']
['retainers', 'ternaries']
['generating', 'greatening']
['resmelts', 'smelters', 'termless']
\end{verbatim}
\afterverb
%
提示:你也许需要构建一个字典满足从字母的集合到这些字母可以组成的单词列表的映射。问题是你如何表示这个字母集合使得它们可以作为字典的键?

\item 修改之前的程序,使程序按照结果集合的大小从大到小输出。

\index{拼字游戏}
\index{bingo}

\item 在拼字游戏中,如果你手头的7个字母和桌面上的1个字母组成一个8个字母的单词,你实现了“bingo”。哪8个字母组成的集合有最大的概率实现“bingo”?提示:有7组。

% (7, ['angriest', 'astringe', 'ganister', 'gantries', 'granites',
% 'ingrates', 'rangiest'])

\index{置换}

\item 我们定义两个单词为“置换对”,如果你能通过交换字母顺序将一个单词转换为另一个单词\footnote{这个练习受\url{puzzlers.org}中一个例子的启发。}。例如,“converse”和“conserve”是一对“置换对”。提示:不要测试所有的单词对,也不要测试所有的交换。

你可以在\url{thinkpython.com/code/anagram_sets.py}下载一个解答。

\end{enumerate}
\end{ex}



\begin{ex}

\index{Car Talk}
\index{难题}

下面是另一个Car Talk难题\footnote{\url{www.cartalk.com/content/puzzler/transcripts/200651}。}:

\begin{quote}
如果每次你从一个单词中删除一个字母它仍是一个有效的英语单词,在英语中满足这样条件的最长的单词是什么?

删除的字母可以位于两端或者中间,但是你不能重新排列字母。每次你删除一个字母,你得到另一个英语单词。最终你得到一个只有一个字母的单词。我要知道最长的单词是什么,它有多少字母?

我要给出一个例子:Sprite。你从sprite开始删除字母,首先删除r,我们得到单词spite,接着删除e,我们得到spit,再删除s,我们得到pit,it和I。
\end{quote}

\index{可缩小的单词}
\index{单词,可缩小的}

编写程序,找出可以按这中方法缩小的所有单词,并找出最长的一个。

这个练习比以往的都更有挑战性,所以给出一些建议:

\begin{enumerate}

\item 你也许需要编写一个函数,参数为一个单词,函数找出所有删除一个字母后仍合法的单词,即原单词的“孩子”。

\index{递归定义}
\index{定义!递归}

\item 递归的看,一个单词是可缩小的,如果它所有的孩子都是可缩小的。作为基本状态,你可以认为空字符串是可以缩小的。

\item 我提供的单词表{\tt words.txt}中不包括单字母的单词。所以你需要添加“I”,“a”和空字符串。

\item 为了提高你的程序的效率,你需要记住已知的可缩小的单词。

\end{enumerate}

你可以在\url{thinkpython.com/code/reducible.py}下载我的解答。

\end{ex}


\chapter{实例学习:数据结构的选择}
\chapter{实例学习:数据结构的实例}


\chapter{文件}
\chapter{文件}

\index{文件}
\index{类型!文件}


\section{持久性}

\index{持久性}

目前我们所见到的大多数的程序都是瞬态的,即它们在短时间内运行并输出一些结果,当它们结束时,数据也就消失了。如果你再次运行程序,它将从一个初始的状态开始。

另一类程序被称为是{\bf 持久的},它们长时间运行(或者时刻运行),至少将一部分数据记录在永久储存设备(如硬盘)上,当程序关闭并重新启动时,它们可以恢复结束前的状态以便继续运行。

一个持久的程序的例子是操作系统,当一台电脑开机后操作系统在绝大多数时间都在运行,对于一个接受网络请求的web服务器,操作系统时刻在运行。

程序维护数据最简单的方法是读写文本文件。我们已经接触过读取文本文件的程序,在本章中我们将接触写入文本文件的程序。

记录程序状态的另一个方式是使用数据库。在本章节中我给出一个简单的数据库和模块{\tt pickle},该模块简化了存储数据的过程。

\index{pickle模块}
\index{模块!pickle}


\section{读取和写入}

\index{文件!读取和写入}

文本文件是储存在类似硬盘、闪存、CD-ROM等永久介质上的字符序列。我们在章节~\ref{单词表}中接触了文件的打开和读取。

\index{open函数}
\index{函数!open}

要写入一个文件,你需要在打开文件时添加第二个参数\verb"'w'":

\beforeverb
\begin{verbatim}
>>> fout = open('output.txt', 'w')
>>> print fout
<open file 'output.txt', mode 'w' at 0xb7eb2410>
\end{verbatim}
\afterverb
%
如果文件已经存在,以写的模式打开该文件将会清空原来的数据并从新的开始,所以要小心!如果文件不存在,那么将创建一个新的文件。

{\tt write}方法将数据写入文件。

\beforeverb
\begin{verbatim}
>>> line1 = "This here's the wattle,\n"
>>> fout.write(line1)
\end{verbatim}
\afterverb
%
文件对象将跟踪位置,如果你再次调用{\tt write},它将在尾部写入新的数据。

\beforeverb
\begin{verbatim}
>>> line2 = "the emblem of our land.\n"
>>> fout.write(line2)
\end{verbatim}
\afterverb
%
当你完成写入,你可以关闭文件。

\beforeverb
\begin{verbatim}
>>> fout.close()
\end{verbatim}
\afterverb
%

\index{close方法}
\index{方法!close}


\section{格式运算符}

\index{格式运算符}
\index{运算符!格式}

{\tt write}的参数必须是字符串,如果我们要将其他值写入文件,我们需要将它们转换为字符串。最简单的方法是使用{\tt str}:

\beforeverb
\begin{verbatim}
>>> x = 52
>>> f.write(str(x))
\end{verbatim}
\afterverb
%
另一个方法是使用{\bf 格式运算符}{\tt \%}。当作用于整数,{\tt \%}是取模运算符,而当第一个运算数是字符串时,{\tt \%}是格式运算符。

\index{格式字符串}

第一个运算数是{\bf 格式字符串},它包含一个或多个{\bf 格式序列},它们指定了第二个运算数是如何格式化的。结果为一个字符串。

\index{格式序列}

例如,格式序列\verb"'%d'"意味着第二个运算数应该被格式化为一个整数({\tt d}代表“decimal”):

\beforeverb
\begin{verbatim}
>>> camels = 42
>>> '%d' % camels
'42'
\end{verbatim}
\afterverb
%
结果是字符串\verb"'42'",需要和整数值{\tt 42}区分开来。

格式序列可以出现在字符串中的任何位置,所以你可以将值嵌入到语句中:

\beforeverb
\begin{verbatim}
>>> camels = 42
>>> 'I have spotted %d camels.' % camels
'I have spotted 42 camels.'
\end{verbatim}
\afterverb
%
如果字符串中有多于一个格式序列,第二个参数必须为一个元组。每个格式序列按次序和元组中的元素对应。

下面的例子中使用\verb"'%d'"来格式化一个整数。
\verb"'%g'" to format
a floating-point number (don't ask why), and \verb"'%s'" to format
a string:

\beforeverb
\begin{verbatim}
>>> 'In %d years I have spotted %g %s.' % (3, 0.1, 'camels')
'In 3 years I have spotted 0.1 camels.'
\end{verbatim}
\afterverb
%
元组中元素的个数必须等于字符串中格式序列的个数。同时,元素的类型必须符合对应的格式序列。

\index{异常!类型错误}
\index{类型错误}

\beforeverb
\begin{verbatim}
>>> '%d %d %d' % (1, 2)
TypeError: not enough arguments for format string
>>> '%d' % 'dollars'
TypeError: illegal argument type for built-in operation
\end{verbatim}
\afterverb
%
在第一个例子中,元组中没有足够的元素,在第二个例子中,元素的类型错误。

格式运算符十分强大,但它很难使用。你可以在\url{docs.python.org/lib/typesseq-strings.html}阅读更多有关的内容。


\section{文件名和路径}
\label{路径}

\index{文件名}
\index{路径}
\index{目录}
\index{文件夹}

文件以{\bf 目录} (也称为“文件夹”)的形式管理。每个运行的程序有一个“当前目录”,它是许多操作的默认目录。例如,当你打开一个文件来读取数据,Python在当前目录下寻找这个文件。

\index{os模块}
\index{模块!os}

{\tt os}模块提供了操作文件和目录的函数(“os”代表“operating system”)。{\tt os.getcwd}返回当前目录的名称:

\index{getcwd函数}
\index{函数!getcwd}

\beforeverb
\begin{verbatim}
>>> import os
>>> cwd = os.getcwd()
>>> print cwd
/home/dinsdale
\end{verbatim}
\afterverb
%
{\tt cwd}代表“current working directory”,即当前工作陌路。在本例中结果是{\tt /home/dinsdale},它是用户名为{\tt dinsdale}的主目录。

\index{工作目录}
\index{目录!工作}
类似{\tt cwd}能够确定一个文件的字符串称为{\bf 路径}。{\bf 相对路径}从当前目录开始,{\bf 绝对路径}从文件系统的根目录开始。

\index{相对路径}
\index{路径!相对}
\index{绝对路径}
\index{路径!绝对}

我们现在看到的路径都是简单的文件名,因此它们是相对当前目录的。要得到一个文件的绝对目录,你可以使用{\tt os.path.abspath}:

\beforeverb
\begin{verbatim}
>>> os.path.abspath('memo.txt')
'/home/dinsdale/memo.txt'
\end{verbatim}
\afterverb
%
{\tt os.path.exists}检查一个文件或者目录是否存在:

\index{exists函数}
\index{函数!exists}

\beforeverb
\begin{verbatim}
>>> os.path.exists('memo.txt')
True
\end{verbatim}
\afterverb
%
如果存在,{\tt os.path.isdir}检查它是否是一个目录:

\beforeverb
\begin{verbatim}
>>> os.path.isdir('memo.txt')
False
>>> os.path.isdir('music')
True
\end{verbatim}
\afterverb
%
类似的,{\tt os.path.isfile}检查是否是一个文件。

{\tt os.listdir}返回给定目录下的文件(以及其他目录):

\beforeverb
\begin{verbatim}
>>> os.listdir(cwd)
['music', 'photos', 'memo.txt']
\end{verbatim}
\afterverb
%
为了演示这些函数,下面的例子“遍历”一个目录,打印所有文件的名字,并对所有目录递归地调用自身。

\index{遍历,目录}
\index{目录!遍历}

\beforeverb
\begin{verbatim}
def walk(dir):
    for name in os.listdir(dir):
        path = os.path.join(dir, name)

        if os.path.isfile(path):
            print path
        else:
            walk(path)
\end{verbatim}
\afterverb
%
{\tt os.path.join}读取一个目录名和一个文件名,并将两者合并为一个完整路径。 
\begin{ex}
修改{\tt walk}函数,使之返回文件名列表,而不是打印这些信息。
\end{ex}

\begin{ex}
{\tt os}模块提供了与我们的{\tt walk}函数类似的函数,但功能更丰富。阅读文档,使用该函数打印给定目录下的文件和子目录。
\end{ex}


\section{捕获异常}
\label{捕获}
当你尝试读写文件时,很多地方会发生错误。如果你试图打开一个不存在的文件,你会得到一个{\tt 输入输出错误}:

\index{open海曙}
\index{函数!open}
\index{异常!输入输出错误}
\index{输入输出错误}

\beforeverb
\begin{verbatim}
>>> fin = open('bad_file')
IOError: [Errno 2] No such file or directory: 'bad_file'
\end{verbatim}
\afterverb
%
如果你没有权限访问一个文件:

\index{文件!权限}
\index{权限,文件}

\beforeverb
\begin{verbatim}
>>> fout = open('/etc/passwd', 'w')
IOError: [Errno 13] Permission denied: '/etc/passwd'
\end{verbatim}
\afterverb
%
如果你试图读取一个目录,你会得到:

\beforeverb
\begin{verbatim}
>>> fin = open('/home')
IOError: [Errno 21] Is a directory
\end{verbatim}
\afterverb
%
为了避免这些错误,你可以使用类似{\tt os.path.exists}和{\tt os.path.isfile}的检查函数,但是检查所有可能的错误会占用很多时间和代码(如果“{\tt Errno 21}”是一个错误信息,那么至少有21种出错情况)。

\index{异常,捕获}
\index{try语句}
\index{语句!try}

更好的方法是当问题出现了才去处理,即{\tt try}语句所做的。它的语法类似{\tt if}语句:

\beforeverb
\begin{verbatim}
try:    
    fin = open('bad_file')
    for line in fin:
        print line
    fin.close()
except:
    print 'Something went wrong.'
\end{verbatim}
\afterverb
%
Python从{\tt try}语句开始执行,如果一切正常,那么{\tt except}将被跳过。如果发生异常,则跳出{\tt try}语句块,执行{\tt except}中的代码。

使用{\tt try}语句处理异常被称为{\bf 捕获}异常。在本例中,{\tt except}语句块中的代码仅仅打印了错误信息。通常,捕获异常给了你修补问题的机会,你可以继续尝试,或者至少可以优雅的结束程序。


\section{数据库}

\index{数据库}

{\bf 数据库}是用于存储数据的文件。大多数的数据库以字典的形式组织,即将键映射为值。数据库是保存在磁盘(或其他永久存储设备)上,因此即使程序结束它们仍然存在。


\index{anydbm模块}
\index{模块!anydbm}

模块{\tt anydbm}提供了创建和跟新数据库文件的接口。作为一个例子,我将创建一个包含图片文件标题的数据库。

\index{open函数}
\index{函数!open}

打开数据库和其他文件类似:

\beforeverb
\begin{verbatim}
>>> import anydbm
>>> db = anydbm.open('captions.db', 'c')
\end{verbatim}
\afterverb
%
模式\verb"'c'"代表如果文件不存在则创建文件。返回结果是一个数据库对象,它可以像字典一样被使用(对于大多数的操作)。如果你创建一个新的项目, {\tt anydbm}将更新数据库文件。

\index{更新!数据库}


\beforeverb
\begin{verbatim}
>>> db['cleese.png'] = 'Photo of John Cleese.'
\end{verbatim}
\afterverb
%
当你访问某个项目是,{\tt anydbm}将读取文件:

\beforeverb
\begin{verbatim}
>>> print db['cleese.png']
Photo of John Cleese.
\end{verbatim}
\afterverb
%
如果你对已有的键再次进行赋值,{\tt anydbm}将替代旧的值:

\beforeverb
\begin{verbatim}
>>> db['cleese.png'] = 'Photo of John Cleese doing a silly walk.'
>>> print db['cleese.png']
Photo of John Cleese doing a silly walk.
\end{verbatim}
\afterverb
%
许多字典的方法,如{\tt keys}和{\tt items},同样适用于数据库对象,包括使用{\tt for}语句实现的迭代:

\index{字典方法!anydbm模块}

\beforeverb
\begin{verbatim}
for key in db:
    print key
\end{verbatim}
\afterverb
%
和其他文件一样,当你完成操作后你需要关闭文件:

\beforeverb
\begin{verbatim}
>>> db.close()
\end{verbatim}
\afterverb
%

\index{close方法}
\index{方法!close}


\section{Pickling}

\index{pickling}

{\tt anydbm}模块的一个限制在于键和值必须是字符串。如果你尝试使用其他数据类型,你会得到一个错误。

\index{pickle模块}
\index{模块!pickle}

{\tt pickle}模块可以解决这个问题。它能将任何类型的对象翻译成适合在数据库中储存的字符窜,同时能将字符串还原成对象。

{\tt pickle.dumps}读取一个对象作为参数,并返回一个表征字符串({\tt dumps}是“dump string”(转储字符串)的缩写):

\beforeverb
\begin{verbatim}
>>> import pickle
>>> t = [1, 2, 3]
>>> pickle.dumps(t)
'(lp0\nI1\naI2\naI3\na.'
\end{verbatim}
\afterverb
%
这个格式对人类读者来说不是很好理解,但是对{\tt pickle}来说很好解释。{\tt pickle.loads}(“载入字符串”)可以重建对象:

\beforeverb
\begin{verbatim}
>>> t1 = [1, 2, 3]
>>> s = pickle.dumps(t1)
>>> t2 = pickle.loads(s)
>>> print t2
[1, 2, 3]
\end{verbatim}
\afterverb
%
虽然新的对象和老的对象有相同的值,它们(通常)不是同一个对象:

\beforeverb
\begin{verbatim}
>>> t1 == t2
True
>>> t1 is t2
False
\end{verbatim}
\afterverb
%
换言之,pickling然后unpickling等效于复制一个对象。

你可以使用{\tt pickle}在数据库中存储一个非字符串对象。事实上,这个组合非常常用,并有一个已经封装好的模块{\tt shelve}。

\index{shelve模块}
\index{模块!shelve}


\begin{ex}

\index{回文集合}
\index{集合!回文}

如果你做了章节~\ref{回文}中的练习,修改你的方案,创建一个数据库,将列表中的单词映射为使用同样字母的单词的列表。

编写另一个程序,读取数据库并以人类适宜阅读的格式打印内容。
\end{ex}


\section{管道}

\index{shell}
\index{管道}

大多数的操作系统提供一个命令行的接口,称为{\bf shell}。shell通常提供浏览文件系统和启动程序的命令。例如,在Unix系统中,你可以使用{\tt cd}改变目录,使用{\tt ls}显示一个目录的内容,通过输入{\tt firefox}来启动一个网页浏览器。

\index{ls (Unix 命令)}
\index{Unix 命令!ls}

任何你在shell中可以启动的程序也可以在Python中通过使用{\bf 管道}来启动。一个管道是一个表示活动进程的对象。

例如,Unix命令{\tt ls -l}将以详细格式显示当前目录下的内容。你可以使用{\tt os.popen}来启动{\tt ls}:

\index{popen函数}
\index{函数!popen}

\beforeverb
\begin{verbatim}
>>> cmd = 'ls -l'
>>> fp = os.popen(cmd)
\end{verbatim}
\afterverb
%
参数是一个包含shell命令的字符串。返回值是一个对象,就像打开一个文件一样。你可以使用{\tt readline}每次从输出读取一样,或者使用{\tt read}一次读取所有内容:

\index{readline方法}
\index{方法!readline}
\index{read方法}
\index{方法!read}

\beforeverb
\begin{verbatim}
>>> res = fp.read()
\end{verbatim}
\afterverb
%
当你完成操作后,你像关闭一个文件一样关闭管道:

\index{close方法}
\index{方法!close}

\beforeverb
\begin{verbatim}
>>> stat = fp.close()
>>> print stat
None
\end{verbatim}
\afterverb
%
返回值是{\tt ls}进程的最终状态,{\tt None}表示它正常结束(没有错误)。

\index{文件!压缩}
\index{压缩!文件}
\index{Unix 命令!gunzip}
\index{gunzip (Unix命令)}

管道的一个常用用法是增量的读取一个压缩文件,即不需要一次解压整个文件。下面的函数读取一个压缩文件名作为参数,使用{\tt gunzip}来解压文件,并返回一个管道:

\beforeverb
\begin{verbatim}
def open_gunzip(filename):
    cmd = 'gunzip -c ' + filename
    fp = os.popen(cmd)
    return fp
\end{verbatim}
\afterverb
%
如果你每次从{\tt fp}读取一行,你不需要将解压的文件保存在内存或磁盘上。


\section{编写模块}
\label{模块}

\index{模块,编写}
\index{单词统计}

任何包含Python代码的文件可以作为模块被导入。例如,假设你有文件{\tt wc.py},内容如下:

\beforeverb
\begin{verbatim}
def linecount(filename):
    count = 0
    for line in open(filename):
        count += 1
    return count

print linecount('wc.py')
\end{verbatim}
\afterverb
%
如果你运行这个程序,它将读取自身,并打印行数,结果是7。你也可以这样导入模块:

\beforeverb
\begin{verbatim}
>>> import wc
7
\end{verbatim}
\afterverb
%
现在你有一个模块对象{\tt wc}:

\index{模块对象}
\index{对象!模块}

\beforeverb
\begin{verbatim}
>>> print wc
<module 'wc' from 'wc.py'>
\end{verbatim}
\afterverb
%
这里提供了一个称为\verb"linecount"的函数:

\beforeverb
\begin{verbatim}
>>> wc.linecount('wc.py')
7
\end{verbatim}
\afterverb
%
以上就是如何编写Python模块。

这个例子中唯一的问题在于当你导入模块后,最后的测试代码被执行。通常当你导入一个模块,它定义新的函数,但并不执行它们。

\index{import语句}
\index{语句!import}


作为模块的程序通常写成一下结构:

\beforeverb
\begin{verbatim}
if __name__ == '__main__':
    print linecount('wc.py')
\end{verbatim}
\afterverb
%
\verb"__name__"是一个程序开始时设置的内建变量。如果程序以脚本的形式运行,\verb"__name__"的值为\verb"__main__",在这种情况下,测试代码将被执行。否则,如果是作为模块被导入,测试代码将被忽略。

\begin{ex}
将例子输入到文件{\tt wc.py}中,并以脚本形式运行。如果在Python解释器中运行{\tt import wc},当模块被导入后,\verb"__name__"的值是什么?

警告:当你再次导入一个已经导入的的模块,Python将什么也不错。它不会重新读取文件,即使文件发生了改变。

\index{模块!重载}
\index{reload函数}
\index{函数!reload}
如果你要重载一个模块,你可以使用内建函数{\tt reload},但它可能会出错,因此最安全的方法是重启解释器然后重新导入模块。
\end{ex}



\section{调试}

\index{调试}
\index{空白}

当你读写文件,你也许会遇到空白带来的问题。由于空格符、tab符和换行符通常是不可见的,这样的错误很难调试:

\beforeverb
\begin{verbatim}
>>> s = '1 2\t 3\n 4'
>>> print s
1 2	 3
 4
\end{verbatim}
\afterverb

\index{repr函数}
\index{函数!repr}
\index{字符串表示}

内建函数{\tt repr}可以用来解决这个问题。它读取一个对象作为参数,并返回一个表示这个对象的字符串。对于字符串,它将空白符号用反斜杠序列表示:

\beforeverb
\begin{verbatim}
>>> print repr(s)
'1 2\t 3\n 4'
\end{verbatim}
\afterverb

这个对调试很有用。

你也许会遇到另一个问题,不同的操作系统使用不同的符号作为换行符。有的系统使用\verb"\n",有的使用\verb"\r",有的使用两者。如果你在不同系统中使用,这些差异会导致问题。

\index{行结束符号}
大多数系统提供了格式转换的程序。你可以在\url{wikipedia.org/wiki/Newline}中找到(并阅读更多相关内容)。当然你可以自己编写一个转换程序。


\section{术语}

\begin{description}

\item[持久性:] 一个长期运行、并至少将一部分数据保存在永久性的储存设备上的程序。
\index{持久性}

\item[格式运算符:] 运算符{\tt \%},参数为一个格式字符串和一个元组,生成一个按格式字符串规定的元组中元素的值的字符串。
\index{格式运算符}
\index{运算符!格式}

\item[格式字符串:] 使用格式运算符,包含格式序列的字符串。
\index{格式字符串}

\item[格式序列:] 格式字符串中的字符序列,类似{\tt \%d},指定了一个值的格式。
\index{格式序列}

\item[文本文件:] 保存在类似硬盘的永久储存设备上的字符序列。
\index{文本文件}

\item[目录:] 一个命名的文件的集合,也称为文件夹。
\index{目录}

\item[路径:] 用于识别文件的字符串。
\index{路径}

\item[相对路径:] 从当前目录开始的路径。
\index{相对路径}

\item[绝对路径:] 从文件系统顶部开始的路径。
\index{绝对路径}

\item[捕获:] 为了防止程序因为异常而终止,使用{\tt try}和{\tt except}语句来捕捉异常。
\index{捕获}

\item[数据库:] 一个类似字典使用键对应值的文件。
\index{数据库}

\end{description}


\section{练习}

\begin{ex}
\label{urllib}

\index{urllib模块}
\index{模块!urllib}
\index{URL}

{\tt urllib}模块提供了操作URL和从互联网下载信息的方法。下面的例子从{\tt thinkpython.com}下载并打印一条秘密信息:

\beforeverb
\begin{verbatim}
import urllib

conn = urllib.urlopen('http://thinkpython.com/secret.html')
for line in conn.fp:
    print line.strip()
\end{verbatim}
\afterverb

运行程序,运行结果将给你下一步指令。

\index{秘密练习}
\index{练习,秘密}

\end{ex}

\begin{ex}
\label{和校验}

\index{MP3}

在一个有很多MP3文件的收藏中,有可能同一首歌有多个拷贝,以不同的名字保存在不同的目录下。这个练习的目的是找出这些拷贝。

\begin{enumerate}

\item 编写程序,递归地搜索一个目录和所有子目录,返回一个完整路径的后缀给定的(如{\tt .mp3})的文件名列表。提示:{\tt os.path}提供了几个有用的操作文件和路径名的函数。

\index{拷贝}
\index{MD5算法}
\index{算法!MD5}
\index{和校验}

\item 为了识别拷贝,你可以使用哈希函数,读取文件并生成一个针对内容的简短的概述。例如,MD5 (Message-Digest algorithm 5)读取一个任意长的“消息”并返回一个128比特的“校验和”。两个不同文件返回相同的校验和的概率非常小。

你可以在\url{wikipedia.org/wiki/Md5}了解有关MD5的知识。在一个Unix系统上你可以使用{\tt md5sum}程序和Python中的管道来计算校验和。

\index{管道}

\end{enumerate}

\end{ex}


\begin{ex}

\index{网络电影数据库(IMDb)}
\index{IMDb (网络电影数据库)}
\index{数据库}

网络电影数据库(IMDb)是一个在线的电影信息收集的网站。它们的数据库可以以纯文本的格式获得,便于Python的读取。在这个练习中,你需要文件{\tt actors.list.gz}和{\tt actresses.list.gz},它们可以从\url{www.imdb.com/interfaces#plain}下载。

\index{纯文本}
\index{文本!纯}
\index{解析}

我编写了一个程序,可以解析这些文件并分割为演员名、电影标题等。你可以在\url{thinkpython.com/code/imdb.py}下载。

如果你已脚本方式运行{\tt imdb.py},程序将读取{\tt actors.list.gz}并在每行答应演员-电影对。或者你可以{\tt import imdb},调用函数\verb"process_file"来处理这些文件。参数为一个文件名,一个函数对象和一个可选的指定处理行数的数。下面给出一个例子:

\beforeverb
\begin{verbatim}
import imdb

def print_info(actor, date, title, role):
    print actor, date, title, role

imdb.process_file('actors.list.gz', print_info)
\end{verbatim}
\afterverb

当你调用\verb"process_file",它打开{\tt filename},读取内容,并对文件中的每行调用\verb"print_info"。\verb"print_info"读取演员、日期、电影标题和角色作为参数,并打印这些信息。

\begin{enumerate}

\item 编写函数,读取{\tt actors.list.gz}和{\tt actresses.list.gz},使用{\tt shelve}来构建一个数据库,将每个演员映射到他或她的电影列表。

\index{shelve模块}
\index{模块!shelve}

\item 两个演员称为是“合演”的,如果他们至少一起演过一部电影。基于上一步建立的数据库,构建第二个数据库,将每个演员映射到他或她“合演”的演员列表。

\index{Bacon, Kevin}
\index{Kevin Bacon Game}

 \item 编写程序,实现“Six Degrees of Kevin Bacon”,你可以在\url{wikipedia.org/wiki/Six_Degrees_of_Kevin_Bacon}中了解相关信息。这个问题的挑战在于你需要在关系图上找到最短路径。你可以在\url{wikipedia.org/wiki/Shortest_path_problem}里阅读最短路径算法相关的资料。

\end{enumerate}

\end{ex}


\chapter{类和对象}
\chapter{类和对象}


\chapter{类和函数}
\chapter{类和函数}
\label{ 时间}


\section{时间}

作为用户定义类型的另一个例子,我们将定义一个{\tt Time}类,记录当前时间,类的定义如下:

\index{用户定义类型}
\index{类型!用户定义}
\index{Time类}
\index{类!Time}

\beforeverb
\begin{verbatim}
class Time(object):
    """represents the time of day.
       attributes: hour, minute, second"""
\end{verbatim}
\afterverb
%
我们可以创建一个新的{\tt Time}对象,并对时、分和秒进行赋值:

\beforeverb
\begin{verbatim}
time = Time()
time.hour = 11
time.minute = 59
time.second = 30
\end{verbatim}
\afterverb
%
{\tt Time}对象的状态图如下:

\index{状态图}
\index{图!状态}
\index{对象图}
\index{图!对象}

\beforefig
\centerline{\includegraphics{figs/time.eps}}
\afterfig

\begin{ex}
\label{printtime}
编写函数\verb"print_time",参数为一个时间对象,以{\tt 时:分:秒}的格式打印时间。提示:格式字符串\verb"'%.2d'"使用至少两位打印一个整数,如果需要则在前面添零。
\end{ex}

\begin{ex}
\label{is_after}

\index{布尔函数}
编写布尔函数\verb"is_after",读取两个时间对象{\tt t1}和{\tt t2},如果{\tt t1}在{\tt t2}之后则返回{\tt True},否则返回{\tt False}。挑战:不使用{\tt if}语句。
\end{ex}


\section{纯函数}

\index{原型和补丁}
\index{开发方案!原型和补丁}
在下面几个章节中,我们将编写两个函数,实现时间相加的功能。它们将展示两种函数:纯函数和修改。同时将给出一个我称为{\bf 原型和补丁}的开发计划,即对于一个复杂的问题,从简单的原型开始,增量地处理其中的复杂问题。

下面给出\verb"add_time"的一个简单原型:

\beforeverb
\begin{verbatim}
def add_time(t1, t2):
    sum = Time()
    sum.hour = t1.hour + t2.hour
    sum.minute = t1.minute + t2.minute
    sum.second = t1.second + t2.second
    return sum
\end{verbatim}
\afterverb
%
这个函数创建一个新的{\tt Time}对象,初始化其属性并作为引用返回给一个新的对象。这被称为{\bf 纯函数},因为它不改变任何作为参数的对象,除了返回一个值它没有其他作用,类似显示一个值或读取用户输入。

\index{纯函数}
\index{函数类型!纯}

我创建了两个时间对象来测试这个函数,{\tt start}包含了一个电影开始的时间,如{\em Monty Python and the Holy Grail},{\tt duration}包含了电影的时间长度,是1小时35分钟。

\index{Monty Python and the Holy Grail}

\verb"add_time"给出电影结束的时间。

\beforeverb
\begin{verbatim}
>>> start = Time()
>>> start.hour = 9
>>> start.minute = 45
>>> start.second =  0

>>> duration = Time()
>>> duration.hour = 1
>>> duration.minute = 35
>>> duration.second = 0

>>> done = add_time(start, duration)
>>> print_time(done)
10:80:00
\end{verbatim}
\afterverb
%
{\tt 10:80:00}不是你所想要的结果。问题在于这个函数没有处理分钟和秒钟加起来超过60的情况。当这个情况发生时,我们需要将多余的秒钟“进位”到分钟,将多余的分钟“进位”到小时。

\index{进位,加法}

下面给出一个改进的版本:

\beforeverb
\begin{verbatim}
def add_time(t1, t2):
    sum = Time()
    sum.hour = t1.hour + t2.hour
    sum.minute = t1.minute + t2.minute
    sum.second = t1.second + t2.second

    if sum.second >= 60:
        sum.second -= 60
        sum.minute += 1

    if sum.minute >= 60:
        sum.minute -= 60
        sum.hour += 1

    return sum
\end{verbatim}
\afterverb
%
虽然这个函数是正确的,但是它开始变得冗长。我们之后会看见一个精简的版本。


\section{修改函数}
\label{增量}

\index{修改函数}
\index{函数类型!修改}

有时让函数修改参数对象是很有用的。这中情况下,修改对调用者是可见的。这样工作的函数被称为{\bf 修改函数}。

\index{increment}

{\tt increment}是将一定秒数加到一个{\tt 时间}对象,可以写成一个修改函数。下面是一个草稿:

\beforeverb
\begin{verbatim}
def increment(time, seconds):
    time.second += seconds

    if time.second >= 60:
        time.second -= 60
        time.minute += 1

    if time.minute >= 60:
        time.minute -= 60
        time.hour += 1
\end{verbatim}
\afterverb
%
第一行执行基本的操作,后面几行处理我们之前遇到过的特殊情况。

\index{特殊情况}

这个函数对吗?如果参数{\tt seconds}大于60会怎么样?

在这种情况下,进位一次是不够的,我们需要不断进位直到{\tt time.second}小于60。一个解决方案是使用{\tt while}语句替换{\tt if}语句。这能是函数工作正常,但不是很有效率。

\begin{ex}
编写一个正确的{\tt increment},不使用任何循环。
\end{ex}

任何修改函数可以做的都可以使用纯函数来实现。事实上有的编程语言只允许纯函数。有些证据证明使用纯函数的程序相比使用修改函数的程序开发更快捷,错误更少。但是修改函数更加方便使用,而函数的编程效率相对较低。

通常,我推荐你使用纯函数,除非修改函数有明显的优势。这个称为{\bf 函数式编程风格}。

\index{函数式编程风格}


\begin{ex}
编写纯函数版本的{\tt increment},创建一个新的时间对象并返回,而不是修改参数。
\end{ex}


\section{原型与计划}
\label{原型}

\index{原型和补丁}
\index{开发计划!原型和补丁}
\index{有计划的开发}
\index{开发计划!有计划的}
我在展示的开发计划被称为“原型和补丁”。对于每个函数,我编写实现基本功能的原型并进行测试,并对错误打补丁。

这个方法会很有效率,尤其是对问题没有一个深入的认识。但是增量的修改会使得代码变得不必要的复杂,因为需要处理不同的特殊情况,同时由于你很难知道是否找到了所有的错误,代码也不可靠。

另一种是{\bf 有计划的开发},从高层次分析问题将会简化程序的设计。在这个例子中,对问题的分析在于认识到时间对象是3个60进制的数(参考\url{wikipedia.org/wiki/Sexagesimal}。)!{\tt 秒}是“第1列”,{\tt minute}是“第60列”,{\tt 小时}是“第360列”。

\index{六十进制}

当我们编写\verb"add_time"和{\tt increment},我们完成了基60的加法,这也是为什么我们需要从一列到另一列进位。

\index{进位,加法}

这个观察给出了解决整个问题的另一个方法,我们可以将时间对象转换为整数,并利用计算机进行整数计算。

下面的函数将时间转换为整数:

\beforeverb
\begin{verbatim}
def time_to_int(time):
    minutes = time.hour * 60 + time.minute
    seconds = minutes * 60 + time.second
    return seconds
\end{verbatim}
\afterverb
%
下面的函数将整数转换为时间(回忆{\tt divmod}将第一个参数除以第二个参数,并返回商和余数的元组)。

\index{divmod}

\beforeverb
\begin{verbatim}
def int_to_time(seconds):
    time = Time()
    minutes, time.second = divmod(seconds, 60)
    time.hour, time.minute = divmod(minutes, 60)
    return time
\end{verbatim}
\afterverb
%
你也许需要一些思考,并运行一些测试来确保这些函数工作正常。一个测试方法是对许多{\tt x}值检查\verb"time_to_int(int_to_time(x)) == x"。这是一个强壮型检查的例子。

\index{强壮型检查}

当你确信它们是正确的,你可以使用它们重写\verb"add_time":

\beforeverb
\begin{verbatim}
def add_time(t1, t2):
    seconds = time_to_int(t1) + time_to_int(t2)
    return int_to_time(seconds)
\end{verbatim}
\afterverb
%
这个版本比原来的简洁,同时也更容易验证。

\begin{ex}
使用\verb"time_to_int"和\verb"int_to_time"重写{\tt increment}。
\end{ex}

有时候,60进制和10进制的相互转换比处理时间更难。基数转换相对更抽象,我们的直觉更擅长处理时间。

但是如果我们将时间看成60进制的数,并编写转换函数(\verb"time_to_int"和\verb"int_to_time"),我们使得程序更简短,更适合阅读和调试,以及更可靠。

同时也方便以后增加新的特性。例如,想象将两个时间相减,得到两者之间的间隔。最直观的方法是实现借位减法。使用转换函数可以更简单,也更容易正确。

\index{借位减法}
\index{借位,减法}
\index{普遍化}

讽刺的是有时候将问题复杂化(或普遍化)实际简化了问题(因为特殊情况变少,同时出错概率减小)。


\section{调试}
\index{调试}

一个时间对象被称为是良好组织的,如果{\tt 分钟}和{\tt 秒钟}位于0到60(包括0但不包括60),{\tt hours}是正的,{\tt 小时}和{\tt 分钟}是整数,但我们可以允许{\tt 秒钟}有小数部分。

\index{约束}

类似这些要求被称为{\bf 约束},它们应该始终为真。换言之,如果它们非真,则有些地方就有错误。

编写程序检查约束可以帮助你检查错误并找出原因。例如,你可以编写函数\verb"valid_time",读取一个时间对象作为参数,如果违反了约束则返回{\tt False}:

\beforeverb
\begin{verbatim}
def valid_time(time):
    if time.hours < 0 or time.minutes < 0 or time.seconds < 0:
        return False
    if time.minutes >= 60 or time.seconds >= 60:
        return False
    return True
\end{verbatim}
\afterverb
%
在每个函数的开头你可以检查参数来保证它们是有效的:

\index{raise语句}
\index{语句!raise}

\beforeverb
\begin{verbatim}
def add_time(t1, t2):
    if not valid_time(t1) or not valid_time(t2):
        raise ValueError, 'invalid Time object in add_time'
    seconds = time_to_int(t1) + time_to_int(t2)
    return int_to_time(seconds)
\end{verbatim}
\afterverb
%
或者你可以使用{\tt assert}语句,它将检查一个给定的约束,如果检查失败则会发出一个异常错误。

\index{assert语句}
\index{语句!assert}

\beforeverb
\begin{verbatim}
def add_time(t1, t2):
    assert valid_time(t1) and valid_time(t2)
    seconds = time_to_int(t1) + time_to_int(t2)
    return int_to_time(seconds)
\end{verbatim}
\afterverb
%
{\tt assert}语句很有用,它们区分普通的条件判断和异常检查。


\section{术语}

\begin{description}

\item[原型和补丁:] 一种开发计划,包括编写程序的草稿、测试、修改发现的错误。
\index{原型和补丁}

\item[有计划的开发:] 一种开发计划,包括从高层次对程序进行分析,相对增量开发或原型开发有更多的计划。
\index{有计划的开发}

\item[纯函数:] 不修改作为参数的对象的函数。
\index{纯函数}

\item[修改函数:] 修改一个或多个作为参数的对象的函数。
\index{修改函数}

\item[函数式编程风格:] 一种程序设计模式,将大多数函数设计为纯函数。
\index{函数时编程风格}

\item[约束:] 在程序执行时必须始终为真的条件。
\index{约束}

\end{description}


\section{练习}

\begin{ex}
编写函数\verb"mul_time",参数为一个时间对象和一个数,返回一个新的时间对象,其值是原时间和数的乘积。

使用\verb"mul_time"编写一个函数,参数为一个时间对象和一个数值,时间对象表示完成一个比赛所用的时间,数值表示距离,返回一个时间对象,其意义是平均速度(英里每单位时间)。

\index{跑步速度}

\end{ex}

\begin{ex}

\index{Date类}
\index{类!Date}
编写日期对象的类定义,含有属性{\tt 日},{\tt 月}和{\tt 年}。编写函数\verb"increment_date",参数为一个日期对象{\tt date}和一个整数{\tt n},返回值为一个新的日期对象,对应{\tt date}后的{\tt n}天。提示:“2月没有30天...”挑战:你的函数在闰年工作正常吗?参考\url{wikipedia.org/wiki/Leap_year}。

\end{ex}


\begin{ex}

\index{datetime模块}
\index{模块!datetime}

{\tt datetime}模块提供了类似本章节中的日期和时间对象,{\tt date}和{\tt time},它们提供了丰富的方法和运算符,阅读\url{docs.python.org/lib/datetime-date.html}中的文档。

\begin{enumerate}

\item 使用{\tt datetime}模块编写程序,读取一个日期,打印这个日期所在的周。

\index{生日}

\item 编写程序,读如一个生日,打印用户的年龄,以及多少天、小时、分钟和秒钟后是下一个生日。
\end{enumerate}

\end{ex}



\chapter{类和方法}
\chapter{类和方法}


\chapter{继承}
\chapter{继承}

在本章中我们将设计一个来玩扑克的类。如果你不玩扑克,你可以在\url{wikipedia.org/wiki/Poker}中阅读相关信息,但你不必这么做,做练习时我会告诉你一些所需要的信息。

\index{玩扑克,Anglo-American}
\index{纸牌,玩}
\index{扑克}

如果你不熟悉如何玩扑克,你可以阅读\url{wikipedia.org/wiki/Playing_cards}中的内容。


\section{扑克对象}

一副牌有52张牌,每一张属于4个花色和13个等级。4中花色是黑桃、红桃、草花和方块。13个等级是Ace,2,3,4,5,6,7,8,9,10,J,Q和K。对于不同的游戏,Ace肯能比2大,也可能比2小。

\index{等级}
\index{花色}

如果我们要定义一个纸牌的对象,显然其属性有{\tt 等级}和{\tt 花色},但是属性的数据类型应该如何选择?一个方法是使用字符串,如用\verb"'Spade'"描述花色, \verb"'Queen'"描述等级。这么做的一个问题在于实现花色或等级的比较不是很容易。

\index{编码}
\index{加密}
\index{映射}
\index{表示}

另一个方法是使用整数对等级和花色{\bf 编码}。在这里,“编码”指我们将定义一个数字和花色以及数字和等级之间的映射。这种映射不是秘密的(区别“加密”)。

例如,下面的表格给出了花色和对应整数编码:

\beforefig
\begin{tabular}{l c l}
Spades & $\mapsto$ & 3 \\
Hearts & $\mapsto$ & 2 \\
Diamonds & $\mapsto$ & 1 \\
Clubs & $\mapsto$ & 0
\end{tabular}
\afterfig

这个编码简化了卡片的比较,由于高的花色映射为大的数,我们可以通过比较编码来比较花色。

等级的映射相对更明显,每个数字等级映射为对应的整数,对于其他卡片:

\beforefig
\begin{tabular}{l c l}
Jack & $\mapsto$ & 11 \\
Queen & $\mapsto$ & 12 \\
King & $\mapsto$ & 13 \\
\end{tabular}
\afterfig

我使用$\mapsto$符号来突出这些映射不是Python程序实现的。它们属于程序的设计,但它们不直接的表现在代码中。

\index{Card类}
\index{类!Card}

{\tt Card}类的定义如下:

\beforeverb
\begin{verbatim}
class Card(object):
    """represents a standard playing card."""

    def __init__(self, suit=0, rank=2):
        self.suit = suit
        self.rank = rank
\end{verbatim}
\afterverb
%
Init方法有两个可选参数,默认的卡片是草花2。

\index{init方法}
\index{方法!init}

如果要创建一个卡片,你可以调用{\tt Card},并传入你要的花色和等级。

\beforeverb
\begin{verbatim}
queen_of_diamonds = Card(1, 12)
\end{verbatim}
\afterverb
%


\section{类属性}

\index{类属性}
\index{属性!类}

为了以适合人类阅读的格式打印卡片对象,我们需要整数代码到对应花色和等级的映射。一个自然的方法是使用字符串列表,我们将这两个列表赋值给{\bf 类属性}:

\beforeverb
\begin{verbatim}
# inside class Card:

    suit_names = ['Clubs', 'Diamonds', 'Hearts', 'Spades']
    rank_names = [None, 'Ace', '2', '3', '4', '5', '6', '7', 
              '8', '9', '10', 'Jack', 'Queen', 'King']

    def __str__(self):
        return '%s of %s' % (Card.rank_names[self.rank],
                             Card.suit_names[self.suit])
\end{verbatim}
\afterverb
%
定义在类内部任何函数外的变量,如\verb"suit_names"和\verb"rank_names",被称为类属性,因为它们和类对象{\tt Card}像关联。

\index{实例属性}
\index{属性!实例}

该术语区别类似{\tt suit}和{\tt rank}的变量,它们被称为{\bf 实例属性},因为它们和一个特定的实例相关联。

\index{点符号}

两种属性都通过点符号访问。例如,在\verb"__str__"里,{\tt self}是一个卡片对象,{\tt self.rank}是它的等级。类似的,{\tt Card}是一个类对象,\verb"Card.rank_names"是一个和类关联的字符串列表。

每个卡片都有自己的{\tt 花色}和{\tt 等级},但是有一个\verb"suit_names"和\verb"rank_names"的备份。

将这些结合起来,表达式\verb"Card.rank_names[self.rank]"意思是“从{\tt self}对象中使用属性{\tt rank}作为下标,访问{\tt Card}类中的\verb"rank_names"列表,选择对应的字符串”。

\verb"rank_names"中的第一个元素是{\tt None},因为没有等级为零的卡片。通过引入{\tt None}作为占位符,我们很好的将下标2映射为字符串\verb"'2'",以此类推。若不使用这种方法,我们可以使用一个字典,而不是一个列表。

基于我们已有的方法,我们可以创建并打印卡片:

\beforeverb
\begin{verbatim}
>>> card1 = Card(2, 11)
>>> print card1
Jack of Hearts
\end{verbatim}
\afterverb
%
下面给出{\tt Card}类对象和一个卡片实例的图:

\index{状态图}
\index{图!状态}
\index{对象图}
\index{图!对象}

\beforefig
\centerline{\includegraphics{figs/card1.eps}}
\afterfig

{\tt Card}是一个类对象,它的数据类型为{\tt type}。{\tt card1}的类型为{\tt Card}。(为了节省空间,我没有画出\verb"suit_names"和\verb"rank_names"的内容)。


\section{比较卡片}
\label{比较卡片}

\index{运算夫!关系}
\index{关系运算夫}

对于内建类型,关系运算符(如{\tt <},{\tt >},{\tt ==}等)可以比较它们的值。对于用户定义的类型,我们可以通过提供\verb"__cmp__"函数来重载内建运算符的行为。


\verb"__cmp__"接受两个参数,{\tt self}和{\tt other},如果第一个对象大则返回一个正数,如果第一个对象小则返回一个负数,如果两者两等则返回0。

\index{重载}
\index{运算符重载}

卡片的正确顺序不是很明显。例如一个草花3和一个方块2哪个大?一个等级更高,而另一个花色更高。为了比较卡片,你需要确定等级和花色哪个更重要。

答案取决于你玩的是什么游戏。为了简单起见,我们任意的选择花色更为重要,因此所有的黑桃都大于任意方块,依此类推。

\index{cmp 方法@\_\_cmp\_\_ 方法}
\index{方法!\_\_cmp\_\_}

当这个决定后,我们可以编写\verb"__cmp__":

\beforeverb
\begin{verbatim}
# inside class Card:

    def __cmp__(self, other):
        # check the suits
        if self.suit > other.suit: return 1
        if self.suit < other.suit: return -1

        # suits are the same... check ranks
        if self.rank > other.rank: return 1
        if self.rank < other.rank: return -1

        # ranks are the same... it's a tie
        return 0    
\end{verbatim}
\afterverb
%
你可以使用元组比较编写更简洁的程序:

\index{元组!比较}
\index{比较!元组}

\beforeverb
\begin{verbatim}
# inside class Card:

    def __cmp__(self, other):
        t1 = self.suit, self.rank
        t2 = other.suit, other.rank
        return cmp(t1, t2)
\end{verbatim}
\afterverb
%
内建函数{\tt cmp}和方法\verb"__cmp__"有相同的接口:它读取两个值,如果第一个更大则返回一个正数,第二个更大则返回一个负数,如果相等则返回0。

\index{cmp函数}
\index{函数!cmp}


\begin{ex}
为时间对象编写方法\verb"__cmp__"。提示:你可以使用元组比较,也可以考虑使用整数减法。

%    def __cmp__(self, other):
%        return time_to_int(self) - time_to_int(other)

%If {\tt self} is later than {\tt other}, the result is
%a positive number.  If {\tt other} is later, the result
%is negative.  And if {\tt self} and {\tt other} are equal
%(but not necessarily identical)
%the result is zero.

\end{ex}


\section{纸牌}
\index{列表!对象的}
\index{纸牌,玩卡片}

现在我们有卡片对象,下一步是定义纸牌。由于纸牌是由卡片组成的,自然的想法是每个纸牌将一个卡片的列表作为它的属性。

\index{init方法}
\index{方法!init}

下面是{\tt 纸牌}的类定义。初始化方法创建{\tt 卡片}属性,并生成标准的52张卡片:

\index{组成}
\index{循环!嵌套}

\index{纸牌类}
\index{类!纸牌}

\beforeverb
\begin{verbatim}
class Deck(object):

    def __init__(self):
        self.cards = []
        for suit in range(4):
            for rank in range(1, 14):
                card = Card(suit, rank)
                self.cards.append(card)
\end{verbatim}
\afterverb
%
生成纸牌最简单的方法是使用一个嵌套的循环。外层循环枚举0到3的花色,内层循环枚举1到13的等级。每次迭代创建一个对应当前花色和等级的新卡片,并附加在{\tt self.cards}后。

\index{append方法}
\index{方法!append}


\section{打印纸牌}
\label{打印纸牌}

\index{str 方法@\_\_str\_\_ 方法}
\index{方法!\_\_str\_\_}

下面是{\tt 纸牌}的\verb"__str__"方法:

\beforeverb
\begin{verbatim}
#inside class Deck:

    def __str__(self):
        res = []
        for card in self.cards:
            res.append(str(card))
        return '\n'.join(res)
\end{verbatim}
\afterverb
%
这个方法演示了汇聚大字符串的有效的方法:建立一个字符串列表,然后使用{\tt join}方法。内建函数{\tt str}对每个卡片调用\verb"__str__"方法,并返回表征的字符串。

\index{汇聚!字符串}
\index{字符串!汇聚}
\index{join方法}
\index{方法!join}
\index{新行}

由于我们在新行符号上调用{\tt join},卡片通过新行符分割。下面是结果:

\beforeverb
\begin{verbatim}
>>> deck = Deck()
>>> print deck
Ace of Clubs
2 of Clubs
3 of Clubs
...
10 of Spades
Jack of Spades
Queen of Spades
King of Spades
\end{verbatim}
\afterverb
%
虽然结果看起来是52行,实际上它是一个包含换行符的一个长字符串。


\section{添加,删除,洗牌,理牌}

对于卡片,我们需要一个从纸牌中删除一张卡片并返回改纸牌的方法。列表的{\tt pop}方法提供了一个便捷的实现方式:

\index{pop方法}
\index{方法!pop}

\beforeverb
\begin{verbatim}
#inside class Deck:

    def pop_card(self):
        return self.cards.pop()
\end{verbatim}
\afterverb
%
由于{\tt pop}删除列表中{\em 最后}一张卡片,我们是纸牌底部进行操作。在现实生活中从底部操作是不可取的\footnote{参考\url{wikipedia.org/wiki/Bottom_dealing}},但在这里也是行得通的。

\index{append方法}
\index{方法!append}

为了添加一张卡片,我们可以使用列表的{\tt append}方法:

\beforeverb
\begin{verbatim}
#inside class Deck:

    def add_card(self, card):
        self.cards.append(card)
\end{verbatim}
\afterverb
%
类似这种调用其他函数,本身不做很多具体工作的方法有时被称为{\bf 饰面}。这个词来自木工行业,通常人们将一片很薄的好木材黏贴在一块便宜的木材的表面。

\index{饰面}

在本例中我们定义个一个“单薄”的方法,使列表的操作适用于纸牌。

作为另一个例子,我们可以编写纸牌方法{\tt shuffle},调用{\tt random}模块中的{\tt shuffle}函数: 

\index{random模块}
\index{模块!random}
\index{shuffle函数}
\index{函数!shuffle}

\beforeverb
\begin{verbatim}
# inside class Deck:
            
    def shuffle(self):
        random.shuffle(self.cards)
\end{verbatim}
\afterverb
%
不要忘了导入{\tt random}模块。

\begin{ex}
\index{sort方法}
\index{方法!sort}

编写纸牌方法{\tt sort},通过调用列表方法{\tt sort}对{\tt 纸牌}中的卡片进行排序。{\tt sort}适用我们定义的\verb"__cmp__"方法来决定排序顺序。
\end{ex}



\section{继成}

\index{继成}
\index{面向对象编程}

和面向对象编程最相关的语言特点是{\bf 继成}。继成是通过修改已有的类来创建新的类的能力。

\index{父类}
\index{子类}
\index{类!子}
\index{子类}
\index{超类}

之所以称为“继成”是因为新的类继成了已有类的方法。换言之,已有类被称为{\bf 父类},新创建的类被称为{\bf 子类}。

例如,我们要创建一个类来表示“手牌”,即一个玩家手中持有的牌。手牌类似纸牌:它们都是卡片的集合,都需要类似添加和删除卡片的操作。

手牌同时区别于纸牌:我们需要手牌有一些操作,但这些操作对纸牌来说没有意义。例如,在扑克中我们需要比较两个手牌来决定谁获胜。在桥牌中,我们需要计算手牌的分数,以便要价。

两个类的相似关系和不同点导致了它们的继成关系。

子类的定义和其他类定义相同,但父类的名字出现在括号中:

\index{括号!父类在}
\index{父类}
\index{类!父}
\index{Hand类}
\index{类!Hand}

\beforeverb
\begin{verbatim}
class Hand(Deck):
    """represents a hand of playing cards"""
\end{verbatim}
\afterverb
%
该定义表示{\tt Hand}继成{\tt Deck},这意味着我们可以想纸牌一样对手牌使用类似\verb"pop_card"和\verb"add_card"的方法。

{\tt Hand}同时继成了{\tt Deck}的\verb"__init__",但这并不是我们想要的:手牌的{\tt cards}属性应该被初始化为一个空的列表,而不是塞满52张新牌。


\index{重载}
\index{init方法}
\index{方法!init}

如果我们给{\tt Hand}提供一个初始化方法,它将重载{\tt Deck}中的方法:

\beforeverb
\begin{verbatim}
# inside class Hand:

    def __init__(self, label=''):
        self.cards = []
        self.label = label
\end{verbatim}
\afterverb
%
因此当你创建了一个手牌对象,Python调用这个初始化方法:

\beforeverb
\begin{verbatim}
>>> hand = Hand('new hand')
>>> print hand.cards
[]
>>> print hand.label
new hand
\end{verbatim}
\afterverb
%
但是其他方法从{\tt Deck}继成,所以我们可以使用\verb"pop_card"和\verb"add_card"来处理卡片:

\beforeverb
\begin{verbatim}
>>> deck = Deck()
>>> card = deck.pop_card()
>>> hand.add_card(card)
>>> print hand
King of Spades
\end{verbatim}
\afterverb
%
自然的下一步是将这些代码封装成\verb"move_cards"方法:

\index{封装}

\beforeverb
\begin{verbatim}
#inside class Deck:

    def move_cards(self, hand, num):
        for i in range(num):
            hand.add_card(self.pop_card())
\end{verbatim}
\afterverb
%
\verb"move_cards"读取两个参数,一个为手牌对象,一个为要处理的卡片数。它同时修改{\tt self}和{\tt hand},并返回{\tt None}。

在有的游戏中,卡片在不同手牌中移动,或者从手牌移动到台面牌。你可以使用\verb"move_cards"来进行任何这些操作:{\tt self}可以是手牌或者桌面牌,{\tt hand}同样可以是{\tt 桌面牌},虽然它名字叫手牌。

\begin{ex}
编写纸牌方法\verb"deal_hands",读取两个参数,分别是手牌的个数和每个手牌的卡片个数,并根据这个数字创建手牌对象,返回一个手牌对象列表。
\end{ex}
继成是一个有用的特性。通过使用继成可以将一些重复行的程序写得更优雅。继成使得代码重用更容易,你可以自定义父类的行为,而不需要去改变它。在某些情况下,继成结构反应了现实中问题的结构,使得问题容易理解。

另一方面,继成会使得程序难以阅读。当一个方法被调用时,有时方法的定义的位置不是很清楚。相关的代码可能散布在多个模块中。同时,很多可以用继成做的事情同样可以不用继成来实现,甚至做得更好。


\section{类图}

目前位置我们见过了栈图,它现实了程序的状态,对象图,它现实了对象的属性和值。这些图是程序执行过程中的快照,因此它们随程序的执行而改变。

同时它们非常详细,有时对某些应用太详细了。类图是个相对抽象的表示程序结构的图,它显示了类的结构和类之间的关系,而不是显示每一个对象。

存在以下几种类的关系:

\begin{itemize}

\item 类中的一个对象包含另一个类的对象的引用。例如,每个长方形包含一个点的应用,每个纸牌包含多个卡片的引用。这类关系被称为{\bf 包含},正如“长方形包含一个点”。

\item 一个类是另一个类的继成。这种关系被称为{\bf 是},正如“手牌是纸牌的一种”。

\item 一个类也许决定于另一个类,即一个类的改动要求其他类的改动。

\end{itemize}

\index{属于关系}
\index{包含关系}
\index{类图}
\index{图!类}
\index{UML}

{\bf 类图}是图形化的表示这些关系的图\footnote{我在这里使用的图类似UML(参考\url{wikipedia.org/wiki/Unified_Modeling_Language})}。例如,下面的图显示了{\tt Card},{\tt Deck}和{\tt Hand}的关系。

\beforefig
\centerline{\includegraphics{figs/class1.eps}}
\afterfig

空心三角箭头表示“属于关系”,在本例中表示手牌继成了纸牌。

标准箭头表示“有关系”,在本例中纸牌有卡片的引用。

\index{多态(在类图中)}

在箭头头部的新号({\tt *})表示{\bf 多态}。它指示了纸牌中有多少张卡片。多态可以是一个简单的数字,如{\tt 52},一个范围,如{\tt 5..7},或一个新号,表示纸牌可以有任何多的卡片。

更详细的图将显示纸牌事实上又一个卡片{\em 列表},但是通常类似列表和字典的内建类型不出现在类图中。

\begin{ex}
阅读{\tt TurtleWorld.py},{\tt World.py}和{\tt Gui.py},绘制类图来显示它们之间的关系。
\end{ex}


\section{调试}
\index{调试}

继承使得调试成为一个挑战,因为当你对一个对象调用方法是,你也许不知道哪个方法被调用。

\index{多态}

假设你编写了一个手牌对象的函数,你希望它能适用于任何手牌,如扑克手牌,桥牌手牌等。假设你调用了类似{\tt shuffle}的方法,你也许调用的是定义在{\tt 纸牌}中的方法,但是如果某个子类重载了这个方法,你会得到那个版本的方法。

\index{执行流程}
当你不确定程序执行的流程,最简单的方法是在相关的方法的开始部分添加打印语句。如果{\tt Deck.shuffle}打印类似{\tt Running Deck.shuffle}的语句,那么当程序执行时它将追踪执行的流程。

另一个方法是使用下面的函数,它接受一个对象和一个方法名(以字符串的形式),返回提供该方法定义的类:

\beforeverb
\begin{verbatim}
def find_defining_class(obj, meth_name):
    for ty in type(obj).mro():
        if meth_name in ty.__dict__:
            return ty
\end{verbatim}
\afterverb
%
下面给出一个例子:

\beforeverb
\begin{verbatim}
>>> hand = Hand()
>>> print find_defining_class(hand, 'shuffle')
<class 'Card.Deck'>
\end{verbatim}
\afterverb
%
因此这个手牌的{\tt shuffle}方法来自{\tt 纸牌}。

\index{mro方法}
\index{方法!mro}
\index{方法解析顺序}

\verb"find_defining_class"使用{\tt mro}方法得到用于查找方法的类对象(类型)列表。“MRO”指“method resolution order”,即方法解析顺序。

\index{重载}
\index{接口}
\index{先决条件}
\index{后决条件}

以下是程序设计的建议:每当你重载一个方法,新方法的接口应该和原方法相同。它们应使用相同的参数,返回相同的类型,符合相同的先决条件和后决条件。如果你遵守这些规则,你会发现任何使用于超类实例(如纸牌)的函数同样使用于子类实例(如手牌或扑克手牌)。

如果你违反了这些规则,你的代码很肯会崩溃。


\section{术语}

\begin{description}

\item[编码:] 通过建立映射使用一个集合的值表示另一个集合的值。
\index{编码}

\item[类属性:] 和类对象关联的属性。类属性定义在类定义内部,方法定义外部。
\index{类属性}
\index{属性!类}

\item[实例属性:] 和类的实例相关的属性。
\index{实例属性}
\index{属性!实例}

\item[饰面:] 为另一个函数提供不同的接口而不进行许多计算的方法或函数。
\index{饰面}

\item[继承:] 通过修改先前定义的类来创建新的类的能力。
\index{继承}

\item[父类:] 被子类继承的类。
\index{父类}

\item[子类:] 通过继承已有的类而创建的新的类。
\index{子类}
\index{类!子}

\item[属于关系:] 子类和父类之间的关系。
\index{属于关系}

\item[包含关系:] 两个类之间的关系,其中的一个类包含另一个类的引用。
\index{包含关系}

\item[类图:] 表示程序中的类以及类之间的关系的图。
\index{类图}
\index{图!类}

\item[多态:] 类图中的标记,对于包含关系,该标记说明了对其他类的引用的数量。
\index{多态(在类图中)}

\end{description}


\section{练习}

\begin{ex}
\index{扑克}

下面是扑克中肯能的手牌,以大小升序排列(概率降序):

\begin{description}

\item[对子:] 等级相同的两张牌。
\vspace{-0.05in}

\item[双对:] 等级相同的两对对子。
\vspace{-0.05in}

\item[相同的3个:] 等级相同的3张卡片。
\vspace{-0.05in}

\item[顺子:] 5张等级连续的卡片(aces可以为高或低,因此{\tt Ace-2-3-4-5}是个顺子,{\tt 10-Jack-Queen-King-Ace}也是顺子,但是{\tt Queen-King-Ace-2-3}不是)。
\vspace{-0.05in}

\item[同花:] 5张花色相同的卡片。
\vspace{-0.05in}

\item[3张相同和2张相同的牌:] 3张牌等级相同,另2张牌等级相同。
\vspace{-0.05in}

\item[相同的4个] 等级相同的4张牌。
\vspace{-0.05in}

\item[同花顺:] 5张花色相同的顺子。
\vspace{-0.05in}

\end{description}
%
本练习的目的是估计抽到不同的手牌的概率。

\begin{enumerate}

\item 从\url{thinkpython.com/code}下载下列文件:

\begin{description}

\item[{\tt Card.py}]: 本章中{\tt Card},{\tt Deck}和{\tt Hand}的完整版本。

\item[{\tt PokerHand.py}]: 未完成的手牌的类,包含一些测试代码。

\end{description}
%
\item 如果你运行{\tt PokerHand.py},它将抽取7张手牌,并检查是否包含同花顺。在你继续前请仔细阅读代码。

\item 在{\tt PokerHand.py}添加方法\verb"has_pair",\verb"has_twopair"等,根据相关准则进行判断并返回True或者False。你的代码应使用于任意张数的手牌(虽然5和7是最常见的数量)。

\item 编写方法{\tt classify},找出手牌中值最大的分类,并记录在{\tt label}属性中。例如,一个7张的手牌可能有一个顺子和一对,那么应该标记为“顺子”。

\item 当你确信你的分类方法工作正常,下一步是估计不同手牌出现的概率。在{\tt PokerHand.py}中编写函数,对一副牌进行洗牌,然后分为若干手牌,对手牌分类,统计不同分类出现的次数。

\item 打印分类和概率的表。多次运行程序,直到输出稳定到一个合理的精度。将你的结果和\url{wikipedia.org/wiki/Hand_rankings}比较。

\end{enumerate}
\end{ex}


\begin{ex}

\index{沼泽}
\index{TurtleWorld}
本练习使用章节~\ref{turtlechap}的TurtleWorld。
你将编写代码,让乌龟玩贴标签的游戏。如果你不熟悉游戏规则,参考\url{wikipedia.org/wiki/Tag_(game)}。

\begin{enumerate}

\item 下载\url{thinkpython.com/code/Wobbler.py}并运行。你将看到乌龟世界里有3只乌龟。如果你按下{\sf Run}按钮,乌龟将随机移动。

\item 阅读代码,了解它是如何工作的。{\tt 闲逛者}类继承{\tt 乌龟}类,这意味着{\tt 乌龟}的方法{\tt lt},{\tt rt},{\tt fd}和{\tt bk}同样适用于闲逛者。

{\tt step}方法由乌龟世界调用。它调用{\tt steer},令乌龟朝向指定的方向。{\tt wobble}根据乌龟的笨拙程度随机的转向,{\tt move}根据乌龟的速度向前移动一定的距离。

\index{跟踪者}

\item 创建{\tt Tagger.py},导入{\tt Wobbler},定义类{\tt Tagger},继承{\tt Wobbler}。以{\tt Tagger}的类对象作为参数的调用\verb"make_world"。

\item 在{\tt Tagger}中添加{\tt steer}方法,重载{\tt Wobbler}中的函数。作为起步练习,编写始终将乌龟指向原点的版本。提示:使用数学函数{\tt atan2}和乌龟属性{\tt x},{\tt y}和{\tt heading}。

\item 修改{\tt steer},将乌龟约束在边界内。为了调试,你可以使用{\sf Step}按钮,它将对每个乌龟调用{\tt step}。

\item 修改{\tt steer},令每只乌龟朝向离它最近的邻居。提示:乌龟有属性{\tt world},是它们所在的乌龟世界的引用,而乌龟世界有属性{\tt animals},对应所有乌龟的列表。

\item 修改{\tt steer}让乌龟玩贴标签游戏。你可以在{\tt Tagger}上添加方法,并重载{\tt steer}和\verb"__init__",但你不能修改或重载{\tt step},{\tt wobble}或{\tt move}。另外{\tt steer}允许改变乌龟的方向,但不能改变位置。

修改规则和你的{\tt steer}方法,增加游戏的质量。例如,慢的乌龟应该有机会给快的乌龟贴上标签。

\end{enumerate}

你可以在\url{thinkpython.com/code/Tagger.py}下载到我的程序。
\end{ex}




\chapter{实例学习:Tkinter}
\chapter{实例学习:Tkinter}


\appendix
\chapter{调试}
\chapter{调试}

\index{调试}

程序中会出现不同的错误,加以区分有助于加快对错误的跟踪。

\begin{itemize}

\item 语法错误是Python在将源代码转换为字节码时产生的。它们通常说明程序的语法有错误。例如:在 {\tt def}语句后忽略冒号会产生类似{\tt SyntaxError: invalid syntax}的信息。

\item 运行时错误由解释器在程序运行出错时产生。大多数的运行时错误消息包含错误发生的位置和正在执行的函数。例如,一个无穷递归的函数最终导致运行时错误“maximum recursion depth exceeded”。

\item 语义错误指程序执行过程中没有产生出错信息,但程序没有做正确的工作。例如,一个表达式没有按照你期望的顺序进行求值,导致错误的结果。

\end{itemize}

\index{语法错误}
\index{运行时错误}
\index{语义错误}
\index{错误!编译时}
\index{错误!语法}
\index{错误!运行时}
\index{错误!语义}
\index{异常}

调试的第一步是找出你遇到了什么类型的错误。虽然下面的章节根据错误类型组织的,但一些方法使用于多种情况。


\section{语法错误}

\index{错误消息}

当你找到原因后语法错误一般很容易修正。不幸的是,错误消息常常不是很有帮助。最常见的消息是{\tt SyntaxError: invalid syntax}和{\tt SyntaxError: invalid token},它们都不包含很多信息量。

另一方面,错误消息告诉你问题发生在程序中的什么位置。事实上,当Python遇到问题时它将告诉你,但这不一定是错误的地方。有时错误在错误消息位置的前面,通常是前几行。

\index{增量开发}
\index{开发计划!增量}

如果你增量的开发程序,你会清楚错误在哪里。它就是你最新添加的代码。

如果你从书中复制一段代码,那么从仔细地检查你的代码和书中的代码开始。检查每一个字符。同时记住书本也会有错误,如果你遇到类似语法错误,也许它就是。

下面给出一些避免常见语法错误的方法:

\index{syntax}

\begin{enumerate}

\item 避免使用Python关键字作为变量名。

\index{关键字}

\item 确保你在每个复合语句头的尾部加上了冒号,包括{\tt for},{\tt while},{\tt if}和{\tt def}。

\index{头部}
\index{冒号}

\item 确保代码中的字符串都有匹配的引号。

\index{引号标记}

\item 如果你用三重引号标记多行字符串,确保你正确的结束该字符串。一个未终止的字符串会导致程序尾部{\tt invalid token}错误,或者程序接下来的部分都被认为字符串,直到下一个字符串。在第二种情况,Python可能不会产生一个错误消息!

\index{多行}
\index{字符创!多行}

\item Python将未封闭的运算符---\verb+(+,\verb+{+,或\verb+[+---的下一行作为当前语句的一部分。通常第二行将产生错误。

\item 在条件语句中使用传统的{\tt =}而不是{\tt ==}。

\index{条件}

\item 确保每行缩进正确。Python可以处理空格和tabs,但如果你混淆它们会产生错误。避免这个问题最好的方法是使用适合Python编辑、会自动缩进的文本编辑器。

\index{缩进}
\index{空白符}

\end{enumerate}

如果这些没有效,继续阅读下一章节...


\subsection{我做了修改但是没有任何区别}

如果解释器报错,但你找不到错误,有可能你和解释器看到不不是相同的代码。检查你的编程环境,确保你正在编辑的文件是Python将要运行的。

如果你不确定,在程序的开始位置添加明显的故意的语法错误,再次运行,如果解释器没有找到这个错误,说明你不在运行新的程序。

下面是一些可能的出错原因:

\begin{itemize}

\item 你编辑了文件,运行前忘记了保存。有的编程环境会为你自动保存,有的不会。

\item 你修改了文件的名字,但仍然运行老的名字。

\item 你的开发环境配置有误。

\item 如果你在编写模块并使用{\tt import},确保你写的模块名不同于Python标准模块名。

\index{模块!reload}
\index{reload函数}
\index{函数!reload}

\item 如果你使用{\tt import}读取一个模块,记得重启解释器或使用{\tt reload}来读取一个修改过的文件。如果你再次导入模块,解释器将不做任何事。

\end{itemize}

如果你陷入困境找不到出错的原因,一个方法是从一个类似“Hello,World!”的新程序开始,确保你从一个已知的可运行的程序开始。然后逐步添加原程序中的代码到新程序。


\section{运行时错误}

当你的程序没有语法错误,Python可以编译并运行。这时可能发生什么错误呢?


\subsection{我的程序什么也没做}

这个问题通常发生在你的文件包含函数和类,但没有任何调用执行的语句。也许你故意这样,因为你仅仅打算导入这个模块来提供函数和类。

如果这不是故意的,确保你调用一个函数来开始执行,或从一个交互的命令提示行下执行。参见下面的“执行流程”章节。


\subsection{我的程序挂起了}
\index{无限循环}
\index{无限递归}
\index{挂起}

如果一个程序停止,但上去什么也没做,我们称“挂起”。通常意味着程序陷入一个无限循环或者无线递归。

\begin{itemize}

\item 如果你怀疑某个循环引起这个问题,在循环开始时添加{\tt print}语句打印“进入循环”,在循环结束时打印“exiting the loop”退出循环。

运行程序,如果你的到第一条消息,但没有第二条消息,你得到一个无限循环。参见“无限循环”章节。

\item 大多数时候,无限循环会令程序运行一段时间,然后产生“RuntimeError: Maximum recursion depth exceeded”的错误。如果是这样,参考下面的“无限递归”章节。

如果你没有得到这样的错误信息,但你怀疑某个递归方法或函数有问题,你仍可以使用“无限递归”章节中提到的方法。

\item 如果这些方法都没有,尝试测试其他的循环、递归的函数和方法。

\item 如果还是没有效果,有可能你不清楚程序执行的流程。参考下面的“执行流程”章节。

\end{itemize}


\subsubsection{无限循环}
\index{无限循环}
\index{循环!无限}
\index{条件}
\index{循环!条件}

如果你有一个无限循环并且你认为这个循环导致了问题,在循环体结束的地方添加{\tt print}语句打印条件语句中的变量值和条件值。

例如:

\beforeverb
\begin{verbatim}
while x > 0 and y < 0 :
    # do something to x
    # do something to y

    print  "x: ", x
    print  "y: ", y
    print  "condition: ", (x > 0 and y < 0)
\end{verbatim}
\afterverb
%
现在当你运行程序,每个循环你都会看到3行输出。对于最后一次循环,条件应该为{\tt false}。如果循环不断执行,你可以看到{\tt x}和{\tt y}的值,也许能够发现为什么它们没有被正确的更新。


\subsubsection{无限递归}
\index{无限递归}
\index{递归!无限}

大多数时候,无限循环会令程序运行一段时间,然后产生“RuntimeError: Maximum recursion depth exceeded”的错误。

如果你怀疑一个函数或方法导致了无限递归,首先检查存在一个基本状态。换言之,应该有一个状态使函数或方法直接返回,而不是再次递归调用。如果不是,你需要重新思考算法,并设计一个基本状态。

如果存在一个基本状态,但程序似乎没有运行到这个状态,你可以在函数或方法开始的部分添加{\tt print}语句打印参数。现在当你运行程序,每次函数或方法被调用时,你将看到几行关于参数的输出。如果参数没有向基本状态靠拢,你也许会发现为什么这样。


\subsubsection{执行流程}
\index{执行流程}

如果你不确定程序的执行流程,在每个函数开始的地方加入{\tt print}语句,打印类似“entering function {\tt foo}”的语句,其中{\tt foo}是函数的名字。

现在当你运行程序,它将打印每个函数被调用的追踪。


\subsection{当我运行程序,我得到一个异常}
\index{异常}
\index{运行时错误}

有时程序运行时出错,Python将打印异常的名字、问题发生的位置和回溯。

\index{回溯}

回溯识别当前运行的函数,然后识别调用该函数的函数,以此类推。换言之,它追踪了你到达这个错误所经过的函数调用。同时它也包括这些调用在文件中的位置。

第一步是检查代码中对应的出错的位置,或许你能发现错误。下面是常见的运行时错误:

\begin{description}

\item[名字错误:]  你试图使用一个不在当前环境下的变量。记住局部变量是局部的,你不能在定义它们的函数的外部引用它们。

\index{名字错误}
\index{类型错误}
\index{异常!名字错误}
\index{异常!类型错误}

\item[类型错误:] 可能的几个原因是:

\begin{itemize}

\item 你试图不适当的使用值。例如:使用一个非整数的值作为字符串、列表或元组的下标。

\index{下标}

\item 格式字符串和用于转换的对象不匹配。这发生在或者数量不相同,或者调用了一个非法的转换。

\index{格式运算符}
\index{运算符!格式}

\item 你调用函数或方法时传递的参数个数有误。对于方法,检查方法定义,确保第一个参数是{\tt self},然后检查方法调用,确保你对一个对象调用方法,并正确的提供其他参数。

\end{itemize}

\item[键错误:]  你试图使用字典中不存在的键访问对应的元素。

\index{键错误}
\index{异常!键错误}
\index{字典}

\item[属性错误:] 你试图访问一个不存在的属性或方法。检查拼写!你可以使用{\tt dir}来列举存在的属性。

如果属性错误指出一个对象是{\tt 空类型},及它是{\tt 空}的。一个常见的原因是函数结束时忘记返回值。如果在返回尾部没有{\tt return}语句,函数将返回一个{\tt None}。另一个原因是使用类似列表中的{\tt sort}方法,它返回{\tt None}。

\index{属性错误}
\index{异常!属性错误}

\item[下标错误:] 用来访问列表、字符串或元组的下标超过长度减1。在错误发生的前一行使用{\tt print}语句显示下标值和序列的长度,检查两个值是否正确。

\index{下标错误}
\index{异常!下标错误}

\end{description}

\index{调试器(pdb)}
\index{Python调试器(pdb)}
\index{pdb(Python调试器)}

Python调试器({\tt pdb})允许你在错误前检查程序状态,有助于追踪异常。你可以在 \url{docs.python.org/lib/module-pdb.html}阅读有关{\tt pdb}的内容。


\subsection{我添加了太多的{\tt print}语句,输出令我应接不暇}

\index{print语句}
\index{语句!print}

使用{\tt print}语句调试的一个问题是你会被大量的输出信息掩埋。有两个处理方法:简化输出或简化程序。

简化输出可以删除或注释不用的{\tt print}语句,或将它们合并,或格式化输出使得它们容易理解。

简化程序你可以做一下几件事。首先缩小程序处理的问题的规模。例如,如果你在搜索一个列表,搜索一个{\em 小的}列表。如果程序从用户读取输入,输入最简单的参数。

\index{死区代码}

第二,整理程序,删除死区代码,使程序易读。例如,如果你认为问题深嵌在程序中,试图用简单的结构重写那部分。如果你怀疑一个大函数有错误,试图将它分割成小函数,然后分别测试。

\index{测试!最小测试案例}
\index{测试案例,最小}

通常寻找最小测试案例的过程让你找到错误。如果你发现一个程序对于一个情况适用,对另一个情况不使用,这将给你一些线索。

同样,重写一段代码有助于发现一些微小的错我。如果你做了一个你认为不会影响程序的修改,但事实上影响了,这个方法可以给你警戒。


\section{语义错误}
\index{语义错误}
\index{错误!语义}

在某种程度上,语义错误时最难调试的,因为解释器不能提供任何错误信息。你所知道的只有程序应该怎么做。

第一步是建立程序代码和你见到的行为之间的联系。你需要假设程序实际上做了什么。一个主要困难在于计算器运行的太快了。

你常希望你可以降低程序的速度,是人类可以跟上,通过使用调试器,你可以实现这步。但是在关键地方加入一些{\tt print}语句所用的时间通常短于建立调试器,加入删除断点,然后“逐步”运行程序直到错误发生。

\subsection{我的程序不工作}

你需要问自己这些问题:

\begin{itemize}

\item 有没有什么程序应该做却没有发生?找到执行该函数的代码段,确保程序被执行。

\item 有没有什么不应该发生的发生了?找到执行该函数的代码段,查看它是否执行了?

\item 有没有代码的执行效果与你期望的不同?确保你理解有问题的代码,尤其是包含调用其他Python模块中的函数或方法。阅读你调用的函数的文档,用一些简单的例子进行测试。

\end{itemize}

在编程时,你心中需要有一个关于程序如何工作的模型。如果你的程序没有按照你期望的工作,很可能问题不在于程序,而在于你心中的模型。

\index{模型,心中的}
\index{心中的模型}

修正你心中的模型的最好的方法是将程序分割为不同部分(通常是函数和方法),并分别测试。一旦你发现了模型和现实的差异,你就可以解决问题。

当然,在开发的过程中尼需要建立并测试组件。如果你遇到了问题,只有一小部分新的代码是不确定正确性的。


\subsection{我写了一个很长的表达式,它没有按照我期望的工作}

\index{表达式!大而复杂}
\index{大而复杂的表达式}

编写复杂的表达式是合理的,如果它们可读。但是它们调试起来很困难。通常我们将一个复杂的表达式分割成一系列的临时变量的赋值。

例如:

\beforeverb
\begin{verbatim}
self.hands[i].addCard(self.hands[self.findNeighbor(i)].popCard())
\end{verbatim}
\afterverb
%
可以重写为:

\beforeverb
\begin{verbatim}
neighbor = self.findNeighbor(i)
pickedCard = self.hands[neighbor].popCard()
self.hands[i].addCard(pickedCard)
\end{verbatim}
\afterverb
%
这个明晰的版本更适于阅读,因为变量名提供了额外的文档,同时调试也更容易,因为你可以检查中间变量的类型和它们的值。

\index{临时变量}
\index{变量!临时}
\index{运算符优先级}
\index{优先级}

大的表达式的另一个问题是计算的顺序不一定是你期望的。例如,如果你将表达式$\frac{x}{2 \pi}$翻译为Python,你也许会写成:

\beforeverb
\begin{verbatim}
y = x / 2 * math.pi
\end{verbatim}
\afterverb
%
这是不正确的,因为乘法和除法有相同的优先级,因此是从左往右计算的,这个表达式计算的是$x \pi / 2$。

调试表达式的一个好的方法是添加括号,使得计算顺序简洁明了:

\beforeverb
\begin{verbatim}
 y = x / (2 * math.pi)
\end{verbatim}
\afterverb
%
当你不确定计算优先级是,使用括号。不仅程序将工作正常(按你的要求执行),同时也让那些没有记住优先级规则的人阅读起来更方便。


\subsection{我的函数或方法没有按照我期望的返回}
\index{return语句}
\index{语句!return}

如果你的{\tt return}语句包含一个复杂的表达式,你没有机会在返回前打印{\tt 返回}值。同样你可以使用临时变量。例如:对于

\beforeverb
\begin{verbatim}
return self.hands[i].removeMatches()
\end{verbatim}
\afterverb
%
你可以写成:

\beforeverb
\begin{verbatim}
count = self.hands[i].removeMatches()
return count
\end{verbatim}
\afterverb
%
现在你在返回前可以打印{\tt count}的值。


\subsection{我实在是卡住了,我需要帮助}

第一,尝试离开电脑几分钟。电脑辐射会对大脑产生影响,导致下列几种症状:

\begin{itemize}

\item 沮丧和愤怒

\index{沮丧}
\index{愤怒}
\index{调试!情绪反应}
\index{带情绪的调试}

\item 迷信的人认为“电脑讨厌我”,并神奇的相信“程序仅当我向后戴着帽子时才工作正常”。

\index{调试!迷信}
\index{迷信的调试}

\item 随机漫步编程(用各种可能的方法编程,并选择工作正常的那个)。

\index{随机漫步编程}
\index{开发计划!随机漫步编程}

\end{itemize}

如果你发现你有以上任意一种症状,站起来走一走。当你心绪平静时,思考一下程序。它是做什么的?什么肯能造成了这种行为?上次可以工作的程序是什么时候?下一步做什么?

有时找到一个错误很费时间。我常常在我离开电脑,让思维游荡的时候找到错误。一些找到错误最好的地方有火车上,浴室里,以及临睡前。


\subsection{不,我真的需要帮助}

即使最好的程序员也会卡住。有时你在一个程序上工作了太长的时间,因此你难以发现错误。而他人可能一眼就发现问题。

在你向其他人寻求帮助前,你需要做好准备。你的程序需要尽可能简洁,你需要最少的输入来重现错误。你需要在合适的位置加入{\tt print}语句,同时输出应可理解。你需要能够以简洁的语言描述问题。

当你想某人求助,你需要提供足够的信息:

\begin{itemize}

\item 是否有出错消息?它是什么?指向程序的哪部分?

\item 错误出现前你做的最后一步是什么?你写的最后几行是什么?什么新的测试导致了错误?

\item 你做了哪些尝试?你学到了什么?

\end{itemize}

当你找到了错误,花时间想一想你怎么能更快的定位它。下一次你遇到类似的问题,你就可以更快的找到问题。

记住,目标不仅仅是让程序工作,而是学会如何让程序工作。

\printindex

\clearemptydoublepage

=======
\chapter{编程的方式}

这本书的的目的是教会大家如何像计算机科学家一样思考。计算机科学用严谨的语言来表明思想,尤其是计算。像工程师,他们设计,把各个组件装配成系统并且在可选方案中评估出折中方案。像科学家,他们研究复杂系统的性能,做出假定并且测试假设。


\index{解决问题}

对一个计算机科学家来说最重要的是解决问题。解决问题意味着清晰明确的阐述问题,积极思考问题答案,并且清楚正确的表达答案的能力。实践证明:学习如何编程是一种很好的机会来练习解决问题的技巧。这也是为什么把这章叫做“编程的方式“。\\

一方面,你将学习编程,一个非常有用的技巧。另一方面你将会把编程作为一种科技。随着我们的深入学习,这点会渐渐明晰。

\section{Python编程语言}
\index{编程语言}
\index{语言!编程}

我们将要学习的编程语言是Python。Python仅是高级语言中的一种,你可能
也听说过其他的高级编程语言,比如C,C++,Perl,和Java。\\

也有一些低级语言,有时也被称为机器语言或者汇编语言。一般来说,计算机
只能执行用低级语言编写的程序。所以,用高级语言编写的程序在执行前必须
做相应的处理。这会花费一定的时间,同时这也是高级语言的“缺点“。\\

\index{移植性}
\index{高级语言}
\index{低级语言}
\index{语言!高级}
\index{语言!低级}

然而,高级语言的优点也是无限的。首先,用高级语言编程是一件非常容易的。
用高级语言编程通常花费的时间比较少,同时编写的程序简短,易读,易纠错
。第二,高级语言是可移植的,这意味着他们可以不加修改(或者修改很少)地运行在不同的平台上。低级语言编写的程序只能在一个机器上运行,如果想
要运行在另外一台机器上,必须得重写。\\

基于这些优点,几乎所有的程序都是用高级语言编写的。低级语言统称仅仅用
在一些专门的应用程序中。

\index{complie}
\index{interpret}

有两种程序把高级语言“处理”成低级语言:解释器和编译器。解释器读取源程序,解释执行,也就是按照程序表达的意思"做事"。解释器一次解释一点,或者说,一次读取一行,然后执行。\\

\beforefig
\centerline{\includegraphics[height=0.77in]{figs/interpret.eps}}
\afterfig

\index{源码}
\index{目标}
\index{可执行代码}

编译器读取程序,完全转换之。在这种情况下,高级语言程序叫做源码,编译后的程序叫做目标代码或者叫可执行代码。一旦程序被编译,就可以直接执行,无须再编译。

\beforefig
\centerline{\includegraphics[height=0.77in]{figs/compile.eps}}
\afterfig

一般地,我们把python当作是解释型语言,因为用Python编写的程序是通过
解释器执行的。有两种使用解释器的方式:交互模式和脚本模式。在交互模式
下,你可以输入Python程序,然后解释器输出结果:

\index{交互模式}
\index{脚本模式}

\beforeverb
\begin{verbatim}    
>>>1 + 1
2
\end{verbatim} %直接输出以上两句,包括空白和断行
\afterverb

锯齿符,{\tt >>>},是提示符,解释器用它来表明自己已经准备好了,
如果你输入{\tt 1 + 1},解释器显示{\tt 2}。\\

\index{提示符}

另外地,我们可以把代码存储在一个文件里,使用解释器执行文件,此时这个
文件被称作脚本。习惯上,Python脚本的扩展名为{\tt .py}。

\index{脚本}

如果要执行Python脚本,我们必须提供给解释器脚本的文件名。在UNIX命令窗口,可以输入{\tt python dinsdale.py}。在其他开发环境中,会有些细节方面的差别。可以在Python官网上(\url{python.org})找到相应的指导。

\index{测试!交互模式}

在交互模式下工作很容易测试一小段代码,因为可以随时输入,并且立刻执行。但如果代码量较大,我们必须把代码存放在脚本里,这样方便我们以后修改执行。 

\section{什么是程序}

程序就是指令集合,这些指令说明了如何执行计算。计算可能是数学上的,例如解决等式组或者计算多项式的平方根。但是也可以是符号计算,比如搜索替换文件的文本或者(非常奇怪)编译一个程序。

\index{程序}

不同的语言有一些细节上的差异。但是他们有一些共有的指令:

\begin{description}

\item[输入:] 从键盘获取数据,文件,或者从其他设备。

\item[输出:] 在显示器上显示数据或者把数据输出到文件或其他设备。

\item[数学运算:] 做基本的数学操作像加法和乘法。

\item[条件执行:]检查条件,然后执行正确的语句。

\item[循环:]重复执行一些动作,通常有些变化。

\end{description}

信不信由你,就是这样。我们用过的任何一个软件,无论多么复杂,基本上都是由与这些相似的指令组成。所以,我们可以这么理解:编程就是把复杂庞大的任务分解为一系列的小任务,知道这些小任务简单到可以用这些基本的指令表示。

\index{算法}

这个有点模糊,但是当我们讲到算法的时候,我们再回过头来聊这个话题。

\section{什么是调试?}
\index{调试}
\index{臭虫}

有三种错误经常在程序中出现:语法错误,运行时错误和语义错误。为了能够快速的跟踪捕捉到他们,区分他们之间的诧异还是很有好处的。

\subsection{语法错误}
\index{语法错误}
\index{错误!语法}
\index{错误信息}

Python只能执行语法正确的程序;否则,解释器就会报错。语法指的是程序的结构和结构的规则。\index{syntax 语法}
比如,括号必须是成对出现,所以{\tt (1 + 2)}是合法的,但{\tt 8)}就是语法错误。

\index{parentheses!matching  括号!匹配}
\index{syntax 语法}
\index{cummings, e. e. 康明思}

在英语中,读者可以忍受大多数语法错误,这就是为什么我们玩味E. E康明思的诗歌,而没有提出任何错误信息的原因。Python不会这么仁慈。如果在你程序的某个地方出现了哪怕是一个语法错误,Python也会显示错误信息然后退出,你也不能再继续执行程序。在你初学编程的几周里,你很可能会花费大量的时间追踪,捕捉语法错误。一旦你有经验了,你犯的错误就更少,并且也能很快的发现他们。

\subsection{运行时错误}
\label{runtime 运行时}
\index{runtime error 运行时错误}
\index{error!runtime 错误!运行时}
\index{exception 异常}
\index{safe language 安全语言}
\index{language!safe 语言!安全}


第二中错误是运行时错误,之所以这么命名是因为从这种错误知道程序开始运行才会出现。这些错误也叫做异常,因为他们通常表明异常的事情发生了。\\

运行时错误在前几章的简短的代码中比较少见,因此你可能会有一段时间才会遇到。

\section{语义错误}
\index{semantics 语义}
\index{semantics errors 语义错误}
\index{error!semantic 错误!语义}
\index{error message 错误信息}

第三中错误是语义错误。如果有语义错误,程序会成功运行(即计算机不会产生任何的错误信息),但是它却没有做对!计算机做了另外的事。确切的说,计算机确实做了你告诉他的指令。

\subsection{试验性的调试}

你必须拥有的一条技能是调试。尽管在这个过程中,你可能很受伤,但,调试是编程中最具有挑战,最有意思,最能考验智力的一部分。\\

\index{experimental debugging 实验性的调试}
\index{debugging!experimental 调试!实验}

某种程度上,调试就像是侦探。你面对着很多线索,必须推断导致你看到的结果的过程和事件。\\

调试也像是一个科学实验。一旦你意识到错误的地方,改正她,再尝试。如果你的假想是正确的,你就可以预测出改变带来的结果,你也就离能够执行的程序更近一步了。如果你的猜想是错误的,你不得不提出一个新的。正如Sherlock Holmoes指出的,“当你移除了不可能的,留下来的无论是什么,也不论多么不可能,都是真理。(A. Conan Doyle, {\em The Sign of Four})\\

\index{Holems, Sherlock}
\index{Doyle, Arthur Conan}

对某些人来说,编程和调试是同时完成的。也就是,编程是不断调试,直到看到想要结果的过程。理念就是:你必须以一个能够工作的程序开始,然后做些小改动,随着进度不断调试他们,这样就总是有一个可工作的程序。\\

比如:Linux是一个包含成千上万行代码的操作系统,但它也是从一个Linux Torvalds用来研究Intel 80386芯片的小程序开始的。按照Larry Greenfield的说法,“Linus的早期项目就是一个在打印AAAA和BBBB之间切换的程序。“({\em The Linux Users' Guide} Beta Version 1)。

\index{Linux}

接下来的章节将介绍更多的调试建议还有其他的编程经验。

\section{正式语言和自然语言}
\index{formal language 正式语言}
\index{natural language 自然语言}
\index{language!formal 语言!正式}
\index{language!natural 语言!自然}

自然语言是人们日常说的语言,比如英语,西班牙语和法语。他们不是人民设计的(尽管人们努力的强加一些规则);他们是自然发展的。\\

正式语言是人们为了特别的应用而设计的语言。比如,数学家使用的符号就是一门正式语言,它很擅长揭示数字和符号之间的联系。化学家用正式语言代表分子的化学结构。最重要的是:

\begin{quote}
{编程语言是正式语言,是被设计来表达计算的。}
\end{quote}

正式语言倾向于有严谨的语法规则。比如,$3 + 3 = 6$是语法争取的数学语句。但是$3 += 3 \mbox{\$} 6$ 不是。$H_2O$是语法正确的化学分子式,但$_2Zz$ 不是。\\

语法规则涉及到两个方面:标记和结构。标记是语言的最基本元素,比如字,数字和化学元素。$3 += 3 \mbox{\$} 6$的一个问题是$\$$不是一个合法的数学标记(至少据我所知)。相似的,$_2Zz$不合法是因为没有元素的缩写是$Zz$。

\index{token 标记}
\index{structure 结构}

第二种语法错误涉及到语句的结构,也就是,标记被安排的方式。语句$3 + = 3 \mbox{\$} 6$是非法的因为尽管$+$和$=$是合法的标记,但我们不能把两个相连。同样的,在化学分子式中,下标必须在元素之后,不是前面。

\begin{ex}
写一个结构正确的英语句子,同时标记也必须合法。然后写一个结构不合理但是标记合法的句子。
\end{ex}

当阅读一个英文句子或者正式语言的一个语句,必须明确句子的结构(尽管对于自然语言来说,这个是潜意识的)。这个过程叫做句法分析。

\index{parse 句法分析}

比如,当你听到一个句子,“一便士硬币掉了”,你理解“一便士硬币”是主语,“掉了”是谓语。一旦你分析了这个句子,你就明确句子的意思。假如你知道一个便士是什么,并且什么是掉了,你就会明白这个句子的一般含意。\\

尽管正式语言和自然语言有很多共同点---标记,结构,语法和语义---也存在一些不同点:

\index{ambiguity 二义性}
\index{redundancy 冗余性}
\index{literalness 无修饰性}

\begin{description}

\item[二义性:]自然语言充满了二义性(模糊性),人们利用上下文来区分。正式语言被设计成近乎没有二义性,这也意味着每个语句都有明确的意思,无论上下文。

\item[冗余性:]为了弥补二义性和减少误解,自然语言设置了很多冗余。因此自然语言是冗长的。自然语言更简短,精确。

\item[无修饰性:]自然语言充满了习语和隐喻。如果我说“一便士硬币掉了”,也许根本没有便士也没有东西掉了\footnote{这个习语意思是某人困惑之后恍然大悟。}。正式语言表达了是精确的意思。

\end{description}

成长过程中,说自然语言的人---每个人---通常在调整自己适应正式语言的过程中都会经历痛苦。某种程度上,正式语言和自然语言之间的区别就像诗歌和散文\footnote{译者:这里的散文不是诗化的散文,像余光中老前辈开启的诗化散文}之间的区别,甚至更多:

\index{poetry 诗歌}
\index{prose 散文}

\begin{description}

\item[诗歌:]单词的运用既是为了语义的需要,也是为了音韵的需要,整首诗创造了一种情感共鸣。二义性不仅很常见,而且常常是故意安排的。

\item[散文:]单词的字面意思更加重要,结构也表达了更多的意思。散文比诗歌更容易分析,但是仍然具有二义性。

\item[程序:]计算机程序是无二义性。可以通过分析标记和结构完全理解。


\end{description}

这里给些读程序时候的一些建议(包括其他正式语言).第一,记住正式语言是比自然语言要晦涩的,所以要花长时间阅读。其次,结构也是非常重要的,所以,从头到位,从左到右阅读通常不是一个好的办法,可以学习在大脑中分析程序,识别标记的意思,然后解释结构。最后,细节也很重要。一些拼写和标点上细小的错误(在自然语言中可以忽略的),有时会在正式语言中掀起大浪。

\section{第一个程序}
\label{hello}
\index{Hello, World}

通常,学习新语言的第一个程序就是"hello world", 应为所做的就是显示单词,"Hello , World!".在Python中,看起来是:

\beforeverb
\begin{verbatim}
print 'Hello, World!'
\end{verbatim}
\afterverb

这是一个print语句的例子\footnote{在Python3.0中,{\tt print}是一个函数,不是一个语句了,所以语法是{\tt print("Hello, World!")}。我们不久就要接触到函数了!  译注:在本书翻译时python 2.7 和python3.1已经发布,python 3.2的release 版也即将发布},没有真正在纸上打印东西。它在显示器上显示了一个值。在这种情况下,结果是单词

\index{Python 3.0}

\beforeverb
\begin{verbatim}
Hello, World!
\end{verbatim}
\afterverb

程序中的引号标志了要被显示的文本的开始和结束,他们不会出现在结果中。

\index{quotation mark 引号}
\index{print statement print 语句}
\index{statement!print 语句!打印}

一些人通过"Hello, World!"程序的简洁程度来判断编程语言的好坏。按照这个标准,Python确实非常好!


\section{调试}
\index{debugging 调试}

坐在电脑前面看这本书是个不错的方法,你可以随时尝试书中的例子。你可以在
交互模式下运行大多数的程序,但是如果你把代码放在一个脚本里,也是很容易尝试改变一些内容的。\footnote{译者注:我的理解是,可以很方便的
改动某些变量或者语句,然后执行}\\

无论何时,尝试一个新的特点的时候,你应该故意的犯些错误。比如,在"Hello, World!"程序中,如果忽略了双引号其中之一,会发生什么?如果把两个引号都忽略了,又会怎样?如果拼错了{\tt print}了呢?\\

\index{error message 错误信息}

这种实验能够有效的帮助你记住你看的内容,同时也对调试有好处,因为你知道了错误信息的意思了。现在故意的犯错误总比以后猝不及防的犯错误要好的多。\\

编程,特别是调试,有时带来很强的情绪。你在一个困难的bug里苦苦挣扎,你可能变得怒不可遏,苦恼不堪,甚至羞愧不已。\\

有证据表明,人们很容易把电脑当成人来对待\footnote{参看Reeves和Nass,{\it The Media Equation:How People Treat Computers, Television, and New Media Like Real People and Places}.}.当电脑工作正常,我们把它们当作是队友,当电脑不给力时,我们把它们当成粗鲁顽固的人。\\

\index{debugging!emotional response 调试!情绪反应}
\index{emotional debugging 情绪调试}

为这些反应作准备也许会帮助你合理的处理。一个方法是把电脑当作一个员工,他既拥有一定力量,比如速度和精度,也会有特别的缺点,比如缺少默契,没有能力理解大的图片。\\

你的工作就是做一个好的经理:发掘有效的方法扬长补短。并且寻找方法利用你的情绪来投入到解决问题中,不要让你的(不良)反应干扰你工作的能力.\\

学习调试是令人沮丧的,但是一种宝贵的技巧,在编程的其他领域也是大有裨益的。在每章的末尾,都有一个调试段落,像这个一样,是我调试经验的总结。我希望他们对你有益!

\section{术语表}
\begin{description}

\item[problem solving 问题解决:]表述问题,发现解,表达解的过程。

\item[high-level language 高级语言:]像Python一样的程序设计语言,被设计让人们易读易写程序。

\item[low-level language 低级语言:]设计让计算机容易执行的程序设计语言;也叫做“机器语言”或者“汇编语言”。

\item[portability 可移植性:]程序可以在一台或多台电脑执行的属性。

\item[interpret 解释:]逐行逐行解释执行用高级语言编写的程序。

\item[compile 编译:]把用高级语言编写的程序转换成低级语言。
\index{compile 编译}

\item[source code 源码:]未编译的高级语言编写的程序。

\item[object code 目标代码:] 编译器转换程序后的输出。
\index{object code 目标代码}

\item[executable 可执行代码:] 目标代码的别名,可以被执行。
\item{executable 可执行代码}
\item[prompt 提示符:] 解释器显示的字符,表明做好准备让用户输入。
\index{prompt 提示符}

\item[script 脚本:]存储在文件中的程序。
\index{script 脚本}

\item[interactive mode 交互模式:] 一种通过输入命令和表达式的使用python解释器的方式。
\index{interactive mode 交互模式}

\item[script mode 脚本模式:] 一种使用Python解释器的方式,Python解释器读取脚本中的语句执行。
\index{script mode 脚本模式}

\item[program 程序:]指明计算的指令集合。
\index{program 程序}


\item[algorithm 算法] 求解一类问题的通用过程。
\index{algorithm}

\item[bug:] 程序的错误。
\index{bug}

\item[debugging 调试:]发现,去除程序错误的过程。
\index{debugging 调试}

\item[syntax 语法] 程序的结构。
\index{syntax 语法} 

\item[syntax error 语法错误:]使程序不能正确解析的错误。
\index{syntax error}


\item[exception 异常:]程序在运行时发现的错误。
\index{exception 异常}

\item[semantics  语义:]程序的含意。
\index{semantics 语义}

\item[semantics error 语义错误:] 程序中的错误,使计算机执行另外的程序。
\index{semantics error 语义错误}

\item[natural language 自然语言:]人们日常交流用的语言,自然发展的。
\index{natural language 自然语言}

\item[formal language 正式语言:]人民为了某种特殊目的设计的语言,比如,代表数学思想或者计算机程序,所有的程序设计语言都是正式语言。
\index{formal language 正式语言}

\item[token 标记:]程序语法结构的最基本元素,类似于自然语言的单词。
\index{token 标记}

\item[parse 句法分析:]检查程序,分析语法结构。
\index{parse 句法分析}

\item[print statement print 语句:]一条指示Python解释器显示一个值的指令。
\index{print statement print 语句}

\index{statement!print 语句!打印}

\end{description}

\section{练习}

\begin{ex}
打开浏览器浏览Python官网\url{python.org}.这个页面包含了Python的一些信息,还有和Python相关的连接。你可以查看Python官方文档。\\

比如,在搜索框里输入{\tt print},第一个链接就是{\tt print}语句的文档。此时,并不是所有的信息对你都有意义,但是知道它们在哪里总是有好处的。

\index{documentation 文档}
\index{python.org}
\end{ex}

\begin{ex}
启动Python的解释器,输入{\tt help()}启动在线帮助工具。或者你也可以输入\verb"help('print')" 获得关于{\tt print}语句的信息。\\

如果没有成功,你或许需要安装额外的Python官方文档,或者设置环境变量。这个依赖于你使用的操作系统和Python解释器版本。

\index{help utility 帮助工具}
\end{ex}


\begin{ex}
打开Python解释器,我们暂且把它作为计算器。关于数学操作的语法,Python和标准的数学符号很相似。比如,符号{\tt +},{\tt -} 和{\tt /}表示加减,除。乘法的符号是{\tt *}。\\

如果43分钟30秒,跑了10公里,每英里花费的时间是多少?你的平均速度是多少英里每小时?(Hint:一英里等于1.61公里)。

\index{caculator 计算器}
\index{running pace 跑步速度}
\end{ex}


























\chapter{变量、表达式和语句}

\section{变量和数据类型}
\index{变量}
\index{类型数据}
\index{字符串}

{\bf 变量}是程序处理的最基本的事物,就像一个字母或者一个数字。至今为止我们见过的变量有{\tt 1},{\tt 2}以及
\verb"'Hello, World!'"。

这些变量属于不同{\bf 数据类型}:{\tt 2}是一个整数,而\verb"'Hello, World!'"是一个{\bf 字符串},之所以这么称呼是因为它包含一“串”字母。因为被引号包围,读者(以及解释器)可以将它们识别为字符串。

\index{引号标记}

print命令同样适用于整数。

\beforeverb
\begin{verbatim}
>>> print 4
4
\end{verbatim}
\afterverb
%
如果你不确定一个变量的数据类型,解释器会告诉你。

\beforeverb
\begin{verbatim}
>>> type('Hello, World!')
<type 'str'>
>>> type(17)
<type 'int'>
\end{verbatim}
\afterverb
%
显然,字符串属于类型{\tt str},整数属于类型{\tt int}。同样,带小数点的数字属于{\tt float}类型,因为这些数字通过一种叫做{\bf floating-point}的数据类型来表示。

\index{数据类型}
\index{字符串类型}
\index{数据类型!字符串}
\index{整数类型}
\index{数据类型!整数}
\index{浮点数类型}
\index{数据类型!浮点数类型}

\beforeverb
\begin{verbatim}
>>> type(3.2)
<type 'float'>
\end{verbatim}
\afterverb
%
那么对于像\verb"'17'"和\verb"'3.2'"呢?
它们虽然看起来想数字,实际上由于它们在引号中,所以是字符串。

\index{引号标记}

\beforeverb
\begin{verbatim}
>>> type('17')
<type 'str'>
>>> type('3.2')
<type 'str'>
\end{verbatim}
\afterverb
%
它们是数字。

当你输入一个大整数时,你会习惯性地在3个数子中间加上逗号,如{\tt 1,000,000}。对于Python来说这不是一个有效的数字,但是这样的表示是合法的:

\beforeverb
\begin{verbatim}
>>> print 1,000,000
1 0 0
\end{verbatim}
\afterverb
%
当然,这不是我们所期待的。Python将{\tt 1,000,000}解释为一个用逗号划分的数字序列,在输出时用空格区分。

\index{语义错误}
\index{错误!语义错误}
\index{错误信息}

这是我们见到的第一个语义错误的例子:代码可以运行,没有报出错信息,但是没有做“正确”的事情。

\section{变量}
\index{变量}
\index{赋值语句}
\index{语句!赋值语句}

对{\bf 变量}的操作是编程语言最强大的特征之一。一个变量是一个对应值的名称。

通过{\bf 赋值语句}创建新的变量,并对它们赋值:

\beforeverb
\begin{verbatim}
>>> message = 'And now for something completely different'
>>> n = 17
>>> pi = 3.1415926535897931
\end{verbatim}
\afterverb
%
这个例子包含三条赋值语句。第一条将一个字符串赋值给一个叫做{\tt message}的变量;第二条将整数{\tt 17}赋值给变量{\tt n};第三条将$\pi$的近似值赋值给变量{\tt pi}。

\index{状态图表}
\index{图表!状态}

变量通常用表示为变量名用一个肩头指向变量的值。这样的图称作{\bf 状态图表},因为它给出了每个变量所处的状态(想象各个变量处在各自的状态)。以下的图表给出了之前例子中的结果:

\beforefig
\centerline{\includegraphics{figs/state2.eps}}
\afterfig

你可以使用一个print语句来显示变量的值:

\beforeverb
\begin{verbatim}
>>> print n
17
>>> print pi
3.14159265359
\end{verbatim}
\afterverb
%
一个变量的数据类型是这个变量对应值的数据类型。

\beforeverb
\begin{verbatim}
>>> type(message)
<type 'str'>
>>> type(n)
<type 'int'>
>>> type(pi)
<type 'float'>
\end{verbatim}
\afterverb
%
\begin{ex}
如果你在输入一个整数时以$0$开头,你会得到一个奇怪的错误:

\beforeverb
\begin{verbatim}
>>> zipcode = 02492
                  ^
SyntaxError: invalid token
\end{verbatim}
\afterverb

其他的数字也许可以工作,但是结果是诡异的:

\beforeverb
\begin{verbatim}
>>> zipcode = 02132
>>> print zipcode
1114
\end{verbatim}
\afterverb

你能指出这是为什么吗?提示:打印数值{\tt 01},{\tt 010},{\tt 0100}和{\tt 01000}。

\index{八进制}

\end{ex}

\section{变量命名和关键字}
\index{关键字}

程序员通常选取有意义的名字作为变量明——变量名说明了变量的作用。

变量名可以任意长。它们可以同时包含字母和数字,但必须以一个字母开头。大写字母的使用是合法的,但是一个好的习惯是使用小写字母作为变量的首字母(后面会说明原因)。

下划线符号(\verb"_")可以出现在一个变量名中,通常使用在有多个单词的名字中,如\verb"my_name"或者\verb"airspeed_of_unladen_swallow"。

\index{下划线字符}

如果你赋予了一个变量非法的名字,你会得到这样的一个语法错误:

\beforeverb
\begin{verbatim}
>>> 76trombones = 'big parade'
SyntaxError: invalid syntax
>>> more@ = 1000000
SyntaxError: invalid syntax
>>> class = 'Advanced Theoretical Zymurgy'
SyntaxError: invalid syntax
\end{verbatim}
\afterverb
%
{\tt 76trombones}是非法的,因为它不是有字母开始。{\tt more@}是非法的,因为它包含了非法字符{\tt @}。但是{\tt class}哪里错了呢?

事实上{\tt class}属于Python的一个{\bf 关键字}。解释器使用这些关键字来识别程序的结构,因此它们不能被用做变量名。

\index{关键字}

Python共有31个关键字\footnote{在Python 3.0中,{\tt exec}不再是一个关键字,但{\tt nonlocal}是。}:

\beforeverb
\begin{verbatim}
and       del       from      not       while    
as        elif      global    or        with     
assert    else      if        pass      yield    
break     except    import    print              
class     exec      in        raise              
continue  finally   is        return             
def       for       lambda    try
\end{verbatim}
\afterverb
%
你可以放一份关键字列表在手边。如果解释器报错说变量名有误,而你有不知道原因,看看这个变量名是否在关键字列表上。

\section{程序语句}

程序语句是一组Python解释器可以执行的代码。我们已经遇到过两种程序语句:print和assignment。

\index{程序语句}
\index{交互模式}
\index{脚本模式}

当你在交互模式下输入一条程序语句,解释器会执行该条语句,并显示结果,如果有结果的话。

一个脚本通常包含一系列程序语句。如果有多于一条程序语句,每执行一条语句将会显示对应的结果。

例如,脚本

\beforeverb
\begin{verbatim}
print 1
x = 2
print x
\end{verbatim}
\afterverb
%
会产生输出

\beforeverb
\begin{verbatim}
1
2
\end{verbatim}
\afterverb
%
赋值语句不产生输出。

\section{运算符和运算数}
\index{运算符,数学}
\index{数学运算符}
\index{运算数}
\index{表达式}

{\bf 运算符}是表示计算的特殊符号,如加法和乘法。运算符作用的数据被称为{\bf 运算数}。

运算符{\tt +}、{\tt -}、{\tt *}、{\tt /}和{\tt **}实现加法、减法、乘法、除法和指数运算,如下面的示例:

\beforeverb
\begin{verbatim}
20+32   hour-1   hour*60+minute   minute/60   5**2   (5+9)*(15-7)
\end{verbatim}
\afterverb
%
在一些其他的编程语言中,\verb"^"被用做指数运算,但是在Python中这是一个位操作,称为XOR。在本书中将不会涉及位操作的知识,你可以在\url{wiki.python.org/moin/BitwiseOperators}中阅读相关知识。

\index{位操作}
\index{运算符!按位}

%When a variable name appears in the place of an operand, it
%is replaced with its value before the operation is
%performed.

除法运算符有可能不按照你期望的运行:

\beforeverb
\begin{verbatim}
>>> minute = 59
>>> minute/60
0
\end{verbatim}
\afterverb
%
{\tt minute}的值是59,在传统的算术中,59除以60的结果是0.98333,而不是0。产生偏差的原因是因为Python执行了{\bf 下取整除法}\footnote{在Python 3.0中,这个除法的结果是一个{\tt 浮点数}。新的运算符{\tt //}执行取整除法。}。

\index{Python 3.0}
\index{下取整除法}
\index{浮点除法}
\index{除法!下取整}
\index{除法!浮点}

当两个运算数都是整数,结果同样是整数;下取整除法舍去了小数部分,因此在这个例子中结果为0。

如果任意一个运算数是一个浮点数,Python执行浮点除法,此时结果为{\tt 浮点数}:

\beforeverb
\begin{verbatim}
>>> minute/60.0
0.98333333333333328
\end{verbatim}
\afterverb


\section{表达式}

{\bf 表达式}是数值、变量和运算符的组合。一个数值是一个表达式,一个变量也是一个表达式,因此一下都是合法的表达式(假设变量{\tt x}已经被赋值):

\index{表达式}
\index{赋值}

\beforeverb
\begin{verbatim}
17
x
x + 17
\end{verbatim}
\afterverb
%
如果你在交互模式下输入一个表达式,解释器立即对它{\bf 赋值}并显示结果:

\beforeverb
\begin{verbatim}
>>> 1 + 1
2
\end{verbatim}
\afterverb
%
但在脚本中,一个表达式本身不做任何事情!初学者通常对此会产生困惑。

\begin{ex}
在交互模式下输入一下表达式,观察结果:

\beforeverb
\begin{verbatim}
5
x = 5
x + 1
\end{verbatim}
\afterverb
%
现在将这些表达式放在一个脚本文件里并运行。此时输出是什么?修改脚本,在每个表达式前添加print语句,然后观察输出。
\end{ex}


\section{运算符的优先级}
\index{运算符的优先级}
\index{优先级规则}
\index{PEMDAS}

当一个表达式中出现多于一个运算符时,求值的顺序由{\bf 优先级规则}决定。Python遵从数学运算符的约定。{\bf PEMDAS}是一个有效的方法来记忆这些规则:

\index{括号!压倒优先级}

\begin{itemize}

\item {\bf P}arentheses(括号)具有最高的优先级,可以强迫表达式按照指定的顺序求值。由于在括号中的表达式先被求值,{\tt 2 * (3-1)}的结果是4,{\tt (1+1)**(5-2)}的结果是8。你可以用括号来使得表达式更容易阅读,如{\tt (minute * 100) / 60},即便这并不改变最终的结果。

\item {\bf E}xponentiation(指数运算)有次高的优先级,因此{\tt 2**1+1}的结果是3,而不是4,{\tt 3*1**3}的结果是3,而不是27。

\item {\bf M}ultiplication(乘法)和{\bf D}ivision(除法)具有相同的优先级,并优于{\bf A}ddition(加法)和{\bf S}ubtraction(减法),它们同样具有相同的优先级。因此{\tt 6+4/2}的结果是8,而不是5。

\item 具有相同优先级的运算符是从左到右进行求值的。因此在表达式{\tt degrees / 2 * pi}中,首先计算除法,然后在结果上乘以{\tt pi}。如果要除以$2 \pi$,你需要使用括号,或者使用{\tt degrees / 2 / pi}。

\end{itemize}


\section{字符串操作}
\index{字符串!操作}
\index{运算符!字符串}

一般情况下,你不能使用算术运算符作用于字符串,即使字符串看起来像数字,因此以下是非法的:

\beforeverb
\begin{verbatim}
'2'-'1'    'eggs'/'easy'    'third'*'a charm'
\end{verbatim}
\afterverb
%
运算符{\tt +}可以作用于字符串,但未必按照你期望的方式工作:它执行的是{\bf 级联},即将字符串首尾相连。例如:

\index{级联}

\beforeverb
\begin{verbatim}
first = 'throat'
second = 'warbler'
print first + second
\end{verbatim}
\afterverb
%
程序的输出是{\tt throatwarbler}。

运算符{\tt *}同样可以作用于字符串,它的功能是重复字符串。如\verb"'Spam'*3"的结果是\verb"'SpamSpamSpam'"。如果一个运算符是字符串,那么另一个必须是一个整数。

运算符{\tt +}和{\tt *}在字符串的的使用类似加法和乘法,如{\tt 4*3}相当于{\tt 4+4+4},因此我们期望\verb"'Spam'*3"相当于\verb"'Spam'+'Spam'+'Spam'",事实如此。当然,字符串的级联和重复相比整数的加法和乘法还是有很大的区别的。你可以找出一个加法有而字符串级联没有的性质吗?


\index{交换性}


\section{注释}
\index{注释}

当程序变得越来越长,越来越复杂,它们变得更难读懂。正是的变成语言是密集的,通常很难通过阅读一小段代码来指出它是做什么的,或者为这么这么写。

由于这个理由,在程序旁添加标记,使用自然语言来说明这段程序是做什么的是一个良好的想法。这些标记被称为{\bf 注释},它们由\verb"#"符号开头:

\beforeverb
\begin{verbatim}
# compute the percentage of the hour that has elapsed
percentage = (minute * 100) / 60
\end{verbatim}
\afterverb
%
在这个例子中,注释以整行的方式出现。你也可以在一行的末尾添加注释:

\beforeverb
\begin{verbatim}
percentage = (minute * 100) / 60     # percentage of an hour
\end{verbatim}
\afterverb
%
{\tt \#}符号后到一行结束中的任何内容都被忽略---它们将不影响程序的运行。

当说明代码中不明显的特点时,注释非常有效。我们可以假设阅读代码的人可以指出代码{\em 做了什么};更有必要的是解释{\em 为什么这么做}。

以下的注释是冗余而无用的:

\beforeverb
\begin{verbatim}
v = 5     # assign 5 to v
\end{verbatim}
\afterverb
%
以下的注释包含了代码中没有的有用信息:

\beforeverb
\begin{verbatim}
v = 5     # velocity in meters/second. 
\end{verbatim}
\afterverb
%
好的变量明可以减少对注释的依赖,但长的变量名会使得表达式难以阅读,因此这是一个折衷。

\section{调试}
\index{调试}

目前位置你遇到的语法错误大多为非法变量明,如关键字{\tt class}和{\tt yield},或者包含非法字符,如\verb"odd~job"和\verb"US$"。

\index{语法错误}
\index{错误!语法}

如果你在一个变量名中加入一个空格,Python会认为这是两个没有运算符的运算数:

\beforeverb
\begin{verbatim}
>>> bad name = 5
SyntaxError: invalid syntax
\end{verbatim}
\afterverb
%
对于语法错误,错误消息没有很大的帮助。通常的消息是{\tt SyntaxError: invalid syntax}和{\tt SyntaxError: invalid token},它们都不包含很多信息量。

\index{错误消息}
\index{未定义而使用}
\index{异常}
\index{运行时错误}
\index{错误!运行时}

运行时错误通常由于“未定义而使用”,即试图使用一个没有赋值的变量。这会在你拼写变量名错误时发生:

\beforeverb
\begin{verbatim}
>>> principal = 327.68
>>> interest = principle * rate
NameError: name 'principle' is not defined
\end{verbatim}
\afterverb
%
变量名是大小写敏感的,因此{\tt LaTeX}和{\tt latex}是不同的。

\index{大小写敏感,变量名}
\index{语义错误}
\index{错误!语义}

语义错误通常发生在运算符的顺序上。例如,求值$\frac{1}{2 \pi}$,你也许会写成:

\beforeverb
\begin{verbatim}
>>> 1.0 / 2.0 * pi
\end{verbatim}
\afterverb
%
但是首先执行了除法,因此你会得到$\pi / 2$,并不是想要的结果!Python没有办法知道你写的是什么含义,因此这种情况下你不会得到一个错误消息;你得到的只是错误的结果。

\index{运算符优先级}

\section{术语表}

\begin{description}

\item[数值:] 程序操作的基本数据单元,如一个数字或字符串。
\index{数值}

\item[类型:] 数据的分类。目前为止我们见过的类型有整数({\tt int}),浮点数({\tt float}),和字符串({\tt str})。
\index{类型}

\item[整数:] 表示所有整数的类型。
\index{整数}

\item[浮点数:] 表示具有小数部分的数的类型。
\index{浮点数}

\item[字符串:] 表示一串字符的数据类型。
\index{字符串}

\item[变量:] 指向一个数值的名称。
\index{变量}

\item[语句:] 一段代表命令或行为的代码。目前为止我们见过的语句有赋值语句和打印语句。
\index{语句}

\item[赋值:] 将数值赋值给变量的语句。
\index{赋值}

\item[状态图:] 表示变量和对应数值的图。
\index{状态图}

\item[关键字:] 一些编译器用来对程序进行语法分析的保留单词,你不能使用例如{\tt if},{\tt  def}和{\tt while}作为变量名。
\index{关键字}

\item[运算符:] 表示一种简单运算的特殊符号,如加法,乘法,或者字符串级联。
\index{运算符}

\item[运算数:] 运算符作用的数值。
\index{运算数}

\item[下取整除法:] 两数相除并舍去小数部分的运算。
\index{下取整除法}

\item[表达式:] 表示单一结果的变量、运算符和数值的组合。
\index{表达式}

\item[求值] 通过执行运算符化简表达式得到一个单一结果。

\item[优先级规则] 定义运算符和运算数求值顺规的规则。
\index{优先级规则}
\index{优先级}

\item[级联:] 将两个运算数首尾现连。
\index{级联}

\item[注释:] 程序中提供给程序员(或者任何阅读程序的人)信息,但不影响程序执行的内容。
\index{注释}

\end{description}


\section{练习}

\begin{ex}
假设我们执行了如下的赋值语句:

\begin{verbatim}
width = 17
height = 12.0
delimiter = '.'
\end{verbatim}

对于下列表达式,写出表达式的数值和数据类型(表达式的数值)。

\begin{enumerate}

\item {\tt width/2}

\item {\tt width/2.0}

\item {\tt height/3}

\item {\tt 1 + 2 * 5}

\item {\tt delimiter * 5}

\end{enumerate}

使用Python解释器来验证你的结果。
\end{ex}

\begin{ex}
练习使用Python解释器作为计算器。
\index{calculator}

\begin{enumerate}

\item 一个半径为$r$的球的体积为$\frac{4}{3} \pi r^3$。一个半径为5的球体的体积是多少?提示:392.6是错误的!

\item 假设一本书的封面价格是\$24.95,书店提供40\%的折扣,对第一本书的运输费用为\$3,往后每增加一本需要75每分,如果购买60本书总共需要多少?

\item 如果我在6:52离开我家,用轻松的步伐跑了1英里(8:15每英里),然后保持节奏跑了3英里(7:12每英里),最后用轻松的步伐跑了1英里,我什么时候回到家吃早饭?

\index{跑步比赛}

\end{enumerate}
\end{ex}


\chapter{函数}
\label{funcchap}
\index{function call 函数调用}

在程序设计中,函数是带有函数名的一系列执行计算的语句,当定义一个函数,我们指
定一个函数名和一系列的语句。然后,就可以通过函数名调用函数。我们其实已经看到
一个函数调用的例子。

\beforeverb
\begin{verbatim}
>>>type(32)
<type 'int'>
\end{verbatim}
\afterverb

函数名是类型,小括号里的表达式称作函数的形式参数(argument).结果是形参的类型。

\index{parenthess!argument in}

通常我们这么说:函数接受一个参数,返回一个结果(叫做返回值)。

\index{argument 形式参数}
\index{return value 返回值}

\section{类型转换函数}
\index{conversion!type 转换!类型}
\index{type conversion 类型转换}

Python提供一些内置函数用来把一种类型的值转换成另一类型。{\\t int}函数接受一
个值,如果可以,就把它转换成整数,否则就会“抱怨”。

\index{int function int 函数}
\index{function!int 函数!int}

\beforeverb
\begin{verbatim}
>>> int('32')
32
>>> int('Hello')
Traceback (most recent call last):
  File "<stdin>", line 1, in <module>
ValueError: invalid literal for int() with base 10: 'Hello'
\end{verbatim}
\afterverb

{\tt int}函数可以把浮点数转换为整数,但是不能向上取整,只能截掉小数部分:

\beforeverb
\begin{verbatim}
>>> int(3.99999)
3
>>> int(-2.3)
-2
\end{verbatim}
\afterverb

{\tt float}函数把整数和字符串转换成浮点数:

\index{float function float函数}
\index{function!float 函数!float}

\beforeverb
\begin{verbatim}
>>>float(32)
32.0
>>>float('3.14159')
3.14159
\end{verbatim}
\afterverb \footnote{在译者的机器上float('3.14159')的输出为:3.1415899999999999(解释器Python2.5和2.6);3.14159(解释器Python3.1)。}

最后,{\tt str}把参数转换为字符串:

\index{str function str函数}
\index{function!str 函数!str}

\beforeverb
\begin{verbatim}
>>>str(32)
'32'
>>>str(3.14159)
'3.14159'
\end{verbatim}
\afterverb

2236




\chapter{实例学习:接口设计}
\label{turtlechap}

\section{TurtleWorld}
\index{TurtleWorld}
\index{Swampy}

为了完成这本书,我写了一个叫做Swamp的模块。其中的一个就是TurtleWorld,它提供了一系列在屏幕上绘制移动的乌龟的函数。

你可以从\url{thinkpython.com/swampy}下载Swamp,然后根据说明将Swampy安装到你的系统上。

打开含有{\tt TurtleWorld.py}的文件夹,创建一个{\tt polygon.py}的文件,然后输入以下代码:

\beforeverb
\begin{verbatim}
from TurtleWorld import *

world = TurtleWorld()
bob = Turtle()
print bob

wait_for_user()
\end{verbatim}
\afterverb
%
第一行是我们先前见过的{\tt import}语句的变种;它直接从模块中导入函数,而不是创建一个模型模块对象,这样你可以直接访问函数,而不需要点符号。

\index{导入指令}
\index{语句!导入}

接下来的几行创建了一个TurtleWorld并赋值给{\tt world},创建了一个Turtle并赋值给{\tt bob},然后打印{\tt bob}:

\beforeverb
\begin{verbatim}
<TurtleWorld.Turtle instance at 0xb7bfbf4c>
\end{verbatim}
\afterverb
%
输出表明{\tt bob}指向一个在模块{\tt TurtleWorld}中定义的Turtle的{\tt 实例}。在这里,“实例”指一个集合中的一个成员;这个Trutle是集合中可能存在的Turtles中的一个。

\index{实例}

\verb"wait_for_user" 告诉TurtleWorld等待用户进行某些操作,在这个例子中,用户除了关闭窗口没有其他的操作。

TurtleWorld提供了多个移动乌龟的函数:{\tt fd}和{\tt bk} 用来前进和后退, {\tt lt} 和 {\tt rt} 用来左转和右转。此外,每只乌龟携带了一支笔,笔或者提起或者放下;如果笔是放下的,乌龟将在它经过的地方留下轨迹。函数 {\tt pu} 和 {\tt pd}代表“提起笔”和“放下笔”。

要画一个直角,下程序中添加以下代码(在创建{\tt bob}之后,调用 \verb"wait_for_user"之前):

\beforeverb
\begin{verbatim}
fd(bob, 100)
lt(bob)
fd(bob, 100)
\end{verbatim}
\afterverb
%
第一行告诉 {\tt bob} 向前移动100步。第二行告诉它左转。

当你运行这个程序是,你可以看见 {\tt bob} 先向东移动,然后向南移动,并留下两条轨迹。

现在修改程序,绘制一个正方形。在未完成之前请不要继续下去!

%\newpage

\section{简单重复}
\label{重复}
\index{重复}
你肯能会写出类似这样的代码(忽略创建TurtleWorld和等待用户的):

\begin{verbatim}
fd(bob, 100)
lt(bob)

fd(bob, 100)
lt(bob)

fd(bob, 100)
lt(bob)

fd(bob, 100)
\end{verbatim}
%
我们可以使用{\tt for}语句使得程序更简洁。在 {\tt polygon.py} 中添加如下代码然后运行:

\index{for 循环}
\index{循环!for}
\index{语句!for}

\beforeverb
\begin{verbatim}
for i in range(4):
    print 'Hello!'
\end{verbatim}
\afterverb
%
你会看到这样的结果:

\beforeverb
\begin{verbatim}
Hello!
Hello!
Hello!
Hello!
\end{verbatim}
\afterverb
%
这是{\tt for}语句最简单的用法;以后我们会看到更多的例子。这些已经足够帮助你重写画正方形的程序了。请在完成后继续阅读。

%\newpage

这里给出使用 {\tt for} 语句画正方形的代码:

\beforeverb
\begin{verbatim}
for i in range(4):
    fd(bob, 100)
    lt(bob)
\end{verbatim}
\afterverb
%
{\tt for}语句的语法类似一个函数的定义。它有一个以冒号结尾的首部以及缩进过的主体部分。主题部分可以包含任意多的语句。

\index{循环}

{\tt for} 常常被称作 {\bf 循环},因为程序执行时穿过主体部分,然后循环回到顶部。在这个例子中,主体部分循环了4次。

这个版本的程序事实上和先前的画正方形的程序有所不同,因为在画最后一条边后多了一次左转。额外的一次旋转仅仅花费很少的时间,但是代码得以简化,如果我们每次都重复循环中的操作。这个版本还让乌龟回到起始位置的同时朝向初始的方向。

\section{练习}
以下是一系列使用TurtleWorld的练习。它们被设计的更有趣,同时含有关键点。当你在练习时,思考关键点是什么。

接下来的章节有练习的解答,所以在你完成前(至少尝试前)不要去翻阅它们。

\begin{enumerate}

\item 写一个叫做 {\tt square} 的函数,读取一个叫做 {\tt t} 的参数,参数是一个乌龟。函数使用这个乌龟画一个正方形。

写一个函数调用将 {\tt bob} 作为参数传给{\tt square},然后运行程序。

\item 在{\tt square}中新增一个叫做 {\tt length} 的参数。修改程序主体使得边长等于 {\tt length},然后修改函数调用,增加第二个参数。运行程序。使用一系列的 {\tt length}测试你的程序。

\item 函数 {\tt lt} 和 {\tt rt} 默认转90度,但是你可以提供第二个参数来指定旋转的角度。例如,{\tt lt(bob, 45)} 令 {\tt bob} 向左转45度。

复制 {\tt square} 并命名为 {\tt polygon}。添加一个新的参数 {\tt n} ,并修改函数使得函数绘制一个n边正多边形。提示:一个n边正多边形的外角为 $360.0 / n$ 度。

\index{多边形函数}
\index{函数!多边形}

\item 写一个 {\tt circle} 函数,输入为一个乌龟 {\tt t} 和半径 {\tt r}作为参数,使用的边长和边数调用{\tt polygon}函数画一个多边形来近似圆。使用一系列 {\tt r}来测试你的程序。

\index{圆函数}
\index{函数!圆}

提示:计算出圆的周长,使得{\tt length * n = circumference}。

额外提示:如果你觉得 {\tt bob} 移动得太慢,你可以设置 {\tt bob.delay},它决定了两次移动的时间间隔。{\tt bob.delay = 0.01} 应该会让它看起来在运动。

\item {\tt circle}的一个更一般化的的版本是 {\tt arc},它接收一个额外的参数 {\tt angle},指定了绘制圆周的多少部分。 {\tt angle} 使用角度作为单位,因此当 {\tt angle=360}时, {\tt arc} 将会绘制一个完整的圆周。

\index{弧函数}
\index{函数!弧}

\end{enumerate}

\section{封装}

第一个练习要求你将画正方形的代码装入一个函数定义,然后传递一个乌龟作为参数,调用该函数。如下所示:

\beforeverb
\begin{verbatim}
def square(t):
    for i in range(4):
        fd(t, 100)
        lt(t)

square(bob)
\end{verbatim}
\afterverb
%
最深处的代码{\tt fd}和{\tt lt}被缩进两次,因此它们处在函数定义中的{\tt for}循环中。在接下来的一行中,{\tt square{bob)}左对齐到初始位置,代表着{\tt for}}循环和函数定义的结束。

在函数中,{\tt t} 和 {\tt bob}指向同一个顾伟,因此 {\tt lt(t)} 和 {\tt lt(bob)} 具有相同的效果。那么为什么不直接使用参数 {\tt bob}呢?主要是考虑到 {\tt t} 可以被用来代表任意的乌龟,不仅仅是 {\tt bob},你可以创建一个新的乌龟,然后把它作为参数传给 {\tt square}:

\beforeverb
\begin{verbatim}
ray = Turtle()
square(ray)
\end{verbatim}
\afterverb
%
将一段代买打包在一个函数的过程叫做 {\bf 封装}。封装的一个好处是给代码附上一个名称,起着文档的作用。另一个好处是你可以重用代码,调用一个函数比复制黏贴代码更明智!

\index{封装}


\section{泛化}

下一步是在{\tt square}函数中添加 {\tt length} 参数,如下所示:

\beforeverb
\begin{verbatim}
def square(t, length):
    for i in range(4):
        fd(t, length)
        lt(t)

square(bob, 100)
\end{verbatim}
\afterverb
%
给一个函数添加一个变量称为 {\bf 泛化},因为它使得函数使用更广泛的情况:在先前的版本中,画的正方形总是相同的大小,在这个版本中可以是任意大小。

\index{泛化}

下一步仍是泛化。{\tt polygon}可以画一个任意边数的正多边形,不仅仅是正方形。如下所示:

\beforeverb
\begin{verbatim}
def polygon(t, n, length):
    angle = 360.0 / n
    for i in range(n):
        fd(t, length)
        lt(t, angle)

polygon(bob, 7, 70)
\end{verbatim}
\afterverb
%
这段代码花了一个边长为70的正七变形。如果函数有较多的参数,人们容易忘记它们分别是什么,或者应该以什么顺序给出。因此将参数名包含在参数列表中是合法的,有事也很有用:

\beforeverb
\begin{verbatim}
polygon(bob, n=7, length=70)
\end{verbatim}
\afterverb
%
这些包括变量名作为“参数”,被称作{\bf 关键字参数}(不要和Python关键字,如 {\tt while} 或 {\tt def}混淆)。

\index{关键字参数}
\index{参数!关键字}

这样的语法使得程序的可读性增强。同时提醒数值和参数是如何工作的:当你调用一个函数,数值将被赋值给参数。


\section{接口设计}

下一步是写函数 {\tt circle},它读取半径{\tt r}作为参数。下面给出一个示例,通过调用{\tt polygon} 画正50边形:

\beforeverb
\begin{verbatim}
def circle(t, r):
    circumference = 2 * math.pi * r
    n = 50
    length = circumference / n
    polygon(t, n, length)
\end{verbatim}
\afterverb
%
第一行使用公式$2 \pi r$计算半径为{\tt r}的圆的周长。 由于使用了{\tt math.pi},我们需要包含 {\tt math}。 通常{\tt import}语句放在脚本的开头。


{\tt n}是用来近似圆的多边形的边数,{\tt length} 是每段边长的长度。因此{\tt polygon}画一个边长为{\tt r}的50边形来近似一个圆。

这个方法有个限制,由于 {\tt n} 是一个常数,当圆非常大的时候,每条边长将变得很长。而对于小的圆,画这些小的线段又浪费了太多的时间。一个解决方法是将函数泛化,增加一个参数{\tt n}。这样给用户(调用{\tt circle}的人)更多的控制,但是函数接口就显得复杂了。

\index{接口}

一个函数的 {\bf 接口} 是函数如何使用的概述:参数是什么?函数做了什么?返回值是什么?如果一个接口“尽可能的简洁,但不简单(Einstein)”,我们称这个接口是“干净的”。

\index{Einstein, Albert}

在这个例子中,{\tt r}属于接口,因为它决定了所画圆的大小。 {\tt n}相对而言不适合作为接口,因此它属于函数{\em 如何}画圆的细节。

与其将接口弄乱,根据{\tt 周长}选择一个合适的{\tt n}是一个更好的方法。

\beforeverb
\begin{verbatim}
def circle(t, r):
    circumference = 2 * math.pi * r
    n = int(circumference / 3) + 1
    length = circumference / n
    polygon(t, n, length)
\end{verbatim}
\afterverb
%
现在多边形的边数大约是 {\tt 周长/3},因此每条边的长度大约为3,即使圆足够好看,又保证绘画的效率,因此适合所有尺寸的圆。

\section{重构}
\label{重构}
\index{重构}

当我写 {\tt circle}时,我可以重用 {\tt polygon}的代码,因为多边形对应和许多代码和圆的代码很相似。但是{\tt arc}并没有那么相似,我们不能使用 {\tt polygon}或者{\tt circle} 来画一段圆弧。

一个解决方法是复制一个 {\tt polygon} 拷贝,并将其改为 {\tt arc},结果看起来应该如下:

\beforeverb
\begin{verbatim}
def arc(t, r, angle):
    arc_length = 2 * math.pi * r * angle / 360
    n = int(arc_length / 3) + 1
    step_length = arc_length / n
    step_angle = float(angle) / n
    
    for i in range(n):
        fd(t, step_length)
        lt(t, step_angle)
\end{verbatim}
\afterverb
%
函数代码的第二部分和 {\tt polygon}非常相似。但是如果不修改接口,我们不能重用 {\tt polygon} 的代码。我们可以泛化 {\tt polygon},让它读入一个角度作为第三个参数,但此时已经不适合用 {\tt polygon}作为函数名了!因此我们称这个泛化的函数为 {\tt polyline}:

\beforeverb
\begin{verbatim}
def polyline(t, n, length, angle):
    for i in range(n):
        fd(t, length)
        lt(t, angle)
\end{verbatim}
\afterverb
%
现在我们使用{\tt polyline}重写{\tt polygon}和{\tt arc}:

\beforeverb
\begin{verbatim}
def polygon(t, n, length):
    angle = 360.0 / n
    polyline(t, n, length, angle)

def arc(t, r, angle):
    arc_length = 2 * math.pi * r * angle / 360
    n = int(arc_length / 3) + 1
    step_length = arc_length / n
    step_angle = float(angle) / n
    polyline(t, n, step_length, step_angle)
\end{verbatim}
\afterverb
%
最后,我们使用{\tt arc}重写{\tt circle}:

\beforeverb
\begin{verbatim}
def circle(t, r):
    arc(t, r, 360)
\end{verbatim}
\afterverb
%
这个重新安排程序以优化程序接口改善程序复用的过程被称为 {\bf 重构}。在本例中,我们注意到{\tt arc}和{\tt polygon}代码中有许多类似的部分,于是我们将这部分“提取”并作为 {\tt polyline}。

\index{重构}

如果我们在一开始的时候设计过,我们会首先写 {\tt polyline}来避免重构,但通常在工程开始的时候你不肯能了解所有的细节并设计所有的接口。当你开始编程后,你对工程会有更深入的了解。有时候重构是你学到某些东西的标志。


\section{一个开发者的计划}
\index{开发计划!封装和泛化}

{\bf 开发计划}是写程序的一个过程。在本例中我们使用的过程是“封装和泛化”。这个过程包括以下步骤:

\begin{enumerate}

\item 开始阶段写一个小的没有函数定义的程序。

\item 当程序可以工作后,将其封装成函数,并赋予函数名。

\item 通过适当添加函数参数泛化函数。

\item 重复步骤1——3,直到你有了一系列可以工作的函数。复制黏贴代码以避免重复输入(和重复调试)。

\item 寻找机会通过重构改善代码。例如,你在不同地方使用了相似的代码,可以考虑将它们重构到一个更一般化的函数中。

\end{enumerate}

这个过程包含一些缺点,我们会在以后介绍改进方法,当当你在一开始不知道如何将程序划分为函数时这是一个有效的方法,让你可以继续你的设计。


\section{文档字符串}
\label{文档字符串}
\index{文档字符串}

{\bf 文档字符串}是在函数头部解释接口的字符串(“doc”是“document”的缩写),以下是一个例子:

\beforeverb
\begin{verbatim}
def polyline(t, length, n, angle):
    """Draw n line segments with the given length and
    angle (in degrees) between them.  t is a turtle.
    """    
    for i in range(n):
        fd(t, length)
        lt(t, angle)
\end{verbatim}
\afterverb
%
文档字符串是一个用引号三次包含的多行字符串,三重引号允许字符串跨越多行。

\index{引号标记}
\index{三重引用字符串}
\index{字符串!三重引用}
\index{多行字符串}
\index{字符串!多行}

文档字符串虽然简洁,但对于需要使用这个函数的人来说包含了必要的信息。它准确的解释了函数做了什么(而不涉及是怎么做的)。它解释了每个参数对函数的影响以及每个参数的数据类型(如果不是很明显)。

这类文档的撰写时接口设计中一个重要的部分。一个良好设计的接口可以用简洁的语言来解释。如果你在解释某个函数时遇到了困难,也许就是接口有待改进的标记。


\section{调试}
\index{调试}
\index{接口}

接口好比函数和调用者间的一个协议。调用者同意提供一定的参数,函数同意完成一定的工作。

例如, {\tt polyline} 需要四个参数。第一个参数是一个乌龟。第二个参数是一个数,理论上这应该是一个正数,事实上即使不是函数也能正常工作。第三个参数是一个整数,否则{\tt range}会报错(报错信息取决于你运行的Python版本)。第四个参数是一个数,决定绘图的角度。

这写要求被称作{\bf 先决条件},在程序执行前首先需要满足这些条件。相对的,在函数尾部的条件被称为{\bf 后决条件}。后决条件包括函数需要的功能(如画线段)以及附带的效果(如移动乌龟和对环境的其他操作)。

\index{先决条件}
\index{后决条件}

调用者需要对先决条件负责。如果调用者违反了(合理设计的)先决条件导致函数工作异常,这个错误是调用者的,而不是函数的。

% Removing this because we haven't seen conditionals yet!
%However, for purposes of debugging it is often a good idea for
%functions to check their preconditions rather than assume they are
%true.  If every function checks its preconditions before starting,
%then if something goes wrong, you will know which function to blame.


\section{术语表}

\begin{description}

\item[实例:] 一个集合中的一员。本章中TurtleWorld是TurtleWorlds集合中的一员。
\index{实例}

\item[循环:] 程序中可以重复执行的部分。
\index{循环}

\item[封装:] 将一系列语句包装成一个函数的过程。
\index{封装}

\item[泛化:] 将某些无需确定的(如一个数字)东西用一些通用的东西(如变量或参数)代替的过程。
\index{ 泛化}

\item[关键字参数:] 将“关键字”作为名字的参数。
\index{关键字参数}
\index{参数!关键字}

\item[接口:] 如何使用一个函数的描述,包括参数名、参数描述和返回值。
\index{接口}

\item[开发计划:] 一个编写程序的过程。
\index{开发计划}

\item[文档字符串:] 在函数定义处的字符串,用来给出函数接口的文档。
\index{文档字符串}

\item[先决条件:] 调用者在调用函数前需要满足的条件。
\index{先决条件}

\item[后决条件:] 函数在返回前需要满足的条件。
\index{后决条件}

\end{description}


\section{练习}

\begin{ex}

下载本章的代码
\url{thinkpython.com/code/polygon.py}.

\begin{enumerate}

\item 给 {\tt polygon}, {\tt arc} 和{\tt circle}编写合适的文档字符串。

\index{栈图}

\item 绘制栈图,给出{\tt circle(bob, radius)}执行时程序的状态。你可以手工计算其中的算术,或者添加{\tt print}语句。

\item 章节\ref{refactoring}中的{\tt arc}的版本不是非常精确,近似的直线总是在实际圆的外围,因此乌龟最终停止的位置偏离正确的位置。我的程序给出了一个减小误差的方法。阅读代码并理解。如果你尝试画图,你可以看到它是如何工作的。

\end{enumerate}

\end{ex}


\begin{ex}
\index{花}

Write an appropriately general set of functions that
can draw flowers like this:

\centerline{\includegraphics[height=1in]{figs/flowers.eps}}

You can download a solution from \url{thinkpython.com/code/flower.py}.

\end{ex}


\begin{ex}
\index{pie}

写一些列合适的函数,绘制以下的图形:

\centerline{\includegraphics[height=0.9in]{figs/pies.eps}}

你可以从这里下载一个解决方案 \url{thinkpython.com/code/pie.py}。

\end{ex}

\begin{ex}
\index{字母}
\index{乌龟打字机}
\index{打字机,乌龟}

字母可以由一些基本元素组成,,如垂直或水平的线和曲线。使用最少的基本元素设计一种字体,并调用函数绘制字母。

你可以为每个字母写一个函数,命名为\verb"draw_a", \verb"draw_b"等,然后把你的函数放在 {\tt letters.py}的文件里。你可以从\url{thinkpython.com/code/typewriter.py}下载一个“乌龟打字机”来帮助你测试你的代码。

你可以从这里下载一个解决方案 \url{thinkpython.com/code/letters.py}。

\end{ex}




\chapter{条件语句和递归}

\section{ 模操作符}

\index{modulus operator 模操作符}
\index{operator!modulus 操作符!模}

模操作符用于两个整数,第一个操作数除以第二个操作数产生余数。在Python
中,模操作符是一个百分号(\verb"%")。语法的格式和其他的操作符相同。

\beforeverb
\begin{verbatim}
>>>quotient = 7 / 3
>>>print quotient
2
>>>remainder = 7 % 3
>>>print remainder
1
\end{verbatim}
\afterverb

7除以3等于2余1。\\

模操作符是非常有用的,比如,你可以查看一个数是否可以被另一个数整除---
如果{\tt x \% y}是0,{\tt x}就可以被{\tt y}整除。\\

\index{divisibility 整除}

你也可以用模运算来提取整数的最右边的数字。比如,{\tt x \% 10 }得到{\tt x}的最右面的一个数字\footnote{译注:个位数}(以十为底)。类似地,
{\tt x \% 100}得到最后的两位数字\footnote{十位和个位的数字}。

\section{布尔表达式}
\index{boolean expression 布尔表达式}
\index{expression!boolean}
\index{logical operator 逻辑运算符}
\index{operator!logical}

布尔表达式的结果要么是真(true),要么为假(false)。下面的例子是使用
{\tt ==}运算符,比较两个操作数,如果相等则结果为{\tt True},否则为{\tt False}:

\beforeverb
\begin{verbatim}
>>> 5 == 5
True
>>> 5 == 6
False
\end{verbatim}
\afterverb

{\tt True}和{\tt False}是两个特殊的值,属于{\tt bool}类型;他们不是
字符串:

\index{True special value True特殊值}
\index{False special value False特殊值}
\index{specail value!True}
\index{special value!False}
\index{bool type bool类型}
\index{type!bool}

\beforeverb
\begin{verbatim}
>>> type(True)
<type 'bool'>
>>> type(False)
<type 'bool'>
\end{verbatim}
\afterverb

{\tt ==}运算符是关系运算符中的一个,其他的还有:

\beforeverb
\begin{verbatim}
      x != y               # x is not equal to y
      x > y                # x is greater than y
      x < y                # x is less than y
      x >= y               # x is greater than or equal to y
      x <= y               # x is less than or equal to y
\end{verbatim}
\afterverb


尽管你可能很熟悉这些运算符,他们在Python中的表示方法和数学中的有很大
的不同。一个常见的错误是只使用一个{\tt =}号,而不是两个{\tt ==}号。
记住{\tt =}是赋值操作符,{\tt ==}是关系运算符。而且,Python中没有这样
的符号{\tt =<}或者{\tt =>}\footnote{在FP(functional programming中可能
会遇到这个符号}。

\index{relational operator 关系运算符}
\index{operator!relational}

\section{逻辑运算符}
\index{logical operator 逻辑运算符}
\index{operator!logical}

有三个逻辑运算符:{\tt and},{\tt or}和{\tt not}。这些操作符的意思和
在英语中的意思差不多。比如,{\tt x > 0 and x < 10}为真,仅当{\tt x}
大于0小于10\footnote{译注:在Python中,更pythonic的写法是 0 < x < 10
。  这样的符号对于c/c++背景的程序员来说,有点陌生,在c/c++等值的分别
是\&\&, || ,!}。

\index{and operator and运算符}
\index{or operator or运算符}
\index{not operator not运算符}
\index{operator!and}
\index{operator!or}
\index{operator!not}

如果{\tt n \% 2 == 0 or n \% 3 == 0}有一个条件语句为真,则表达式的值就为真,亦即n可以被2或3整除。\\

最后,{\tt not}运算符对一个布尔表达式取反,所以如果{\tt (
x > y)为假,则{\tt not (x > y)}为真,亦即,{\tt x}小于或等于{\tt y}。\\

严格来说,逻辑运算符的操作数只能是布尔表达式,但是Python
对此可没什么严格要求。任何不为0的整数也被解释成{\tt True}
\footnote{这个也可扩展到任何其他的类型,比如后面要涉及到的
list,tuple,dict,set还有str。}

\beforeverb
\begin{verbatim}
>>> 17 and True
True
\end{verbatim}
\afterverb

这个灵活性是很有用处的,但是可能会产生一些微妙的问题。我们
要尽可能的避免他们(除非知道自己在做什么)。

\section{条件执行}
\label{conditional execution}

\index{conditional statement 条件语句}
\index{statement!conditional}
\index{if statement if语句}
\index{statement!if}
\index{conditional execution 条件执行}

考虑到要写一些有用的程序,我们几乎总是需要检查条件,并改变相应的改变程序的行为。条件语句给了我们这个能力。最简单的要属{\tt if}语句了:



\beforeverb
\begin{verbatim}
if x > 0:
    print 'x is positive'
\end{verbatim}
\afterverb
{\tt if}语句后面的布尔表达式叫做条件。如果条件为真,则下面
缩进的语句就被执行。反之,则什么也不发生。\footnote{这是
针对本例而言,因为本例只有一条语句。在其他的情况下,可能
会有诸如{\tt else}之类的语句。}

\index{condition 条件}
\index{compound statment}
\index{statment!compound}

{\tt if}语句和函数定义有着相同的结构:
一个头,后面跟着一个缩进的语句块。这样的语句叫做复合语句。\\

虽然对复合语句里面可以含有的语句数量不限,但是必须至少有一
条\footnote{c/c++中没有这样的限制}。偶然地,可能在语句体里
暂时不需要语句(通常作为一个占位符)。在这种情况下,我们可以
使用{\tt pass}语句,它什么也不做。



\index{pass statement}
\index{statement!pass}

\beforeverb
\begin{verbatim}
if x < 0:
    pass          # need to handle negative values!
\end{verbatim}
\afterverb

\section{选择执行}
\label {alternative execution}

\index{alternative executive 选择执行}
\index{else keyword else关键字}
\index{keyword!else}

{\tt if}语句的第二种形式是选择执行,此时,有两种可能性,
条件决定了哪一个可能性被执行。语法是这样:

\beforeverb
\begin{verbatim}
if x%2 == 0:
    print 'x is even'
else:
    print 'x is odd'
\end{verbatim}
\afterverb

如果{\tt x}除以2的余数是0,我们可以判定{\tt x}是偶数,程序
就输出这个效果。如果条件为假,第二个语句就被执行。因为条件
必须为真或假,其中一个必定会被执行。选择项叫做分支,因为
它们是执行流的分支。

\index{branch 分支}


\section{链式条件}
\index{chained conditional 链式条件}
\index{conditional!chained}

有时,可能会有不止两种可能性,我们就需要更多的分支。一种方式是使用链式条件语句。

\beforeverb
\begin{verbatim}
if x < y:
    print 'x is less than y'
elif x > y:
    print 'x is greater than y'
else:
    print 'x and y are equal'
\end{verbatim}
\afterverb

{\tt elif}是"else if"的缩写形式。再次说明一下,只有一条
语句被执行。{\tt elif}语句的数目也是没有限制的。如果要写
{\tt else}语句,必须是在链式条件的最后,但是如果没有,也是
允许的。

\index{elif keyword elif关键字}
\index{keyword!elif}

\beforeverb
\begin{verbatim}
if choice == 'a':
    draw_a()
elif choice == 'b':
    draw_b()
elif choice == 'c':
    draw_c()
\end{verbatim}
\afterverb

每个条件按顺序被检查。如果第一个为假,下一个就被检查,如此
如此。如果有一个为真,相应的分支就被执行,链式语句也就终止
。尽管可能有多个条件为真,也只有第一个为真的分支被执行。\\


\section{嵌套的条件语句}
\index{nested conditional 嵌套的条件语句}
\index{conditional!nested}

一条条件语句也可以嵌套在另一个语句之中。我们写一个典型的例子:

\beforeverb
\begin{verbatim}
if x == y:
    print 'x and y are equal'
else:
    if x < y:
        print 'x is less than y'
    else:
        print 'x is greater than y'
\end{verbatim}
\afterverb
外层的条件包两个分支。第一个分支包含一个简单语句。第二个分支包含例外一个
{\tt if}语句,同时,这个{\tt if}语句也有两个分支。这两个简单的分之都是简
单语句,尽管他们本也可能是条件语句。\\

尽管缩进使得代码结构清晰,嵌套语句还是难以快速的理解。一般来说,尽可能的
避免使用嵌套条件语句。\\

逻辑运算符可以简化嵌套条件语句。比如,下面的代码可以只用一条条件语句:

\beforeverb
\begin{verbatim}
if 0 < x:
    if x < 10:
        print 'x is a positive single-digit number.'
\end{verbatim}
\afterverb

只有当我们“通过”了两个条件时,{\tt print}语句才会被执行,所以,我们可以用{\tt and}运算符达到同样的效果。

\beforeverb
\begin{verbatim}
if 0 < x and x < 10:
    print 'x is a positive single-digit number.'
\end{verbatim}
\afterverb
 


\section{递归}
\label{recusion}
\index{recursion 递归}


函数调用\footnote{译注:台湾的书籍一般翻译为呼叫,这个很形象}另外一个函数是合法的;函数调用他自身也是合法的。很难一眼看出这样做有什么好处\footnote{译者注:我猜,这种情况就像是自己把自己提起来一样~~},但实践证明,
这是程序能做的最具有魔力的事情之一。比如,看下面的函数:

\beforeverb
\begin{verbatim}
def countdown(n):
    if n <= 0:
        print 'Blastoff!'
    else:
        print n
        countdown(n-1)
\end{verbatim}
\afterverb

如果{\tt n}是非正数,程序输出,“Blastoff!“,否则,输出{\tt n},然后调用
{\tt countdown}函数---也就是它自己---同时把{\tt n-1}当作参数传递给它。

如果我们调用这个函数,究竟发生了什么?

\beforeverb
\begin{verbatim}
>>> countdown(3)
\end{verbatim}
\afterverb
%
{\tt countdown}从{\tt n=3}开始执行,{\tt n}此时大于0,于是输出3,接着
调用自身。。。。。。

\begin{quote}
{\tt countdown}从{\tt n=2}开始执行,{\tt n}此时大于0,于是输出2,接着调用自身。。。。。。

\begin{quote}
{\tt countdown}从{\tt n=1}开始执行,{\tt n}此时大于0,于是输出1,接着调用自身。。。。。。

\begin{quote}
{\tt countdown}从{\tt n=0}开始执行,{\tt n}此时不大于0,输出“Blastoff!"
然后返回。
\end{quote}

接受{\tt n=1}的{\tt countdown}返回。
\edn{quote}

接受{\tt n=2}的{\tt countdown}返回。
\end{quote}

接受{\tt n=2}的{\tt countdown}返回。
\end{quote}

{\tt countdown}接受{\tt n=3}的函数返回。

然后,我们就会到\verb"__main__"里了。整个输出如下:

\beforeverb
\begin{verbatim}
3
2
1
Blastoff!
\end{verbatim}
\afterverb

 调用自身的函数称作递归函数;调用的过程叫做递归。

 \index{recursion 递归}
 \index{function!recursive}

另外一个例子,我们写一个打印一个字符串{\tt n}次的函数。

\beforeverb
\begin{verbatim}
def print_n(s, n):
    if n <= 0:
        return
    print s
    print_n(s, n-1)
\end{verbatim}
\afterverb

如果{\tt n <= 0},{\tt return}语句退出函数。执行流立刻返回到调用者,
剩余的部分就不再被执行了。

\index{return statement return语句}
\index{statement!return}

函数的剩余部分和{\tt countdown}还书相似:如果{\tt n}大于0,输出{\tt s},然后调用自身显示{\tt s} $n-1$次。所以,输出的行数是{\tt 1 + (n-1)},
也就是{\tt n}次。\\

这样简单的例子,其实可以很容易用一个{\tt for}循环来实现。但我们以后将会看
到很难写成{\tt for}循环形式,但是很容易用递归实现的例子,我们现在就
开始认识递归是有好处的。

{\section{递归函数的堆栈图}
\index{stack diagram 堆栈图}
\index{function frame 函数图}
\index{frame 图}






\chapter{ “结果”函数}

\label{fruitchap}

\section{返回值}
\index{返回值}

我们使用的一些内建函数,如数学函数,会返回结果。调用这些函数的过程中会产生一个数值,通常我们将它赋值给一个变量,或者作为表达式的一部分。

\beforeverb
\begin{verbatim}
e = math.exp(1.0)
height = radius * math.sin(radians)
\end{verbatim}
\afterverb
%
目前为止我们所写的函数都是空的:它们打印一些东西或者移动乌龟,但它们的返回值都是{\tt 空}。

在本章中,我们开始编写卓有成效的函数。第一个例子是{\tt area},它返回给定半径的圆的面积:

\beforeverb
\begin{verbatim}
def area(radius):
    temp = math.pi * radius**2
    return temp
\end{verbatim}
\afterverb
%
我们先前见过  {\tt return} 语句,但在卓有成效的函数中, {\tt return} 语句包含一个表达式。返回语句的含义为:“立即从函数返回,并将表达式的值作为返回值”。表达式可以任意复杂,因此我们可以这样更简洁的写这个函数:

\index{返回语句}
\index{语句!返回}

\beforeverb
\begin{verbatim}
def area(radius):
    return math.pi * radius**2
\end{verbatim}
\afterverb
%
同时,{\bf 临时变量} 如 {\tt temp} 通常使得调试更为方便。

\index{临时变量}
\index{变量!临时}

有时一个函数会有多个返回语句,每一条位于一个条件分支中:

\beforeverb
\begin{verbatim}
def absolute_value(x):
    if x < 0:
        return -x
    else:
        return x
\end{verbatim}
\afterverb
%
由于 {\tt return} 语句位于一个选择的条件中,只有一条会被执行。

当一条返回语句被执行时,函数立刻返回,而不再执行后面的语句。在 {\tt return} 语句后的代码,或者存在于任何程序流执行不到的地方的代码,称为 {\bf 死区代码}。

\index{死区代码}

在一个卓有成效的函数中,确保每个可能的路径都对应一个{\tt return} 语句是一个良好的习惯。例如:

\beforeverb
\begin{verbatim}
def absolute_value(x):
    if x < 0:
        return -x
    if x > 0:
        return x
\end{verbatim}
\afterverb
%
这个程序是错误的,因为当 {\tt x} 恰为0的时候,没有一个条件语句为真,函数结束时没有遇到{\tt return} 语句。如果一个程序执行到函数的末尾,它将返回一个 {\tt None}值,而不是绝对值0.

\index{None特殊值}
\index{特殊值!None}

\beforeverb
\begin{verbatim}
>>> print absolute_value(0)
None
\end{verbatim}
\afterverb
%
顺便一提,Python提供了一个内建函数{\tt abs},可以计算一个数的绝对值。

\index{绝对值函数}
\index{函数!绝对值}

\begin{ex}

\index{比较函数}
\index{函数!比较}

编写 {\tt 比较} 函数,函数返回 {\tt 1} 如果 {\tt x > y},返回{\tt 0} 如果 {\tt x == y},返回 {\tt -1} 如果 {\tt x < y}。
\end{ex}


\section{增量开发}
\label{增量开发}
\index{开发计划!增量}

当你编写越来越大的函数时,你发现在调试上会花更多的时间。

对于越加复杂的程序,你可以尝试{\bf 增量开发}的方法。增量开发是通过每次测试一小部分代码的方法来避免过长的调试过程。

\index{测试!增量开发}
\index{毕达哥拉斯理论}

例如,假设你需要计算给定坐标$(x_1, y_1)$和$(x_2, y_2)$两点间的距离。根据毕达哥拉斯理论,距离为:

\begin{displaymath}
\mathrm{distance} = \sqrt{(x_2 - x_1)^2 + (y_2 - y_1)^2}
\end{displaymath}
%
首先考虑 {\tt distance} 函数在Python中应该是如何的。换言之,函数的输入(参数)和输出(返回值)应该是什么?

在本例中,输入是两个点,你可以用4个数来表示。返回值是距离,可以用一个浮点数来表示。

现在你可以写出函数的框架:

\beforeverb
\begin{verbatim}
def distance(x1, y1, x2, y2):
    return 0.0
\end{verbatim}
\afterverb
%
显然,这个版本并没有计算距离;它总是返回零。但是它的语法是正确的,可以正常运行,这意味着你在把它修改的更复杂前可以测试它。

为了测试新函数,我们使用样例参数来调用它:

\beforeverb
\begin{verbatim}
>>> distance(1, 2, 4, 6)
0.0
\end{verbatim}
\afterverb
%
我选择的两个点水平距离为3,垂直距离为4,因此两点距离为5(斜边为3-4-5的三角形)。当测试一个函数时,直到正确答案是很有用的。

\index{测试!直到答案}

现在我们确定函数的语法是正确的,我们可以开始添加代码。下一步是计算$x_2 - x_1$ 和 $y_2 - y_1$的差,下一个版本将这两个值储存在临时变量中,并打印。

\beforeverb
\begin{verbatim}
def distance(x1, y1, x2, y2):
    dx = x2 - x1
    dy = y2 - y1
    print 'dx is', dx
    print 'dy is', dy
    return 0.0
\end{verbatim}
\afterverb
%
如果函数可以工作,它应该显示 \verb"'dx is 3'" 以及 {\tt 'dy is 4'}。如果没错,我们知道函数得到了正确的参数,并正确的执行了第一步计算。如果有错,我们只需要检查几行代码。

下一步我们计算 {\tt dx} 和 {\tt dy}的平方和:

\beforeverb
\begin{verbatim}
def distance(x1, y1, x2, y2):
    dx = x2 - x1
    dy = y2 - y1
    dsquared = dx**2 + dy**2
    print 'dsquared is: ', dsquared
    return 0.0
\end{verbatim}
\afterverb
%
同样,你可以运行程序,并检查结果(应该是25)。最后你可以使用 {\tt math.sqrt} 来计算并返回结果。

\index{开根}
\index{函数!开根}

\beforeverb
\begin{verbatim}
def distance(x1, y1, x2, y2):
    dx = x2 - x1
    dy = y2 - y1
    dsquared = dx**2 + dy**2
    result = math.sqrt(dsquared)
    return result
\end{verbatim}
\afterverb
%
如果运行正确,你就完成了这个程序。否则,你需要在返回语句前打印 {\tt result}。

函数的最终版本在执行过程中并不显示任何内容。{\tt print}语句在调试的过程中非常有用,但是一旦程序工作正常了,你需要删除它们。这样的代码被称为{\bf 脚手架代码},它们对编写程序有帮助,但并不是最终结果的一部分。

\index{脚手架代码}

当你刚开始时,你应该每次只添加几行代码。当你变得熟练时,你发现你可以调试大段的代码。通常,增量开发可以节约你很多调试的时间。

这个过程的关键在于:

\begin{enumerate}

\item 编写程序时每次只做少量修改,在任何时刻如果出错,你可以很方便的定位错误。

\item 使用临时变量记录中间过程值,这样你可以显示并检查它们。

\item 当程序正常工作后,你可能需要删除一些脚手架代码或者合并多个语句为一个复合语句。注意保持程序的可读性。

\end{enumerate}

\begin{ex}

\index{斜边}

使用增量开发编写一个 {\tt 斜边} 函数,读取三角形两条直角边作为参数,返回斜边长度。记录下开发过程中的每一个阶段。
\end{ex}

\section{Composition}

\index{复合}
\index{函数复合}

现在你应该可以预料,你可以在一个函数中调用另一个函数。这种能力被称为 {\bf 复合}。

例如,我们将写一个函数,它读入两个点作为参数,一个是圆心,另一个是圆周上的一个点,该函数将计算圆周的面积。

假设圆心对应的点保存在变量 {\tt xc} 和{\tt yc},圆周上的点的坐标为 {\tt xp} 和 {\tt yp}。第一步是计算圆的半径,即两点之间的距离。我们刚写了函数 {\tt distance},它实现了这个功能:

\beforeverb
\begin{verbatim}
radius = distance(xc, yc, xp, yp)
\end{verbatim}
\afterverb
%
第二部是通过半径计算圆的面积,我们刚才也实现了:

\beforeverb
\begin{verbatim}
result = area(radius)
\end{verbatim}
\afterverb
%
将这两步封装在一个函数,我们得到:

\index{封装}

\beforeverb
\begin{verbatim}
def circle_area(xc, yc, xp, yp):
    radius = distance(xc, yc, xp, yp)
    result = area(radius)
    return result
\end{verbatim}
\afterverb
%
临时变量 {\tt radius} 和 {\tt result} 用来开发和调试,但是当程序正常工作后,我们可以通过复合这些函数调用来使得程序看起来更简洁:

\beforeverb
\begin{verbatim}
def circle_area(xc, yc, xp, yp):
    return area(distance(xc, yc, xp, yp))
\end{verbatim}
\afterverb
%

\section{布尔函数}
\label{布尔}

\index{布尔函数}

函数可以返回布尔值,这样便于隐藏函数中复杂的细节。例如:

\beforeverb
\begin{verbatim}
def is_divisible(x, y):
    if x % y == 0:
        return True
    else:
        return False
\end{verbatim}
\afterverb
%
通常布尔函数的函数名类似yes/no的问题; \verb"is_divisible" 根据{\tt x}是否可以被{\tt y}整除,返回 {\tt True} 或 {\tt False}。

下面给出一个例子:

\beforeverb
\begin{verbatim}
>>>   is_divisible(6, 4)
False
>>>   is_divisible(6, 3)
True
\end{verbatim}
\afterverb
%
运算符 {\tt ==} 的结果返回一个布尔值,所以我们可以直接返回结果,这样更加简洁:

\beforeverb
\begin{verbatim}
def is_divisible(x, y):
    return x % y == 0
\end{verbatim}
\afterverb
%
布尔函数通常用在条件语句中:

\index{条件语句}
\index{语句!条件}

\beforeverb
\begin{verbatim}
if is_divisible(x, y):
    print 'x is divisible by y'
\end{verbatim}
\afterverb
%
也许会写成类似如下的代码:

\beforeverb
\begin{verbatim}
if is_divisible(x, y) == True:
    print 'x is divisible by y'
\end{verbatim}
\afterverb
%
但是多余的比较是不必要的。

\begin{ex}
写一个函数 \verb"is_between(x, y, z)" ,如果$x \le y \le z$则返回 {\tt True},否则返回 {\tt False} 。
\end{ex}


\section{更多递归}

\index{递归}
\index{图灵完备语言}
\index{语言!图灵完备}
\index{Turing, Alan}
\index{图灵理论}

我们仅仅覆盖了Python的一小部分,但也许你会对这是一个{\em 完备}的编程语言集感兴趣,即任何可以被计算的东西都可以用这用语言来表示。任何现有的程序都可以用你现在学到的语言特征重写(事实上你还需要一些命令来控制键盘、鼠标、磁盘等设备,但仅此而已)。

这种说法首先由早期计算机科学家Alan Turing提出(有人会争论他是一个数学家,但许多早期的计算机科学家是数学家起家的),被称为图灵理论。如果你想对图灵理论有更完全(更精确)的讨论,我推荐Michael Sipser's的书 {\em Introduction to the Theory of Computation}。

为了让你认识到你可以用到目前为止学到的工具做什么,我们来计算一些递归定义的数学函数。递归定义类似循环定义,定义的本身包含对所定义的东西的引用。一个真正的循环定义不是很有用:

\begin{description}

\item[frabjous:] 一个描述frabjous东西的形容词。

\end{description}

\index{frabjous}
\index{循环定义}
\index{定义!循环}

如果在字典里看到这样的解释,你也许会感到恼怒。然而,如果你查看阶乘的定义(表示为$!$),你会的到类似这样的东西:

\vspace{-0.35in}
\begin{eqnarray*}
&&  0! = 1 \\
&&  n! = n (n-1)!
\end{eqnarray*}
\vspace{-0.25in}

这个定义规定0的阶乘是1,对于其他的任何数 $n$的阶乘,其值为 $n$ 乘以 $n-1$的阶乘。

因此 $3!$ 的阶乘是 3 乘以 $2!$,$2!$的阶乘是  2 乘以 $1!$,$1!$的阶乘是 1 乘以$0!$。将这些放在一起,$3!$ 等于3乘以2乘以1乘以1,结果为6。

\index{阶乘函数}
\index{函数!阶乘}
\index{递归定义}

如果你可以写出某个东西的递归定义,你通常可以通过写一个Python程序来进行求值。第一步是决定参数。很明显在本例中 {\tt factorial} 读取一个整数作为参数:

\beforeverb
\begin{verbatim}
def factorial(n):
\end{verbatim}
\afterverb
%
如果参数等于0,我们所要做的就是返回1:

\beforeverb
\begin{verbatim}
def factorial(n):
    if n == 0:
        return 1
\end{verbatim}
\afterverb
%
否则,将开始有趣的部分,我们递归的调用阶乘函数 $n-1$ 然后乘以$n$:

\beforeverb
\begin{verbatim}
def factorial(n):
    if n == 0:
        return 1
    else:
        recurse = factorial(n-1)
        result = n * recurse
        return result
\end{verbatim}
\afterverb
%
程序的执行流程类似章节~\ref{递归}中 {\tt countdown} 。如果我们使用参数3调用 {\tt factorial}:

由于3不是0,我们得到第二个计算{\tt n-1}阶乘的分支...

\begin{quote}
由于2不是0,我们得到第二个计算{\tt n-1}阶乘的分支...


  \begin{quote}
  由于1不是0,我们得到第二个计算{\tt n-1}阶乘的分支...


    \begin{quote}
    由于0{\em 等于}0,我们得到第一个分支,返回结果1,不再进行更多的递归调用。
    \end{quote}


  返回值(1)被乘以 $n$,对应1,然后结果返回。
  \end{quote}


返回值(1)被乘以 $n$,对应2,然后结果返回。 
\end{quote}


返回值(2)被乘以 $n$,对应3,结果为6,这是整个函数最终返回的结果。

\index{栈图}
下面给出这一系列函数调用对应的栈图:

\vspace{0.1in}
\beforefig
\centerline{\includegraphics{figs/stack3.eps}}
\afterfig
\vspace{0.1in}

如图所示,返回值通过栈向上传递。在每一帧中,返回值是{\tt result}的值,即 {\tt n} 和 {\tt recurse}的乘积。

\index{帧}

在最后一帧中,局部变量 {\tt recurse} 和 {\tt result} 并不存在,因为该分支并没有创建它们。


\section{信心的飞跃}
\index{递归}
\index{信心的飞跃}

按照程序执行流程是阅读程序的一个方法,但这么做很容易走入迷宫。这里给出另一个方法,我称之为“信心的飞跃”。当你遇到一个函数调用,你{\em 假设}函数执行无误,并返回正确的结果,而不是追溯程序执行流程。

事实上,当你调用内建函数时,你已经体验过信心的飞跃。当你调用 {\tt math.cos} 和 {\tt math.exp}时,你并不检查这些函数的主体。你仅仅假设它们是由优秀的程序员编写的,可以正常的工作。

这同样适用于你自己写的函数。例如,在章节~\ref{布尔}中,我们编写了一个叫做\verb"is_divisible"的函数,判断一个数是否可以被另一个数整除。一旦我们通过检查测试函数确信它可以正常工作,我们可以直接调用该函数,而不用再去查看函数主体。

\index{测试!信心的飞跃}

以上思想同样适用于递归程序。当你遇到递归调用,你应该假设递归调用工作正常(并返回正确结果),而不是跟着程序执行流程。你可以这么问你自己,“假设我可以得到$n-1$的阶乘,我应该怎么计算$n$的阶乘?”很明显,通过乘以$n$就可以了。

当然,假设程序工作正常,而事实上你还没有完成这个函数,这里有一些诡异,但这就是为什么称之为信心的飞跃!


\section{另一个例子}
\label{另一个例子}

\index{斐波纳契函数}
\index{函数!斐波纳契}

除了 {\tt 阶乘}外,最常见的递归定义的数学函数是 {\tt 斐波纳契}函数,其定义如下:\footnote{参见 \url{wikipedia.org/wiki/Fibonacci_number}。}:

\vspace{-0.25in}
\begin{eqnarray*}
&& \mathrm{fibonacci}(0) = 0 \\
&& \mathrm{fibonacci}(1) = 1 \\
&& \mathrm{fibonacci}(n) = \mathrm{fibonacci}(n-1) + \mathrm{fibonacci}(n-2);
\end{eqnarray*}
%
翻译成Python为:

\beforeverb
\begin{verbatim}
def fibonacci (n):
    if n == 0:
        return 0
    elif  n == 1:
        return 1
    else:
        return fibonacci(n-1) + fibonacci(n-2)
\end{verbatim}
\afterverb
%
如果你试图跟随执行过程,即使很小的数$n$,你的脑袋也会爆炸。但是根据信心的飞跃,假设两个递归调用工作正确,就很容易看出通过将两者相加你将得到正确的结果。

\index{执行流程}


\section{类型检查}
\label{监护人}

\index{类型检查}
\index{错误检查}
\index{阶乘函数}

如果我们调用 {\tt factorial} 函数时提供的参数是1.5会怎么样?

\index{运行时错误}

\beforeverb
\begin{verbatim}
>>> factorial(1.5)
RuntimeError: Maximum recursion depth exceeded
\end{verbatim}
\afterverb
%
看起来这将会是一个无穷的递归。这事怎么发生的?递归函数有一个初始情况————当 {\tt n == 0}。但是如果 {\tt n} 不是一个整数,我们将 {\em 错过} 初始状态并不断的递归。

\index{无穷递归}
\index{递归!无穷}

在第一次递归调用中, {\tt n} 的值是0.5。在下一次调用时,它变为 -0.5。接着,它变得越来越小(更朝负方向),永远不会为0。

我们有两种选择。我们可以推广 {\tt factorial}函数使之可以工作在浮点数,或者我们可以让 {\tt factorial} 检查参数的数据类型。第一个方法可以参考gamma函数\footnote{参考 \url{wikipedia.org/wiki/Gamma_function}。},超出了本书的范围。让我们看看第二个方法。

\index{gamma函数}

我们可以使用内建函数 {\tt isinstance} 来验证参数的数据类型。同时我们可以确保参数的正的:

\index{isinstance函数}
\index{函数!isinstance}

\beforeverb
\begin{verbatim}
def factorial (n):
    if not isinstance(n, int):
        print 'Factorial is only defined for integers.'
        return None
    elif n < 0:
        print 'Factorial is only defined for positive integers.'
        return None
    elif n == 0:
        return 1
    else:
        return n * factorial(n-1)
\end{verbatim}
\afterverb
%
第一部分处理非整数;第二部分捕获负数。在这两种情况下程序将打印错误信息,并返回 {\tt None}表示有错误发生:

\beforeverb
\begin{verbatim}
>>> factorial('fred')
Factorial is only defined for integers.
None
>>> factorial(-2)
Factorial is only defined for positive integers.
None
\end{verbatim}
\afterverb
%
如果通过了两个检查,我们知道 $n$ 是一个正整数,可以保证递归会终止。

\index{守护人模式}
\index{模式!监护人}

这个程序展示了一个称为{\bf 监护人}的模式。前两个条件扮演了监护人的角色,防止错误的参数造成代码运行的错误。监护代码保证了程序的正确性。


\section{调试}
\label{factdebug}

\index{调试}

将大的程序拆成小的函数的过程给出了合理的调试测试点。如果一个函数工作不正常,有三种情况需要考虑:

\begin{itemize}

\item 函数得到的参数有误,即先决条件没有满足。

\item 函数本身有误,即后决条件没有满足。

\item 返回之有误,或者使用方式不正确。

\end{itemize}

为了排除第一种可能性,你可以在函数开始部分添加{\tt print}语句,打印参数值(数据类型)。或和你可以直接编写代码检查先决条件。

\index{先决条件}
\index{后决条件}

如果参数没有问题,在每条{\tt return}语句前添加{\tt print}语句,打印返回值。可能的话手工检查结果。试图使用容易检查的参数来调用函数(参见章节~\ref{增量开发})。

如果函数工作正常,检查函数调用,确保返回值被正确使用(或被使用!)。

\index{执行流程}

在函数开始和结尾添加打印语句可以使程序流程更加明显。例如,下面给出带打印语句的{\tt factorial}函数:

\beforeverb
\begin{verbatim}
def factorial(n):
    space = ' ' * (4 * n)
    print space, 'factorial', n
    if n == 0:
        print space, 'returning 1'
        return 1
    else:
        recurse = factorial(n-1)
        result = n * recurse
        print space, 'returning', result
        return result
\end{verbatim}
\afterverb
%
{\tt space}是空格的字符串,用来缩进输出结果。下面给出{\tt factorial(5)}的输出结果:

\beforeverb
\begin{verbatim}
                     factorial 5
                 factorial 4
             factorial 3
         factorial 2
     factorial 1
 factorial 0
 returning 1
     returning 1
         returning 2
             returning 6
                 returning 24
                     returning 120
\end{verbatim}
\afterverb
%
如果你不清楚程序执行的流程,这样的输出信息很有帮助。搭建脚手架会花费一些时间,但会帮助节省很多调试时间。


\section{术语}

\begin{description}

\item[临时变量:] 在复杂的计算过程中用来记录中间值的变量。
\index{临时变量}
\index{变量!临时}

\item[死区代码:] 程序中无法执行到的代码,通常因为出现在 {\tt return} 语句的后面。
\index{死区代码}

\item[{\tt None}:]  一个特殊的函数返回值,表明函数没有返回语句,或者返回语句没有参数。
\index{None特殊值}
\index{特殊值!None}

\item[增量开发:] 一种程序开发的方法,通过每次添加测试少量代码来避免程序的调试。
\index{增量开发}

\item[脚手架:] 在程序开发过程中使用的代码,并不出现在最终版本中。
\index{脚手架}

\item[守护人:] 使用条件语句检查错误并处理可能出错的情况的编程模式。
\index{守护人模式}
\index{模式!守护人}

\end{description}


\section{练习}

\begin{ex}
\index{栈图}

画下面程序的栈图。程序执行时会打印什么内容?

\beforeverb
\begin{verbatim}
def b(z):
    prod = a(z, z)
    print z, prod
    return prod

def a(x, y):
    x = x + 1
    return x * y

def c(x, y, z):
    sum = x + y + z
    pow = b(sum)**2
    return pow

x = 1
y = x + 1
print c(x, y+3, x+y)
\end{verbatim}
\afterverb

\end{ex}


\begin{ex}
\index{Ackerman函数}
\index{函数!ack}

Ackermann函数,$A(m, n)$,定义为\footnote{See \url{wikipedia.org/wiki/Ackermann_function}。}:

\begin{eqnarray}
A(m, n) = \begin{cases} 
              n+1 & \mbox{if } m = 0 \\ 
        A(m-1, 1) & \mbox{if } m > 0 \mbox{ and } n = 0 \\ 
A(m-1, A(m, n-1)) & \mbox{if } m > 0 \mbox{ and } n > 0.
\end{cases} 
\end{eqnarray}
%
编写一个 {\tt ack}的函数来计算 Ackerman's 公式。使用你的函数来计算 {\tt ack(3, 4)},结果应该是125。如果选择大的 {\tt m} 和 {\tt n}会发生什么?

\end{ex}


\begin{ex}
\label{回文}

\index{回文}

回文是一个从前往后和从后往前拼写相同的单词,如“noon”和“redivider”。从递归的角度来看,如果一个单词第一个字母和最后一个字母相同,且中间是回文的,那么这个单词是回文的。

下面的函数读取一个字符串作为参数,分别返回第一个、最后一个和中间的字母:

\beforeverb
\begin{verbatim}
def first(word):
    return word[0]

def last(word):
    return word[-1]

def middle(word):
    return word[1:-1]
\end{verbatim}
\afterverb
%
我们会在章节~\ref{strings}学习它们是如何工作的。

\begin{enumerate}

\item 在{\tt palindrome.py}文件中输入这些函数并测试。如果你输入两个字母,调用{\tt middle}会返回什么?一个字母呢?空字符串呢(使用\verb"''"输入,不包含任何字母)?

\item 编写函数\verb"is_palindrome",读取一个字符串作为参数,如果是回文则返回{\tt True},否则返回{\tt False}。你可以使用内建的{\tt len}函数来检查字符串的长度。

\end{enumerate}

\end{ex}

\begin{ex}
数$a$称为数$b$的幂,如果$a$可以被$b$整除,同时$a/b$是$b$的幂。编写函数\verb"is_power",读取{\tt a}和{\tt b}作为参数,如果{\tt a}是{\tt b}的幂则返回{\tt True}。
\end{ex}


\begin{ex}

\index{最大公约数(GCD)}
\index{GCD(最大公约数)}

最大公约数(GCD)是可以整除$a$和$b$最大的书\footnote{本练习基于Abelson and Sussman's {\em Structure and Interpretation of Computer Programs}的习题。}。

寻找两个数的GCD的一个方法是欧几里得算法,算法指出如果$r$是$a$除以$b$的余数,那么$gcd(a, b) = gcd(b, r)$。对于基本情况,我们考虑$gcd(a, 0) = a$。

\index{欧几里得算法}
\index{算法!欧几里得}

编写函数\verb"gcd" ,参数为{\tt a}和{\tt b},返回值为两者的最大公约数。如果你需要帮助,参考\url{wikipedia.org/wiki/Euclidean_algorithm}。

\end{ex}



\chapter{迭代器}
\index{iteration 迭代器}

\section{多重赋值}
\index{statement!assignment}
\index{mutiple assignment 多重赋值}

你可能已经发现,给一个变量多次赋值是合法的。一次新的赋值使得已存在的变量指向一个新值(当然也就不指向原来的值)。

\beforeverb
\begin{verbatim}
bruce = 5
print bruce,
bruce = 7
print bruce
\end{verbatim}
\afterverb

程序的输出是{\tt 5 7},因为第一次{\tt bruce}被输出时,它的值是5,第二次是7。
第一个{\tt print}语句末尾的逗号抑制了换行,这也是为什么两个输出在同一行的原因。

\index{newline 换行}

下图是多重赋值的状态图:

\index{state diagram}
\index{diagram!state}

\beforefig
\centerline{\includegraphics{figs/assign2.eps}}
\afterfig

对于多重赋值,很有必要分清赋值操作符和关系运算符中的等号。因为Python使用
等于号({\tt =})来表示赋值。很容易,误把这样的语句{\tt a = b}当作判断相等的语句,
实际不是的!\\

\index{相等与赋值}


第一,相等是对称关系,赋值不是。比如,在数学中,如果$a = 7$, 那么$7 =
a$。但在Python中,语句{\tt a = 7}是合法的,{\tt 7 = a}是非法的。\\

另外,在数学中,相等语句总是要么为真要么为假。如果,$a = b$,则,$a$总
是等于$b$。在Python中,赋值语句可以使得两个变量相等,但是他们不总是保持相等:

\beforeverb
\begin{verbatim}
a = 5
b = a    # a and b are now equal
a = 3    # a and b are no longer equal
\end{verbatim}
\afterverb
%

第三行改变了{\tt a}的值,但是没有改变{\tt b}的值。所以他们不再相等。

尽管多重赋值通常是有益的,使用时也要小心。如果变量的值经常改变,代码
会变得很脑阅读和调试。

\section{更新变量}
\label{update}

\index{update 更新}
\index{variable!updating}

多重赋值的最常见的形式之一就是更新(update),变量的新值依赖于原有值。

\beforeverb
\begin{verbatim}
x = x+1
\end{verbatim}
\afterverb

含义是:获取{\tt x}的当前值,加一,然后用新值更新变量{\tt x}。

如果试着更新一个不存在的变量,将会得到一个错误,因为Python在把值赋给
{\tt x}之前会计算右边的值。

\beforeverb
\begin{verbatim}
>>> x = x+1
NameError: name 'x' is not defined
\end{verbatim}
\afterverb
%

在更新一个变量之前,必须得初始化(initialize)它,通常的做法就是一个
简单的赋值。

\index{initialization (before update) (更新前)初始化}

\beforeverb
\begin{verbatim}
>>> x = 0
>>> x = x+1
\end{verbatim}
\afterverb
%

仅仅通过加一来更新变量叫做增量 (increment);减一叫做减量(decrement)。

\index{increment  增量}
\index{decrement 减量}

\section{{\tt while 语句}}

\index{statement!while}
\index{while loop while循环}
\index{loop!while}
\index{iteration 迭代}

计算机通常被用来自动完成重复性的任务。计算机很擅长于重复相同或相似的任务。人类却恰恰相反\footnote{太枯燥了,回顾一下小学时,老师天天要求抄生字......}。

我们已经看到两个程序,{\tt countdown}和\verb"print_n",它们使用递归
实现重复,这也可以乘坐迭代{\bf iteration}。因为迭代是如此的常见,以致
于Python提供了几个特有的方式来简化使用。其中之一就是我们在\ref{重复}
部分看到的{\tt for}语句。我们不久将回来重新研究它。\\

另外一个就是{\tt while}语句。这里是一个使用{\tt while}语句的{\tt coutdown}版本。

\beforeverb
\begin{verbatim}
def countdown(n):
    while n > 0:
        print n
        n = n-1
    print 'Blastoff!'
\end{verbatim}
\afterverb

我们几乎可以把{\tt while}语句当成英语来读了。含义是:当{\tt n}大于0时
,显示{\tt n}的值,并把{\tt n}的值减1。当{\tt n}的值为0的时候,显示{\tt Blastoff!}。

\index{flow of execution 执行流}

更正式地,下面的是{\tt while}语句的执行流。

\begin{enumerate}

\item 计算条件的值,产生结果{\tt True}或者{\tt False}。

\item 如果条件为假,退出{\tt while循环},继续执行下一条语句。

\item 如果条件为真,执行语句体里的语句,然后回到步骤一。

\end{enumerate}

执行流的类型称作是循环(loop)的原因是因为第三步循环返回至第一步。

\index{condition 条件}
\index{loop 循环}
\index{body 体}

循环体应该改变一个或多个变量的值,使得最终条件为假,循环终止。否则,
循环将永远重复,也就产生了无限循环(infinite loop)。计算机科学家的
一个永远的谈资就是看到洗发水的说明"泡沫,漂洗,重复",是一个无限循环。

\index{infinite loop 无限循环}
\index{loop!infinite}

在{\tt countdown}的例子里,我们可以证明循环一定会终止,因为我们知道
{\tt n}的值是有限的,并且每一次循环{\tt n}的值都会减小,最终,{\tt n}的值肯定是0。在其他情况下,就不一定这么容易辨别了:

beforeverb
\begin{verbatim}
def sequence(n):
    while n != 1:
        print n,
        if n%2 == 0:        # n is even
            n = n/2
        else:               # n is odd
            n = n*3+1
\end{verbatim}
\afterverb

这个循环的条件是{\tt n != 1},所以循环会一直执行到{\tt n}是{\tt 1},
此时条件为假。\\

每一次循环,程序输出{\tt n}的值,然后检查是否为偶数或奇数。如果是偶数
{\tt n}就除以2。如果为奇数,{\tt n}的值就被{\tt n*3+1}代替。比如,
如果传递3给{\tt sequence},产生的结果是3, 10, 5, 16, 8, 4, 2, 1。

因为{\tt n}时增时减,没有一个明显的办法确定{\tt n}是否会为1,也就是
程序是否会正常终止。对于某些特别的{\tt n},我们可以证明终止。比如,
如果{\tt n}的值是2的倍数,每次循环时{\tt n}的值,都是偶数直到为1。
前面的例子从16开始都是这种情况。

\index{Collatz conjecture Collatz猜想}

困难的是我们是否可以证明对所有的正整数,程序都能终止。迄今为止\footnote{参看\url{wikipedia.org/wiki/Collatz_conjecture}.}没有人可以证明
可以,也没有人证明不可以。

\begin{ex}
重写\ref{递归}部分的\verb"print_n"函数,要求使用迭代器,而不是递归。
\end{ex}

\section{{\tt break}语句}
\index{break statement break语句}
\index{statement!break}

有时,直到执行到循环体里面的时候,才直到需要跳出循环。此时,我们可以
使用{\tt break}语句跳出循环。

比如,假设一直想从用户那里得到输入,直到用户输入{\tt done}。我们可以
这么写:

\beforeverb
\begin{verbatim}
while True:
    line = raw_input('> ')
    if line == 'done':
        break
    print line

print 'Done!'
\end{verbatim}
\afterverb

循环条件为{\tt True},也就是永远为真,所以循环直到遇到break statement
才终止执行。

每次循环,用尖括号提示用户。如果用户输入{\tt done},{\tt break}语句
终止了循环。否则,程序输出用户输入的内容,会到循环的顶部。下面是一个例子:

\beforeverb
\begin{verbatim}
> not done
not done
> done
Done!
\end{verbatim}
\afterverb
%

这种使用{\tt while}循环的方式很常见,因为我们可以在循环的任何地方检查
条件(不仅仅是在顶部),同时也积极的表达了结束的条件(当这个发生时,终止),而不是消极地("一直运行,直到这个发生")。

\section{平方根}
\index{square root}

循环经常用在计算数值的程序中,通常都是以一个相近的值开始,然后迭代,逐渐提高。

\index{Newton's method 牛顿方法}

比如,有一种计算平方根的算法叫牛顿方法。假设,我们想得到$a$的平方根。如果以任意猜测的
一个值开始,我们可以用下面的公式,计算一个更好的猜测值:

\[ y = \frac{x + a/x}{2} \]
比如,如果$a$是4, $x$是3:

\beforeverb
\begin{verbatim}
>>> a = 4.0
>>> x = 3.0
>>> y = (x + a/x) / 2
>>> print y
2.16666666667
\end{verbatim}
\afterverb
%
结果已经接近正确答案了($\sqrt{4} = 2$)。如果我们重复这个过程,就更接近了:

\beforeverb
\begin{verbatim}
>>> x = y
>>> y = (x + a/x) / 2
>>> print y
2.00641025641
\end{verbatim}
\afterverb
经过几次更新,猜测值基本上等于精确值了:
\index{update 更新}

\beforeverb
\begin{verbatim}
>>> x = y
>>> y = (x + a/x) / 2
>>> print y
2.00001024003
>>> x = y
>>> y = (x + a/x) / 2
>>> print y
2.00000000003
\end{verbatim}
\afterverb

一般来说,我们实现并不知道经过多少步才能得到正确的结果,但是我们知道什么时候得到
正确的结果,应为猜测值成定值了。

\beforeverb
\begin{verbatim}
>>> x = y
>>> y = (x + a/x) / 2
>>> print y
2.0
>>> x = y
>>> y = (x + a/x) / 2
>>> print y
2.0
\end{verbatim}
\afterverb

当{\tt y == x},我们就可以停止了。下面是一个循环,从一个初始值开始,然后逐步逼近,直到猜测值为定值。

\beforeverb
\begin{verbatim}
while True:
    print x
    y = (x + a/x) / 2
    if y == x:
        break
    x = y
\end{verbatim}
\afterverb
%

对于大多数的{\tt a},这个都适用。但是,一般说来,测试浮点数是否相等是危险地。浮点值
只是近似相等:大多数有理数,像$1/3$,和无理数,像$\sqrt{2}$,不能用一个浮点数
精确的表示。

\index{floating-point 浮点}
\index{epsilon }

与其检查{\tt x}和{\tt y}是否精确相等,不如安全的采用内置的函数{\tt abs}计算
差的绝对值:

\beforeverb
\begin{verbatim}
    if abs(y-x) < epsilon:
        break
\end{verbatim}
\afterverb
%
这里,\verb"epsilon"是一个极小值,像{\tt 0.0000001} ,表示了什么样的接近才是
非常接近了。

\begin{ex}
\label{square_root}
\index{encapsulation 封装}

把这个循环封装在函数\verb"square_root"里,接受{\tt a}为参数,选择一个合适的{\tt x},返回{\tt a}的近似平方根。
\end{ex}

\section{算法}
\index{algorithm 算法}

牛顿方法是算法的一个例子:它是解决一类问题的方法(在这个例子里是计算平方根)。

 定义一个算法可不是件容易的事。从不是算法的方面入手或许会有帮助。当你学习
 单位数相乘时,你背了乘法表。事实上,你记住了100个具体的解法。这种知识不是
 算法。\\

但是,如果你比较“懒”,你可能作些小弊。比如,计算$n$和9的乘积,你可以把十位写成$n-1$,个位写成$10-n$。这个就是解决任何单位数乘以9的一般方法\footnote{译注:比如
$8 * 9 = 72 = (8-1)(10-8)$}。对了,这就是算法。

\index{addition with carrying}
\index{carrying, addition with}
\index{subtraction!with borrowing}
\index{borrowing, subtraction with}

类似地,我们学到的技巧,比如加法进位,减法借位,长除法都是算法。这些算法的共有的
特点就是他们不需要任何的智力来实施。他们是机械的过程,每一步都依靠简单的规则。

从我的观点来看,人们花费大量的时间在学校里学习---不夸张地说,不需要任何智力的算法是非常不值得的。\\

从另一方面来说,设计算法的过程确实有趣的,挑战智力的,也是程序设计的中心内容。

有些事情,人们作起来很自然,没有困难也无须苦思冥想,但却是最难用算法表达的。理解自然语言是一个好的例子。我们都有这种能力,但是至今没有人能解释为什么我们可以,至少我们不能以算法的形式表达出来。

\section{调试}

当我们开始编写大型的程序时,就会发现调试将会花费我们大量的时间。代码越多,意味着
犯错的机会就越大,隐藏bug(s)的地方就越多。

\index{debugging!by bisection}
\index{bisection, debugging by}

减少调试时间的一种方法就是“二分法”。例如,程序有100行代码,每次检查一个,
需要100步。

换一种思路,把问题分成两半。查看程序的中间部分,或者接近中间部分,寻找一个
中间值来检查。添加一个{\tt print}语句(或者其他的能产生验证效果的语句),然后执行程序。

如果中间检查有问题,那么必定是程序的前半部分有问题。反之,则在第二部分。

每次按照这样检查的话,我们缩减了需要检查的代码。至少理论上来说,六步以后,
(这远远小于100),我们就可以缩减到一两行代码了。

实际中,很难界定程序的中间部分是哪里,也不总可能检查它。数代码的行数然后计算精确的中间点是没有任何意义的。相反,仔细想象什么地方最有可能出现错误,什么地方容易插入一个语句来检查。然后,选择一处你认为bug出现在检查点前后几率近似相等的地方。

\section{术语表}

\begin{description}

\item [multiple assignment 多重赋值:] 在程序的执行过程中给同一个变量赋多次值。
\index{mutiple assignement 多重赋值}
\index{assignment!multiple}

\item [update 更新:] 变量的新值依赖原来值的赋值。
\index{update 更新}

\item [initialization 初始化:]给一个将被更新的变量初始值的赋值。
\index{initialization!variable}

\item [increment 增量:] 增加一个变量的更新(通常是1)。
\index{increment}

\item [decrement 减量:]减小一个变量的更新(通常是1)。
\index{decrement}

\item [iteration 迭代:]使用递归或者循环的重复性执行的语句集合。
\index{iteration 迭代}

\item [infinite loop:无限循环]终止条件永远不满足的循环。
\index{infinite loop 无限循环}

\end{description}


\section{练习}

\begin{ex}

\index{algorithm!square root}

测试这章的平方根算法,你可以把结果和{\tt math.sqrt}的结果比较。写一个函数
\verb"test_square_root"打印一个如下的表:

\beforeverb
\begin{verbatim}
1.0 1.0           1.0           0.0
2.0 1.41421356237 1.41421356237 2.22044604925e-16
3.0 1.73205080757 1.73205080757 0.0
4.0 2.0           2.0           0.0
5.0 2.2360679775  2.2360679775  0.0
6.0 2.44948974278 2.44948974278 0.0
7.0 2.64575131106 2.64575131106 0.0
8.0 2.82842712475 2.82842712475 4.4408920985e-16
9.0 3.0           3.0           0.0

\end{verbatim}
\afterverb

第一列是数字$a$,第二列是用\ref{square_root}计算的$a$的平方根,第三列是用
{\tt math.sqrt}计算的平方根,第四列是两个结果的差值。
\end{ex}

\begin{ex}

\index{eval function eval函数}
\index{function!eval}

内置函数{\tt eval}接受一个字符串,然后调用python解释器计算字符串的值,例如:

\beforeverb
\begin{verbatim}
>>> eval('1 + 2 * 3')
7
>>> import math
>>> eval('math.sqrt(5)')
2.2360679774997898
>>> eval('type(math.pi)')
<type 'float'>
\end{verbatim}
\afterverb

编写一个函数\verb"eval_loop"重复的提示用户,接受用户输入,然后调用
{\tt eval}计算值,并打印结果。

程序必须知道用户舒服\verb"'done'"才结束循环,然后返回最后计算的
表达式的值。
\end{ex}

\begin{ex}

\index{Ramanujan, Srinivasa}

杰出的数学家Srinivasa Ramanujan 发现了无穷序列\footnote{参看\url{wikipedia.org/wiki/Pi}.}可以用来产生近似的$pi$值。

\index{pi}

\[\frac{1}{\pi} = \frac{2\sqrt{2}}{9801} 
\sum^\infty_{k=0} \frac{(4k)!(1103+26390k)}{(k!)^4 396^{4k}} \]

编写函数\verb"estimate_pi",函数使用上面的公式计算$pi$值,并返回之,
函数应该使用一个{\tt while}循环计算项的和知道最后一项小鱼{\tt le-15}
(Python里的记法为$10^{-15})$。你可以拿它和{\tt math.pi}比较一下。

可以参看我的解答\url{thinkpython.com/code/pi.py}。
\end{ex}








\chapter{字符串}


\chapter{实例学习:字符处理}

\section{读取单词表}
\label{wordlist}

我们本章的联系需要一个英语单词表。在网络上有数以万计的单词表,但是
最适合我们的一个是贡献给公共域的单词表,它是由Grady Ward搜集整理作为Moby词典工程的一部分\footnote{\url{wikipedia.org/wiki/Moby_Project}.}。
它由113,809个官方纵横组合字谜的单词组成,也就是在纵横组合字谜中和其他字谜游戏中存在的单词组成。在Moby的搜集中,文件名是{\tt 113089.fic};我
复制了这个文件,命名为{\tt words.txt},并把它包含在了Swampy里了。

\index{Swampy}
\index{crosswords 纵横组合字谜}


这个文件是纯文本文件,你可以用一个编辑器打开它,当然,你也可以用python
来读取它。内置函数{\tt open}接受一个文件名作为参数,返回一个文件对象,
用它可以读取文件。

\index{oepn function open函数}
\index{function!open}
\index{plain text 纯文本文件}
\index{text!plain}
\index{object!file}
\index{file object  文件对象}

\beforeverb
\begin{verbatim}
>>> fin = open('words.txt')
>>> print fin
<open file 'words.txt', mode 'r' at 0xb7f4b380>
\end{verbatim}
\afterverb

{\tt fin}是赋给用来输入的文件对象的常见名称。大开模式\verb"'r'"表明
文件以只读方式大开(和\verb"'w'"以只写模式打开相反)。

\index{redline method readline方法}
\index{methods!readline}

文件对象提供了多种方式来读取文件,其中就包括{\tt readline},该函数从
文件中读取字符知道遇到换行符,然后以字符串的形式返回结果:

\beforeverb
\begin{verbatim}
>>> fin.readline()
'aa\r\n'
\end{verbatim}
\afterverb
%

这个单词表的第一个单词是"aa,"是一种熔岩的名称。序列\verb"\r\n"代表两个
空白字符,回车和换行,它们把这个单词下一行分隔开来。

文件对象自己跟踪达到了文件的什么地方,所以当再次调用{\tt readline}函数时,就会得到下一个单词:

\beforeverb
\begin{verbatim}
>>> fin.readline()
'aah\r\n'
\end{verbatim}
\afterverb

下一个单词是"aah,"---绝对合法的一个单词,不要以那么怪的眼神看我\footnote{译者:我没有笑话你的意思哦}。如果那个whitespace看上去很不给力,我们
使用{\tt strip}函数剔除它:

\index{strip method strip方法}
\index{method!strip}

\beforeverb
\begin{verbatim}
>>> line = fin.readline()
>>> word = line.strip()
>>> print word
aahed
\end{verbatim}
\afterverb

也可以在{\tt for}循环中使用文件对象。下面的程序对去{\tt word.txt},打印每一个单词,一行一个:

\index{open function open函数}
\index{function!open}

\beforeverb
\begin{verbatim}
fin = open('words.txt')
for line in fin:
    word = line.strip()
    print word
\end{verbatim}
\afterverb

\begin{ex}
编写一个程序,读取{\tt words.txt},打印包含超过20个字符的单词(不包括空白)。

\index{whitespace 空格}

\end{ex}

\section{练习}

下个部分有这些练习的答案。你至少得在参看答案之前尝试做一做每一道题。

\begin{ex}

1939年,Ernest Vincent Wright发表了一本50,000字的小说{\em Gadsby},在这本书中
不包含字母“e“。"e"在英语中是非常常见的字母,所以这件事很难做到。

事实上,如果不使用最常见的符号是很难想象那样的情况的。开始进展比较缓慢,但是
谨慎地训练几个小时之后,你就会逐步的掌握要领。

好的,进入正题!

编写一个函数\verb"has_no_e",如果给定的单词不含有字母"e",返回{\tt True}。

修改上一部分的程序,打印不含有"e"的单词,并计算不含有"e"的单词的百分比。

\index{lipogram 避讳}

\end{ex}

\begin{ex}

编写一个函数{\tt avoids}接受一个单词和一串“禁止”字母,如果单词没有使用禁止
字母中的任何一个,返回{\tt True}

修改你的程序提示用户输入一串禁止字母,然后打印不含有它们中任意一个的单词数目。你能找出由5个“禁止”字母组成的单词,不包括最小数目的单词吗?

\end{ex}

\begin{ex}
编写一个函数\verb"uses_only",接受一个单词和一串字母,如果单词仅仅包含列表
字符串里的字母,返回{\tt True}。除了"Hoe alfalfa"你能用{\tt acefhlo}里的字母
造一个句子吗?
\end{ex}

\begin{ex}

编写一个函数\verb"uses_all",接受一个单词和一串字母,如果单词使用一串字母里
的所有字母(至少一次),返回{\tt True}。有多少单词包含了全部的元音{\tt aeiou}?{\tt aeiouy}呢?
\end{ex}


\begin{ex}
编写一个函数\verb"is_abecedarian",如果单词里的字母以字典顺序出现,返回{\tt True}。有多少{\tt abecedarian}式的单词?
\end{ex}

\index{abecedarian}

\section{搜索}

\index{search pattern 搜索模式}
\index{pattern!search}






前一部分的所有联系都有一个共同点:他们都可以通过一个可搜索模式来解决,我们
在\ref{find}部分遇到过。最简单的例子是:

\beforeverb
\begin{verbatim}
def has_no_e(word):
    for letter in word:
        if letter == 'e':
            return False
    return True
\end{verbatim}
\afterverb

{\tt for}循环遍历{\tt word}中的每一个字母;如果不匹配,就进入下一个字母。
如果我们正常退出循环,就意味着我们没有找到“e“,于是返回{\tt True}。

\index{traversal 遍历}

\index{generalization}

{\tt avoids}是一个更一般的\verb"has_no_e"版本,但是有着相同的结构:

\beforeverb
\begin{verbatim}
def avoids(word, forbidden):
    for letter in word:
        if letter in forbidden:
            return False
    return True
\end{verbatim}
\afterverb

如果发现一个“禁止”字母,我们就返回{\tt False};如果我们达到了循环的结尾,
就返回{\tt True}。

\verb"uses_only"和它类似,出了条件的意思是相反的。

\beforeverb
\begin{verbatim}
def uses_only(word, available):
    for letter in word: 
        if letter not in available:
            return False
    return True
\end{verbatim}
\afterverb

除了使用一串“禁止”字母,我们也可以使用可获得字母(avaliable)。如果我们在、
{\tt word}中发现一个字母不再可获得字母中,就返回{\tt False}。

\verb"uses_all"函数也类似,出了我们掉换了word 和一串字母的角色。
\beforeverb
\begin{verbatim}
def uses_all(word, required):
    for letter in required: 
        if letter not in word:
            return False
    return True
\end{verbatim}
\afterverb

我们没有遍历{\tt word}中的字母,循环遍历了“要求“的字母,如果任意一个“要求”
字母没有出现在{\tt word}中,我们返回{\tt False}。

\index{traversal 遍历}

如果你想像一个计算机科学家一样思考,你可能已经认出\verb"uses_all"是前面已经
解决过的问题的实例,你就会编写:

\beforeverb
\begin{verbatim}
def uses_all(word, required):
    return uses_only(required, word)
\end{verbatim}
\afterverb

这是程序设计方法中的一个实例---叫做问题识别,含义是你认识到现在解决的问题
是已经解决问题的一个实例,然后修改应用以前的解法。

\index{problem recognition 问题识别}
\index{development plan!problem recognition}

\section{使用索引循环}

\index{looping!with indices}
\index{index!looping with}

我用{\tt for}循环编写以前的函数,因为我只需要用到字符串里的字符;我没有必要
使用索引来解决问题。

对于\verb"is_abecedarian"我们必须要比较相邻的字母,{\tt for}循环就有点力不
从心了。




\beforeverb
\begin{verbatim}
def is_abecedarian(word):
    previous = word[0]
    for c in word:
        if c < previous:
            return False
        previous = c
    return True
\end{verbatim}
\afterverb

另外一种方法就是使用递归:

\beforeverb
\begin{verbatim}
def is_abecedarian(word):
    if len(word) <= 1:
        return True
    if word[0] > word[1]:
        return False
    return is_abecedarian(word[1:])
\end{verbatim}
\afterverb

也还可以使用{\tt while}循环:

\beforeverb
\begin{verbatim}
def is_abecedarian(word):
    i = 0
    while i < len(word)-1:
        if word[i+1] < word[i]:
            return False
        i = i+1
    return True
\end{verbatim}
\afterverb


循环从{\tt i = 0}开始,以{\tt i = len(word) -1}收尾。每次循环时,比较第$i$
(你可以把它当成当前的字符)和$i+1$字符(你可以把它当作第二个字符)。

如果下一个字符小于当前的字符(字典顺序是在前),字母次序就被打断了,我们
返回{\tt False}。

如果我们到达了循环的末尾并且没有发现错误,这个{\tt word}“通过“了测试。
为了让自己信服循环正确的结束了,考虑这样一个例子\verb"'flossy'"。这个单词
的长度是6,所以最后一次循环执行的时候{\tt i} 是4---正好是倒数第二个字符的索引。
在最后一次迭代中,程序比较倒数第二个字符和最后一个字符---这就是我们希望的~。

\index{palindrome 回文}

下面是一个\verb"is_palindrome"的版本(参看练习\ref{回文}) ,函数使用了两个
索引;一个从头部开始,逐步递增,另外一个从尾部开始,逐步递减。


\beforeverb
\begin{verbatim}
def is_palindrome(word):
    i = 0
    j = len(word)-1

    while i<j:
        if word[i] != word[j]:
            return False
        i = i+1
        j = j-1

    return True
\end{verbatim}
\afterverb

或者,你也注意到了这又是已解决问题的一个实例,你或许会这么写:

\beforeverb
\begin{verbatim}
def is_palindrome(word):
    return is_reverse(word, word)
\end{verbatim}
\afterverb

\index{problem recognition 问题识别}
\index{development plan!problem recognition}

我认为你做了练习\ref{is_reverse}。

\section{调试}

\index{debugging 调试}
\index{testing!is hard}
\index{program testing 程序测试}

测试程序是一种“折磨”。本章的函数相对来说比较容易测试,因为你可以手动的检查
输出结果。尽管如此,某些地方还是很难甚至不可能选择一个合适的单词表来测试
所有可能的错误。\\


拿\verb"has_no_e"做例子,有两种明显的情况需要测试:含有"e"的单词应该返回{\tt False};不含有的返回{\tt True}。对于这两种情况,应该都没有任何错误。\\


在这两中情况下又有一些不太明显的子情况。在含有“e“的单词中,我们应该测试
"e"在单词前,在单词尾,在中间的某个地方。也应该测试长单词,短单词,和非常
短的单词,像空字符串。空字符串是特殊情况中的一个例子,也是非常不明显的情况,而错误恰恰经常藏身此处。

\index{special case 特别情况}

除了测试你自己想出的情况,你也可以用单词表来测试,比如使用{\tt word.txt}。
通过查看输出,你就可以抓住错误,但是小心:你或许能抓住一种错误(本不该包括
的单词被包括了)但不是另外一种(本该包含的单词没有被包含)。

一般说来,测试能帮助我们发现bugs,但是产生好的测试案例不是一件容易的事,
而且即使你测试了,你也不能保证你的程序就是正确的。

\index{testing!and absence of bugs}

一位传奇式的计算机科学家这么说过:

\begin{quote}
程序测试能够发现bugs是存在的,但是永远发现不了bugs不存在。

---Edsger W.Dijkstra
\end{quote}

\index{Dijkstra, Edsger}

\section{术语表}

\begin{description}

\item [file object 文件对象:] 代表打开文件的数值。
\index{file object 文件对象}
\index{object!file}

\item [problem recognition 问题识别:]通过把问题阐释成已经解决问题的一个
实例来解决问题的方法。
\index{problem recognition 问题识别}

\item [special case 特殊情况:]非典型的或不明显的测试例子(很难完美解决)。
\index{special case 特殊情况}

\end{description}

\section{练习}

\begin{ex}


\index{Car Talk 汽车谈话}
\index{Puzzler 难题}
\index{double letters }

这个问题是基于广播节目{\em Car Talk}播发的一个难题\footnote{\url{www.cartalk.com/content/puzzler/transcripts/200725}.}:

\begin{quote}

 给我一个含有三个连续的双字母单词,我能给你几个几乎符合条件的单词,但是不是
 完全符合。比如,单词{\tt committee},c-o-m-m-i-t-t-e-e。如果没有了$i$在那里,会更完美;Mississippi也是:M-i-s-s-i-s-s-i-p-p-i。如果能够去掉这些$i$,
 结果就很完美。有一个单词有三个连续的双字母,据我所知,只有这一个。当然,
 也有可能有500多个,但是我只能想到这一个。那么这个单词是什么?
 \end{quote}

 编写一个程序,寻找它。你可以参看我的答案\url{thinkpython.com/code/cartalk.py}.

\end{ex}



\begin{ex}
下面的也是一个{\em Car Talk}难题\footnote{\url{www.cartalk.com/content/puzzler/transcripts/200803}.}:

\index{Car Talk}
\index{Puzzler 难题}
\index{odometer 里程表}
\index{palindrome 回文}


\begin{quote}
“有一天,我驾车在高速公路上行使,我偶然看到了我的里程表。像大多数的里程表
一样,它显示了六个数字,全部是里数。比如,如果我的小汽车行使额300,000里,
我将会看到3-0-0-0-0-0。\\

“现在,我那天看到的却是非常的有意思。我注意到后四个数字是回文的;也就是说
从前往后读和从后往前读是一样的。比如,5-4-4-5是回文,所以我的里程表可能显示3-1-5-4-4-5。\\

“行使了一里之后,后五位数是回文的了。比如,可能是3-6-5-4-5-6。再行使一里后
中间的四位又是回文的了。你或许猜到了,一公里后,六位数字是回文的了!\\

“问题是,当我第一次看里程表的时候,显示的是什么?”
\end{quote}

编写一个Python程序,测试所有的六位数,打印任何满足条件的数字。你可以参看我的答案\url{thinkpython.com/code/cartalk.py}.

\end{ex}

\begin{ex}

下面的又是一个{\em Car Talk}难题,你可以用搜索来解决它\footnote{\url{www.cartalk.com/content/puzzler/transcripts/200813}}:

\index{Car Talk}
\index{Puzzler  难题}
\index{palindrome 回文}

\begin{quote}
“最近,我拜访了妈妈。我们认识到组成我年龄的两位数倒过来时是她的年龄。
比如,如果她是73,我就是37。我们想知道这种事发生的频率是多少?但是我们
转移到了另外一个话题,我们也就没有得到答案。\\

“当我会到家的时候,我计算出组成我们年龄的两位数已经有六次是像上面那样
了。我也计算出,如果幸运的话,几年后又会发生了,如果我们真的幸运的话,
在那之后,还会一次。也就是说,总共会有8次。现在的问题是,我今年多大了?”

\end{quote}

编写一个Python程序,搜索这个难题的答案。Hint:你或许会发现字符串方法{\tt zfill}对你有些帮助。

可以参看我的答案\url{thinkpython.com/code/cartalk.py}.

\end{ex}






























\chapter{列表}

\index{列表}
\index{类型!列表}


\section{列表是一个序列}

类似字符串,{\bf 列表}一个序列的值。在字符串中,每个值是字符;在一个列表中可以是任何数据类型。列表中的数值称为{\bf 元素},有时也称为{\bf 项目}。

\index{元素}
\index{序列}
\index{项目}

有多种方法可以创建一个新的列表;最简单的方法是用方括号(\verb"["和\verb"]")将元素包括起来:

\beforeverb
\begin{verbatim}
[10, 20, 30, 40]
['crunchy frog', 'ram bladder', 'lark vomit']
\end{verbatim}
\afterverb
%
第一个例子是包含4个整数的列表。第二个例子是包含3个字符串的列表。一个列表中的元素不需要是相同数据类型。下面的列表包含一个字符串、一个浮点数、一个整数和另一个列表:

\beforeverb
\begin{verbatim}
['spam', 2.0, 5, [10, 20]]
\end{verbatim}
\afterverb
%
在一个列表中的列表称为{\bf 嵌套}。

\index{嵌套列表}
\index{列表!嵌套}

没有任何元素的列表称为空列表。你可以使用空的括号\verb"[]"创建一个空列表。

\index{空列表}
\index{列表!空}

正如你可能期望的,列表的值可以被赋值给变量:

\beforeverb
\begin{verbatim}
>>> cheeses = ['Cheddar', 'Edam', 'Gouda']
>>> numbers = [17, 123]
>>> empty = []
>>> print cheeses, numbers, empty
['Cheddar', 'Edam', 'Gouda'] [17, 123] []
\end{verbatim}
\afterverb
%

\index{赋值}

% From Jeff: write sum for a nested list?


\section{列表是可改变的}

\index{列表!元素}
\index{访问}
\index{下标}
\index{括号运算符}
\index{运算符!括号}

访问列表中元素的语法和访问字符串中字符的语法相同,都是通过括号运算符实现的。括号中的表达式指定了下标。记住下标从0开始:

\beforeverb
\begin{verbatim}
>>> print cheeses[0]
Cheddar
\end{verbatim}
\afterverb
%
与字符串不同,列表是可以改变的。当括号运算符出现在赋值语句的左边,它指向列表中将被赋值的元素。

\index{可改变的}

\beforeverb
\begin{verbatim}
>>> numbers = [17, 123]
>>> numbers[1] = 5
>>> print numbers
[17, 5]
\end{verbatim}
\afterverb
%
{\tt numbers}中的第一个元素,原来是123,现在是5。
used to be 123, is now 5.

\index{下标!从0开始}
\index{0,下标开始于}

你可以将列表看成下标和元素的对应关系。这种关系成为{\bf 映射}。每个下标“对应”一个元素。这里给出{\tt cheeses},{\tt numbers}和{\tt empty}的状态图:

\index{状态图}
\index{图!状态}
\index{映射}

\beforefig
\centerline{\includegraphics{figs/list_state.eps}}
\afterfig

列表用外部标有“list”的盒子表示,内部是列表中的元素。{\tt cheeses}是一个有3个元素的列表,下标分别是0,1和2。{\tt numbers}包含2个元素。状态图显示了第二个元素原来是123,被重新赋值为5。{\tt empty}对应一个没有元素的列表。

\index{项目赋值}
\index{赋值!项目}

列表下标的工作原理和字符串的相同:

\begin{itemize}

\item 任何整数表达式可以作为下标。

\item 试图读写一个不存在的元素将得到{\tt 下标错误}。

\index{异常!下标错误}
\index{下标错误}

\item 下标可以取负数,它将从后往前访问列表。

\end{itemize}

\index{列表!下标}


\index{列表!成员}
\index{成员!列表}
\index{in运算符}
\index{运算符!in}

{\tt in}运算符同样使用列表。

\beforeverb
\begin{verbatim}
>>> cheeses = ['Cheddar', 'Edam', 'Gouda']
>>> 'Edam' in cheeses
True
>>> 'Brie' in cheeses
False
\end{verbatim}
\afterverb


\section{遍历列表}
\index{列表!遍历}
\index{遍历!列表}
\index{for循环}
\index{循环!for}
\index{语句!for}

最常用的遍历列表的方式是使用{\tt for}循环。语法类似字符串:

\beforeverb
\begin{verbatim}
for cheese in cheeses:
    print cheese
\end{verbatim}
\afterverb
%
这种写法适用于只读列表中的元素。如果你需要写或者更新元素,你需要通过下标访问。一个常用的做法是结合{\tt range}和{\tt len}函数:

\index{循环!下标}
\index{下标!循环}

\beforeverb
\begin{verbatim}
for i in range(len(numbers)):
    numbers[i] = numbers[i] * 2
\end{verbatim}
\afterverb
%
这个循环遍历列表并更新每个元素。{\tt len}函数返回列表中元素个数。{\tt range}函数返回一个从0到$n-1$的下标的列表,其中$n$是列表的长度。每次循环中,{\tt i}得到下一个元素的下标。循环主体中的赋值语句使用{\tt i}读取老的值并赋值新的值。

\index{项目更新}
\index{更新!项目}

对于一个空列表的{\tt for}循环将不会执行循环的主体:

\beforeverb
\begin{verbatim}
for x in empty:
    print 'This never happens.'
\end{verbatim}
\afterverb
%
虽然一个列表可以包含另一个列表,被嵌套的列表作为单独的一个元素。以下列表的长度为4:

\index{嵌套列表}
\index{列表!嵌套}

\beforeverb
\begin{verbatim}
['spam', 1, ['Brie', 'Roquefort', 'Pol le Veq'], [1, 2, 3]]
\end{verbatim}
\afterverb



\section{列表操作}
\index{列表!操作}

运算符{\tt +}连接列表:

\index{连接!列表}
\index{列表!连接}

\beforeverb
\begin{verbatim}
>>> a = [1, 2, 3]
>>> b = [4, 5, 6]
>>> c = a + b
>>> print c
[1, 2, 3, 4, 5, 6]
\end{verbatim}
\afterverb
%
类似的,运算符{\tt *}给定次数地重复列表:

\index{重复!列表}
\index{列表!重复}

\beforeverb
\begin{verbatim}
>>> [0] * 4
[0, 0, 0, 0]
>>> [1, 2, 3] * 3
[1, 2, 3, 1, 2, 3, 1, 2, 3]
\end{verbatim}
\afterverb
%
第一个例子重复{\tt [0]}4次。第二个例子重复列表{\tt [1, 2, 3]}3次。


\section{列表切片}

\index{切片运算符}
\index{运算符!切片}
\index{下标!切片}
\index{列表!切片}
\index{切片!列表}

切片运算符同样适用于列表:

\beforeverb
\begin{verbatim}
>>> t = ['a', 'b', 'c', 'd', 'e', 'f']
>>> t[1:3]
['b', 'c']
>>> t[:4]
['a', 'b', 'c', 'd']
>>> t[3:]
['d', 'e', 'f']
\end{verbatim}
\afterverb
%
如果你忽略第一个下标,切片从列表头开始。如果你忽略第二个,切片到列表尾部结束。因此如果你忽略两个,切片为整个列表的拷贝。

\index{列表!复制}
\index{切片!复制}
\index{复制!切片}

\beforeverb
\begin{verbatim}
>>> t[:]
['a', 'b', 'c', 'd', 'e', 'f']
\end{verbatim}
\afterverb
%
由于列表是可改变的,有必要在折叠、旋转或切断操作前复制列表。

\index{可改变}

赋值语句左边的切片运算符可以更新多个元素:

\index{slice!update}
\index{update!slice}

\beforeverb
\begin{verbatim}
>>> t = ['a', 'b', 'c', 'd', 'e', 'f']
>>> t[1:3] = ['x', 'y']
>>> print t
['a', 'x', 'y', 'd', 'e', 'f']
\end{verbatim}
\afterverb
%

% You can add elements to a list by squeezing them into an empty
% slice:

% \beforeverb
% \begin{verbatim}
% >>> t = ['a', 'd', 'e', 'f']
% >>> t[1:1] = ['b', 'c']
% >>> print t
% ['a', 'b', 'c', 'd', 'e', 'f']
% \end{verbatim}
% \afterverb
%
% And you can remove elements from a list by assigning the empty list to
% them:

% \beforeverb
% \begin{verbatim}
% >>> t = ['a', 'b', 'c', 'd', 'e', 'f']
% >>> t[1:3] = []
% >>> print t
% ['a', 'd', 'e', 'f']
% \end{verbatim}
% \afterverb
%
% But both of those operations can be expressed more clearly
% with list methods.


\section{列表方法}

\index{列表!方法}
\index{方法,列表}

Python提供了一个列表的方法。例如,{\tt append}方法将新的元素添加到列表尾部:

\index{append方法}
\index{方法!append}

\beforeverb
\begin{verbatim}
>>> t = ['a', 'b', 'c']
>>> t.append('d')
>>> print t
['a', 'b', 'c', 'd']
\end{verbatim}
\afterverb
%
{\tt extend}方法读取一个列表作为参数,并附加其中所有的元素:

\index{extend方法}
\index{方法!extend}

\beforeverb
\begin{verbatim}
>>> t1 = ['a', 'b', 'c']
>>> t2 = ['d', 'e']
>>> t1.extend(t2)
>>> print t1
['a', 'b', 'c', 'd', 'e']
\end{verbatim}
\afterverb
%
这个例子中{\tt t2}没有改变。

{\tt sort}方法从小到大对列表中的元素进行排序:

\index{sort方法}
\index{方法!sort}

\beforeverb
\begin{verbatim}
>>> t = ['d', 'c', 'e', 'b', 'a']
>>> t.sort()
>>> print t
['a', 'b', 'c', 'd', 'e']
\end{verbatim}
\afterverb
%
列表的方法都是空的,它们对列表进行修改并返回{\tt None}。如果你写了{\tt t = t.sort()},你不会得到你想要的结果。

\index{空方法}
\index{方法!空}
\index{特殊值None}
\index{特殊值!None}


\section{映射,筛选和归并}

对列表中所有元素求和,你可以这么使用循环:

% see add.py

\beforeverb
\begin{verbatim}
def add_all(t):
    total = 0
    for x in t:
        total += x
    return total
\end{verbatim}
\afterverb
%
{\tt total}被初始化为0。每次经过循环,{\tt x}从列表中读取一个元素。运算符{\tt +=}提供可一个快捷的更新变量的方法。这是{\bf 增量赋值语句}:

\index{更新运算符}
\index{运算符!更新}

\index{赋值!增量}
\index{增量赋值}

\beforeverb
\begin{verbatim}
    total += x
\end{verbatim}
\afterverb
%
相当于:

\beforeverb
\begin{verbatim}
    total = total + x
\end{verbatim}
\afterverb
%
当循环执行时,{\tt total}记录了元素的和。这样的变量称为{\bf 累加器}。

\index{累加器!和}

对列表中元素求和是一个普通的操作,Python提供了内建函数{\tt sum}:

\beforeverb
\begin{verbatim}
>>> t = [1, 2, 3]
>>> sum(t)
6
\end{verbatim}
\afterverb
%
类似将一个序列的元素合并到一个单一的数值的操作称为{\bf 归并}。

\index{归并模式}
\index{模式!归并}
\index{遍历}


有时你需要遍历一个列表来构建另一个列表。例如,下面的函数读取一个字符串列表作为参数,返回大写后的新列表:

\beforeverb
\begin{verbatim}
def capitalize_all(t):
    res = []
    for s in t:
        res.append(s.capitalize())
    return res
\end{verbatim}
\afterverb
%
{\tt res}被初始化为一个空的列表。每次循环中,我们附加下一个元素。因此{\tt res}是另一种累加器。

\index{累加器!列表}
类似\verb"capitalize_all"有时被称为{\bf 映射},因为它对序列中的每个元素“映射”某个函数(在本例中是方法{\tt capitalize})。

\index{映射模式}
\index{模式!映射}
\index{筛选模式}
\index{模式!筛选}

另一个常见的操作是从列表中选择一些元素,并返回一个子列表。例如,下面的函数读取一个字符串列表,并返回仅包含大写字符串的列表:

\beforeverb
\begin{verbatim}
def only_upper(t):
    res = []
    for s in t:
        if s.isupper():
            res.append(s)
    return res
\end{verbatim}
\afterverb
%
{\tt isupper}是一个字符串的方法,如果字符串仅含有大写字母,则返回{\tt True}。

类似\verb"only_upper"的操作被称为{\bf 筛选},它选出一部分元素,而过滤其他元素。

绝大多数列表操作可以表示为映射、筛选和归并的组合。由于这些操作很常用,Python提供了语言特点的支持,包括内建函数{\tt map}和一个称为“列表理解”的运算符。

\index{列表!理解}

\begin{ex}
\label{累积}
\index{累积求和}

编写函数,读取一个数字列表作为参数,返回累积求和值。即第$i$个元素是原列表中前$i+1$个元素的和。例如,{\tt [1, 2, 3]}的累积求和为{\tt [1, 3, 6]}。
\end{ex}


\section{删除元素}

\index{元素删除}
\index{删除,列表中的元素}

有多种方法从列表中删除一个元素。如果你知道元素的下标,你可以使用{\tt pop}:

\index{pop方法}
\index{方法!pop}

\beforeverb
\begin{verbatim}
>>> t = ['a', 'b', 'c']
>>> x = t.pop(1)
>>> print t
['a', 'c']
>>> print x
b
\end{verbatim}
\afterverb
%
{\tt pop}修改列表,并返回被删除的元素。如果你不提供下标,它将删除最后一个元素。

如果你不需要被删除的值,你可以使用{\tt del}运算符:

\index{del运算符}
\index{运算符!del}

\beforeverb
\begin{verbatim}
>>> t = ['a', 'b', 'c']
>>> del t[1]
>>> print t
['a', 'c']
\end{verbatim}
\afterverb
%
如果你知道你要删除的元素(但不知道下标),你可以使用{\tt remove}:

\index{remove方法}
\index{方法!remove}

\beforeverb
\begin{verbatim}
>>> t = ['a', 'b', 'c']
>>> t.remove('b')
>>> print t
['a', 'c']
\end{verbatim}
\afterverb
%
{\tt remove}的返回值是{\tt None}。

\index{特殊值None}
\index{特殊值!None}

要删除多于一个元素,你可以对{\tt del}使用切片下标:

\beforeverb
\begin{verbatim}
>>> t = ['a', 'b', 'c', 'd', 'e', 'f']
>>> del t[1:5]
>>> print t
['a', 'f']
\end{verbatim}
\afterverb
%
照常,切片选择从第一个下标到第二个下标(不包括第二个下标)中的所有元素。


\section{列表和字符串}

\index{列表}
\index{字符串}
\index{序列}

字符串是字符的序列,列表是值的序列,但是字符的列表不同于字符串。你可以使用{\tt list}将字符串转化为字符的列表:

\index{list!函数}
\index{函数!list}

\beforeverb
\begin{verbatim}
>>> s = 'spam'
>>> t = list(s)
>>> print t
['s', 'p', 'a', 'm']
\end{verbatim}
\afterverb
%
由于{\tt list}是内建函数名,你应该避免将它作为变量名。我同样避免使用{\tt l}因为它看起来像{\tt 1}。于是我用了{\tt t}。

{\tt list}函数将字符串分割成单独的字符。如果你要将字符串分割成单词,你可以使用{\tt split}方法:

\index{split方法}
\index{方法!split}

\beforeverb
\begin{verbatim}
>>> s = 'pining for the fjords'
>>> t = s.split()
>>> print t
['pining', 'for', 'the', 'fjords']
\end{verbatim}
\afterverb
%
一个可选的参数称为{\bf 分割符},它指定了什么字符作为分界线。下面的例子使用连字符作为分割符:

\index{可选参数}
\index{参数!可选}
\index{分割服}

\beforeverb
\begin{verbatim}
>>> s = 'spam-spam-spam'
>>> delimiter = '-'
>>> s.split(delimiter)
['spam', 'spam', 'spam']
\end{verbatim}
\afterverb
%
{\tt join}功能和{\tt split}相反。它将一个列表字符串连接起来。{\tt join}是一个字符串方法,因此你需要在一个分割符上调用它,并传递一个列表作为参数:

\index{join方法}
\index{方法!join}
\index{连接}

\beforeverb
\begin{verbatim}
>>> t = ['pining', 'for', 'the', 'fjords']
>>> delimiter = ' '
>>> delimiter.join(t)
'pining for the fjords'
\end{verbatim}
\afterverb
%
在这个例子中分割符是一个空格,{\tt join}在每个单词中间添加一个空格。如果不需要使用空格连接,你可以使用一个空的字符串\verb"''"作为分割符。

\index{空字符串}
\index{字符串!空}


\section{对象和值}

\index{对象}
\index{值}

如果我们执行以下的赋值语句:

\beforeverb
\begin{verbatim}
a = 'banana'
b = 'banana'
\end{verbatim}
\afterverb
%
我们知道{\tt a}和{\tt b}都指向一个字符串,但我们不知道他们是否指向一个{\em 相同的}字符串。有些两种可能:

\index{别名}

\beforefig
\centerline{\includegraphics{figs/list1.eps}}
\afterfig

在第一种情况中,{\tt a}和{\tt b}指向两个有相同值的不同对象。在第二种情况中,它们指向同一个对象。

\index{is运算符}
\index{运算符!is}

你可以使用{\tt is}运算符检查两个变量是否指向同一个对象:

\beforeverb
\begin{verbatim}
>>> a = 'banana'
>>> b = 'banana'
>>> a is b
True
\end{verbatim}
\afterverb
%
在这个例子中,Python只产生了一个字符串对象,{\tt a}和{\tt b}都指向它。

但是如果你创建两个列表,你得到两个对象:

\beforeverb
\begin{verbatim}
>>> a = [1, 2, 3]
>>> b = [1, 2, 3]
>>> a is b
False
\end{verbatim}
\afterverb
%
状态图看起来是这样的:

\index{状态图}
\index{图!状态}

\beforefig
\centerline{\includegraphics{figs/list2.eps}}
\afterfig

在本例中,我们称这两个列表是{\bf 相等的},因为它们有相同的元素,但不是{\bf 相同的},因为它们不是同一个对象。如果两个对象是相同的,它们也是相等的,但是如果它们是相等的,它们不一定相同。

\index{相等}
\index{相同}

目前为止,我们可以交换地使用“对象”和“值”,但更精确地说是对象包含一个值。如果你执行{\tt [1,2,3]},你会得到一个整数序列的对象。如果另一个列表有相同的元素,我们称它有相同的值,但它不是相同的对象。

\index{对象}
\index{值}


\section{别名}

\index{别名}
\index{引用!别名}

如果{\tt a}指向一个对象,然后你进行赋值{\tt b = a},那么两个变量都指向同一个对象:

\beforeverb
\begin{verbatim}
>>> a = [1, 2, 3]
>>> b = a
>>> b is a
True
\end{verbatim}
\afterverb
%
状态图如图所示:

\index{状态图}
\index{图!状态}

\beforefig
\centerline{\includegraphics{figs/list3.eps}}
\afterfig

一个变量和一个对象的关联称为{\bf 引用}。在这个例子中,同一个对象有两个引用。

\index{引用}

如果一个对象有多于一个引用,我们称这个对象是有{\bf 别名}的。

\index{可改变}
有别名的对象是可改变的,对一个别名的改动会影响另一个:

\beforeverb
\begin{verbatim}
>>> b[0] = 17
>>> print a
[17, 2, 3]
\end{verbatim}
\afterverb
%
这个行为虽然很有用,但容易造成错误。通常,对于可改变的对象避免使用别名相对更安全。

\index{不可改变}

对于不可改变的对象,如字符串,使用别名没有任何问题。例如:

\beforeverb
\begin{verbatim}
a = 'banana'
b = 'banana'
\end{verbatim}
\afterverb
%
使用{\tt a}或者{\tt b}指向同一个字符串基本上没有任何区别。


\section{列表参数}

\index{列表!作为参数}
\index{参数}
\index{参数!列表}
\index{引用}
\index{参数}

当你将一个列表作为参数传给一个函数,函数将得到这个列表的一个引用。如果函数对这个列表参数进行了修改,调用者会看见变动。例如,\verb"delete_head"删除列表的第一个元素:

\beforeverb
\begin{verbatim}
def delete_head(t):
    del t[0]
\end{verbatim}
\afterverb
%
它是这么使用的:

\beforeverb
\begin{verbatim}
>>> letters = ['a', 'b', 'c']
>>> delete_head(letters)
>>> print letters
['b', 'c']
\end{verbatim}
\afterverb
%
参数{\tt t}和变量{\tt letters}是同一个对象的别名。栈图如下:

\index{栈图}
\index{图!栈}

\beforefig
\centerline{\includegraphics{figs/stack5.eps}}
\afterfig

由于列表被两个帧共享,我把它画在它们中间。

需要注意的是修改列表操作和创建列表操作间的区别,例如,{\tt append}方法是修改一个列表,而{\tt +}运算符是创建一个新的列表:

\index{append方法}
\index{方法!append}
\index{列表!连接}
\index{连接!列表}

\beforeverb
\begin{verbatim}
>>> t1 = [1, 2]
>>> t2 = t1.append(3)
>>> print t1
[1, 2, 3]
>>> print t2
None

>>> t3 = t1 + [3]
>>> print t3
[1, 2, 3]
>>> t2 is t3
False
\end{verbatim}
\afterverb

如果你要编写函数修改列表,这个区别就很重要。例如,下面函数{\em 没有}删除列表的第一个元素:

\beforeverb
\begin{verbatim}
def bad_delete_head(t):
    t = t[1:]              # WRONG!
\end{verbatim}
\afterverb
切片操作创建一个新的列表,并使{\tt t}指向它。但这些操作对作为参数的列表都没有影响。

\index{切片运算符}
\index{运算符!切片}

一个替代的写法是创建并返回一个新的列表。例如,{\tt tail}返回不包含第一个元素的列表:

\beforeverb
\begin{verbatim}
def tail(t):
    return t[1:]
\end{verbatim}
\afterverb
%
这个函数并不修改原来的列表。下面给出如何使用这个函数:

\beforeverb
\begin{verbatim}
>>> letters = ['a', 'b', 'c']
>>> rest = tail(letters)
>>> print rest
['b', 'c']
\end{verbatim}
\afterverb


\begin{ex}

编写函数{\tt chop},读取一个列表并进行修改,删除第一个和最后一个元素,并返回{\tt None}。

再编写函数{\tt middle},读取一个列表作为参数,返回一个包含除了第一个和最后一个元素的新列表。

\end{ex}


\section{调试}
\index{调试}

粗心的使用列表(以及其他可改变的对象)会导致长时间的调试。下面给出一些常见的陷阱以及避免它们的方法:

\begin{enumerate}

\item 记住大多数的列表的方法对参数进行修改,然后返回{\tt None}。字符串的方法则相反,它们保留原始的字符串并返回一个新的字符串。

如果你习惯编写处理字符串的代码,如:

\beforeverb
\begin{verbatim}
word = word.strip()
\end{verbatim}
\afterverb

你可能会写出下面的代码:

\beforeverb
\begin{verbatim}
t = t.sort()           # WRONG!
\end{verbatim}
\afterverb

\index{sort方法}
\index{方法!sort}

由于{\tt sort}返回{\tt None},你对{\tt t}的下一个操作可能会失败。

在使用列表的方法和运算符前,你应该仔细阅读文档,并在交互模式下测试。列表和其他序列(如字符串)共有的方法和运算符的文档在\url{docs.python.org/lib/typesseq.html}。可改变的序列独有的方法和运算符的文档在\url{docs.python.org/lib/typesseq-mutable.html}。


\item 养成自己的编码习惯。

列表中的一个问题是有太多的途径做相同的事。例如,要删除列表中的一个元素,你可以使用{\tt pop},{\tt remove},{\tt del}甚至切片赋值。

要添加一个元素,你可以使用{\tt append}方法或{\tt +}运算符。但记住什么是正确的:

\beforeverb
\begin{verbatim}
t.append(x)
t = t + [x]
\end{verbatim}
\afterverb

以下是错误的:

\beforeverb
\begin{verbatim}
t.append([x])          # WRONG!
t = t.append(x)        # WRONG!
t + [x]                # WRONG!
t = t + x              # WRONG!
\end{verbatim}
\afterverb

在交互模式下测试每一个例子,保证你明白它们做了什么。注意只有最后一个会导致运行时错误,其他的都是合法的,但做了错误的事情。


\item 复制拷贝,避免别名。

\index{别名!复制以避免}
\index{赋值!以避免别名}

如果你要使用类似{\tt sort}的方法来修改参数,但同时有要保留原列表,你可以复制一个拷贝。

\beforeverb
\begin{verbatim}
orig = t[:]
t.sort()
\end{verbatim}
\afterverb

在这个例子中你还可以使用内建函数{\tt sorted},它将返回一个新的已排序的列表,原列表将保持不变。注意你需要避免使用{\tt sorted}作为变量名!

\end{enumerate}



\section{术语}

\begin{description}

\item[列表:] 一个序列的值。
\index{列表}

\item[元素:] 列表(或序列)中的一个值,也称为项目。
\index{元素}

\item[下标:] 对应列表中的元素的整数。
\index{下标}

\item[嵌套列表:] 列表中的元素是另一个列表。
\index{嵌套列表}

\item[列表遍历:] 对列表中的元素按顺序访问。
\index{列表!遍历}

\item[映射:] 一个集合中的元素和另一个集合中的元素的对应关系。例如,列表是下标到元素的映射。
\index{映射}

\item[累加器:] 循环中用于相加或累积的变量。
\index{累加器}

\item[增量赋值:] 更新变量的语句,使用类似\verb"+="的运算符。
\index{赋值!增量}
\index{增量赋值}

\index{遍历}

\item[归并:] 遍历序列,将所有元素求和为一个值的处理模式。
\index{归并模式}
\index{模式!归并}

\item[映射:] 遍历序列,对每个元素执行操作的处理模式。
\index{映射模式}
\index{模式!映射}

\item[筛选:] 遍历序列,选出满足一定标准的处理模式。
\index{筛选模式i}
\index{模式!筛选}

\item[对象:] 变量可以指向的东西。一个对象有其数据类型和值。
\index{对象}

\item[相等:] 有相同的值。
\index{相等}

\item[相同:] 是用一个对象(隐含相等)。
\index{相同}

\item[引用:] 变量和值间的关联。
\index{引用}

\item[别名:] 两个或两个以上的变量指向同一个对象。
\index{别名}

\item[分割符:] 用于指示字符串分割位置的字符或者字符串。
\index{分割符}

\end{description}


\section{练习}

\begin{ex}
编写函数\verb"is_sorted",读取一个列表作为参数,如果列表是升序排序的,则返回{\tt True},否则返回{\tt False}。你可以假设(作为先决条件)列表中的元素可以用关系运算符如{\tt <},{\tt >}等比较。

\index{先决条件}

例如,\verb"is_sorted([1,2,2])"将返回{\tt True},\verb"is_sorted(['b','a'])"将返回{\tt False}。
\end{ex}


\begin{ex}
\label{回文}

\index{回文}

两个单词是回文的,如果你可以重新排列一个的字符后可以拼写出另一个。编写函数\verb"is_anagram",读取两个字符串,如果它们是回文的则返回{\tt True}。
\end{ex}


\begin{ex}
\label{复制}

这也被称为生日悖论:

\begin{enumerate}

\index{生日悖论}
\index{赋值}

\item 编写函数\verb"has_duplicates",读取一个列表作为参数,如果任何元素出现超过一次,则返回{\tt True}。函数不能改变原列表。

\item 如果你班级上有23个学生,2个学生生日相同的概率是多少?你可以通过随即产生23个生日并检查匹配来估计概率。提示:你可以使用{\tt random}模块中的{\tt randint}函数来生成随即生日。

\index{random模块}
\index{模块!random}
\index{randint函数}
\index{函数!randint}

\end{enumerate}

你可以在\url{wikipedia.org/wiki/Birthday_paradox}了解这个问题,你可以在\url{thinkpython.com/code/birthday.py}找到我的程序。

\end{ex}


\begin{ex}

\index{赋值}
\index{唯一}

编写函数\verb"remove_duplicates",参数为一个列表,返回一个新的列表,其中只包含原列表中唯一的元素。提示:列表中的元素不一定按照原来的顺序。
\end{ex}


\begin{ex}
\index{append方法}
\index{方法append}
\index{列表!连接}
\index{连接!列表}

编写函数,读取文件{\tt words.txt},建立一个列表,每个单词为一个元素。编写两个版本函数,一个使用{\tt append}方法,另一个使用{\tt t = t + [x]}。那个版本运行得慢?为什么?

你可以在\url{thinkpython.com/code/wordlist.py}中找到我的程序。
\end{ex}


\begin{ex}
\label{单词表1}
\label{两分法}

\index{成员!两分法搜索}
\index{两分法搜索}
\index{搜索,两分法}

\index{成员!二进制搜索}
\index{二进制搜索}
\index{搜索,二进制}

检查一个单词是否在单词表中,你可以使用{\tt in}运算符,但这很慢,因为它按顺序查找单词。

由于单词是按照字母顺序排序的,我们可以使用两分法(也称二进制搜索)来加快速度,类似你在字典中查找单词的方法。你从中间开始,如果你要找的单词在中间的单词之前,你查找前半部分,否则你查找后半部分。

每次查找,你将搜索范围减小一半。如果单词表有113,809个单词,你只需要17步来找到这个单词,或着知道单词不存在。

编写函数{\tt bisect},参数为一个已排序的列表和一个目标值,返回该值在列表中的位置,如果不存在则返回{\tt None}。

\index{bisect模块}
\index{模块!bisect}

或者你可以阅读{\tt bisect}模块的文档并使用!
\end{ex}

\begin{ex}
\index{反转词对}

两个单词被称为是“反转词对”,如果一个是另一个的反转。编写函数,找出单词表中所有的反转词对。
\end{ex}

\begin{ex}
\index{连锁词}

两个单词被称为是“连锁词”,如果交替的从两个单词中取出字符将组成一个新的单词\footnote{这个练习来自\url{puzzlers.org}中的一个例子。}。例如,“shoe”和“cold”连锁后成为“schooled”。

\begin{enumerate}

\item 编写程序,找出所有的连锁词。提示:不用列举所有的单词对。

\item 你能够找到三重连锁的单词吗?即每个字母依次从3个单词得到。

\end{enumerate}
\end{ex}


\chapter{字典}
\index{dictionary 字典}

\index{dictionary 字典}
\index{type!dict}
\index{key 关键字}
\index{key-value pair 关键字-值对}
\index{index 索引}

字典像列表一样,但是更一般。列表的索引必须是整数,但字典的索引(键)几乎可以是任何类型。\\

可以把字典当作是索引集合(关键字)和值集合之间的映射。每一个关键字对应
一个值。关键字和对应的值称为键-值对,或者项。

我们将构造一个英语单词及其对应西班牙语单词的字典,关键字和关键字值都
是字符串。\\

{\tt dict}函数创建一个空字典。由于{\tt dict}是内建函数名,所以,我们
应该避免使用它作为变量名。\\

\index{dict function dict函数}
\index{function!dict}

\beforeverb
\begin{verbatim}
>>> eng2sp = dict()
>>> print eng2sp
{}
\end{verbatim}
\afterverb

大括号\verb"{}",代表空字典。向字典中添加一个项,可以使用方括号:

\index{squiggly bracket 大括号}
\index{bracket!squiggly}

\beforeverb
\begin{verbatim}
>>> eng2sp['one'] = 'uno'
\end{verbatim}
\afterverb

这行代码创建了一个从关键字{\tt 'one'}到值\verb"'uno'"的映射。如果
再次输出字典,可以看到关键字-值对,和他们之间的冒号:

\beforeverb
\begin{verbatim}
>>> print eng2sp
{'one': 'uno'}
\end{verbatim}
\afterverb

上面输出的格式,也是输入格式。比如,可以创建一个拥有三项的字典:

\beforeverb
\begin{verbatim}
>>> eng2sp = {'one': 'uno', 'two': 'dos', 'three': 'tres'}
\end{verbatim}
\afterverb
%

但是如果输出{\tt eng2sp},你可能会感到惊讶:

\beforeverb
\begin{verbatim}
>>> print eng2sp
{'one': 'uno', 'three': 'tres', 'two': 'dos'}
\end{verbatim}
\afterverb

关键字-值对的顺序和输入的不一样。事实上,如果在读者的机器上尝试这个例子,
也可能得到不同的结果。一般来说,字典项的顺序是随机的。\\

但这个也不会有什么问题,因为字典的元素是不是通过索引来获取的。可以使用
关键字来查询对应的值:

\beforeverb
\begin{verbatim}
>>> print eng2sp['two']
'dos'
\end{verbatim}
\afterverb

关键字{\tt 'two'}总是对应值\verb"'dos'",所以项的顺序没有什么关系。

如果关键字不在字典中,就会抛出异常:

\index{exception!KeyError}
\index{KeyError}

\beforeverb
\begin{verbatim}
>>> print eng2sp['four']
KeyError: 'four'
\end{verbatim}
\afterverb

{\tt len}函数对于字典也是适用的。它返回键-值对的数目:

\index{len function len函数}
\index{function!len}

\beforeverb
\begin{verbatim}
>>> len(eng2sp)
3
\end{verbatim}
\afterverb

{\tt in}运算符对字典也同样使用。它显示某个键是否在字典中作为关键字(.

\index{membership!dictionary}
\index{in operator in运算符}
\index{operator!in}


\beforeverb
\begin{verbatim}
>>> 'one' in eng2sp
True
>>> 'uno' in eng2sp
False
\end{verbatim}
\afterverb

如果想查看某个值是否在字典中,可以使用方法{\tt values},返回包含关键字值
的列表,然后使用{\tt in}运算符:

\index{values method values方法}
\index{method!values}

\beforeverb
\begin{verbatim}
>>> vals = eng2sp.values()
>>> 'uno' in vals
True
\end{verbatim}
\afterverb

{\tt in}运算符操作列表和字典时使用不同的算法。对于列表,使用搜索算法,,
参考\ref{find}部分。随着列表变长,搜索时间成比例增加;最于字典,Python
使用{\bf 散列}算法,带来一个很显著的效果:无论字典有多少项,{\tt in}
运算符花费近乎同样的时间。在这里,我不对此做出更多的解释,详情参考
\url{wikipedia.org/wiki/Hash_table} 。

\index{hashtable 散列}

\begin{ex}
\label{wordlist2}

\index{set membership}
\index{membership!set}

编写函数,读取{\tt words.txt}文件里的单词,把他们作为字典键存储在字典里,
关键字值随便是什么。然后,使用{\tt in}运算符查看某个字符串是否在字典里。\\


如果做了练习\ref{wordlist1},可以比较一下这个实现和列表{\tt
	in}运算符与二分搜索的速度。

\end{ex}

\section{把字典作为计数器}
\label{histogram}

\index{counter 计数器}


假设给定一个字符串,统计每个字母出现的次数。可以有好几种:

\begin{enumerate}

\item
创建26个变量,每个代表一个字母。然后遍历字符串,对每一个字母,增加对应对应的计数器,可以使用链条件语句。

\item 船舰一个26元素的列表。然后把每个字符变换为数字(使用内建{\tt ord}函数),
把数字作为列表的索引,增加相应的计数器。

\item 创建一个字典,字母作为关键字,计数器作为对应的值。第一次遇到衣蛾
字符,把它加入字典。然后可以增加相应项的值。

\end{enumerate}


以上的每个方法实现同样的计算,但是实现的方法不同。

\index{implementation 实现}

实现是实施计算的一种方法,存在某些实现比其他的要好。比如, 用字典实现
的好处是我们不必要实现知道哪个字母会出现在字符串里,我们只需为出现的
字母分配空间。

下面是字典实现的代码:

\beforeverb
\begin{verbatim}
def histogram(s):
    d = dict()
    for c in s:
        if c not in d:
            d[c] = 1
        else:
            d[c] += 1
    return d
\end{verbatim}
\afterverb

函数名是{\bf histogram},是统计学术语,用来直观表示频率。

\index{histogram 直方图}
\index{frequency 频率}
\index{traversal 遍历}

函数的第一行,创建了一个空字典。{\tt for}循环遍历字符串。每次循环,如果字符{\tt
	c}
	不在字典里,我们创建一个键为{\tt c},初值为1的项。如果{\tt c}
	已经在字典里,我们增加{\tt d[c]}的值。

\index{histogram 直方图}

看看它是如何工作的:

\beforeverb
\begin{verbatim}
>>> h = histogram('brontosaurus')
>>> print h
{'a': 1, 'b': 1, 'o': 2, 'n': 1, 's': 2, 'r': 2, 'u': 2, 't': 1}
\end{verbatim}
\afterverb

直方图({\tt histogram})显示,字母{\tt
	'a'}和\verb"'b'"出现一次,\verb"'o'"出现两次,等等。


\begin{ex}

\index{get method get方法}
\index{method!get}

字典有一个{\tt get}函数,接受一个键和一个缺省值。如果键在字典里,{\tt
	get}返回对应的值;否则返回缺省值。比如:

\beforeverb
\begin{verbatim}
>>> h = histogram('a')
>>> print h
{'a': 1}
>>> h.get('a', 0)
1
>>> h.get('b', 0)
0
\end{verbatim}
\afterverb

使用{\tt get}编写一个精巧的{\tt histogram}函数。应该不使用{\tt if}语句就可
以实现。
\end{ex}


\section{循环和字典}

\index{dictionary!looping with}
\index{looping!with dictionaries}
\index{traversal 遍历}

如果在{\tt
	for}语句中使用字典,程序遍历字典的关键字。比如,\verb"print_hist"输出每个关键字和对应的值:


\beforeverb
\begin{verbatim}
def print_hist(h):
    for c in h:
        print c, h[c]
\end{verbatim}
\afterverb

下面是输出结果:

\beforeverb
\begin{verbatim}
>>> h = histogram('parrot')
>>> print_hist(h)
a 1
p 1
r 2
t 1
o 1
\end{verbatim}
\afterverb

可以再一次看到,关键字是无序的。


\begin{ex}

\index{keys method keys方法}
\index{method!keys}

字典有一个方法{\tt keys},以列表形式返回字典的关键字。

修改\verb"print_hist"以字典顺序\footnote{译注:此处的字典顺序指的是字母顺序,因为英文的字典是按照字母的顺序排列的。}打印关键字和对应的值。

\end{ex}



\section{颠倒查询}

\index{dictionary!lookup}
\index{dictionary!reverse lookup}
\index{lookup,dictionary 查询,字典}
\index{reverse lookup,dictionary 颠倒查询,字典}

给定一个字典{\tt d}和关键字{\tt k},可以很容易的查询对应的值{\tt v = d[k]}。
这个操作叫做查询。

但是如果,给定一个{\tt v},想查找对应的{\tt k}该怎么办?有两个问题:
第一,可能有多个关键字映射到同一个值{\tt
	v}。根据实际应用,可能需要选择其中一个,也可能创建一个列表来容纳所有的关键字。第二,没有一个简单的语法来实施颠倒查询,必须使用搜索。


下面是一个函数,接受一个值,返回第一个对应于该值的关键字:

\beforeverb
\begin{verbatim}
def reverse_lookup(d, v):
    for k in d:
        if d[k] == v:
            return k
    raise ValueError
\end{verbatim}
\afterverb
%

这个函数也是搜索的一个例子,但是使用了一个我们没有见过的特性,{\tt raise}.
{\tt raise}语句引发一个异常;此处,引发一个{\tt ValueError}异常,一般意味着,
参数的值出现了问题。

\index{search}
\index{pattern!search}
\index{raise statement 引发异常}
\index{statement!raise}
\index{exception!ValueError}
\index{ValueError}


如果我们到达了循环的末尾,意味着{\tt v}没有出现在字典关键字值里,所以我们
引发一个异常。

下面是一个成功颠倒查询的例子:

\beforeverb
\begin{verbatim}
>>> h = histogram('parrot')
>>> k = reverse_lookup(h, 2)
>>> print k
r
\end{verbatim}
\afterverb

一个查询失败的例子:

\beforeverb
\begin{verbatim}
>>> k = reverse_lookup(h, 3)
Traceback (most recent call last):
  File "<stdin>", line 1, in ?
  File "<stdin>", line 5, in reverse_lookup
ValueError
\end{verbatim}
\afterverb

%

手动引发一个异常和Python引发异常是相同的:输出回溯路径和错误信息。

\index{traceback 回溯}
\index{optional argument}
\index{argument!optional}

{\tt raise}语句接受一个详细的错误信息作为可选参数。比如:

\beforeverb
\begin{verbatim}
>>> raise ValueError, 'value does not appear in the dictionary'
Traceback (most recent call last):
  File "<stdin>", line 1, in ?
ValueError: value does not appear in the dictionary
\end{verbatim}
\afterverb

颠倒查询比正常查询慢很多。如国必须经常颠倒查询,或者当字典变的很大时,程序
的表现会很糟糕。

\begin{ex}
修改\verb"reverse_lookup",让它一列表形式返回一个对应于值{\tt v}的所有关键字。
, 如果没有,返回一个空的列表。
\end{ex}



\section{字典和列表}

 列表可以作为字典的关键字值。比如,给定一个从字母到频率映射的字典,可能
 想反转它,也就是说,创建一个从频率到字母映射的字典。因为可能有好几个字母
 的频率是一样的,所以,反转字典的关键字值应该表示成由字母构成的列表。

 \index{invert dictionary 反转字典}
 \index{dictionary!invert}

 下面是一个反转字典的函数:

 \beforeverb
\begin{verbatim}
def invert_dict(d):
    inv = dict()
    for key in d:
        val = d[key]
        if val not in inv:
            inv[val] = [key]
        else:
            inv[val].append(key)
    return inv
\end{verbatim}
\afterverb


每次循环,{\tt key}从{\tt d}得到一个关键字,{\tt val}得到对应的值。如果{\tt val}
不在{\tt inv}里,意味着,我们之前还没有见过它,所以我们创建一个
新的项,用包含一个值的列表初始化它。否则,我们把对应的关键字追加到列表里。

\index{singleton }

下面是一个例子:

\beforeverb
\begin{verbatim}
>>> hist = histogram('parrot')
>>> print hist
{'a': 1, 'p': 1, 'r': 2, 't': 1, 'o': 1}
>>> inv = invert_dict(hist)
>>> print inv
{1: ['a', 'p', 't', 'o'], 2: ['r']}
\end{verbatim}
\afterverb

这里有个图表,显示了{\tt hist}和{\tt inv}的变化:

\index{state diagram 状态图}
\index{diagram!state}

\beforefig
\centerline{\includegraphics{figs/dict1.eps}}
\afterfig

字典用盒子来表示,类型{\tt dict}在盒子上方,键-值对在盒子里面。
如果值为整型,浮点型或者字符串,通常画在盒子里面,如果是列表,则画在
盒子外面,这样做仅仅是为了保持图的简洁。

列表可以作为字典的关键字值,正如上面的例子演示的。但是,不能作为字典的关键字。请看下面:

\index{TypeError}
\index{exception!TypeError}


\beforeverb
\begin{verbatim}
>>> t = [1, 2, 3]
>>> d = dict()
>>> d[t] = 'oops'
Traceback (most recent call last):
  File "<stdin>", line 1, in ?
TypeError: list objects are unhashable
\end{verbatim}
\afterverb
%

我在之前提到过,字典是用散列方法实现的,这就意味着,关键字必须是可散列的。

\index{hash function 散列函数}
\index{hashable 可散列的}

散列函数,接受一个任何类型的值,返回一个整数。字典使用这些
整数(散列值)存储,查询键-值对。

\index{immutability 不可变性}


如果关键字是不可变的,则一切正常。但是,如果关键字是可变的,比如
列表,麻烦来了。比如,当创建一个键-值对,Python散列关键字,并且
把它存放在相应的位置。如果修改关键字,然后再一次散列,就会散列到
另外一个位置。这种情况下,将会得到两个有着相同键的项,或者可能
无法获取关键字。无论那种情况,字典都不能正常工作。

这就是为什么关键字必须是可散列的,可变数据类型像列表不能作为关键字的原因。最
最简单的解决方式是使用元组,我们在下一章将会遇到。

虽然字典是可变的,不能作为关键字,但是可以作为关键字值。

\begin{ex}
查阅字典方法{\tt setdefault}的文档,用它编写一个更简洁的\verb"invert_dict"。

\index{setdefault method setdefault方法}
\index{method!setdefault}

\end{ex}




\section{备忘录}

如果尝试过\ref{另外一个例子}部分的{\tt fibonacci}函数,可能会
注意到,提供的参数越大,函数运行花费的时间越长。确切地说,运行时间增长
长的很快。

\index{fibonacci function fibonacci函数}
\index{function!fibonacci}

为了一探究竟,看看这个调用图,其中{\tt n = 4}:

\beforefig
\centerline{\includegraphics[height=2in]{figs/fibonacci.eps}}
\afterfig


调用图包含了一系列函数框图,每个框图及其调用的函数框图用直线连接。
在调用图的顶部,{\tt n=4}的{\tt fibonacci}调用{\tt n=3} 的
{\tt fibonacci}和{\tt n=2}的{\tt fibonacci}。依次地,{\tt n=3}的{\tt fibonacci}
调用{\tt n=2}和{\tt n=1}的{\tt fibonacci}函数......

\index{function frame 函数框图}
\index{frame}
\index{call graph}

计算一下{\tt fibonacci(0)}和{\tt fibonaci(1)}分别被调用了多少次。可以看出,
这不是解决这个问题的高效方式,并且随着参数变大,效率会更低。

\index{memo 备忘录}

另外一种方法就是跟踪已经被计算的值---把它存储在字典里。存储已经计算的留作后用叫做备忘录\footnote{参考\url{wikipedia.org/wiki/Memoization}。}。下面是使用备忘录实现
的{\tt fibonacci}:

\beforeverb
\begin{verbatim}
known = {0:0, 1:1}

def fibonacci(n):
    if n in known:
        return known[n]

    res = fibonacci(n-1) + fibonacci(n-2)
    known[n] = res
    return res
\end{verbatim}
\afterverb

{\tt known}是一个字典,跟踪我们已经计算出的Fibonacci数字。起始项为:
0映射到0,1映射到1。

无论何时调用{\tt fibonacci},函数检查{\tt known}字典。如果字典包含需要的结果,
函数立刻返回,否则函数计算新值,并把它加入字典,然后返回。

\begin{ex}
运行此版本的{\tt fibonacci}函数和原先的版本,多传递几个参数,然后比较运行的时间。
\end{ex}


\section{全局变量}

\index{global variable 全局变量}
\index{variabl!global 全局}

在以往的例子中,我们在函数外面创建{\tt
	known},所以它属于特殊的框图\verb"__mian__"。
\verb"__main__"内的变量有时称为全局变量,因为任何函数都可以访问它们。不像局部、
变量,在函数返回时销毁,全局变量在函数调用过程中,都是存在的。

\index{flag 标记}

把全局变量作为标记来使用是很常见的,也就是说,布尔变量(“标记”)表明条件是否为真。
比如,有些程序使用{\tt verbose}标记控制输出的层次:

\beforeverb
\begin{verbatim}
verbose = True

def example1():
    if verbose:
        print 'Running example1'
\end{verbatim}
\afterverb

如果试着给一个全局变量重新赋值\footnote{译注:这个地方使用的术语是不准确的。
我们知道Python中使用的是引用,所以不存在赋值,我们可以说重新绑定(rebind)},我们可能会很惊讶。下面的例子跟踪函数是否被
调用:

\index{mutiple assignment 多重赋值}
\index{assignment!multiple 多次}

\beforeverb
\begin{verbatim}
been_called = False

def example2():
    been_called = True         # WRONG
\end{verbatim}
\afterverb

如果运行程序,将会看到\verb"been_called"的值并没有改变。问题在于
{\tt example2}创建了一个新的局部变量\verb"been_called"。局部变量随着函数的终结
而销毁,而对全局变量没有任何影响。

\index{global statement 全局语句}
\index{statement!global 全局}
\index{declaration 声明}

如果要在函数体里为全局变量重新赋值,我们必须在使用前声明全局变量:

\beforeverb
\begin{verbatim}
been_called = False

def example2():
    global been_called 
    been_called = True
\end{verbatim}
\afterverb

{\tt
	global}语句告诉解释器“在这个函数里,当我说\verb"been_called",它就是全局变量,不要创建一个局部变量了。“

\index{update!global variable 全局变量}
\index{global variable!update 更新}

下面是一个更新全局变量值的例子:

\beforeverb
\begin{verbatim}
count = 0

def example3():
    count = count + 1          # WRONG
\end{verbatim}
\afterverb
如果运行它,会得到:

\index{UnboundLocalError}
\index{exception!UnboundLocalError}

\beforeverb
\begin{verbatim}
UnboundLocalError: local variable 'count' referenced before assignment
\end{verbatim}
\afterverb
Python认为{\tt count}是一个局部变量,这也意味着在写之前必须读取它。
解决的方法是,声明{\tt count}为全局变量。

\index{counter 计数器}

beforeverb
\begin{verbatim}
def example3():
    global count
    count += 1
\end{verbatim}
\afterverb
%

如果全局变量是可变的,就可以不声明而修改它:

\index{mutability 可变}

\beforeverb
\begin{verbatim}
known = {0:0, 1:1}

def example4():
    known[2] = 1
\end{verbatim}
\afterverb
%

所以,我们可以添加,删除,替换全局列表和全局字典的元素,但是如果想要
给变量重新赋值,必须声明为全局变量:

\beforeverb
\begin{verbatim}
def example5():
    global known
    known = dict()
\end{verbatim}
\afterverb
%

\section{长整数}

\index{long integer 长整数}
\index{integer!long 长}
\index{type!long 长}

如果计算{\tt fibonacci(50)},我们得到:
\beforeverb
\begin{verbatim}
>>> fibonacci(50)
12586269025L
\end{verbatim}
\afterverb
%

结果尾部的{\tt L}表明该数是一个长整型\footnote{在Python 3.0中,类型{\tt
	long}不存在了;所有整数,即使非常大,也是{\tt int}类型,或者
{\tt long}类型}。

\index{Python 3.0}

{\tt
	int}类型的值是有范围的;长整型值可以是任意大,但是值越大,消耗的空间和时间就越大。

基本数学运算符对长整型适用,{\tt math}模块的函数也如此,所以,一般来说,
{\tt int}的代码也同样适用于{\tt long}。

任何时候,计算结果太大,而整型无法表示的时候,Python自动把结果转换为
长整型:

\beforeverb
\begin{verbatim}
>>> 1000 * 1000
1000000
>>> 100000 * 100000
10000000000L
\end{verbatim}
\afterverb

第一种情况下,结果是{\tt int}型;第二种情况,是{\tt long}型。

\begin{ex}

\index{encryption 加密}
\index{RSA algorithm RSA算法}
\index{algorithm!RSA}

大整数幂是一般公钥加密算法的基础。阅读维基百科中RSA算法页面\footnote{
	\url{wikipedia.org/wiki/RSA}.},然后编写一个函数加密解密信息。
\end{ex}

\section{调试}
\index{debugging 调试}

随着程序数据的增加,通过手动输出并检查数据变得越来越笨拙。下面是关于此种情况
的一些建议:

\begin{description}

\item [缩小输入数据量:]如果可以,尽量减少数据量。比如,如果程序想要读取一个
文本文档,就可以仅仅读取开始的10行,或者其他的你认为比较小的部分。可以修改
文件本身,或者(最好这样)修改程序,让程序只读取文件的前{\tt n}行。

如果出现错误,可以缩小{\tt n}的值,到出现错误的地方,然后逐渐增加它,直到找放到
并更正错误。

\item [检查概要和类型:]不必要输出检查全部数据,考虑输出数据的概要:比如,
字典项的数目或者列表中数字的个数。

一个常见的运行时错误是数据类型错误。调试这种粗无,通常输出值的类型就足够了。

\item [编写自我测试:]有时,可以编写代码自动检查错误。比如,计算列表中数字的
平均数,可以检查结果应该不大于列表的最大元素,不大于最小元素。这个叫做“健康测试”,因为测试发现不健康的结果。


\index{sanity check 健康测试}
\index{consistency check 连贯性测试}

另外一种测试比较两个不同的计算结果,看看他们是否连贯。这叫做“连贯性测试”。

\item
[精巧的打印输出:]格式化调试输出可以更容易发现粗无。参看\ref{factdebug}部分。
{\tt pprint}模块提供{\tt pprint}函数,以人类可读格式显示内置类型。

\index{pretty print 精巧的输出}
\index{pprint module pprint模块}
\index{module!pprint}

\end{description}

再次提醒:花费在搭建“脚手架”的时间越多,用来调试的时间就越少。

\index{scaffolding 脚手架}




\section{术语表}

\begin{description}

\item [dictionary 字典:]键集合到对应值的映射。
\index{dictionary 字典}

\item [key-value pair 键值对:]键--值映射的表示。
\index{key-value pair 键值对}

\item [item 项:]键值对的别名。
\index{item!dictionary 字典}

\item [key 键:] 字典中,键值对的第一个部分。
\index{key 键}

\item [value 关键字值:]字典中,键值对的第二部分。比我们以前使用的单词“值”,
更特殊。
\index{value 关键字值}

\item [implementation 实现:]计算的方式。
\index{implementation 实现}

\item [hashtable 散列表:]实现Python字典的一种算法。
\index{hashtable 散列表}

\item [hash function 散列函数:]计算键位置的函数。
\index{hash function 散列函数}

\item [hashable 散列体:]拥有散列函数的数据类型。不可变类型,像整型,浮点型,
字符串都是散列体;可变类型,比如列表和字典不是。
\index{hashable 散列体}

\item [lookup 查询:]通过关键字查找关键字值的字典操作。
\index{lookup 查询}

\item [reverse lookup 颠倒查询:]通过关键字值查找对应关键字的字典操作。
\index{reverse lookup ,dictionary 颠倒查询,字典}

\item[singleton 独子体]只有一个元素的线性数据结构.
\index{singleton 独子体}

\item [call graph 调用框图:]显示在程序执行时,从调用函数到被调用函数框图的箭
头的图表。
\index{call graph 调用框图}
\index{diagram!call graph 调用框图}

\item [histogram 直方图:]计数器的集合。
index{histogram 直方图}

\item [memo 备忘录:]存储已经计算出的值,留作后用的方法。
\index{memo 备忘录}

\item [global variable 全局变量:]函数体外定义的变量。全局变量可以从任何
函数中访问。
\index{global variable 全局变量}

\item [flag 标记:]表明条件是否成立的布尔变量。
\index{flag 标记}

\item [declaration 声明:] 像{\tt global}语句一样,告诉解释器关于变量的一些
信息。
\index{declaration 声明}

\end{description}



section{练习}

\begin{ex}
\index{duplicate}

如果做了练习\ref{duplicate},已经编写了一个\verb"has_duplicates"函数,它接受
一个列表作为参数,如果有一个对象在列表中出现多次,就返回{\tt True}。

使用字典编写一个更快,更简洁的\verb"has_duplicates"函数。
\end{ex}



\begin{ex}
\label{exroptatepairs}

\index{letter rotation }
\index{rotation!letters}

两个单词,如果“旋转”其中一个可以得到第二个就称为“旋转对”(参看练习\ref{exrotate} \verb"rotate_word")

编写一个程序,读取单词列表,发现所有的旋转对。
\end{ex}



\begin{ex}
\index{Car Talk}
\index{Puzzler 难题}

下面又是一个难题,来自{\em Car
	Talk}\footnote{\url{www.cartalk.com/content/puzzler/transcripts/200717}.}:


\begin{quote}
这个难题来自于Dan O'Leary。他遇到一个常见的由一个音节,五个字母构成的
单词。这个单词具有如下的性质:当移除第一个字母,剩下的字母组成了一个
与原来单词发音形同的单词。如果去掉第二个字母,其他字母不变,剩下的又是
一个同音词。问这个单词是什么?

现在给个不成功的例子。我们随手举个五个字母的单词为例,“wrack“。W-R-A-C-K,之
,比如`wrack with pain' 。如果去掉第一个字母,剩下'R-A-C-K'。就像,`Holy
cow, did you see the rack on that buck!It must have been a nine-pointer!'.
它确实是一个很完美的同音词。如果去掉'r',剩下`wack',它确实是一个单词,但是不
不是同音词了。

但是确实至少有一个符合这个条件的单词---去掉前两个字母中的一个都可以构成
原来单词的同音词。问题是,这个单词是什么?
\end{quote}

\index{homophone 同音词}
\index{reducible world 可化简单词}
\index{word,reducible 单词,可化简}

可以使用练习\ref{wordlist2}的字典,检查字符串是否在字典列表里。

如果要检查两个单词是否是同音词,可以使用卡内基.梅隆大学发音辞典。可以
从这里下载:
\url{www.speech.cs.cmu.edu/cgi-bin/cmudict}或者从这里\url{thinkpython.com/code/c06d} 
也可以下载\url{thinkpython.com/code/pronounce.py},这个模块提供了
函数\verb"read_dictionary",读取发音辞典,返回Python字典,提供了从单词到
对应发音(以字符串表示)的映射。

编写程序,列举能够解决这个难题的所有单词。参看我的解答\url{thinkpython.com/code/homophone.py}。


\end{ex}





































































































































































\chapter{元组}
\label{元组}

\section{元组是不可变的}

\index{元组}
\index{类型!元组}
\index{序列}

元组是一组序列的值。元组中的值可以是任何数据类型,使用整数作为下标,在这个方面元组很想列表。但是一个主要的区别是元组是不可改变的。

\index{可改变的}
\index{不可改变的}

从语法构成上来看,元组是用逗号隔开的值的序列:

\beforeverb
\begin{verbatim}
>>> t = 'a', 'b', 'c', 'd', 'e'
\end{verbatim}
\afterverb
%
通常用括号包含元组,虽然这不是必要的:

\index{括号!元组位于}

\beforeverb
\begin{verbatim}
>>> t = ('a', 'b', 'c', 'd', 'e')
\end{verbatim}
\afterverb
%
要创建只含一个元素的元组,你需要包含最后的逗号:

\index{单一}
\index{元组!单一}

\beforeverb
\begin{verbatim}
>>> t1 = 'a',
>>> type(t1)
<type 'tuple'>
\end{verbatim}
\afterverb
%
在括号中的值不是元组:

\beforeverb
\begin{verbatim}
>>> t2 = ('a')
>>> type(t2)
<type 'str'>
\end{verbatim}
\afterverb
%
创建元组的另一个方式是使用内减函数{\tt tuple}。当没有参数时,函数创建一个空的元组:

\index{tuple函数}
\index{函数!tuple}

\beforeverb
\begin{verbatim}
>>> t = tuple()
>>> print t
()
\end{verbatim}
\afterverb
%
如果参数是一个序列(字符串、列表或元组),返回结果是使用序列中的元素构成的元组:

\beforeverb
\begin{verbatim}
>>> t = tuple('lupins')
>>> print t
('l', 'u', 'p', 'i', 'n', 's')
\end{verbatim}
\afterverb
%
由于{\tt tuple}是一个内建函数的名字,你应该避免使用它作为变量名。

大多数的列表运算符适用于元组。括号运算符可以索引一个元素:

\index{括号运算符}
\index{运算符!括号}

\beforeverb
\begin{verbatim}
>>> t = ('a', 'b', 'c', 'd', 'e')
>>> print t[0]
'a'
\end{verbatim}
\afterverb
%
切片运算符选择一个范围内的元素。

\index{切片运算符}
\index{运算符!切片}
\index{元组!切片}
\index{切片!元组}

\beforeverb
\begin{verbatim}
>>> print t[1:3]
('b', 'c')
\end{verbatim}
\afterverb
%
但是如果你试图修改元组中的元素,你会得到错误:

\index{异常!类型错误}
\index{类型错误}
\index{项目赋值}
\index{赋值!项目}

\beforeverb
\begin{verbatim}
>>> t[0] = 'A'
TypeError: object doesn't support item assignment
\end{verbatim}
\afterverb
%
你不能修改一个元组的元素,但是你可以使用另一个元组代替原来的元组:

\beforeverb
\begin{verbatim}
>>> t = ('A',) + t[1:]
>>> print t
('A', 'b', 'c', 'd', 'e')
\end{verbatim}
\afterverb
%

\section{元组赋值}
\label{元组赋值}

\index{元组!赋值}
\index{赋值!元组}
\index{交换模式}
\index{模式!交换}

我们经常会用到两个变量间的值的交换。对于传统的赋值,你需要使用一个临时变量。例如,要交换{\tt a}和{\tt b}:

\beforeverb
\begin{verbatim}
>>> temp = a
>>> a = b
>>> b = temp
\end{verbatim}
\afterverb
%
这个方法显得笨拙,{\bf 元组赋值}就优雅许多:

\beforeverb
\begin{verbatim}
>>> a, b = b, a
\end{verbatim}
\afterverb
%
左侧是变量组成的元组,右侧是表达式组成的元组。每个值被赋值给对应的变量。右侧所有的表达式被计算后再赋值。

左侧的变量个数必须等于右侧值的个数:

\index{异常!值错误}
\index{值错误}

\beforeverb
\begin{verbatim}
>>> a, b = 1, 2, 3
ValueError: too many values to unpack
\end{verbatim}
\afterverb
%
更一般的,右侧可以是任何序列(字符串、列表或元组)。例如,要分离一个email地址的用户名和域,你可以这么写:

\index{split方法}
\index{方法!split}
\index{email地址}

\beforeverb
\begin{verbatim}
>>> addr = 'monty@python.org'
>>> uname, domain = addr.split('@')
\end{verbatim}
\afterverb
%
{\tt split}的返回值是两个元素的列表。第一个元素被赋值给{\tt uname},第二个被赋值给{\tt domain}。

\beforeverb
\begin{verbatim}
>>> print uname
monty
>>> print domain
python.org
\end{verbatim}
\afterverb
%

\section{元组作为返回值}

\index{元组}
\index{值!元组}
\index{返回值!元组}
\index{函数,元组作为返回值}

严格地说,一个函数只能返回一个值,但是如果这个值是一个元组,等效于返回多个值。例如,如果你要计算两个整数的除法并得到商和余数,分别计算{\tt x/y}和{\tt x\%y}的效率太低,一个好的方法是同时计算这两个值。

\index{divmod}

内建函数{\tt divmod}读取两个参数,返回有两个值的元组,分别是商和余数。你可以将结果保存在一个元组中:

\beforeverb
\begin{verbatim}
>>> t = divmod(7, 3)
>>> print t
(2, 1)
\end{verbatim}
\afterverb
%
或者使用元组赋值将元素分开保存:

\index{元组赋值}
\index{赋值!元组}

\beforeverb
\begin{verbatim}
>>> quot, rem = divmod(7, 3)
>>> print quot
2
>>> print rem
1
\end{verbatim}
\afterverb
%
下面给出一个返回元组的函数的例子:

\beforeverb
\begin{verbatim}
def min_max(t):
    return min(t), max(t)
\end{verbatim}
\afterverb
%
{\tt max}和{\tt min}是内建函数,分别查找序列中最大和最小的元素, \verb"min_max"计算两者并返回包含两个值的元组。

\index{max函数}
\index{函数!max}
\index{min函数}
\index{函数!min}


\section{变长参数元组}

\index{变长参数元组}
\index{参数!变长元组}
\index{聚集}
\index{参数!聚集}
\index{参数!聚集}

函数可以读取一个变长的参数。一个以{\tt *}开头的参数将参数{\bf 聚集}为一个元组。例如,{\tt printall}可以接收任意个数的参数,并打印它们:

\beforeverb
\begin{verbatim}
def printall(*args):
    print args
\end{verbatim}
\afterverb
%
聚集的参数可以取任何你喜欢的名字,但是习惯上使用{\tt args}。下面给出函数是如何工作的:

\beforeverb
\begin{verbatim}
>>> printall(1, 2.0, '3')
(1, 2.0, '3')
\end{verbatim}
\afterverb
%
和聚集相对应的是{\bf 散布}。如果你有一个值的序列要作为多个参数传给一个函数,你可以使用{\tt *}运算符。例如,{\tt divmod}需要两个参数,而不是一个元组:

\index{散布}
\index{参数散布}

\index{类型错误}
\index{异常!类型错误}

\beforeverb
\begin{verbatim}
>>> t = (7, 3)
>>> divmod(t)
TypeError: divmod expected 2 arguments, got 1
\end{verbatim}
\afterverb
%
但是如果你散布元组,它将工作正常:

\beforeverb
\begin{verbatim}
>>> divmod(*t)
(2, 1)
\end{verbatim}
\afterverb
%
\begin{ex}
许多内建函数使用变长参数元组。例如,{\tt max}和{\tt min}可以读取任意个数的参数:

\index{max函数}
\index{函数!max}
\index{min函数}
\index{函数!min}

\beforeverb
\begin{verbatim}
>>> max(1,2,3)
3
\end{verbatim}
\afterverb
%
但是{\tt sum}函数不是这样。

\index{sum函数}
\index{函数!sum}

\beforeverb
\begin{verbatim}
>>> sum(1,2,3)
TypeError: sum expected at most 2 arguments, got 3
\end{verbatim}
\afterverb
%
编写函数{\tt sumall},可以接收任意多个参数,并返回它们的和。

\end{ex}


\section{列表和元组}

\index{zip函数}
\index{函数!zip}

{\tt zip}是一个内建函数,参数为两个或两个以上的序列,并将它们“拉链”成一个元组的列表,每个元组包含每个序列中的一个元素\footnote{在Python 3.0中,{\tt zip}返回一个元组的迭代器,但是对于大多数情况,迭代器表现的像一个列表。}。

\index{Python 3.0}

下面是一个字符串和一个列表的拉链的例子:

\beforeverb
\begin{verbatim}
>>> s = 'abc'
>>> t = [0, 1, 2]
>>> zip(s, t)
[('a', 0), ('b', 1), ('c', 2)]
\end{verbatim}
\afterverb
%
结果是一个元组的列表,每个元组包含字符串中的一个字符和列表中对应的元素。

\index{列表!元组的}

如果序列的长度不同,结果的长度和较短的序列相同。

\beforeverb
\begin{verbatim}
>>> zip('Anne', 'Elk')
[('A', 'E'), ('n', 'l'), ('n', 'k')]
\end{verbatim}
\afterverb
%
你可以在{\tt for}循环中使用元组赋值来遍历一个元组的列表:

\index{遍历}
\index{元组赋值}
\index{赋值!元组}

\beforeverb
\begin{verbatim}
t = [('a', 0), ('b', 1), ('c', 2)]
for letter, number in t:
    print number, letter
\end{verbatim}
\afterverb
%
对于每次循环,Python选择列表中下一个元组并将元素赋值给{\tt letter}和{\tt number}。循环的输出是:

\index{循环}

\beforeverb
\begin{verbatim}
0 a
1 b
2 c
\end{verbatim}
\afterverb
%
如果你结合{\tt zip},{\tt for}和元组赋值,你得到一个同时遍历两个(或多个)序列的常用写法。例如,\verb"has_match"读取两个序列,{\tt t1}和{\tt t2},如果有下标{\tt i}使得{\tt t1[i] == t2[i]},则返回{\tt True} :

\index{for循环}

\beforeverb
\begin{verbatim}
def has_match(t1, t2):
    for x, y in zip(t1, t2):
        if x == y:
            return True
    return False
\end{verbatim}
\afterverb
%
如果你要遍历一个序列中的元素和它们的下标,你可以使用内建函数{\tt enumerate}:

\index{遍历}
\index{enumerate函数}
\index{函数!enumerate}

\beforeverb
\begin{verbatim}
for index, element in enumerate('abc'):
    print index, element
\end{verbatim}
\afterverb
%
循环的输出为:

\beforeverb
\begin{verbatim}
0 a
1 b
2 c
\end{verbatim}
\afterverb
%


\section{字典和元组}

\index{字典}
\index{items方法}
\index{方法!items}
\index{键-值对}

字典有个方法称为{\tt items},返回一个元组的列表,每个元组是键-值对\footnote{在Python 3.0中稍有不同。}。

\beforeverb
\begin{verbatim}
>>> d = {'a':0, 'b':1, 'c':2}
>>> t = d.items()
>>> print t
[('a', 0), ('c', 2), ('b', 1)]
\end{verbatim}
\afterverb
%
正如你对字典所期望的,列表中的项目没有固定的顺序。

\index{字典!初始化}

相反的,你可以使用一个元组列表来初始化一个字典:

\beforeverb
\begin{verbatim}
>>> t = [('a', 0), ('c', 2), ('b', 1)]
>>> d = dict(t)
>>> print d
{'a': 0, 'c': 2, 'b': 1}
\end{verbatim}
\afterverb

结合{\tt dict}和{\tt zip}给出了一个简洁的创建字典的方法: 

\index{zip函数!和dict一同使用}

\beforeverb
\begin{verbatim}
>>> d = dict(zip('abc', range(3)))
>>> print d
{'a': 0, 'c': 2, 'b': 1}
\end{verbatim}
\afterverb
%
字典的另一个方法{\tt update}读取一个元组列表,将它作为键-值对加入现有的字典。

\index{update方法}
\index{方法!update}

\index{遍历!字典}
\index{字典!遍历}

结合{\tt items},元组赋值和{\tt for}循环,你可以得到遍历字典的键-值对的常用写法:

\beforeverb
\begin{verbatim}
for key, val in d.items():
    print val, key
\end{verbatim}
\afterverb
%
循环的输出为:

\beforeverb
\begin{verbatim}
0 a
2 c
1 b
\end{verbatim}
\afterverb
%

\index{元组!作为字典中的键}
\index{哈希表}

通常使用元组作为字典的键(主要因为你不能使用列表)。例如,电话簿是姓-名对到电话号码的映射。假设我们已经定义了{\tt last},{\tt first}和{\tt number},我们可以这么写:

\beforeverb
\begin{verbatim}
directory[last,first] = number
\end{verbatim}
\afterverb
%
括号中的表达式是一个元组。我们可以使用元组赋值来遍历字典。

\index{元组!在括号中}

\beforeverb
\begin{verbatim}
for last, first in directory:
    print first, last, directory[last,first]
\end{verbatim}
\afterverb
%
这个循环遍历{\tt directory}中作为键的元组。将每个元组中的元素赋值给{\tt last}和{\tt first},然后打印姓名和对应的电话号码。

有两种在状态图中标示元组的方式。下面的更详细的版本给出了类似列表的下标和元素。例如,元组\verb"('Cleese', 'John')"将会标示成这样:

\index{状态图}
\index{图!状态}

\beforefig
\centerline{\includegraphics{figs/tuple1.eps}}
\afterfig
但是在一个更大的图中你也许希望忽略细节。例如,一个电话簿的状态图会是这样:

\beforefig
\centerline{\includegraphics{figs/dict2.eps}}
\afterfig

这里元组根据Python的语法以图形速记的形式给出。

图中的电话号码是BBC的投诉电话,请不要拨打。



\section{元组比较}

\index{比较!元组}
\index{元组!比较}
\index{sort方法}
\index{方法!sort}

关系运算符使用于元组和其他序列。Python从每个序列的第一个元素开始比较。如果它们相同,则继续比较下一个元素,依次类推,直到找到有区别的元素,以后的元素将不被考虑(即使它们很大)。

\beforeverb
\begin{verbatim}
>>> (0, 1, 2) < (0, 3, 4)
True
>>> (0, 1, 2000000) < (0, 3, 4)
True
\end{verbatim}
\afterverb
%
{\tt sort}函数的工作原理类似,它根据第一个元素排序,如果第一个元素相同,则对第二个元素进行排序,依次类推。

这个特点对应{\bf DSU}模式: 

\begin{description}

\item[Decorate] 装饰序列,生成元组列表,将一个或多个排序关键字放在元素的最前面。

\item[Sort] 对元组列表排序

\item[Undecorate] 通过从已排序的序列中抽出元素来还原。

\end{description}

\label{DSU}
\index{DSU模式}
\index{模式!DSU}
\index{装饰-排序-还原模式}
\index{模式!装饰-排序-还原}

例如,假设你有列单词,你需要从最长到最短将它们排序:

\beforeverb
\begin{verbatim}
def sort_by_length(words):
    t = []
    for word in words:
       t.append((len(word), word))

    t.sort(reverse=True)

    res = []
    for length, word in t:
        res.append(word)
    return res
\end{verbatim}
\afterverb
%
第一个循环构造了一个元组列表,每个元组有单词长度和单词组成。

{\tt sort}函数首先比较第一个元素,即单词长度,仅当单词长度相同时才考虑第二个元素。关键字参数{\tt reverse=True}告诉{\tt sort}使用降序排序。

\index{关键字参数}
\index{参数!关键字}
\index{遍历}

第二个循环遍历元组列表,构建一个按长度递减排列的单词列表。

\begin{ex}
在这个例子中,长度相同的单词是按照字母顺序排序的。对于一些其他的应用,你也许需要它们是随机排序的。修改例子是长度相同的单词随机排序。提示:参考{\tt random}模块中的{\tt random}函数。


\index{random模块}
\index{模块!random}
\index{random函数}
\index{函数!random}

\end{ex}


\section{序列的序列}
\index{序列}

我主要使用元组的列表,事实上几乎本章所有的例子都还可以使用列表的列表,元组的元组和列表的元组来实现。为了避免枚举可能的组合,有时说成序列的序列更为方便。

在很多情况下,不同的序列(字符串,列表和元组)可以交换的使用。那么该如何选择,为什么这么选择呢?

\index{字符串}
\index{列表}
\index{元组}
\index{可改变}
\index{不可改变}

显而易见的是,字符串相比其他序列受到更多的限制,因为元素必须是字符。同时它们是不可改变的。如果你需要能够修改字符串中的字符(而非创建一个新的字符串),你需要使用字符的列表。

列表比元组更为常用,主要因为它们是可改变的。但有一些情况你或许会更倾向于元组:

\begin{enumerate}

\item 在某些情况下,如{\tt return}语句,语法上创建一个元组比创建一个列表更方便。在其他情况下,你也许更倾向列表。

\item 如果你要使用一个类似字典关键字的序列,你必须使用类似元组或字符串的不可改变的数据类型。

\item 如果你将序列作为函数参数,使用元组会减小潜在的因为别名而造成的意外的行为。

\end{enumerate}

由于元组是不可改变的,它们不提供类似{\tt sort}或{\tt reverse}等修改列表的方法。但是Python提供内建函数{\tt sorted}和{\tt reversed},它们读取任何序列作为参数,并返回一个新的排序后的列表。

\index{sorted函数}
\index{函数!sorted}
\index{reversed函数}
\index{函数!reversed}


\section{调试}

\index{调试}
\index{数据结构}
\index{形状错误}
\index{错误!形状}
列表、字典和元组通常被成为{\bf 数据结构}。本章中我们开始接触复合数据结构,如元组的列表、以元组作为键列表作为值的字典。复合数据结构十分有用,但它们容易造成我习惯称呼的{\bf 形状错误},即由于数据结构含有错误的类型、大小或符合而造成的错误。例如,你期望得到一个只含一个整数的列表,而我给你的是一个整数(不是一个列表),这将导致错误。

\index{structshape模块}
\index{模块!structshape}

为了帮助调试此类错误,我编写了一个叫做{\tt structshape}的模块,提供一个{\tt structshape}函数,它可以读取任何数据结构作为参数,并返回一个描述该结构的字符串。你可以在\url{thinkpython.com/code/structshape.py}下载。

下面给出一个简单列表的结果:

\beforeverb
\begin{verbatim}
>>> from structshape import structshape
>>> t = [1,2,3]
>>> print structshape(t)
list of 3 int
\end{verbatim}
\afterverb
%
更好的程序也许会输出``list of 3 int{\em s},'',但不考虑复数相对简单。下面是一个列表的列表:

\beforeverb
\begin{verbatim}
>>> t2 = [[1,2], [3,4], [5,6]]
>>> print structshape(t2)
list of 3 list of 2 int
\end{verbatim}
\afterverb
%
如果列表中的元素不是同一数据类型,{\tt structshape}按类型聚合它们:

\beforeverb
\begin{verbatim}
>>> t3 = [1, 2, 3, 4.0, '5', '6', [7], [8], 9]
>>> print structshape(t3)
list of (3 int, float, 2 str, 2 list of int, int)
\end{verbatim}
\afterverb
%
下面是一个元素的列表:

\beforeverb
\begin{verbatim}
>>> s = 'abc'
>>> lt = zip(t, s)
>>> print structshape(lt)
list of 3 tuple of (int, str)
\end{verbatim}
\afterverb
%
下面是有3项从整数到字符串映射的字典。

\beforeverb
\begin{verbatim}
>>> d = dict(lt) 
>>> print structshape(d)
dict of 3 int->str
\end{verbatim}
\afterverb
%
如果你在跟踪数据结构时遇到了问题,{\tt structshape}可以帮助你。


\section{术语}

\begin{description}

\item[元组:] 不可改变的元素的序列。
\index{元组}

\item[元组赋值:] 一个序列在右、一个元组变量在左的赋值。右边首先计算值,然后将其元素赋值给左边的变量。
\index{元组赋值}
\index{赋值!元组}

\item[聚集:] 对变长参数元组的集合操作。
\index{聚集}

\item[散布:] 将序列作为参数列表的操作。
\index{散布}

\item[DSU:] ``decorate-sort-undecorate''的简称,一个构建元组列表、排序、提取结果的模式。
\index{DSU模式}

\item[数据结构:] 相关数据的集合,通常组织为列表、字典、元组等。
\index{数据结构}

\item[形状(数据结构的):] 对数据结构类型、大小和组成的概述。
\index{形状}

\end{description}


\section{练习}

\begin{ex}
编写函数\verb"most_frequent",参数为一个字符串,以频率降序打印字符出现的次数。从不同语言的测试文本中寻找频率的不同。将你的结果和\url{wikipedia.org/wiki/Letter_frequencies}中的表格作比较。

\index{字母频率}
\index{频率!字母}

\end{ex}


\begin{ex}
\label{回文}

\index{回文集合}
\index{集合!回文}

更多关于回文的练习!

\begin{enumerate}

\item 编写函数,从文件中读取一个单词表(参考章节~\ref{单词表}),打印所有回文的单词。

下面的例子给出可能的输出结果:

\beforeverb
\begin{verbatim}
['deltas', 'desalt', 'lasted', 'salted', 'slated', 'staled']
['retainers', 'ternaries']
['generating', 'greatening']
['resmelts', 'smelters', 'termless']
\end{verbatim}
\afterverb
%
提示:你也许需要构建一个字典满足从字母的集合到这些字母可以组成的单词列表的映射。问题是你如何表示这个字母集合使得它们可以作为字典的键?

\item 修改之前的程序,使程序按照结果集合的大小从大到小输出。

\index{拼字游戏}
\index{bingo}

\item 在拼字游戏中,如果你手头的7个字母和桌面上的1个字母组成一个8个字母的单词,你实现了“bingo”。哪8个字母组成的集合有最大的概率实现“bingo”?提示:有7组。

% (7, ['angriest', 'astringe', 'ganister', 'gantries', 'granites',
% 'ingrates', 'rangiest'])

\index{置换}

\item 我们定义两个单词为“置换对”,如果你能通过交换字母顺序将一个单词转换为另一个单词\footnote{这个练习受\url{puzzlers.org}中一个例子的启发。}。例如,“converse”和“conserve”是一对“置换对”。提示:不要测试所有的单词对,也不要测试所有的交换。

你可以在\url{thinkpython.com/code/anagram_sets.py}下载一个解答。

\end{enumerate}
\end{ex}



\begin{ex}

\index{Car Talk}
\index{难题}

下面是另一个Car Talk难题\footnote{\url{www.cartalk.com/content/puzzler/transcripts/200651}。}:

\begin{quote}
如果每次你从一个单词中删除一个字母它仍是一个有效的英语单词,在英语中满足这样条件的最长的单词是什么?

删除的字母可以位于两端或者中间,但是你不能重新排列字母。每次你删除一个字母,你得到另一个英语单词。最终你得到一个只有一个字母的单词。我要知道最长的单词是什么,它有多少字母?

我要给出一个例子:Sprite。你从sprite开始删除字母,首先删除r,我们得到单词spite,接着删除e,我们得到spit,再删除s,我们得到pit,it和I。
\end{quote}

\index{可缩小的单词}
\index{单词,可缩小的}

编写程序,找出可以按这中方法缩小的所有单词,并找出最长的一个。

这个练习比以往的都更有挑战性,所以给出一些建议:

\begin{enumerate}

\item 你也许需要编写一个函数,参数为一个单词,函数找出所有删除一个字母后仍合法的单词,即原单词的“孩子”。

\index{递归定义}
\index{定义!递归}

\item 递归的看,一个单词是可缩小的,如果它所有的孩子都是可缩小的。作为基本状态,你可以认为空字符串是可以缩小的。

\item 我提供的单词表{\tt words.txt}中不包括单字母的单词。所以你需要添加“I”,“a”和空字符串。

\item 为了提高你的程序的效率,你需要记住已知的可缩小的单词。

\end{enumerate}

你可以在\url{thinkpython.com/code/reducible.py}下载我的解答。

\end{ex}


\chapter{实例学习:数据结构的实例}


\chapter{文件}

\index{文件}
\index{类型!文件}


\section{持久性}

\index{持久性}

目前我们所见到的大多数的程序都是瞬态的,即它们在短时间内运行并输出一些结果,当它们结束时,数据也就消失了。如果你再次运行程序,它将从一个初始的状态开始。

另一类程序被称为是{\bf 持久的},它们长时间运行(或者时刻运行),至少将一部分数据记录在永久储存设备(如硬盘)上,当程序关闭并重新启动时,它们可以恢复结束前的状态以便继续运行。

一个持久的程序的例子是操作系统,当一台电脑开机后操作系统在绝大多数时间都在运行,对于一个接受网络请求的web服务器,操作系统时刻在运行。

程序维护数据最简单的方法是读写文本文件。我们已经接触过读取文本文件的程序,在本章中我们将接触写入文本文件的程序。

记录程序状态的另一个方式是使用数据库。在本章节中我给出一个简单的数据库和模块{\tt pickle},该模块简化了存储数据的过程。

\index{pickle模块}
\index{模块!pickle}


\section{读取和写入}

\index{文件!读取和写入}

文本文件是储存在类似硬盘、闪存、CD-ROM等永久介质上的字符序列。我们在章节~\ref{单词表}中接触了文件的打开和读取。

\index{open函数}
\index{函数!open}

要写入一个文件,你需要在打开文件时添加第二个参数\verb"'w'":

\beforeverb
\begin{verbatim}
>>> fout = open('output.txt', 'w')
>>> print fout
<open file 'output.txt', mode 'w' at 0xb7eb2410>
\end{verbatim}
\afterverb
%
如果文件已经存在,以写的模式打开该文件将会清空原来的数据并从新的开始,所以要小心!如果文件不存在,那么将创建一个新的文件。

{\tt write}方法将数据写入文件。

\beforeverb
\begin{verbatim}
>>> line1 = "This here's the wattle,\n"
>>> fout.write(line1)
\end{verbatim}
\afterverb
%
文件对象将跟踪位置,如果你再次调用{\tt write},它将在尾部写入新的数据。

\beforeverb
\begin{verbatim}
>>> line2 = "the emblem of our land.\n"
>>> fout.write(line2)
\end{verbatim}
\afterverb
%
当你完成写入,你可以关闭文件。

\beforeverb
\begin{verbatim}
>>> fout.close()
\end{verbatim}
\afterverb
%

\index{close方法}
\index{方法!close}


\section{格式运算符}

\index{格式运算符}
\index{运算符!格式}

{\tt write}的参数必须是字符串,如果我们要将其他值写入文件,我们需要将它们转换为字符串。最简单的方法是使用{\tt str}:

\beforeverb
\begin{verbatim}
>>> x = 52
>>> f.write(str(x))
\end{verbatim}
\afterverb
%
另一个方法是使用{\bf 格式运算符}{\tt \%}。当作用于整数,{\tt \%}是取模运算符,而当第一个运算数是字符串时,{\tt \%}是格式运算符。

\index{格式字符串}

第一个运算数是{\bf 格式字符串},它包含一个或多个{\bf 格式序列},它们指定了第二个运算数是如何格式化的。结果为一个字符串。

\index{格式序列}

例如,格式序列\verb"'%d'"意味着第二个运算数应该被格式化为一个整数({\tt d}代表“decimal”):

\beforeverb
\begin{verbatim}
>>> camels = 42
>>> '%d' % camels
'42'
\end{verbatim}
\afterverb
%
结果是字符串\verb"'42'",需要和整数值{\tt 42}区分开来。

格式序列可以出现在字符串中的任何位置,所以你可以将值嵌入到语句中:

\beforeverb
\begin{verbatim}
>>> camels = 42
>>> 'I have spotted %d camels.' % camels
'I have spotted 42 camels.'
\end{verbatim}
\afterverb
%
如果字符串中有多于一个格式序列,第二个参数必须为一个元组。每个格式序列按次序和元组中的元素对应。

下面的例子中使用\verb"'%d'"来格式化一个整数。
\verb"'%g'" to format
a floating-point number (don't ask why), and \verb"'%s'" to format
a string:

\beforeverb
\begin{verbatim}
>>> 'In %d years I have spotted %g %s.' % (3, 0.1, 'camels')
'In 3 years I have spotted 0.1 camels.'
\end{verbatim}
\afterverb
%
元组中元素的个数必须等于字符串中格式序列的个数。同时,元素的类型必须符合对应的格式序列。

\index{异常!类型错误}
\index{类型错误}

\beforeverb
\begin{verbatim}
>>> '%d %d %d' % (1, 2)
TypeError: not enough arguments for format string
>>> '%d' % 'dollars'
TypeError: illegal argument type for built-in operation
\end{verbatim}
\afterverb
%
在第一个例子中,元组中没有足够的元素,在第二个例子中,元素的类型错误。

格式运算符十分强大,但它很难使用。你可以在\url{docs.python.org/lib/typesseq-strings.html}阅读更多有关的内容。


\section{文件名和路径}
\label{路径}

\index{文件名}
\index{路径}
\index{目录}
\index{文件夹}

文件以{\bf 目录} (也称为“文件夹”)的形式管理。每个运行的程序有一个“当前目录”,它是许多操作的默认目录。例如,当你打开一个文件来读取数据,Python在当前目录下寻找这个文件。

\index{os模块}
\index{模块!os}

{\tt os}模块提供了操作文件和目录的函数(“os”代表“operating system”)。{\tt os.getcwd}返回当前目录的名称:

\index{getcwd函数}
\index{函数!getcwd}

\beforeverb
\begin{verbatim}
>>> import os
>>> cwd = os.getcwd()
>>> print cwd
/home/dinsdale
\end{verbatim}
\afterverb
%
{\tt cwd}代表“current working directory”,即当前工作陌路。在本例中结果是{\tt /home/dinsdale},它是用户名为{\tt dinsdale}的主目录。

\index{工作目录}
\index{目录!工作}
类似{\tt cwd}能够确定一个文件的字符串称为{\bf 路径}。{\bf 相对路径}从当前目录开始,{\bf 绝对路径}从文件系统的根目录开始。

\index{相对路径}
\index{路径!相对}
\index{绝对路径}
\index{路径!绝对}

我们现在看到的路径都是简单的文件名,因此它们是相对当前目录的。要得到一个文件的绝对目录,你可以使用{\tt os.path.abspath}:

\beforeverb
\begin{verbatim}
>>> os.path.abspath('memo.txt')
'/home/dinsdale/memo.txt'
\end{verbatim}
\afterverb
%
{\tt os.path.exists}检查一个文件或者目录是否存在:

\index{exists函数}
\index{函数!exists}

\beforeverb
\begin{verbatim}
>>> os.path.exists('memo.txt')
True
\end{verbatim}
\afterverb
%
如果存在,{\tt os.path.isdir}检查它是否是一个目录:

\beforeverb
\begin{verbatim}
>>> os.path.isdir('memo.txt')
False
>>> os.path.isdir('music')
True
\end{verbatim}
\afterverb
%
类似的,{\tt os.path.isfile}检查是否是一个文件。

{\tt os.listdir}返回给定目录下的文件(以及其他目录):

\beforeverb
\begin{verbatim}
>>> os.listdir(cwd)
['music', 'photos', 'memo.txt']
\end{verbatim}
\afterverb
%
为了演示这些函数,下面的例子“遍历”一个目录,打印所有文件的名字,并对所有目录递归地调用自身。

\index{遍历,目录}
\index{目录!遍历}

\beforeverb
\begin{verbatim}
def walk(dir):
    for name in os.listdir(dir):
        path = os.path.join(dir, name)

        if os.path.isfile(path):
            print path
        else:
            walk(path)
\end{verbatim}
\afterverb
%
{\tt os.path.join}读取一个目录名和一个文件名,并将两者合并为一个完整路径。 
\begin{ex}
修改{\tt walk}函数,使之返回文件名列表,而不是打印这些信息。
\end{ex}

\begin{ex}
{\tt os}模块提供了与我们的{\tt walk}函数类似的函数,但功能更丰富。阅读文档,使用该函数打印给定目录下的文件和子目录。
\end{ex}


\section{捕获异常}
\label{捕获}
当你尝试读写文件时,很多地方会发生错误。如果你试图打开一个不存在的文件,你会得到一个{\tt 输入输出错误}:

\index{open海曙}
\index{函数!open}
\index{异常!输入输出错误}
\index{输入输出错误}

\beforeverb
\begin{verbatim}
>>> fin = open('bad_file')
IOError: [Errno 2] No such file or directory: 'bad_file'
\end{verbatim}
\afterverb
%
如果你没有权限访问一个文件:

\index{文件!权限}
\index{权限,文件}

\beforeverb
\begin{verbatim}
>>> fout = open('/etc/passwd', 'w')
IOError: [Errno 13] Permission denied: '/etc/passwd'
\end{verbatim}
\afterverb
%
如果你试图读取一个目录,你会得到:

\beforeverb
\begin{verbatim}
>>> fin = open('/home')
IOError: [Errno 21] Is a directory
\end{verbatim}
\afterverb
%
为了避免这些错误,你可以使用类似{\tt os.path.exists}和{\tt os.path.isfile}的检查函数,但是检查所有可能的错误会占用很多时间和代码(如果“{\tt Errno 21}”是一个错误信息,那么至少有21种出错情况)。

\index{异常,捕获}
\index{try语句}
\index{语句!try}

更好的方法是当问题出现了才去处理,即{\tt try}语句所做的。它的语法类似{\tt if}语句:

\beforeverb
\begin{verbatim}
try:    
    fin = open('bad_file')
    for line in fin:
        print line
    fin.close()
except:
    print 'Something went wrong.'
\end{verbatim}
\afterverb
%
Python从{\tt try}语句开始执行,如果一切正常,那么{\tt except}将被跳过。如果发生异常,则跳出{\tt try}语句块,执行{\tt except}中的代码。

使用{\tt try}语句处理异常被称为{\bf 捕获}异常。在本例中,{\tt except}语句块中的代码仅仅打印了错误信息。通常,捕获异常给了你修补问题的机会,你可以继续尝试,或者至少可以优雅的结束程序。


\section{数据库}

\index{数据库}

{\bf 数据库}是用于存储数据的文件。大多数的数据库以字典的形式组织,即将键映射为值。数据库是保存在磁盘(或其他永久存储设备)上,因此即使程序结束它们仍然存在。


\index{anydbm模块}
\index{模块!anydbm}

模块{\tt anydbm}提供了创建和跟新数据库文件的接口。作为一个例子,我将创建一个包含图片文件标题的数据库。

\index{open函数}
\index{函数!open}

打开数据库和其他文件类似:

\beforeverb
\begin{verbatim}
>>> import anydbm
>>> db = anydbm.open('captions.db', 'c')
\end{verbatim}
\afterverb
%
模式\verb"'c'"代表如果文件不存在则创建文件。返回结果是一个数据库对象,它可以像字典一样被使用(对于大多数的操作)。如果你创建一个新的项目, {\tt anydbm}将更新数据库文件。

\index{更新!数据库}


\beforeverb
\begin{verbatim}
>>> db['cleese.png'] = 'Photo of John Cleese.'
\end{verbatim}
\afterverb
%
当你访问某个项目是,{\tt anydbm}将读取文件:

\beforeverb
\begin{verbatim}
>>> print db['cleese.png']
Photo of John Cleese.
\end{verbatim}
\afterverb
%
如果你对已有的键再次进行赋值,{\tt anydbm}将替代旧的值:

\beforeverb
\begin{verbatim}
>>> db['cleese.png'] = 'Photo of John Cleese doing a silly walk.'
>>> print db['cleese.png']
Photo of John Cleese doing a silly walk.
\end{verbatim}
\afterverb
%
许多字典的方法,如{\tt keys}和{\tt items},同样适用于数据库对象,包括使用{\tt for}语句实现的迭代:

\index{字典方法!anydbm模块}

\beforeverb
\begin{verbatim}
for key in db:
    print key
\end{verbatim}
\afterverb
%
和其他文件一样,当你完成操作后你需要关闭文件:

\beforeverb
\begin{verbatim}
>>> db.close()
\end{verbatim}
\afterverb
%

\index{close方法}
\index{方法!close}


\section{Pickling}

\index{pickling}

{\tt anydbm}模块的一个限制在于键和值必须是字符串。如果你尝试使用其他数据类型,你会得到一个错误。

\index{pickle模块}
\index{模块!pickle}

{\tt pickle}模块可以解决这个问题。它能将任何类型的对象翻译成适合在数据库中储存的字符窜,同时能将字符串还原成对象。

{\tt pickle.dumps}读取一个对象作为参数,并返回一个表征字符串({\tt dumps}是“dump string”(转储字符串)的缩写):

\beforeverb
\begin{verbatim}
>>> import pickle
>>> t = [1, 2, 3]
>>> pickle.dumps(t)
'(lp0\nI1\naI2\naI3\na.'
\end{verbatim}
\afterverb
%
这个格式对人类读者来说不是很好理解,但是对{\tt pickle}来说很好解释。{\tt pickle.loads}(“载入字符串”)可以重建对象:

\beforeverb
\begin{verbatim}
>>> t1 = [1, 2, 3]
>>> s = pickle.dumps(t1)
>>> t2 = pickle.loads(s)
>>> print t2
[1, 2, 3]
\end{verbatim}
\afterverb
%
虽然新的对象和老的对象有相同的值,它们(通常)不是同一个对象:

\beforeverb
\begin{verbatim}
>>> t1 == t2
True
>>> t1 is t2
False
\end{verbatim}
\afterverb
%
换言之,pickling然后unpickling等效于复制一个对象。

你可以使用{\tt pickle}在数据库中存储一个非字符串对象。事实上,这个组合非常常用,并有一个已经封装好的模块{\tt shelve}。

\index{shelve模块}
\index{模块!shelve}


\begin{ex}

\index{回文集合}
\index{集合!回文}

如果你做了章节~\ref{回文}中的练习,修改你的方案,创建一个数据库,将列表中的单词映射为使用同样字母的单词的列表。

编写另一个程序,读取数据库并以人类适宜阅读的格式打印内容。
\end{ex}


\section{管道}

\index{shell}
\index{管道}

大多数的操作系统提供一个命令行的接口,称为{\bf shell}。shell通常提供浏览文件系统和启动程序的命令。例如,在Unix系统中,你可以使用{\tt cd}改变目录,使用{\tt ls}显示一个目录的内容,通过输入{\tt firefox}来启动一个网页浏览器。

\index{ls (Unix 命令)}
\index{Unix 命令!ls}

任何你在shell中可以启动的程序也可以在Python中通过使用{\bf 管道}来启动。一个管道是一个表示活动进程的对象。

例如,Unix命令{\tt ls -l}将以详细格式显示当前目录下的内容。你可以使用{\tt os.popen}来启动{\tt ls}:

\index{popen函数}
\index{函数!popen}

\beforeverb
\begin{verbatim}
>>> cmd = 'ls -l'
>>> fp = os.popen(cmd)
\end{verbatim}
\afterverb
%
参数是一个包含shell命令的字符串。返回值是一个对象,就像打开一个文件一样。你可以使用{\tt readline}每次从输出读取一样,或者使用{\tt read}一次读取所有内容:

\index{readline方法}
\index{方法!readline}
\index{read方法}
\index{方法!read}

\beforeverb
\begin{verbatim}
>>> res = fp.read()
\end{verbatim}
\afterverb
%
当你完成操作后,你像关闭一个文件一样关闭管道:

\index{close方法}
\index{方法!close}

\beforeverb
\begin{verbatim}
>>> stat = fp.close()
>>> print stat
None
\end{verbatim}
\afterverb
%
返回值是{\tt ls}进程的最终状态,{\tt None}表示它正常结束(没有错误)。

\index{文件!压缩}
\index{压缩!文件}
\index{Unix 命令!gunzip}
\index{gunzip (Unix命令)}

管道的一个常用用法是增量的读取一个压缩文件,即不需要一次解压整个文件。下面的函数读取一个压缩文件名作为参数,使用{\tt gunzip}来解压文件,并返回一个管道:

\beforeverb
\begin{verbatim}
def open_gunzip(filename):
    cmd = 'gunzip -c ' + filename
    fp = os.popen(cmd)
    return fp
\end{verbatim}
\afterverb
%
如果你每次从{\tt fp}读取一行,你不需要将解压的文件保存在内存或磁盘上。


\section{编写模块}
\label{模块}

\index{模块,编写}
\index{单词统计}

任何包含Python代码的文件可以作为模块被导入。例如,假设你有文件{\tt wc.py},内容如下:

\beforeverb
\begin{verbatim}
def linecount(filename):
    count = 0
    for line in open(filename):
        count += 1
    return count

print linecount('wc.py')
\end{verbatim}
\afterverb
%
如果你运行这个程序,它将读取自身,并打印行数,结果是7。你也可以这样导入模块:

\beforeverb
\begin{verbatim}
>>> import wc
7
\end{verbatim}
\afterverb
%
现在你有一个模块对象{\tt wc}:

\index{模块对象}
\index{对象!模块}

\beforeverb
\begin{verbatim}
>>> print wc
<module 'wc' from 'wc.py'>
\end{verbatim}
\afterverb
%
这里提供了一个称为\verb"linecount"的函数:

\beforeverb
\begin{verbatim}
>>> wc.linecount('wc.py')
7
\end{verbatim}
\afterverb
%
以上就是如何编写Python模块。

这个例子中唯一的问题在于当你导入模块后,最后的测试代码被执行。通常当你导入一个模块,它定义新的函数,但并不执行它们。

\index{import语句}
\index{语句!import}


作为模块的程序通常写成一下结构:

\beforeverb
\begin{verbatim}
if __name__ == '__main__':
    print linecount('wc.py')
\end{verbatim}
\afterverb
%
\verb"__name__"是一个程序开始时设置的内建变量。如果程序以脚本的形式运行,\verb"__name__"的值为\verb"__main__",在这种情况下,测试代码将被执行。否则,如果是作为模块被导入,测试代码将被忽略。

\begin{ex}
将例子输入到文件{\tt wc.py}中,并以脚本形式运行。如果在Python解释器中运行{\tt import wc},当模块被导入后,\verb"__name__"的值是什么?

警告:当你再次导入一个已经导入的的模块,Python将什么也不错。它不会重新读取文件,即使文件发生了改变。

\index{模块!重载}
\index{reload函数}
\index{函数!reload}
如果你要重载一个模块,你可以使用内建函数{\tt reload},但它可能会出错,因此最安全的方法是重启解释器然后重新导入模块。
\end{ex}



\section{调试}

\index{调试}
\index{空白}

当你读写文件,你也许会遇到空白带来的问题。由于空格符、tab符和换行符通常是不可见的,这样的错误很难调试:

\beforeverb
\begin{verbatim}
>>> s = '1 2\t 3\n 4'
>>> print s
1 2	 3
 4
\end{verbatim}
\afterverb

\index{repr函数}
\index{函数!repr}
\index{字符串表示}

内建函数{\tt repr}可以用来解决这个问题。它读取一个对象作为参数,并返回一个表示这个对象的字符串。对于字符串,它将空白符号用反斜杠序列表示:

\beforeverb
\begin{verbatim}
>>> print repr(s)
'1 2\t 3\n 4'
\end{verbatim}
\afterverb

这个对调试很有用。

你也许会遇到另一个问题,不同的操作系统使用不同的符号作为换行符。有的系统使用\verb"\n",有的使用\verb"\r",有的使用两者。如果你在不同系统中使用,这些差异会导致问题。

\index{行结束符号}
大多数系统提供了格式转换的程序。你可以在\url{wikipedia.org/wiki/Newline}中找到(并阅读更多相关内容)。当然你可以自己编写一个转换程序。


\section{术语}

\begin{description}

\item[持久性:] 一个长期运行、并至少将一部分数据保存在永久性的储存设备上的程序。
\index{持久性}

\item[格式运算符:] 运算符{\tt \%},参数为一个格式字符串和一个元组,生成一个按格式字符串规定的元组中元素的值的字符串。
\index{格式运算符}
\index{运算符!格式}

\item[格式字符串:] 使用格式运算符,包含格式序列的字符串。
\index{格式字符串}

\item[格式序列:] 格式字符串中的字符序列,类似{\tt \%d},指定了一个值的格式。
\index{格式序列}

\item[文本文件:] 保存在类似硬盘的永久储存设备上的字符序列。
\index{文本文件}

\item[目录:] 一个命名的文件的集合,也称为文件夹。
\index{目录}

\item[路径:] 用于识别文件的字符串。
\index{路径}

\item[相对路径:] 从当前目录开始的路径。
\index{相对路径}

\item[绝对路径:] 从文件系统顶部开始的路径。
\index{绝对路径}

\item[捕获:] 为了防止程序因为异常而终止,使用{\tt try}和{\tt except}语句来捕捉异常。
\index{捕获}

\item[数据库:] 一个类似字典使用键对应值的文件。
\index{数据库}

\end{description}


\section{练习}

\begin{ex}
\label{urllib}

\index{urllib模块}
\index{模块!urllib}
\index{URL}

{\tt urllib}模块提供了操作URL和从互联网下载信息的方法。下面的例子从{\tt thinkpython.com}下载并打印一条秘密信息:

\beforeverb
\begin{verbatim}
import urllib

conn = urllib.urlopen('http://thinkpython.com/secret.html')
for line in conn.fp:
    print line.strip()
\end{verbatim}
\afterverb

运行程序,运行结果将给你下一步指令。

\index{秘密练习}
\index{练习,秘密}

\end{ex}

\begin{ex}
\label{和校验}

\index{MP3}

在一个有很多MP3文件的收藏中,有可能同一首歌有多个拷贝,以不同的名字保存在不同的目录下。这个练习的目的是找出这些拷贝。

\begin{enumerate}

\item 编写程序,递归地搜索一个目录和所有子目录,返回一个完整路径的后缀给定的(如{\tt .mp3})的文件名列表。提示:{\tt os.path}提供了几个有用的操作文件和路径名的函数。

\index{拷贝}
\index{MD5算法}
\index{算法!MD5}
\index{和校验}

\item 为了识别拷贝,你可以使用哈希函数,读取文件并生成一个针对内容的简短的概述。例如,MD5 (Message-Digest algorithm 5)读取一个任意长的“消息”并返回一个128比特的“校验和”。两个不同文件返回相同的校验和的概率非常小。

你可以在\url{wikipedia.org/wiki/Md5}了解有关MD5的知识。在一个Unix系统上你可以使用{\tt md5sum}程序和Python中的管道来计算校验和。

\index{管道}

\end{enumerate}

\end{ex}


\begin{ex}

\index{网络电影数据库(IMDb)}
\index{IMDb (网络电影数据库)}
\index{数据库}

网络电影数据库(IMDb)是一个在线的电影信息收集的网站。它们的数据库可以以纯文本的格式获得,便于Python的读取。在这个练习中,你需要文件{\tt actors.list.gz}和{\tt actresses.list.gz},它们可以从\url{www.imdb.com/interfaces#plain}下载。

\index{纯文本}
\index{文本!纯}
\index{解析}

我编写了一个程序,可以解析这些文件并分割为演员名、电影标题等。你可以在\url{thinkpython.com/code/imdb.py}下载。

如果你已脚本方式运行{\tt imdb.py},程序将读取{\tt actors.list.gz}并在每行答应演员-电影对。或者你可以{\tt import imdb},调用函数\verb"process_file"来处理这些文件。参数为一个文件名,一个函数对象和一个可选的指定处理行数的数。下面给出一个例子:

\beforeverb
\begin{verbatim}
import imdb

def print_info(actor, date, title, role):
    print actor, date, title, role

imdb.process_file('actors.list.gz', print_info)
\end{verbatim}
\afterverb

当你调用\verb"process_file",它打开{\tt filename},读取内容,并对文件中的每行调用\verb"print_info"。\verb"print_info"读取演员、日期、电影标题和角色作为参数,并打印这些信息。

\begin{enumerate}

\item 编写函数,读取{\tt actors.list.gz}和{\tt actresses.list.gz},使用{\tt shelve}来构建一个数据库,将每个演员映射到他或她的电影列表。

\index{shelve模块}
\index{模块!shelve}

\item 两个演员称为是“合演”的,如果他们至少一起演过一部电影。基于上一步建立的数据库,构建第二个数据库,将每个演员映射到他或她“合演”的演员列表。

\index{Bacon, Kevin}
\index{Kevin Bacon Game}

 \item 编写程序,实现“Six Degrees of Kevin Bacon”,你可以在\url{wikipedia.org/wiki/Six_Degrees_of_Kevin_Bacon}中了解相关信息。这个问题的挑战在于你需要在关系图上找到最短路径。你可以在\url{wikipedia.org/wiki/Shortest_path_problem}里阅读最短路径算法相关的资料。

\end{enumerate}

\end{ex}


\chapter{类和对象}


\chapter{类和函数}
\label{ 时间}


\section{时间}

作为用户定义类型的另一个例子,我们将定义一个{\tt Time}类,记录当前时间,类的定义如下:

\index{用户定义类型}
\index{类型!用户定义}
\index{Time类}
\index{类!Time}

\beforeverb
\begin{verbatim}
class Time(object):
    """represents the time of day.
       attributes: hour, minute, second"""
\end{verbatim}
\afterverb
%
我们可以创建一个新的{\tt Time}对象,并对时、分和秒进行赋值:

\beforeverb
\begin{verbatim}
time = Time()
time.hour = 11
time.minute = 59
time.second = 30
\end{verbatim}
\afterverb
%
{\tt Time}对象的状态图如下:

\index{状态图}
\index{图!状态}
\index{对象图}
\index{图!对象}

\beforefig
\centerline{\includegraphics{figs/time.eps}}
\afterfig

\begin{ex}
\label{printtime}
编写函数\verb"print_time",参数为一个时间对象,以{\tt 时:分:秒}的格式打印时间。提示:格式字符串\verb"'%.2d'"使用至少两位打印一个整数,如果需要则在前面添零。
\end{ex}

\begin{ex}
\label{is_after}

\index{布尔函数}
编写布尔函数\verb"is_after",读取两个时间对象{\tt t1}和{\tt t2},如果{\tt t1}在{\tt t2}之后则返回{\tt True},否则返回{\tt False}。挑战:不使用{\tt if}语句。
\end{ex}


\section{纯函数}

\index{原型和补丁}
\index{开发方案!原型和补丁}
在下面几个章节中,我们将编写两个函数,实现时间相加的功能。它们将展示两种函数:纯函数和修改。同时将给出一个我称为{\bf 原型和补丁}的开发计划,即对于一个复杂的问题,从简单的原型开始,增量地处理其中的复杂问题。

下面给出\verb"add_time"的一个简单原型:

\beforeverb
\begin{verbatim}
def add_time(t1, t2):
    sum = Time()
    sum.hour = t1.hour + t2.hour
    sum.minute = t1.minute + t2.minute
    sum.second = t1.second + t2.second
    return sum
\end{verbatim}
\afterverb
%
这个函数创建一个新的{\tt Time}对象,初始化其属性并作为引用返回给一个新的对象。这被称为{\bf 纯函数},因为它不改变任何作为参数的对象,除了返回一个值它没有其他作用,类似显示一个值或读取用户输入。

\index{纯函数}
\index{函数类型!纯}

我创建了两个时间对象来测试这个函数,{\tt start}包含了一个电影开始的时间,如{\em Monty Python and the Holy Grail},{\tt duration}包含了电影的时间长度,是1小时35分钟。

\index{Monty Python and the Holy Grail}

\verb"add_time"给出电影结束的时间。

\beforeverb
\begin{verbatim}
>>> start = Time()
>>> start.hour = 9
>>> start.minute = 45
>>> start.second =  0

>>> duration = Time()
>>> duration.hour = 1
>>> duration.minute = 35
>>> duration.second = 0

>>> done = add_time(start, duration)
>>> print_time(done)
10:80:00
\end{verbatim}
\afterverb
%
{\tt 10:80:00}不是你所想要的结果。问题在于这个函数没有处理分钟和秒钟加起来超过60的情况。当这个情况发生时,我们需要将多余的秒钟“进位”到分钟,将多余的分钟“进位”到小时。

\index{进位,加法}

下面给出一个改进的版本:

\beforeverb
\begin{verbatim}
def add_time(t1, t2):
    sum = Time()
    sum.hour = t1.hour + t2.hour
    sum.minute = t1.minute + t2.minute
    sum.second = t1.second + t2.second

    if sum.second >= 60:
        sum.second -= 60
        sum.minute += 1

    if sum.minute >= 60:
        sum.minute -= 60
        sum.hour += 1

    return sum
\end{verbatim}
\afterverb
%
虽然这个函数是正确的,但是它开始变得冗长。我们之后会看见一个精简的版本。


\section{修改函数}
\label{增量}

\index{修改函数}
\index{函数类型!修改}

有时让函数修改参数对象是很有用的。这中情况下,修改对调用者是可见的。这样工作的函数被称为{\bf 修改函数}。

\index{increment}

{\tt increment}是将一定秒数加到一个{\tt 时间}对象,可以写成一个修改函数。下面是一个草稿:

\beforeverb
\begin{verbatim}
def increment(time, seconds):
    time.second += seconds

    if time.second >= 60:
        time.second -= 60
        time.minute += 1

    if time.minute >= 60:
        time.minute -= 60
        time.hour += 1
\end{verbatim}
\afterverb
%
第一行执行基本的操作,后面几行处理我们之前遇到过的特殊情况。

\index{特殊情况}

这个函数对吗?如果参数{\tt seconds}大于60会怎么样?

在这种情况下,进位一次是不够的,我们需要不断进位直到{\tt time.second}小于60。一个解决方案是使用{\tt while}语句替换{\tt if}语句。这能是函数工作正常,但不是很有效率。

\begin{ex}
编写一个正确的{\tt increment},不使用任何循环。
\end{ex}

任何修改函数可以做的都可以使用纯函数来实现。事实上有的编程语言只允许纯函数。有些证据证明使用纯函数的程序相比使用修改函数的程序开发更快捷,错误更少。但是修改函数更加方便使用,而函数的编程效率相对较低。

通常,我推荐你使用纯函数,除非修改函数有明显的优势。这个称为{\bf 函数式编程风格}。

\index{函数式编程风格}


\begin{ex}
编写纯函数版本的{\tt increment},创建一个新的时间对象并返回,而不是修改参数。
\end{ex}


\section{原型与计划}
\label{原型}

\index{原型和补丁}
\index{开发计划!原型和补丁}
\index{有计划的开发}
\index{开发计划!有计划的}
我在展示的开发计划被称为“原型和补丁”。对于每个函数,我编写实现基本功能的原型并进行测试,并对错误打补丁。

这个方法会很有效率,尤其是对问题没有一个深入的认识。但是增量的修改会使得代码变得不必要的复杂,因为需要处理不同的特殊情况,同时由于你很难知道是否找到了所有的错误,代码也不可靠。

另一种是{\bf 有计划的开发},从高层次分析问题将会简化程序的设计。在这个例子中,对问题的分析在于认识到时间对象是3个60进制的数(参考\url{wikipedia.org/wiki/Sexagesimal}。)!{\tt 秒}是“第1列”,{\tt minute}是“第60列”,{\tt 小时}是“第360列”。

\index{六十进制}

当我们编写\verb"add_time"和{\tt increment},我们完成了基60的加法,这也是为什么我们需要从一列到另一列进位。

\index{进位,加法}

这个观察给出了解决整个问题的另一个方法,我们可以将时间对象转换为整数,并利用计算机进行整数计算。

下面的函数将时间转换为整数:

\beforeverb
\begin{verbatim}
def time_to_int(time):
    minutes = time.hour * 60 + time.minute
    seconds = minutes * 60 + time.second
    return seconds
\end{verbatim}
\afterverb
%
下面的函数将整数转换为时间(回忆{\tt divmod}将第一个参数除以第二个参数,并返回商和余数的元组)。

\index{divmod}

\beforeverb
\begin{verbatim}
def int_to_time(seconds):
    time = Time()
    minutes, time.second = divmod(seconds, 60)
    time.hour, time.minute = divmod(minutes, 60)
    return time
\end{verbatim}
\afterverb
%
你也许需要一些思考,并运行一些测试来确保这些函数工作正常。一个测试方法是对许多{\tt x}值检查\verb"time_to_int(int_to_time(x)) == x"。这是一个强壮型检查的例子。

\index{强壮型检查}

当你确信它们是正确的,你可以使用它们重写\verb"add_time":

\beforeverb
\begin{verbatim}
def add_time(t1, t2):
    seconds = time_to_int(t1) + time_to_int(t2)
    return int_to_time(seconds)
\end{verbatim}
\afterverb
%
这个版本比原来的简洁,同时也更容易验证。

\begin{ex}
使用\verb"time_to_int"和\verb"int_to_time"重写{\tt increment}。
\end{ex}

有时候,60进制和10进制的相互转换比处理时间更难。基数转换相对更抽象,我们的直觉更擅长处理时间。

但是如果我们将时间看成60进制的数,并编写转换函数(\verb"time_to_int"和\verb"int_to_time"),我们使得程序更简短,更适合阅读和调试,以及更可靠。

同时也方便以后增加新的特性。例如,想象将两个时间相减,得到两者之间的间隔。最直观的方法是实现借位减法。使用转换函数可以更简单,也更容易正确。

\index{借位减法}
\index{借位,减法}
\index{普遍化}

讽刺的是有时候将问题复杂化(或普遍化)实际简化了问题(因为特殊情况变少,同时出错概率减小)。


\section{调试}
\index{调试}

一个时间对象被称为是良好组织的,如果{\tt 分钟}和{\tt 秒钟}位于0到60(包括0但不包括60),{\tt hours}是正的,{\tt 小时}和{\tt 分钟}是整数,但我们可以允许{\tt 秒钟}有小数部分。

\index{约束}

类似这些要求被称为{\bf 约束},它们应该始终为真。换言之,如果它们非真,则有些地方就有错误。

编写程序检查约束可以帮助你检查错误并找出原因。例如,你可以编写函数\verb"valid_time",读取一个时间对象作为参数,如果违反了约束则返回{\tt False}:

\beforeverb
\begin{verbatim}
def valid_time(time):
    if time.hours < 0 or time.minutes < 0 or time.seconds < 0:
        return False
    if time.minutes >= 60 or time.seconds >= 60:
        return False
    return True
\end{verbatim}
\afterverb
%
在每个函数的开头你可以检查参数来保证它们是有效的:

\index{raise语句}
\index{语句!raise}

\beforeverb
\begin{verbatim}
def add_time(t1, t2):
    if not valid_time(t1) or not valid_time(t2):
        raise ValueError, 'invalid Time object in add_time'
    seconds = time_to_int(t1) + time_to_int(t2)
    return int_to_time(seconds)
\end{verbatim}
\afterverb
%
或者你可以使用{\tt assert}语句,它将检查一个给定的约束,如果检查失败则会发出一个异常错误。

\index{assert语句}
\index{语句!assert}

\beforeverb
\begin{verbatim}
def add_time(t1, t2):
    assert valid_time(t1) and valid_time(t2)
    seconds = time_to_int(t1) + time_to_int(t2)
    return int_to_time(seconds)
\end{verbatim}
\afterverb
%
{\tt assert}语句很有用,它们区分普通的条件判断和异常检查。


\section{术语}

\begin{description}

\item[原型和补丁:] 一种开发计划,包括编写程序的草稿、测试、修改发现的错误。
\index{原型和补丁}

\item[有计划的开发:] 一种开发计划,包括从高层次对程序进行分析,相对增量开发或原型开发有更多的计划。
\index{有计划的开发}

\item[纯函数:] 不修改作为参数的对象的函数。
\index{纯函数}

\item[修改函数:] 修改一个或多个作为参数的对象的函数。
\index{修改函数}

\item[函数式编程风格:] 一种程序设计模式,将大多数函数设计为纯函数。
\index{函数时编程风格}

\item[约束:] 在程序执行时必须始终为真的条件。
\index{约束}

\end{description}


\section{练习}

\begin{ex}
编写函数\verb"mul_time",参数为一个时间对象和一个数,返回一个新的时间对象,其值是原时间和数的乘积。

使用\verb"mul_time"编写一个函数,参数为一个时间对象和一个数值,时间对象表示完成一个比赛所用的时间,数值表示距离,返回一个时间对象,其意义是平均速度(英里每单位时间)。

\index{跑步速度}

\end{ex}

\begin{ex}

\index{Date类}
\index{类!Date}
编写日期对象的类定义,含有属性{\tt 日},{\tt 月}和{\tt 年}。编写函数\verb"increment_date",参数为一个日期对象{\tt date}和一个整数{\tt n},返回值为一个新的日期对象,对应{\tt date}后的{\tt n}天。提示:“2月没有30天...”挑战:你的函数在闰年工作正常吗?参考\url{wikipedia.org/wiki/Leap_year}。

\end{ex}


\begin{ex}

\index{datetime模块}
\index{模块!datetime}

{\tt datetime}模块提供了类似本章节中的日期和时间对象,{\tt date}和{\tt time},它们提供了丰富的方法和运算符,阅读\url{docs.python.org/lib/datetime-date.html}中的文档。

\begin{enumerate}

\item 使用{\tt datetime}模块编写程序,读取一个日期,打印这个日期所在的周。

\index{生日}

\item 编写程序,读如一个生日,打印用户的年龄,以及多少天、小时、分钟和秒钟后是下一个生日。
\end{enumerate}

\end{ex}



\chapter{类和方法}


\chapter{继承}

在本章中我们将设计一个来玩扑克的类。如果你不玩扑克,你可以在\url{wikipedia.org/wiki/Poker}中阅读相关信息,但你不必这么做,做练习时我会告诉你一些所需要的信息。

\index{玩扑克,Anglo-American}
\index{纸牌,玩}
\index{扑克}

如果你不熟悉如何玩扑克,你可以阅读\url{wikipedia.org/wiki/Playing_cards}中的内容。


\section{扑克对象}

一副牌有52张牌,每一张属于4个花色和13个等级。4中花色是黑桃、红桃、草花和方块。13个等级是Ace,2,3,4,5,6,7,8,9,10,J,Q和K。对于不同的游戏,Ace肯能比2大,也可能比2小。

\index{等级}
\index{花色}

如果我们要定义一个纸牌的对象,显然其属性有{\tt 等级}和{\tt 花色},但是属性的数据类型应该如何选择?一个方法是使用字符串,如用\verb"'Spade'"描述花色, \verb"'Queen'"描述等级。这么做的一个问题在于实现花色或等级的比较不是很容易。

\index{编码}
\index{加密}
\index{映射}
\index{表示}

另一个方法是使用整数对等级和花色{\bf 编码}。在这里,“编码”指我们将定义一个数字和花色以及数字和等级之间的映射。这种映射不是秘密的(区别“加密”)。

例如,下面的表格给出了花色和对应整数编码:

\beforefig
\begin{tabular}{l c l}
Spades & $\mapsto$ & 3 \\
Hearts & $\mapsto$ & 2 \\
Diamonds & $\mapsto$ & 1 \\
Clubs & $\mapsto$ & 0
\end{tabular}
\afterfig

这个编码简化了卡片的比较,由于高的花色映射为大的数,我们可以通过比较编码来比较花色。

等级的映射相对更明显,每个数字等级映射为对应的整数,对于其他卡片:

\beforefig
\begin{tabular}{l c l}
Jack & $\mapsto$ & 11 \\
Queen & $\mapsto$ & 12 \\
King & $\mapsto$ & 13 \\
\end{tabular}
\afterfig

我使用$\mapsto$符号来突出这些映射不是Python程序实现的。它们属于程序的设计,但它们不直接的表现在代码中。

\index{Card类}
\index{类!Card}

{\tt Card}类的定义如下:

\beforeverb
\begin{verbatim}
class Card(object):
    """represents a standard playing card."""

    def __init__(self, suit=0, rank=2):
        self.suit = suit
        self.rank = rank
\end{verbatim}
\afterverb
%
Init方法有两个可选参数,默认的卡片是草花2。

\index{init方法}
\index{方法!init}

如果要创建一个卡片,你可以调用{\tt Card},并传入你要的花色和等级。

\beforeverb
\begin{verbatim}
queen_of_diamonds = Card(1, 12)
\end{verbatim}
\afterverb
%


\section{类属性}

\index{类属性}
\index{属性!类}

为了以适合人类阅读的格式打印卡片对象,我们需要整数代码到对应花色和等级的映射。一个自然的方法是使用字符串列表,我们将这两个列表赋值给{\bf 类属性}:

\beforeverb
\begin{verbatim}
# inside class Card:

    suit_names = ['Clubs', 'Diamonds', 'Hearts', 'Spades']
    rank_names = [None, 'Ace', '2', '3', '4', '5', '6', '7', 
              '8', '9', '10', 'Jack', 'Queen', 'King']

    def __str__(self):
        return '%s of %s' % (Card.rank_names[self.rank],
                             Card.suit_names[self.suit])
\end{verbatim}
\afterverb
%
定义在类内部任何函数外的变量,如\verb"suit_names"和\verb"rank_names",被称为类属性,因为它们和类对象{\tt Card}像关联。

\index{实例属性}
\index{属性!实例}

该术语区别类似{\tt suit}和{\tt rank}的变量,它们被称为{\bf 实例属性},因为它们和一个特定的实例相关联。

\index{点符号}

两种属性都通过点符号访问。例如,在\verb"__str__"里,{\tt self}是一个卡片对象,{\tt self.rank}是它的等级。类似的,{\tt Card}是一个类对象,\verb"Card.rank_names"是一个和类关联的字符串列表。

每个卡片都有自己的{\tt 花色}和{\tt 等级},但是有一个\verb"suit_names"和\verb"rank_names"的备份。

将这些结合起来,表达式\verb"Card.rank_names[self.rank]"意思是“从{\tt self}对象中使用属性{\tt rank}作为下标,访问{\tt Card}类中的\verb"rank_names"列表,选择对应的字符串”。

\verb"rank_names"中的第一个元素是{\tt None},因为没有等级为零的卡片。通过引入{\tt None}作为占位符,我们很好的将下标2映射为字符串\verb"'2'",以此类推。若不使用这种方法,我们可以使用一个字典,而不是一个列表。

基于我们已有的方法,我们可以创建并打印卡片:

\beforeverb
\begin{verbatim}
>>> card1 = Card(2, 11)
>>> print card1
Jack of Hearts
\end{verbatim}
\afterverb
%
下面给出{\tt Card}类对象和一个卡片实例的图:

\index{状态图}
\index{图!状态}
\index{对象图}
\index{图!对象}

\beforefig
\centerline{\includegraphics{figs/card1.eps}}
\afterfig

{\tt Card}是一个类对象,它的数据类型为{\tt type}。{\tt card1}的类型为{\tt Card}。(为了节省空间,我没有画出\verb"suit_names"和\verb"rank_names"的内容)。


\section{比较卡片}
\label{比较卡片}

\index{运算夫!关系}
\index{关系运算夫}

对于内建类型,关系运算符(如{\tt <},{\tt >},{\tt ==}等)可以比较它们的值。对于用户定义的类型,我们可以通过提供\verb"__cmp__"函数来重载内建运算符的行为。


\verb"__cmp__"接受两个参数,{\tt self}和{\tt other},如果第一个对象大则返回一个正数,如果第一个对象小则返回一个负数,如果两者两等则返回0。

\index{重载}
\index{运算符重载}

卡片的正确顺序不是很明显。例如一个草花3和一个方块2哪个大?一个等级更高,而另一个花色更高。为了比较卡片,你需要确定等级和花色哪个更重要。

答案取决于你玩的是什么游戏。为了简单起见,我们任意的选择花色更为重要,因此所有的黑桃都大于任意方块,依此类推。

\index{cmp 方法@\_\_cmp\_\_ 方法}
\index{方法!\_\_cmp\_\_}

当这个决定后,我们可以编写\verb"__cmp__":

\beforeverb
\begin{verbatim}
# inside class Card:

    def __cmp__(self, other):
        # check the suits
        if self.suit > other.suit: return 1
        if self.suit < other.suit: return -1

        # suits are the same... check ranks
        if self.rank > other.rank: return 1
        if self.rank < other.rank: return -1

        # ranks are the same... it's a tie
        return 0    
\end{verbatim}
\afterverb
%
你可以使用元组比较编写更简洁的程序:

\index{元组!比较}
\index{比较!元组}

\beforeverb
\begin{verbatim}
# inside class Card:

    def __cmp__(self, other):
        t1 = self.suit, self.rank
        t2 = other.suit, other.rank
        return cmp(t1, t2)
\end{verbatim}
\afterverb
%
内建函数{\tt cmp}和方法\verb"__cmp__"有相同的接口:它读取两个值,如果第一个更大则返回一个正数,第二个更大则返回一个负数,如果相等则返回0。

\index{cmp函数}
\index{函数!cmp}


\begin{ex}
为时间对象编写方法\verb"__cmp__"。提示:你可以使用元组比较,也可以考虑使用整数减法。

%    def __cmp__(self, other):
%        return time_to_int(self) - time_to_int(other)

%If {\tt self} is later than {\tt other}, the result is
%a positive number.  If {\tt other} is later, the result
%is negative.  And if {\tt self} and {\tt other} are equal
%(but not necessarily identical)
%the result is zero.

\end{ex}


\section{纸牌}
\index{列表!对象的}
\index{纸牌,玩卡片}

现在我们有卡片对象,下一步是定义纸牌。由于纸牌是由卡片组成的,自然的想法是每个纸牌将一个卡片的列表作为它的属性。

\index{init方法}
\index{方法!init}

下面是{\tt 纸牌}的类定义。初始化方法创建{\tt 卡片}属性,并生成标准的52张卡片:

\index{组成}
\index{循环!嵌套}

\index{纸牌类}
\index{类!纸牌}

\beforeverb
\begin{verbatim}
class Deck(object):

    def __init__(self):
        self.cards = []
        for suit in range(4):
            for rank in range(1, 14):
                card = Card(suit, rank)
                self.cards.append(card)
\end{verbatim}
\afterverb
%
生成纸牌最简单的方法是使用一个嵌套的循环。外层循环枚举0到3的花色,内层循环枚举1到13的等级。每次迭代创建一个对应当前花色和等级的新卡片,并附加在{\tt self.cards}后。

\index{append方法}
\index{方法!append}


\section{打印纸牌}
\label{打印纸牌}

\index{str 方法@\_\_str\_\_ 方法}
\index{方法!\_\_str\_\_}

下面是{\tt 纸牌}的\verb"__str__"方法:

\beforeverb
\begin{verbatim}
#inside class Deck:

    def __str__(self):
        res = []
        for card in self.cards:
            res.append(str(card))
        return '\n'.join(res)
\end{verbatim}
\afterverb
%
这个方法演示了汇聚大字符串的有效的方法:建立一个字符串列表,然后使用{\tt join}方法。内建函数{\tt str}对每个卡片调用\verb"__str__"方法,并返回表征的字符串。

\index{汇聚!字符串}
\index{字符串!汇聚}
\index{join方法}
\index{方法!join}
\index{新行}

由于我们在新行符号上调用{\tt join},卡片通过新行符分割。下面是结果:

\beforeverb
\begin{verbatim}
>>> deck = Deck()
>>> print deck
Ace of Clubs
2 of Clubs
3 of Clubs
...
10 of Spades
Jack of Spades
Queen of Spades
King of Spades
\end{verbatim}
\afterverb
%
虽然结果看起来是52行,实际上它是一个包含换行符的一个长字符串。


\section{添加,删除,洗牌,理牌}

对于卡片,我们需要一个从纸牌中删除一张卡片并返回改纸牌的方法。列表的{\tt pop}方法提供了一个便捷的实现方式:

\index{pop方法}
\index{方法!pop}

\beforeverb
\begin{verbatim}
#inside class Deck:

    def pop_card(self):
        return self.cards.pop()
\end{verbatim}
\afterverb
%
由于{\tt pop}删除列表中{\em 最后}一张卡片,我们是纸牌底部进行操作。在现实生活中从底部操作是不可取的\footnote{参考\url{wikipedia.org/wiki/Bottom_dealing}},但在这里也是行得通的。

\index{append方法}
\index{方法!append}

为了添加一张卡片,我们可以使用列表的{\tt append}方法:

\beforeverb
\begin{verbatim}
#inside class Deck:

    def add_card(self, card):
        self.cards.append(card)
\end{verbatim}
\afterverb
%
类似这种调用其他函数,本身不做很多具体工作的方法有时被称为{\bf 饰面}。这个词来自木工行业,通常人们将一片很薄的好木材黏贴在一块便宜的木材的表面。

\index{饰面}

在本例中我们定义个一个“单薄”的方法,使列表的操作适用于纸牌。

作为另一个例子,我们可以编写纸牌方法{\tt shuffle},调用{\tt random}模块中的{\tt shuffle}函数: 

\index{random模块}
\index{模块!random}
\index{shuffle函数}
\index{函数!shuffle}

\beforeverb
\begin{verbatim}
# inside class Deck:
            
    def shuffle(self):
        random.shuffle(self.cards)
\end{verbatim}
\afterverb
%
不要忘了导入{\tt random}模块。

\begin{ex}
\index{sort方法}
\index{方法!sort}

编写纸牌方法{\tt sort},通过调用列表方法{\tt sort}对{\tt 纸牌}中的卡片进行排序。{\tt sort}适用我们定义的\verb"__cmp__"方法来决定排序顺序。
\end{ex}



\section{继成}

\index{继成}
\index{面向对象编程}

和面向对象编程最相关的语言特点是{\bf 继成}。继成是通过修改已有的类来创建新的类的能力。

\index{父类}
\index{子类}
\index{类!子}
\index{子类}
\index{超类}

之所以称为“继成”是因为新的类继成了已有类的方法。换言之,已有类被称为{\bf 父类},新创建的类被称为{\bf 子类}。

例如,我们要创建一个类来表示“手牌”,即一个玩家手中持有的牌。手牌类似纸牌:它们都是卡片的集合,都需要类似添加和删除卡片的操作。

手牌同时区别于纸牌:我们需要手牌有一些操作,但这些操作对纸牌来说没有意义。例如,在扑克中我们需要比较两个手牌来决定谁获胜。在桥牌中,我们需要计算手牌的分数,以便要价。

两个类的相似关系和不同点导致了它们的继成关系。

子类的定义和其他类定义相同,但父类的名字出现在括号中:

\index{括号!父类在}
\index{父类}
\index{类!父}
\index{Hand类}
\index{类!Hand}

\beforeverb
\begin{verbatim}
class Hand(Deck):
    """represents a hand of playing cards"""
\end{verbatim}
\afterverb
%
该定义表示{\tt Hand}继成{\tt Deck},这意味着我们可以想纸牌一样对手牌使用类似\verb"pop_card"和\verb"add_card"的方法。

{\tt Hand}同时继成了{\tt Deck}的\verb"__init__",但这并不是我们想要的:手牌的{\tt cards}属性应该被初始化为一个空的列表,而不是塞满52张新牌。


\index{重载}
\index{init方法}
\index{方法!init}

如果我们给{\tt Hand}提供一个初始化方法,它将重载{\tt Deck}中的方法:

\beforeverb
\begin{verbatim}
# inside class Hand:

    def __init__(self, label=''):
        self.cards = []
        self.label = label
\end{verbatim}
\afterverb
%
因此当你创建了一个手牌对象,Python调用这个初始化方法:

\beforeverb
\begin{verbatim}
>>> hand = Hand('new hand')
>>> print hand.cards
[]
>>> print hand.label
new hand
\end{verbatim}
\afterverb
%
但是其他方法从{\tt Deck}继成,所以我们可以使用\verb"pop_card"和\verb"add_card"来处理卡片:

\beforeverb
\begin{verbatim}
>>> deck = Deck()
>>> card = deck.pop_card()
>>> hand.add_card(card)
>>> print hand
King of Spades
\end{verbatim}
\afterverb
%
自然的下一步是将这些代码封装成\verb"move_cards"方法:

\index{封装}

\beforeverb
\begin{verbatim}
#inside class Deck:

    def move_cards(self, hand, num):
        for i in range(num):
            hand.add_card(self.pop_card())
\end{verbatim}
\afterverb
%
\verb"move_cards"读取两个参数,一个为手牌对象,一个为要处理的卡片数。它同时修改{\tt self}和{\tt hand},并返回{\tt None}。

在有的游戏中,卡片在不同手牌中移动,或者从手牌移动到台面牌。你可以使用\verb"move_cards"来进行任何这些操作:{\tt self}可以是手牌或者桌面牌,{\tt hand}同样可以是{\tt 桌面牌},虽然它名字叫手牌。

\begin{ex}
编写纸牌方法\verb"deal_hands",读取两个参数,分别是手牌的个数和每个手牌的卡片个数,并根据这个数字创建手牌对象,返回一个手牌对象列表。
\end{ex}
继成是一个有用的特性。通过使用继成可以将一些重复行的程序写得更优雅。继成使得代码重用更容易,你可以自定义父类的行为,而不需要去改变它。在某些情况下,继成结构反应了现实中问题的结构,使得问题容易理解。

另一方面,继成会使得程序难以阅读。当一个方法被调用时,有时方法的定义的位置不是很清楚。相关的代码可能散布在多个模块中。同时,很多可以用继成做的事情同样可以不用继成来实现,甚至做得更好。


\section{类图}

目前位置我们见过了栈图,它现实了程序的状态,对象图,它现实了对象的属性和值。这些图是程序执行过程中的快照,因此它们随程序的执行而改变。

同时它们非常详细,有时对某些应用太详细了。类图是个相对抽象的表示程序结构的图,它显示了类的结构和类之间的关系,而不是显示每一个对象。

存在以下几种类的关系:

\begin{itemize}

\item 类中的一个对象包含另一个类的对象的引用。例如,每个长方形包含一个点的应用,每个纸牌包含多个卡片的引用。这类关系被称为{\bf 包含},正如“长方形包含一个点”。

\item 一个类是另一个类的继成。这种关系被称为{\bf 是},正如“手牌是纸牌的一种”。

\item 一个类也许决定于另一个类,即一个类的改动要求其他类的改动。

\end{itemize}

\index{属于关系}
\index{包含关系}
\index{类图}
\index{图!类}
\index{UML}

{\bf 类图}是图形化的表示这些关系的图\footnote{我在这里使用的图类似UML(参考\url{wikipedia.org/wiki/Unified_Modeling_Language})}。例如,下面的图显示了{\tt Card},{\tt Deck}和{\tt Hand}的关系。

\beforefig
\centerline{\includegraphics{figs/class1.eps}}
\afterfig

空心三角箭头表示“属于关系”,在本例中表示手牌继成了纸牌。

标准箭头表示“有关系”,在本例中纸牌有卡片的引用。

\index{多态(在类图中)}

在箭头头部的新号({\tt *})表示{\bf 多态}。它指示了纸牌中有多少张卡片。多态可以是一个简单的数字,如{\tt 52},一个范围,如{\tt 5..7},或一个新号,表示纸牌可以有任何多的卡片。

更详细的图将显示纸牌事实上又一个卡片{\em 列表},但是通常类似列表和字典的内建类型不出现在类图中。

\begin{ex}
阅读{\tt TurtleWorld.py},{\tt World.py}和{\tt Gui.py},绘制类图来显示它们之间的关系。
\end{ex}


\section{调试}
\index{调试}

继承使得调试成为一个挑战,因为当你对一个对象调用方法是,你也许不知道哪个方法被调用。

\index{多态}

假设你编写了一个手牌对象的函数,你希望它能适用于任何手牌,如扑克手牌,桥牌手牌等。假设你调用了类似{\tt shuffle}的方法,你也许调用的是定义在{\tt 纸牌}中的方法,但是如果某个子类重载了这个方法,你会得到那个版本的方法。

\index{执行流程}
当你不确定程序执行的流程,最简单的方法是在相关的方法的开始部分添加打印语句。如果{\tt Deck.shuffle}打印类似{\tt Running Deck.shuffle}的语句,那么当程序执行时它将追踪执行的流程。

另一个方法是使用下面的函数,它接受一个对象和一个方法名(以字符串的形式),返回提供该方法定义的类:

\beforeverb
\begin{verbatim}
def find_defining_class(obj, meth_name):
    for ty in type(obj).mro():
        if meth_name in ty.__dict__:
            return ty
\end{verbatim}
\afterverb
%
下面给出一个例子:

\beforeverb
\begin{verbatim}
>>> hand = Hand()
>>> print find_defining_class(hand, 'shuffle')
<class 'Card.Deck'>
\end{verbatim}
\afterverb
%
因此这个手牌的{\tt shuffle}方法来自{\tt 纸牌}。

\index{mro方法}
\index{方法!mro}
\index{方法解析顺序}

\verb"find_defining_class"使用{\tt mro}方法得到用于查找方法的类对象(类型)列表。“MRO”指“method resolution order”,即方法解析顺序。

\index{重载}
\index{接口}
\index{先决条件}
\index{后决条件}

以下是程序设计的建议:每当你重载一个方法,新方法的接口应该和原方法相同。它们应使用相同的参数,返回相同的类型,符合相同的先决条件和后决条件。如果你遵守这些规则,你会发现任何使用于超类实例(如纸牌)的函数同样使用于子类实例(如手牌或扑克手牌)。

如果你违反了这些规则,你的代码很肯会崩溃。


\section{术语}

\begin{description}

\item[编码:] 通过建立映射使用一个集合的值表示另一个集合的值。
\index{编码}

\item[类属性:] 和类对象关联的属性。类属性定义在类定义内部,方法定义外部。
\index{类属性}
\index{属性!类}

\item[实例属性:] 和类的实例相关的属性。
\index{实例属性}
\index{属性!实例}

\item[饰面:] 为另一个函数提供不同的接口而不进行许多计算的方法或函数。
\index{饰面}

\item[继承:] 通过修改先前定义的类来创建新的类的能力。
\index{继承}

\item[父类:] 被子类继承的类。
\index{父类}

\item[子类:] 通过继承已有的类而创建的新的类。
\index{子类}
\index{类!子}

\item[属于关系:] 子类和父类之间的关系。
\index{属于关系}

\item[包含关系:] 两个类之间的关系,其中的一个类包含另一个类的引用。
\index{包含关系}

\item[类图:] 表示程序中的类以及类之间的关系的图。
\index{类图}
\index{图!类}

\item[多态:] 类图中的标记,对于包含关系,该标记说明了对其他类的引用的数量。
\index{多态(在类图中)}

\end{description}


\section{练习}

\begin{ex}
\index{扑克}

下面是扑克中肯能的手牌,以大小升序排列(概率降序):

\begin{description}

\item[对子:] 等级相同的两张牌。
\vspace{-0.05in}

\item[双对:] 等级相同的两对对子。
\vspace{-0.05in}

\item[相同的3个:] 等级相同的3张卡片。
\vspace{-0.05in}

\item[顺子:] 5张等级连续的卡片(aces可以为高或低,因此{\tt Ace-2-3-4-5}是个顺子,{\tt 10-Jack-Queen-King-Ace}也是顺子,但是{\tt Queen-King-Ace-2-3}不是)。
\vspace{-0.05in}

\item[同花:] 5张花色相同的卡片。
\vspace{-0.05in}

\item[3张相同和2张相同的牌:] 3张牌等级相同,另2张牌等级相同。
\vspace{-0.05in}

\item[相同的4个] 等级相同的4张牌。
\vspace{-0.05in}

\item[同花顺:] 5张花色相同的顺子。
\vspace{-0.05in}

\end{description}
%
本练习的目的是估计抽到不同的手牌的概率。

\begin{enumerate}

\item 从\url{thinkpython.com/code}下载下列文件:

\begin{description}

\item[{\tt Card.py}]: 本章中{\tt Card},{\tt Deck}和{\tt Hand}的完整版本。

\item[{\tt PokerHand.py}]: 未完成的手牌的类,包含一些测试代码。

\end{description}
%
\item 如果你运行{\tt PokerHand.py},它将抽取7张手牌,并检查是否包含同花顺。在你继续前请仔细阅读代码。

\item 在{\tt PokerHand.py}添加方法\verb"has_pair",\verb"has_twopair"等,根据相关准则进行判断并返回True或者False。你的代码应使用于任意张数的手牌(虽然5和7是最常见的数量)。

\item 编写方法{\tt classify},找出手牌中值最大的分类,并记录在{\tt label}属性中。例如,一个7张的手牌可能有一个顺子和一对,那么应该标记为“顺子”。

\item 当你确信你的分类方法工作正常,下一步是估计不同手牌出现的概率。在{\tt PokerHand.py}中编写函数,对一副牌进行洗牌,然后分为若干手牌,对手牌分类,统计不同分类出现的次数。

\item 打印分类和概率的表。多次运行程序,直到输出稳定到一个合理的精度。将你的结果和\url{wikipedia.org/wiki/Hand_rankings}比较。

\end{enumerate}
\end{ex}


\begin{ex}

\index{沼泽}
\index{TurtleWorld}
本练习使用章节~\ref{turtlechap}的TurtleWorld。
你将编写代码,让乌龟玩贴标签的游戏。如果你不熟悉游戏规则,参考\url{wikipedia.org/wiki/Tag_(game)}。

\begin{enumerate}

\item 下载\url{thinkpython.com/code/Wobbler.py}并运行。你将看到乌龟世界里有3只乌龟。如果你按下{\sf Run}按钮,乌龟将随机移动。

\item 阅读代码,了解它是如何工作的。{\tt 闲逛者}类继承{\tt 乌龟}类,这意味着{\tt 乌龟}的方法{\tt lt},{\tt rt},{\tt fd}和{\tt bk}同样适用于闲逛者。

{\tt step}方法由乌龟世界调用。它调用{\tt steer},令乌龟朝向指定的方向。{\tt wobble}根据乌龟的笨拙程度随机的转向,{\tt move}根据乌龟的速度向前移动一定的距离。

\index{跟踪者}

\item 创建{\tt Tagger.py},导入{\tt Wobbler},定义类{\tt Tagger},继承{\tt Wobbler}。以{\tt Tagger}的类对象作为参数的调用\verb"make_world"。

\item 在{\tt Tagger}中添加{\tt steer}方法,重载{\tt Wobbler}中的函数。作为起步练习,编写始终将乌龟指向原点的版本。提示:使用数学函数{\tt atan2}和乌龟属性{\tt x},{\tt y}和{\tt heading}。

\item 修改{\tt steer},将乌龟约束在边界内。为了调试,你可以使用{\sf Step}按钮,它将对每个乌龟调用{\tt step}。

\item 修改{\tt steer},令每只乌龟朝向离它最近的邻居。提示:乌龟有属性{\tt world},是它们所在的乌龟世界的引用,而乌龟世界有属性{\tt animals},对应所有乌龟的列表。

\item 修改{\tt steer}让乌龟玩贴标签游戏。你可以在{\tt Tagger}上添加方法,并重载{\tt steer}和\verb"__init__",但你不能修改或重载{\tt step},{\tt wobble}或{\tt move}。另外{\tt steer}允许改变乌龟的方向,但不能改变位置。

修改规则和你的{\tt steer}方法,增加游戏的质量。例如,慢的乌龟应该有机会给快的乌龟贴上标签。

\end{enumerate}

你可以在\url{thinkpython.com/code/Tagger.py}下载到我的程序。
\end{ex}




\chapter{实例学习:Tkinter}


\appendix

\chapter{调试}

\index{调试}

程序中会出现不同的错误,加以区分有助于加快对错误的跟踪。

\begin{itemize}

\item 语法错误是Python在将源代码转换为字节码时产生的。它们通常说明程序的语法有错误。例如:在 {\tt def}语句后忽略冒号会产生类似{\tt SyntaxError: invalid syntax}的信息。

\item 运行时错误由解释器在程序运行出错时产生。大多数的运行时错误消息包含错误发生的位置和正在执行的函数。例如,一个无穷递归的函数最终导致运行时错误“maximum recursion depth exceeded”。

\item 语义错误指程序执行过程中没有产生出错信息,但程序没有做正确的工作。例如,一个表达式没有按照你期望的顺序进行求值,导致错误的结果。

\end{itemize}

\index{语法错误}
\index{运行时错误}
\index{语义错误}
\index{错误!编译时}
\index{错误!语法}
\index{错误!运行时}
\index{错误!语义}
\index{异常}

调试的第一步是找出你遇到了什么类型的错误。虽然下面的章节根据错误类型组织的,但一些方法使用于多种情况。


\section{语法错误}

\index{错误消息}

当你找到原因后语法错误一般很容易修正。不幸的是,错误消息常常不是很有帮助。最常见的消息是{\tt SyntaxError: invalid syntax}和{\tt SyntaxError: invalid token},它们都不包含很多信息量。

另一方面,错误消息告诉你问题发生在程序中的什么位置。事实上,当Python遇到问题时它将告诉你,但这不一定是错误的地方。有时错误在错误消息位置的前面,通常是前几行。

\index{增量开发}
\index{开发计划!增量}

如果你增量的开发程序,你会清楚错误在哪里。它就是你最新添加的代码。

如果你从书中复制一段代码,那么从仔细地检查你的代码和书中的代码开始。检查每一个字符。同时记住书本也会有错误,如果你遇到类似语法错误,也许它就是。

下面给出一些避免常见语法错误的方法:

\index{syntax}

\begin{enumerate}

\item 避免使用Python关键字作为变量名。

\index{关键字}

\item 确保你在每个复合语句头的尾部加上了冒号,包括{\tt for},{\tt while},{\tt if}和{\tt def}。

\index{头部}
\index{冒号}

\item 确保代码中的字符串都有匹配的引号。

\index{引号标记}

\item 如果你用三重引号标记多行字符串,确保你正确的结束该字符串。一个未终止的字符串会导致程序尾部{\tt invalid token}错误,或者程序接下来的部分都被认为字符串,直到下一个字符串。在第二种情况,Python可能不会产生一个错误消息!

\index{多行}
\index{字符创!多行}

\item Python将未封闭的运算符---\verb+(+,\verb+{+,或\verb+[+---的下一行作为当前语句的一部分。通常第二行将产生错误。

\item 在条件语句中使用传统的{\tt =}而不是{\tt ==}。

\index{条件}

\item 确保每行缩进正确。Python可以处理空格和tabs,但如果你混淆它们会产生错误。避免这个问题最好的方法是使用适合Python编辑、会自动缩进的文本编辑器。

\index{缩进}
\index{空白符}

\end{enumerate}

如果这些没有效,继续阅读下一章节...


\subsection{我做了修改但是没有任何区别}

如果解释器报错,但你找不到错误,有可能你和解释器看到不不是相同的代码。检查你的编程环境,确保你正在编辑的文件是Python将要运行的。

如果你不确定,在程序的开始位置添加明显的故意的语法错误,再次运行,如果解释器没有找到这个错误,说明你不在运行新的程序。

下面是一些可能的出错原因:

\begin{itemize}

\item 你编辑了文件,运行前忘记了保存。有的编程环境会为你自动保存,有的不会。

\item 你修改了文件的名字,但仍然运行老的名字。

\item 你的开发环境配置有误。

\item 如果你在编写模块并使用{\tt import},确保你写的模块名不同于Python标准模块名。

\index{模块!reload}
\index{reload函数}
\index{函数!reload}

\item 如果你使用{\tt import}读取一个模块,记得重启解释器或使用{\tt reload}来读取一个修改过的文件。如果你再次导入模块,解释器将不做任何事。

\end{itemize}

如果你陷入困境找不到出错的原因,一个方法是从一个类似“Hello,World!”的新程序开始,确保你从一个已知的可运行的程序开始。然后逐步添加原程序中的代码到新程序。


\section{运行时错误}

当你的程序没有语法错误,Python可以编译并运行。这时可能发生什么错误呢?


\subsection{我的程序什么也没做}

这个问题通常发生在你的文件包含函数和类,但没有任何调用执行的语句。也许你故意这样,因为你仅仅打算导入这个模块来提供函数和类。

如果这不是故意的,确保你调用一个函数来开始执行,或从一个交互的命令提示行下执行。参见下面的“执行流程”章节。


\subsection{我的程序挂起了}
\index{无限循环}
\index{无限递归}
\index{挂起}

如果一个程序停止,但上去什么也没做,我们称“挂起”。通常意味着程序陷入一个无限循环或者无线递归。

\begin{itemize}

\item 如果你怀疑某个循环引起这个问题,在循环开始时添加{\tt print}语句打印“进入循环”,在循环结束时打印“exiting the loop”退出循环。

运行程序,如果你的到第一条消息,但没有第二条消息,你得到一个无限循环。参见“无限循环”章节。

\item 大多数时候,无限循环会令程序运行一段时间,然后产生“RuntimeError: Maximum recursion depth exceeded”的错误。如果是这样,参考下面的“无限递归”章节。

如果你没有得到这样的错误信息,但你怀疑某个递归方法或函数有问题,你仍可以使用“无限递归”章节中提到的方法。

\item 如果这些方法都没有,尝试测试其他的循环、递归的函数和方法。

\item 如果还是没有效果,有可能你不清楚程序执行的流程。参考下面的“执行流程”章节。

\end{itemize}


\subsubsection{无限循环}
\index{无限循环}
\index{循环!无限}
\index{条件}
\index{循环!条件}

如果你有一个无限循环并且你认为这个循环导致了问题,在循环体结束的地方添加{\tt print}语句打印条件语句中的变量值和条件值。

例如:

\beforeverb
\begin{verbatim}
while x > 0 and y < 0 :
    # do something to x
    # do something to y

    print  "x: ", x
    print  "y: ", y
    print  "condition: ", (x > 0 and y < 0)
\end{verbatim}
\afterverb
%
现在当你运行程序,每个循环你都会看到3行输出。对于最后一次循环,条件应该为{\tt false}。如果循环不断执行,你可以看到{\tt x}和{\tt y}的值,也许能够发现为什么它们没有被正确的更新。


\subsubsection{无限递归}
\index{无限递归}
\index{递归!无限}

大多数时候,无限循环会令程序运行一段时间,然后产生“RuntimeError: Maximum recursion depth exceeded”的错误。

如果你怀疑一个函数或方法导致了无限递归,首先检查存在一个基本状态。换言之,应该有一个状态使函数或方法直接返回,而不是再次递归调用。如果不是,你需要重新思考算法,并设计一个基本状态。

如果存在一个基本状态,但程序似乎没有运行到这个状态,你可以在函数或方法开始的部分添加{\tt print}语句打印参数。现在当你运行程序,每次函数或方法被调用时,你将看到几行关于参数的输出。如果参数没有向基本状态靠拢,你也许会发现为什么这样。


\subsubsection{执行流程}
\index{执行流程}

如果你不确定程序的执行流程,在每个函数开始的地方加入{\tt print}语句,打印类似“entering function {\tt foo}”的语句,其中{\tt foo}是函数的名字。

现在当你运行程序,它将打印每个函数被调用的追踪。


\subsection{当我运行程序,我得到一个异常}
\index{异常}
\index{运行时错误}

有时程序运行时出错,Python将打印异常的名字、问题发生的位置和回溯。

\index{回溯}

回溯识别当前运行的函数,然后识别调用该函数的函数,以此类推。换言之,它追踪了你到达这个错误所经过的函数调用。同时它也包括这些调用在文件中的位置。

第一步是检查代码中对应的出错的位置,或许你能发现错误。下面是常见的运行时错误:

\begin{description}

\item[名字错误:]  你试图使用一个不在当前环境下的变量。记住局部变量是局部的,你不能在定义它们的函数的外部引用它们。

\index{名字错误}
\index{类型错误}
\index{异常!名字错误}
\index{异常!类型错误}

\item[类型错误:] 可能的几个原因是:

\begin{itemize}

\item 你试图不适当的使用值。例如:使用一个非整数的值作为字符串、列表或元组的下标。

\index{下标}

\item 格式字符串和用于转换的对象不匹配。这发生在或者数量不相同,或者调用了一个非法的转换。

\index{格式运算符}
\index{运算符!格式}

\item 你调用函数或方法时传递的参数个数有误。对于方法,检查方法定义,确保第一个参数是{\tt self},然后检查方法调用,确保你对一个对象调用方法,并正确的提供其他参数。

\end{itemize}

\item[键错误:]  你试图使用字典中不存在的键访问对应的元素。

\index{键错误}
\index{异常!键错误}
\index{字典}

\item[属性错误:] 你试图访问一个不存在的属性或方法。检查拼写!你可以使用{\tt dir}来列举存在的属性。

如果属性错误指出一个对象是{\tt 空类型},及它是{\tt 空}的。一个常见的原因是函数结束时忘记返回值。如果在返回尾部没有{\tt return}语句,函数将返回一个{\tt None}。另一个原因是使用类似列表中的{\tt sort}方法,它返回{\tt None}。

\index{属性错误}
\index{异常!属性错误}

\item[下标错误:] 用来访问列表、字符串或元组的下标超过长度减1。在错误发生的前一行使用{\tt print}语句显示下标值和序列的长度,检查两个值是否正确。

\index{下标错误}
\index{异常!下标错误}

\end{description}

\index{调试器(pdb)}
\index{Python调试器(pdb)}
\index{pdb(Python调试器)}

Python调试器({\tt pdb})允许你在错误前检查程序状态,有助于追踪异常。你可以在 \url{docs.python.org/lib/module-pdb.html}阅读有关{\tt pdb}的内容。


\subsection{我添加了太多的{\tt print}语句,输出令我应接不暇}

\index{print语句}
\index{语句!print}

使用{\tt print}语句调试的一个问题是你会被大量的输出信息掩埋。有两个处理方法:简化输出或简化程序。

简化输出可以删除或注释不用的{\tt print}语句,或将它们合并,或格式化输出使得它们容易理解。

简化程序你可以做一下几件事。首先缩小程序处理的问题的规模。例如,如果你在搜索一个列表,搜索一个{\em 小的}列表。如果程序从用户读取输入,输入最简单的参数。

\index{死区代码}

第二,整理程序,删除死区代码,使程序易读。例如,如果你认为问题深嵌在程序中,试图用简单的结构重写那部分。如果你怀疑一个大函数有错误,试图将它分割成小函数,然后分别测试。

\index{测试!最小测试案例}
\index{测试案例,最小}

通常寻找最小测试案例的过程让你找到错误。如果你发现一个程序对于一个情况适用,对另一个情况不使用,这将给你一些线索。

同样,重写一段代码有助于发现一些微小的错我。如果你做了一个你认为不会影响程序的修改,但事实上影响了,这个方法可以给你警戒。


\section{语义错误}
\index{语义错误}
\index{错误!语义}

在某种程度上,语义错误时最难调试的,因为解释器不能提供任何错误信息。你所知道的只有程序应该怎么做。

第一步是建立程序代码和你见到的行为之间的联系。你需要假设程序实际上做了什么。一个主要困难在于计算器运行的太快了。

你常希望你可以降低程序的速度,是人类可以跟上,通过使用调试器,你可以实现这步。但是在关键地方加入一些{\tt print}语句所用的时间通常短于建立调试器,加入删除断点,然后“逐步”运行程序直到错误发生。

\subsection{我的程序不工作}

你需要问自己这些问题:

\begin{itemize}

\item 有没有什么程序应该做却没有发生?找到执行该函数的代码段,确保程序被执行。

\item 有没有什么不应该发生的发生了?找到执行该函数的代码段,查看它是否执行了?

\item 有没有代码的执行效果与你期望的不同?确保你理解有问题的代码,尤其是包含调用其他Python模块中的函数或方法。阅读你调用的函数的文档,用一些简单的例子进行测试。

\end{itemize}

在编程时,你心中需要有一个关于程序如何工作的模型。如果你的程序没有按照你期望的工作,很可能问题不在于程序,而在于你心中的模型。

\index{模型,心中的}
\index{心中的模型}

修正你心中的模型的最好的方法是将程序分割为不同部分(通常是函数和方法),并分别测试。一旦你发现了模型和现实的差异,你就可以解决问题。

当然,在开发的过程中尼需要建立并测试组件。如果你遇到了问题,只有一小部分新的代码是不确定正确性的。


\subsection{我写了一个很长的表达式,它没有按照我期望的工作}

\index{表达式!大而复杂}
\index{大而复杂的表达式}

编写复杂的表达式是合理的,如果它们可读。但是它们调试起来很困难。通常我们将一个复杂的表达式分割成一系列的临时变量的赋值。

例如:

\beforeverb
\begin{verbatim}
self.hands[i].addCard(self.hands[self.findNeighbor(i)].popCard())
\end{verbatim}
\afterverb
%
可以重写为:

\beforeverb
\begin{verbatim}
neighbor = self.findNeighbor(i)
pickedCard = self.hands[neighbor].popCard()
self.hands[i].addCard(pickedCard)
\end{verbatim}
\afterverb
%
这个明晰的版本更适于阅读,因为变量名提供了额外的文档,同时调试也更容易,因为你可以检查中间变量的类型和它们的值。

\index{临时变量}
\index{变量!临时}
\index{运算符优先级}
\index{优先级}

大的表达式的另一个问题是计算的顺序不一定是你期望的。例如,如果你将表达式$\frac{x}{2 \pi}$翻译为Python,你也许会写成:

\beforeverb
\begin{verbatim}
y = x / 2 * math.pi
\end{verbatim}
\afterverb
%
这是不正确的,因为乘法和除法有相同的优先级,因此是从左往右计算的,这个表达式计算的是$x \pi / 2$。

调试表达式的一个好的方法是添加括号,使得计算顺序简洁明了:

\beforeverb
\begin{verbatim}
 y = x / (2 * math.pi)
\end{verbatim}
\afterverb
%
当你不确定计算优先级是,使用括号。不仅程序将工作正常(按你的要求执行),同时也让那些没有记住优先级规则的人阅读起来更方便。


\subsection{我的函数或方法没有按照我期望的返回}
\index{return语句}
\index{语句!return}

如果你的{\tt return}语句包含一个复杂的表达式,你没有机会在返回前打印{\tt 返回}值。同样你可以使用临时变量。例如:对于

\beforeverb
\begin{verbatim}
return self.hands[i].removeMatches()
\end{verbatim}
\afterverb
%
你可以写成:

\beforeverb
\begin{verbatim}
count = self.hands[i].removeMatches()
return count
\end{verbatim}
\afterverb
%
现在你在返回前可以打印{\tt count}的值。


\subsection{我实在是卡住了,我需要帮助}

第一,尝试离开电脑几分钟。电脑辐射会对大脑产生影响,导致下列几种症状:

\begin{itemize}

\item 沮丧和愤怒

\index{沮丧}
\index{愤怒}
\index{调试!情绪反应}
\index{带情绪的调试}

\item 迷信的人认为“电脑讨厌我”,并神奇的相信“程序仅当我向后戴着帽子时才工作正常”。

\index{调试!迷信}
\index{迷信的调试}

\item 随机漫步编程(用各种可能的方法编程,并选择工作正常的那个)。

\index{随机漫步编程}
\index{开发计划!随机漫步编程}

\end{itemize}

如果你发现你有以上任意一种症状,站起来走一走。当你心绪平静时,思考一下程序。它是做什么的?什么肯能造成了这种行为?上次可以工作的程序是什么时候?下一步做什么?

有时找到一个错误很费时间。我常常在我离开电脑,让思维游荡的时候找到错误。一些找到错误最好的地方有火车上,浴室里,以及临睡前。


\subsection{不,我真的需要帮助}

即使最好的程序员也会卡住。有时你在一个程序上工作了太长的时间,因此你难以发现错误。而他人可能一眼就发现问题。

在你向其他人寻求帮助前,你需要做好准备。你的程序需要尽可能简洁,你需要最少的输入来重现错误。你需要在合适的位置加入{\tt print}语句,同时输出应可理解。你需要能够以简洁的语言描述问题。

当你想某人求助,你需要提供足够的信息:

\begin{itemize}

\item 是否有出错消息?它是什么?指向程序的哪部分?

\item 错误出现前你做的最后一步是什么?你写的最后几行是什么?什么新的测试导致了错误?

\item 你做了哪些尝试?你学到了什么?

\end{itemize}

当你找到了错误,花时间想一想你怎么能更快的定位它。下一次你遇到类似的问题,你就可以更快的找到问题。

记住,目标不仅仅是让程序工作,而是学会如何让程序工作。

\printindex

\clearemptydoublepage

>>>>>>> hanjunchao/master

\end{document}
