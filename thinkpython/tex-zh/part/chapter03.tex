\chapter{函数}
\label{funcchap}
\index{function call 函数调用}

在程序设计中,函数是带有函数名的一系列执行计算的语句,当定义一个函数,我们指
定一个函数名和一系列的语句。然后,就可以通过函数名调用函数。我们其实已经看到
一个函数调用的例子。

\beforeverb
\begin{verbatim}
>>>type(32)
<type 'int'>
\end{verbatim}
\afterverb

函数名是类型,小括号里的表达式称作函数的形式参数(argument).结果是形参的类型。

\index{parenthess!argument in}

通常我们这么说:函数接受一个参数,返回一个结果(叫做返回值)。

\index{argument 形式参数}
\index{return value 返回值}

\section{类型转换函数}
\index{conversion!type 转换!类型}
\index{type conversion 类型转换}

Python提供一些内置函数用来把一种类型的值转换成另一类型。{\\t int}函数接受一
个值,如果可以,就把它转换成整数,否则就会“抱怨”。

\index{int function int 函数}
\index{function!int 函数!int}

\beforeverb
\begin{verbatim}
>>> int('32')
32
>>> int('Hello')
Traceback (most recent call last):
  File "<stdin>", line 1, in <module>
ValueError: invalid literal for int() with base 10: 'Hello'
\end{verbatim}
\afterverb

{\tt int}函数可以把浮点数转换为整数,但是不能向上取整,只能截掉小数部分:

\beforeverb
\begin{verbatim}
>>> int(3.99999)
3
>>> int(-2.3)
-2
\end{verbatim}
\afterverb

{\tt float}函数把整数和字符串转换成浮点数:

\index{float function float函数}
\index{function!float 函数!float}

\beforeverb
\begin{verbatim}
>>>float(32)
32.0
>>>float('3.14159')
3.14159
\end{verbatim}
\afterverb \footnote{在译者的机器上float('3.14159')的输出为:3.1415899999999999(解释器Python2.5和2.6);3.14159(解释器Python3.1)。}

最后,{\tt str}把参数转换为字符串:

\index{str function str函数}
\index{function!str 函数!str}

\beforeverb
\begin{verbatim}
>>>str(32)
'32'
>>>str(3.14159)
'3.14159'
\end{verbatim}
\afterverb

2236


