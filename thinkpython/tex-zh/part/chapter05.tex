\chapter{条件语句和递归}

\section{ 模操作符}

\index{modulus operator 模操作符}
\index{operator!modulus 操作符!模}

模操作符用于两个整数,第一个操作数除以第二个操作数产生余数。在Python
中,模操作符是一个百分号(\verb"%")。语法的格式和其他的操作符相同。

\beforeverb
\begin{verbatim}
>>>quotient = 7 / 3
>>>print quotient
2
>>>remainder = 7 % 3
>>>print remainder
1
\end{verbatim}
\afterverb

7除以3等于2余1。\\

模操作符是非常有用的,比如,你可以查看一个数是否可以被另一个数整除---
如果{\tt x \% y}是0,{\tt x}就可以被{\tt y}整除。\\

\index{divisibility 整除}

你也可以用模运算来提取整数的最右边的数字。比如,{\tt x \% 10 }得到{\tt x}的最右面的一个数字\footnote{译注:个位数}(以十为底)。类似地,
{\tt x \% 100}得到最后的两位数字\footnote{十位和个位的数字}。

\section{布尔表达式}
\index{boolean expression 布尔表达式}
\index{expression!boolean}
\index{logical operator 逻辑运算符}
\index{operator!logical}

布尔表达式的结果要么是真(true),要么为假(false)。下面的例子是使用
{\tt ==}运算符,比较两个操作数,如果相等则结果为{\tt True},否则为{\tt False}:

\beforeverb
\begin{verbatim}
>>> 5 == 5
True
>>> 5 == 6
False
\end{verbatim}
\afterverb

{\tt True}和{\tt False}是两个特殊的值,属于{\tt bool}类型;他们不是
字符串:

\index{True special value True特殊值}
\index{False special value False特殊值}
\index{specail value!True}
\index{special value!False}
\index{bool type bool类型}
\index{type!bool}

\beforeverb
\begin{verbatim}
>>> type(True)
<type 'bool'>
>>> type(False)
<type 'bool'>
\end{verbatim}
\afterverb

{\tt ==}运算符是关系运算符中的一个,其他的还有:

\beforeverb
\begin{verbatim}
      x != y               # x is not equal to y
      x > y                # x is greater than y
      x < y                # x is less than y
      x >= y               # x is greater than or equal to y
      x <= y               # x is less than or equal to y
\end{verbatim}
\afterverb


尽管你可能很熟悉这些运算符,他们在Python中的表示方法和数学中的有很大
的不同。一个常见的错误是只使用一个{\tt =}号,而不是两个{\tt ==}号。
记住{\tt =}是赋值操作符,{\tt ==}是关系运算符。而且,Python中没有这样
的符号{\tt =<}或者{\tt =>}\footnote{在FP(functional programming中可能
会遇到这个符号}。

\index{relational operator 关系运算符}
\index{operator!relational}

\section{逻辑运算符}
\index{logical operator 逻辑运算符}
\index{operator!logical}

有三个逻辑运算符:{\tt and},{\tt or}和{\tt not}。这些操作符的意思和
在英语中的意思差不多。比如,{\tt x > 0 and x < 10}为真,仅当{\tt x}
大于0小于10\footnote{译注:在Python中,更pythonic的写法是 0 < x < 10
。  这样的符号对于c/c++背景的程序员来说,有点陌生,在c/c++等值的分别
是\&\&, || ,!}。

\index{and operator and运算符}
\index{or operator or运算符}
\index{not operator not运算符}
\index{operator!and}
\index{operator!or}
\index{operator!not}


