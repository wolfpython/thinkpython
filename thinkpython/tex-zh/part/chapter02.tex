\chapter{变量、表达式和语句}

\section{变量和数据类型}
\index{变量}
\index{类型数据}
\index{字符串}

{\bf 变量}是程序处理的最基本的事物,就像一个字母或者一个数字。至今为止我们见过的变量有{\tt 1},{\tt 2}以及
\verb"'Hello, World!'"。

这些变量属于不同{\bf 数据类型}:{\tt 2}是一个整数,而\verb"'Hello, World!'"是一个{\bf 字符串},之所以这么称呼是因为它包含一“串”字母。因为被引号包围,读者(以及解释器)可以将它们识别为字符串。

\index{引号标记}

print命令同样适用于整数。

\beforeverb
\begin{verbatim}
>>> print 4
4
\end{verbatim}
\afterverb
%
如果你不确定一个变量的数据类型,解释器会告诉你。

\beforeverb
\begin{verbatim}
>>> type('Hello, World!')
<type 'str'>
>>> type(17)
<type 'int'>
\end{verbatim}
\afterverb
%
显然,字符串属于类型{\tt str},整数属于类型{\tt int}。同样,带小数点的数字属于{\tt float}类型,因为这些数字通过一种叫做{\bf floating-point}的数据类型来表示。

\index{数据类型}
\index{字符串类型}
\index{数据类型!字符串}
\index{整数类型}
\index{数据类型!整数}
\index{浮点数类型}
\index{数据类型!浮点数类型}

\beforeverb
\begin{verbatim}
>>> type(3.2)
<type 'float'>
\end{verbatim}
\afterverb
%
那么对于像\verb"'17'"和\verb"'3.2'"呢?
它们虽然看起来想数字,实际上由于它们在引号中,所以是字符串。

\index{引号标记}

\beforeverb
\begin{verbatim}
>>> type('17')
<type 'str'>
>>> type('3.2')
<type 'str'>
\end{verbatim}
\afterverb
%
它们是数字。

当你输入一个大整数时,你会习惯性地在3个数子中间加上逗号,如{\tt 1,000,000}。对于Python来说这不是一个有效的数字,但是这样的表示是合法的:

\beforeverb
\begin{verbatim}
>>> print 1,000,000
1 0 0
\end{verbatim}
\afterverb
%
当然,这不是我们所期待的。Python将{\tt 1,000,000}解释为一个用逗号划分的数字序列,在输出时用空格区分。

\index{语义错误}
\index{错误!语义错误}
\index{错误信息}

这是我们见到的第一个语义错误的例子:代码可以运行,没有报出错信息,但是没有做“正确”的事情。

\section{变量}
\index{变量}
\index{赋值语句}
\index{语句!赋值语句}

对{\bf 变量}的操作是编程语言最强大的特征之一。一个变量是一个对应值的名称。

通过{\bf 赋值语句}创建新的变量,并对它们赋值:

\beforeverb
\begin{verbatim}
>>> message = 'And now for something completely different'
>>> n = 17
>>> pi = 3.1415926535897931
\end{verbatim}
\afterverb
%
这个例子包含三条赋值语句。第一条将一个字符串赋值给一个叫做{\tt message}的变量;第二条将整数{\tt 17}赋值给变量{\tt n};第三条将$\pi$的近似值赋值给变量{\tt pi}。

\index{状态图表}
\index{图表!状态}

变量通常用表示为变量名用一个肩头指向变量的值。这样的图称作{\bf 状态图表},因为它给出了每个变量所处的状态(想象各个变量处在各自的状态)。以下的图表给出了之前例子中的结果:

\beforefig
\centerline{\includegraphics{figs/state2.eps}}
\afterfig

你可以使用一个print语句来显示变量的值:

\beforeverb
\begin{verbatim}
>>> print n
17
>>> print pi
3.14159265359
\end{verbatim}
\afterverb
%
一个变量的数据类型是这个变量对应值的数据类型。

\beforeverb
\begin{verbatim}
>>> type(message)
<type 'str'>
>>> type(n)
<type 'int'>
>>> type(pi)
<type 'float'>
\end{verbatim}
\afterverb
%
\begin{ex}
如果你在输入一个整数时以$0$开头,你会得到一个奇怪的错误:

\beforeverb
\begin{verbatim}
>>> zipcode = 02492
                  ^
SyntaxError: invalid token
\end{verbatim}
\afterverb

其他的数字也许可以工作,但是结果是诡异的:

\beforeverb
\begin{verbatim}
>>> zipcode = 02132
>>> print zipcode
1114
\end{verbatim}
\afterverb

你能指出这是为什么吗?提示:打印数值{\tt 01},{\tt 010},{\tt 0100}和{\tt 01000}。

\index{八进制}

\end{ex}

\section{变量命名和关键字}
\index{关键字}

程序员通常选取有意义的名字作为变量明——变量名说明了变量的作用。

变量名可以任意长。它们可以同时包含字母和数字,但必须以一个字母开头。大写字母的使用是合法的,但是一个好的习惯是使用小写字母作为变量的首字母(后面会说明原因)。

下划线符号(\verb"_")可以出现在一个变量名中,通常使用在有多个单词的名字中,如\verb"my_name"或者\verb"airspeed_of_unladen_swallow"。

\index{下划线字符}

如果你赋予了一个变量非法的名字,你会得到这样的一个语法错误:

\beforeverb
\begin{verbatim}
>>> 76trombones = 'big parade'
SyntaxError: invalid syntax
>>> more@ = 1000000
SyntaxError: invalid syntax
>>> class = 'Advanced Theoretical Zymurgy'
SyntaxError: invalid syntax
\end{verbatim}
\afterverb
%
{\tt 76trombones}是非法的,因为它不是有字母开始。{\tt more@}是非法的,因为它包含了非法字符{\tt @}。但是{\tt class}哪里错了呢?

事实上{\tt class}属于Python的一个{\bf 关键字}。解释器使用这些关键字来识别程序的结构,因此它们不能被用做变量名。

\index{关键字}

Python共有31个关键字\footnote{在Python 3.0中,{\tt exec}不再是一个关键字,但{\tt nonlocal}是。}:

\beforeverb
\begin{verbatim}
and       del       from      not       while    
as        elif      global    or        with     
assert    else      if        pass      yield    
break     except    import    print              
class     exec      in        raise              
continue  finally   is        return             
def       for       lambda    try
\end{verbatim}
\afterverb
%
你可以放一份关键字列表在手边。如果解释器报错说变量名有误,而你有不知道原因,看看这个变量名是否在关键字列表上。

\section{程序语句}

程序语句是一组Python解释器可以执行的代码。我们已经遇到过两种程序语句:print和assignment。

\index{程序语句}
\index{交互模式}
\index{脚本模式}

当你在交互模式下输入一条程序语句,解释器会执行该条语句,并显示结果,如果有结果的话。

一个脚本通常包含一系列程序语句。如果有多于一条程序语句,每执行一条语句将会显示对应的结果。

例如,脚本

\beforeverb
\begin{verbatim}
print 1
x = 2
print x
\end{verbatim}
\afterverb
%
会产生输出

\beforeverb
\begin{verbatim}
1
2
\end{verbatim}
\afterverb
%
赋值语句不产生输出。

\section{运算符和运算数}
\index{运算符,数学}
\index{数学运算符}
\index{运算数}
\index{表达式}

{\bf 运算符}是表示计算的特殊符号,如加法和乘法。运算符作用的数据被称为{\bf 运算数}。

运算符{\tt +}、{\tt -}、{\tt *}、{\tt /}和{\tt **}实现加法、减法、乘法、除法和指数运算,如下面的示例:

\beforeverb
\begin{verbatim}
20+32   hour-1   hour*60+minute   minute/60   5**2   (5+9)*(15-7)
\end{verbatim}
\afterverb
%
在一些其他的编程语言中,\verb"^"被用做指数运算,但是在Python中这是一个位操作,称为XOR。在本书中将不会涉及位操作的知识,你可以在\url{wiki.python.org/moin/BitwiseOperators}中阅读相关知识。

\index{位操作}
\index{运算符!按位}

%When a variable name appears in the place of an operand, it
%is replaced with its value before the operation is
%performed.

除法运算符有可能不按照你期望的运行:

\beforeverb
\begin{verbatim}
>>> minute = 59
>>> minute/60
0
\end{verbatim}
\afterverb
%
{\tt minute}的值是59,在传统的算术中,59除以60的结果是0.98333,而不是0。产生偏差的原因是因为Python执行了{\bf 下取整除法}\footnote{在Python 3.0中,这个除法的结果是一个{\tt 浮点数}。新的运算符{\tt //}执行取整除法。}。

\index{Python 3.0}
\index{下取整除法}
\index{浮点除法}
\index{除法!下取整}
\index{除法!浮点}

当两个运算数都是整数,结果同样是整数;下取整除法舍去了小数部分,因此在这个例子中结果为0。

如果任意一个运算数是一个浮点数,Python执行浮点除法,此时结果为{\tt 浮点数}:

\beforeverb
\begin{verbatim}
>>> minute/60.0
0.98333333333333328
\end{verbatim}
\afterverb


\section{表达式}

{\bf 表达式}是数值、变量和运算符的组合。一个数值是一个表达式,一个变量也是一个表达式,因此一下都是合法的表达式(假设变量{\tt x}已经被赋值):

\index{表达式}
\index{赋值}

\beforeverb
\begin{verbatim}
17
x
x + 17
\end{verbatim}
\afterverb
%
如果你在交互模式下输入一个表达式,解释器立即对它{\bf 赋值}并显示结果:

\beforeverb
\begin{verbatim}
>>> 1 + 1
2
\end{verbatim}
\afterverb
%
但在脚本中,一个表达式本身不做任何事情!初学者通常对此会产生困惑。

\begin{ex}
在交互模式下输入一下表达式,观察结果:

\beforeverb
\begin{verbatim}
5
x = 5
x + 1
\end{verbatim}
\afterverb
%
现在将这些表达式放在一个脚本文件里并运行。此时输出是什么?修改脚本,在每个表达式前添加print语句,然后观察输出。
\end{ex}


\section{运算符的优先级}
\index{运算符的优先级}
\index{优先级规则}
\index{PEMDAS}

当一个表达式中出现多于一个运算符时,求值的顺序由{\bf 优先级规则}决定。Python遵从数学运算符的约定。{\bf PEMDAS}是一个有效的方法来记忆这些规则:

\index{括号!压倒优先级}

\begin{itemize}

\item {\bf P}arentheses(括号)具有最高的优先级,可以强迫表达式按照指定的顺序求值。由于在括号中的表达式先被求值,{\tt 2 * (3-1)}的结果是4,{\tt (1+1)**(5-2)}的结果是8。你可以用括号来使得表达式更容易阅读,如{\tt (minute * 100) / 60},即便这并不改变最终的结果。

\item {\bf E}xponentiation(指数运算)有次高的优先级,因此{\tt 2**1+1}的结果是3,而不是4,{\tt 3*1**3}的结果是3,而不是27。

\item {\bf M}ultiplication(乘法)和{\bf D}ivision(除法)具有相同的优先级,并优于{\bf A}ddition(加法)和{\bf S}ubtraction(减法),它们同样具有相同的优先级。因此{\tt 6+4/2}的结果是8,而不是5。

\item 具有相同优先级的运算符是从左到右进行求值的。因此在表达式{\tt degrees / 2 * pi}中,首先计算除法,然后在结果上乘以{\tt pi}。如果要除以$2 \pi$,你需要使用括号,或者使用{\tt degrees / 2 / pi}。

\end{itemize}


\section{字符串操作}
\index{字符串!操作}
\index{运算符!字符串}

一般情况下,你不能使用算术运算符作用于字符串,即使字符串看起来像数字,因此以下是非法的:

\beforeverb
\begin{verbatim}
'2'-'1'    'eggs'/'easy'    'third'*'a charm'
\end{verbatim}
\afterverb
%
运算符{\tt +}可以作用于字符串,但未必按照你期望的方式工作:它执行的是{\bf 级联},即将字符串首尾相连。例如:

\index{级联}

\beforeverb
\begin{verbatim}
first = 'throat'
second = 'warbler'
print first + second
\end{verbatim}
\afterverb
%
程序的输出是{\tt throatwarbler}。

运算符{\tt *}同样可以作用于字符串,它的功能是重复字符串。如\verb"'Spam'*3"的结果是\verb"'SpamSpamSpam'"。如果一个运算符是字符串,那么另一个必须是一个整数。

运算符{\tt +}和{\tt *}在字符串的的使用类似加法和乘法,如{\tt 4*3}相当于{\tt 4+4+4},因此我们期望\verb"'Spam'*3"相当于\verb"'Spam'+'Spam'+'Spam'",事实如此。当然,字符串的级联和重复相比整数的加法和乘法还是有很大的区别的。你可以找出一个加法有而字符串级联没有的性质吗?


\index{交换性}


\section{注释}
\index{注释}

当程序变得越来越长,越来越复杂,它们变得更难读懂。正是的变成语言是密集的,通常很难通过阅读一小段代码来指出它是做什么的,或者为这么这么写。

由于这个理由,在程序旁添加标记,使用自然语言来说明这段程序是做什么的是一个良好的想法。这些标记被称为{\bf 注释},它们由\verb"#"符号开头:

\beforeverb
\begin{verbatim}
# compute the percentage of the hour that has elapsed
percentage = (minute * 100) / 60
\end{verbatim}
\afterverb
%
在这个例子中,注释以整行的方式出现。你也可以在一行的末尾添加注释:

\beforeverb
\begin{verbatim}
percentage = (minute * 100) / 60     # percentage of an hour
\end{verbatim}
\afterverb
%
{\tt \#}符号后到一行结束中的任何内容都被忽略---它们将不影响程序的运行。

当说明代码中不明显的特点时,注释非常有效。我们可以假设阅读代码的人可以指出代码{\em 做了什么};更有必要的是解释{\em 为什么这么做}。

以下的注释是冗余而无用的:

\beforeverb
\begin{verbatim}
v = 5     # assign 5 to v
\end{verbatim}
\afterverb
%
以下的注释包含了代码中没有的有用信息:

\beforeverb
\begin{verbatim}
v = 5     # velocity in meters/second. 
\end{verbatim}
\afterverb
%
好的变量明可以减少对注释的依赖,但长的变量名会使得表达式难以阅读,因此这是一个折衷。

\section{调试}
\index{调试}

目前位置你遇到的语法错误大多为非法变量明,如关键字{\tt class}和{\tt yield},或者包含非法字符,如\verb"odd~job"和\verb"US$"。

\index{语法错误}
\index{错误!语法}

如果你在一个变量名中加入一个空格,Python会认为这是两个没有运算符的运算数:

\beforeverb
\begin{verbatim}
>>> bad name = 5
SyntaxError: invalid syntax
\end{verbatim}
\afterverb
%
对于语法错误,错误消息没有很大的帮助。通常的消息是{\tt SyntaxError: invalid syntax}和{\tt SyntaxError: invalid token},它们都不包含很多信息量。

\index{错误消息}
\index{未定义而使用}
\index{异常}
\index{运行时错误}
\index{错误!运行时}

运行时错误通常由于“未定义而使用”,即试图使用一个没有赋值的变量。这会在你拼写变量名错误时发生:

\beforeverb
\begin{verbatim}
>>> principal = 327.68
>>> interest = principle * rate
NameError: name 'principle' is not defined
\end{verbatim}
\afterverb
%
变量名是大小写敏感的,因此{\tt LaTeX}和{\tt latex}是不同的。

\index{大小写敏感,变量名}
\index{语义错误}
\index{错误!语义}

语义错误通常发生在运算符的顺序上。例如,求值$\frac{1}{2 \pi}$,你也许会写成:

\beforeverb
\begin{verbatim}
>>> 1.0 / 2.0 * pi
\end{verbatim}
\afterverb
%
但是首先执行了除法,因此你会得到$\pi / 2$,并不是想要的结果!Python没有办法知道你写的是什么含义,因此这种情况下你不会得到一个错误消息;你得到的只是错误的结果。

\index{运算符优先级}

\section{术语表}

\begin{description}

\item[数值:] 程序操作的基本数据单元,如一个数字或字符串。
\index{数值}

\item[类型:] 数据的分类。目前为止我们见过的类型有整数({\tt int}),浮点数({\tt float}),和字符串({\tt str})。
\index{类型}

\item[整数:] 表示所有整数的类型。
\index{整数}

\item[浮点数:] 表示具有小数部分的数的类型。
\index{浮点数}

\item[字符串:] 表示一串字符的数据类型。
\index{字符串}

\item[变量:] 指向一个数值的名称。
\index{变量}

\item[语句:] 一段代表命令或行为的代码。目前为止我们见过的语句有赋值语句和打印语句。
\index{语句}

\item[赋值:] 将数值赋值给变量的语句。
\index{赋值}

\item[状态图:] 表示变量和对应数值的图。
\index{状态图}

\item[关键字:] 一些编译器用来对程序进行语法分析的保留单词,你不能使用例如{\tt if},{\tt  def}和{\tt while}作为变量名。
\index{关键字}

\item[运算符:] 表示一种简单运算的特殊符号,如加法,乘法,或者字符串级联。
\index{运算符}

\item[运算数:] 运算符作用的数值。
\index{运算数}

\item[下取整除法:] 两数相除并舍去小数部分的运算。
\index{下取整除法}

\item[表达式:] 表示单一结果的变量、运算符和数值的组合。
\index{表达式}

\item[求值] 通过执行运算符化简表达式得到一个单一结果。

\item[优先级规则] 定义运算符和运算数求值顺规的规则。
\index{优先级规则}
\index{优先级}

\item[级联:] 将两个运算数首尾现连。
\index{级联}

\item[注释:] 程序中提供给程序员(或者任何阅读程序的人)信息,但不影响程序执行的内容。
\index{注释}

\end{description}


\section{练习}

\begin{ex}
假设我们执行了如下的赋值语句:

\begin{verbatim}
width = 17
height = 12.0
delimiter = '.'
\end{verbatim}

对于下列表达式,写出表达式的数值和数据类型(表达式的数值)。

\begin{enumerate}

\item {\tt width/2}

\item {\tt width/2.0}

\item {\tt height/3}

\item {\tt 1 + 2 * 5}

\item {\tt delimiter * 5}

\end{enumerate}

使用Python解释器来验证你的结果。
\end{ex}

\begin{ex}
练习使用Python解释器作为计算器。
\index{calculator}

\begin{enumerate}

\item 一个半径为$r$的球的体积为$\frac{4}{3} \pi r^3$。一个半径为5的球体的体积是多少?提示:392.6是错误的!

\item 假设一本书的封面价格是\$24.95,书店提供40\%的折扣,对第一本书的运输费用为\$3,往后每增加一本需要75每分,如果购买60本书总共需要多少?

\item 如果我在6:52离开我家,用轻松的步伐跑了1英里(8:15每英里),然后保持节奏跑了3英里(7:12每英里),最后用轻松的步伐跑了1英里,我什么时候回到家吃早饭?

\index{跑步比赛}

\end{enumerate}
\end{ex}
