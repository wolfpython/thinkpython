\chapter{迭代器}
\index{iteration 迭代器}

\section{多重赋值}
\index{statement!assignment}
\index{mutiple assignment 多重赋值}

你可能已经发现,给一个变量多次赋值是合法的。一次新的赋值使得已存在的变量指向一个新值(当然也就不指向原来的值)。

\beforeverb
\begin{verbatim}
bruce = 5
print bruce,
bruce = 7
print bruce
\end{verbatim}
\afterverb

程序的输出是{\tt 5 7},因为第一次{\tt bruce}被输出时,它的值是5,第二次是7。
第一个{\tt print}语句末尾的逗号抑制了换行,这也是为什么两个输出在同一行的原因。

\index{newline 换行}

下图是多重赋值的状态图:

\index{state diagram}
\index{diagram!state}

\beforefig
\centerline{\includegraphics{figs/assign2.eps}}
\afterfig

对于多重赋值,很有必要分清赋值操作符和关系运算符中的等号。因为Python使用
等于号({\tt =})来表示赋值。很容易,误把这样的语句{\tt a = b}当作判断相等的语句,
实际不是的!\\

\index{相等与赋值}


第一,相等是对称关系,赋值不是。比如,在数学中,如果$a = 7$, 那么$7 =
a$。但在Python中,语句{\tt a = 7}是合法的,{\tt 7 = a}是非法的。\\

另外,在数学中,相等语句总是要么为真要么为假。如果,$a = b$,则,$a$总
是等于$b$。在Python中,赋值语句可以使得两个变量相等,但是他们不总是保持相等:

\beforeverb
\begin{verbatim}
a = 5
b = a    # a and b are now equal
a = 3    # a and b are no longer equal
\end{verbatim}
\afterverb
%

第三行改变了{\tt a}的值,但是没有改变{\tt b}的值。所以他们不再相等。

尽管多重赋值通常是有益的,使用时也要小心。如果变量的值经常改变,代码
会变得很脑阅读和调试。

\section{更新变量}
\label{update}

\index{update 更新}
\index{variable!updating}

多重赋值的最常见的形式之一就是更新(update),变量的新值依赖于原有值。

\beforeverb
\begin{verbatim}
x = x+1
\end{verbatim}
\afterverb

含义是:获取{\tt x}的当前值,加一,然后用新值更新变量{\tt x}。

如果试着更新一个不存在的变量,将会得到一个错误,因为Python在把值赋给
{\tt x}之前会计算右边的值。

\beforeverb
\begin{verbatim}
>>> x = x+1
NameError: name 'x' is not defined
\end{verbatim}
\afterverb
%

在更新一个变量之前,必须得初始化(initialize)它,通常的做法就是一个
简单的赋值。

\index{initialization (before update) (更新前)初始化}

\beforeverb
\begin{verbatim}
>>> x = 0
>>> x = x+1
\end{verbatim}
\afterverb
%

仅仅通过加一来更新变量叫做增量 (increment);减一叫做减量(decrement)。

\index{increment  增量}
\index{decrement 减量}

\section{{\tt while 语句}}

\index{statement!while}
\index{while loop while循环}
\index{loop!while}
\index{iteration 迭代}

计算机通常被用来自动完成重复性的任务。计算机很擅长于重复相同或相似的任务。人类却恰恰相反\footnote{太枯燥了,回顾一下小学时,老师天天要求抄生字......}。

我们已经看到两个程序,{\tt countdown}和\verb"print_n",它们使用递归
实现重复,这也可以乘坐迭代{\bf iteration}。因为迭代是如此的常见,以致
于Python提供了几个特有的方式来简化使用。其中之一就是我们在\ref{重复}
部分看到的{\tt for}语句。我们不久将回来重新研究它。\\

另外一个就是{\tt while}语句。这里是一个使用{\tt while}语句的{\tt coutdown}版本。

\beforeverb
\begin{verbatim}
def countdown(n):
    while n > 0:
        print n
        n = n-1
    print 'Blastoff!'
\end{verbatim}
\afterverb

我们几乎可以把{\tt while}语句当成英语来读了。含义是:当{\tt n}大于0时
,显示{\tt n}的值,并把{\tt n}的值减1。当{\tt n}的值为0的时候,显示{\tt Blastoff!}。

\index{flow of execution 执行流}

更正式地,下面的是{\tt while}语句的执行流。

\begin{enumerate}

\item 计算条件的值,产生结果{\tt True}或者{\tt False}。

\item 如果条件为假,退出{\tt while循环},继续执行下一条语句。

\item 如果条件为真,执行语句体里的语句,然后回到步骤一。

\end{enumerate}

执行流的类型称作是循环(loop)的原因是因为第三步循环返回至第一步。

\index{condition 条件}
\index{loop 循环}
\index{body 体}

循环体应该改变一个或多个变量的值,使得最终条件为假,循环终止。否则,
循环将永远重复,也就产生了无限循环(infinite loop)。计算机科学家的
一个永远的谈资就是看到洗发水的说明"泡沫,漂洗,重复",是一个无限循环。

\index{infinite loop 无限循环}
\index{loop!infinite}

在{\tt countdown}的例子里,我们可以证明循环一定会终止,因为我们知道
{\tt n}的值是有限的,并且每一次循环{\tt n}的值都会减小,最终,{\tt n}的值肯定是0。在其他情况下,就不一定这么容易辨别了:

beforeverb
\begin{verbatim}
def sequence(n):
    while n != 1:
        print n,
        if n%2 == 0:        # n is even
            n = n/2
        else:               # n is odd
            n = n*3+1
\end{verbatim}
\afterverb

这个循环的条件是{\tt n != 1},所以循环会一直执行到{\tt n}是{\tt 1},
此时条件为假。\\

每一次循环,程序输出{\tt n}的值,然后检查是否为偶数或奇数。如果是偶数
{\tt n}就除以2。如果为奇数,{\tt n}的值就被{\tt n*3+1}代替。比如,
如果传递3给{\tt sequence},产生的结果是3, 10, 5, 16, 8, 4, 2, 1。

因为{\tt n}时增时减,没有一个明显的办法确定{\tt n}是否会为1,也就是
程序是否会正常终止。对于某些特别的{\tt n},我们可以证明终止。比如,
如果{\tt n}的值是2的倍数,每次循环时{\tt n}的值,都是偶数直到为1。
前面的例子从16开始都是这种情况。

\index{Collatz conjecture Collatz猜想}

困难的是我们是否可以证明对所有的正整数,程序都能终止。迄今为止\footnote{参看\url{wikipedia.org/wiki/Collatz_conjecture}.}没有人可以证明
可以,也没有人证明不可以。

\begin{ex}
重写\ref{递归}部分的\verb"print_n"函数,要求使用迭代器,而不是递归。
\end{ex}

\section{{\tt break}语句}
\index{break statement break语句}
\index{statement!break}

有时,直到执行到循环体里面的时候,才直到需要跳出循环。此时,我们可以
使用{\tt break}语句跳出循环。

比如,假设一直想从用户那里得到输入,直到用户输入{\tt done}。我们可以
这么写:

\beforeverb
\begin{verbatim}
while True:
    line = raw_input('> ')
    if line == 'done':
        break
    print line

print 'Done!'
\end{verbatim}
\afterverb

循环条件为{\tt True},也就是永远为真,所以循环直到遇到break statement
才终止执行。

每次循环,用尖括号提示用户。如果用户输入{\tt done},{\tt break}语句
终止了循环。否则,程序输出用户输入的内容,会到循环的顶部。下面是一个例子:

\beforeverb
\begin{verbatim}
> not done
not done
> done
Done!
\end{verbatim}
\afterverb
%

这种使用{\tt while}循环的方式很常见,因为我们可以在循环的任何地方检查
条件(不仅仅是在顶部),同时也积极的表达了结束的条件(当这个发生时,终止),而不是消极地("一直运行,直到这个发生")。

\section{平方根}
\index{square root}


