\chapter{迭代器}
\index{iteration 迭代器}

\section{多重赋值}
\index{statement!assignment}
\index{mutiple assignment 多重赋值}

你可能已经发现,给一个变量多次赋值是合法的。一次新的赋值使得已存在的变量指向一个新值(当然也就不指向原来的值)。

\beforeverb
\begin{verbatim}
bruce = 5
print bruce,
bruce = 7
print bruce
\end{verbatim}
\afterverb

程序的输出是{\tt 5 7},因为第一次{\tt bruce}被输出时,它的值是5,第二次是7。
第一个{\tt print}语句末尾的逗号抑制了换行,这也是为什么两个输出在同一行的原因。

\index{newline 换行}

下图是多重赋值的状态图:

\index{state diagram}
\index{diagram!state}

\beforefig
\centerline{\includegraphics{figs/assign2.eps}}
\afterfig

对于多重赋值,很有必要分清赋值操作符和关系运算符中的等号。因为Python使用
等于号({\tt =})来表示赋值。很容易,误把这样的语句{\tt a = b}当作判断相等的语句,
实际不是的!\\

\index{相等与赋值}


第一,相等是对称关系,赋值不是。比如,在数学中,如果$a = 7$, 那么$7 =
a$。但在Python中,语句{\tt a = 7}是合法的,{\tt 7 = a}是非法的。\\

另外,在数学中,相等语句总是要么为真要么为假。如果,$a = b$,则,$a$总
是等于$b$。在Python中,赋值语句可以使得两个变量相等,但是他们不总是保持相等:

\beforeverb
\begin{verbatim}
a = 5
b = a    # a and b are now equal
a = 3    # a and b are no longer equal
\end{verbatim}
\afterverb
%

第三行改变了{\tt a}的值,但是没有改变{\tt b}的值。所以他们不再相等。

尽管多重赋值通常是有益的,使用时也要小心。如果变量的值经常改变,代码
会变得很脑阅读和调试。

\section{更新变量}
\label{update}

\index{update 更新}
\index{variable!updating}

多重赋值的最常见的形式之一就是更新(update),变量的新值依赖于原有值。

\beforeverb
\begin{verbatim}
x = x+1
\end{verbatim}
\afterverb

含义是:获取{\tt x}的当前值,加一,然后用新值更新变量{\tt x}。

如果试着更新一个不存在的变量,将会得到一个错误,因为Python在把值赋给
{\tt x}之前会计算右边的值。

\beforeverb
\begin{verbatim}
>>> x = x+1
NameError: name 'x' is not defined
\end{verbatim}
\afterverb
%

在更新一个变量之前,必须得初始化(initialize)它,通常的做法就是一个
简单的赋值。

\index{initialization (before update) (更新前)初始化}

\beforeverb
\begin{verbatim}
>>> x = 0
>>> x = x+1
\end{verbatim}
\afterverb
%

仅仅通过加一来更新变量叫做增量 (increment);减一叫做减量(decrement)。

\index{increment  增量}
\index{decrement 减量}

\section{{\tt while 语句}}

\index{statement!while}
\index{while loop while循环}
\index{loop!while}
\index{iteration 迭代}

计算机通常被用来自动完成重复性的任务。计算机很擅长于重复相同或相似的任务。人类却恰恰相反\footnote{太枯燥了,回顾一下小学时,老师天天要求抄生字......}。

我们已经看到两个程序,{\tt countdown}和\verb"print_n",它们使用递归
实现重复,这也可以乘坐迭代{\bf iteration}。因为迭代是如此的常见,以致
于Python提供了几个特有的方式来简化使用。其中之一就是我们在\ref{重复}
部分看到的{\tt for}语句。我们不久将回来重新研究它。\\

另外一个就是{\tt while}语句。这里是一个使用{\tt while}语句的{\tt coutdown}版本。

\beforeverb
\begin{verbatim}
def countdown(n):
    while n > 0:
        print n
        n = n-1
    print 'Blastoff!'
\end{verbatim}
\afterverb

我们几乎可以把{\tt while}语句当成英语来读了。含义是:当{\tt n}大于0时
,显示{\tt n}的值,并把{\tt n}的值减1。当{\tt n}的值为0的时候,显示{\tt Blastoff!}。

\index{flow of execution 执行流}

更正式地,下面的是{\tt while}语句的执行流。

\begin{enumerate}

\item 计算条件的值,产生结果{\tt True}或者{\tt False}。

\item 如果条件为假,退出{\tt while循环},继续执行下一条语句。

\item 如果条件为真,执行语句体里的语句,然后回到步骤一。

\end{enumerate}

执行流的类型称作是循环(loop)的原因是因为第三步循环返回至第一步。

\index{condition 条件}
\index{loop 循环}
\index{body 体}

循环体应该改变一个或多个变量的值,使得最终条件为假,循环终止。否则,
循环将永远重复,也就产生了无限循环(infinite loop)。计算机科学家的
一个永远的谈资就是看到洗发水的说明"泡沫,漂洗,重复",是一个无限循环。

\index{infinite loop 无限循环}
\index{loop!infinite}

在{\tt countdown}的例子里,我们可以证明循环一定会终止,因为我们知道
{\tt n}的值是有限的,并且每一次循环{\tt n}的值都会减小,最终,{\tt n}的值肯定是0。在其他情况下,就不一定这么容易辨别了:

beforeverb
\begin{verbatim}
def sequence(n):
    while n != 1:
        print n,
        if n%2 == 0:        # n is even
            n = n/2
        else:               # n is odd
            n = n*3+1
\end{verbatim}
\afterverb

这个循环的条件是{\tt n != 1},所以循环会一直执行到{\tt n}是{\tt 1},
此时条件为假。\\

每一次循环,程序输出{\tt n}的值,然后检查是否为偶数或奇数。如果是偶数
{\tt n}就除以2。如果为奇数,{\tt n}的值就被{\tt n*3+1}代替。比如,
如果传递3给{\tt sequence},产生的结果是3, 10, 5, 16, 8, 4, 2, 1。

因为{\tt n}时增时减,没有一个明显的办法确定{\tt n}是否会为1,也就是
程序是否会正常终止。对于某些特别的{\tt n},我们可以证明终止。比如,
如果{\tt n}的值是2的倍数,每次循环时{\tt n}的值,都是偶数直到为1。
前面的例子从16开始都是这种情况。

\index{Collatz conjecture Collatz猜想}

困难的是我们是否可以证明对所有的正整数,程序都能终止。迄今为止\footnote{参看\url{wikipedia.org/wiki/Collatz_conjecture}.}没有人可以证明
可以,也没有人证明不可以。

\begin{ex}
重写\ref{递归}部分的\verb"print_n"函数,要求使用迭代器,而不是递归。
\end{ex}

\section{{\tt break}语句}
\index{break statement break语句}
\index{statement!break}

有时,直到执行到循环体里面的时候,才直到需要跳出循环。此时,我们可以
使用{\tt break}语句跳出循环。

比如,假设一直想从用户那里得到输入,直到用户输入{\tt done}。我们可以
这么写:

\beforeverb
\begin{verbatim}
while True:
    line = raw_input('> ')
    if line == 'done':
        break
    print line

print 'Done!'
\end{verbatim}
\afterverb

循环条件为{\tt True},也就是永远为真,所以循环直到遇到break statement
才终止执行。

每次循环,用尖括号提示用户。如果用户输入{\tt done},{\tt break}语句
终止了循环。否则,程序输出用户输入的内容,会到循环的顶部。下面是一个例子:

\beforeverb
\begin{verbatim}
> not done
not done
> done
Done!
\end{verbatim}
\afterverb
%

这种使用{\tt while}循环的方式很常见,因为我们可以在循环的任何地方检查
条件(不仅仅是在顶部),同时也积极的表达了结束的条件(当这个发生时,终止),而不是消极地("一直运行,直到这个发生")。

\section{平方根}
\index{square root}

循环经常用在计算数值的程序中,通常都是以一个相近的值开始,然后迭代,逐渐提高。

\index{Newton's method 牛顿方法}

比如,有一种计算平方根的算法叫牛顿方法。假设,我们想得到$a$的平方根。如果以任意猜测的
一个值开始,我们可以用下面的公式,计算一个更好的猜测值:

\[ y = \frac{x + a/x}{2} \]
比如,如果$a$是4, $x$是3:

\beforeverb
\begin{verbatim}
>>> a = 4.0
>>> x = 3.0
>>> y = (x + a/x) / 2
>>> print y
2.16666666667
\end{verbatim}
\afterverb
%
结果已经接近正确答案了($\sqrt{4} = 2$)。如果我们重复这个过程,就更接近了:

\beforeverb
\begin{verbatim}
>>> x = y
>>> y = (x + a/x) / 2
>>> print y
2.00641025641
\end{verbatim}
\afterverb
经过几次更新,猜测值基本上等于精确值了:
\index{update 更新}

\beforeverb
\begin{verbatim}
>>> x = y
>>> y = (x + a/x) / 2
>>> print y
2.00001024003
>>> x = y
>>> y = (x + a/x) / 2
>>> print y
2.00000000003
\end{verbatim}
\afterverb

一般来说,我们实现并不知道经过多少步才能得到正确的结果,但是我们知道什么时候得到
正确的结果,应为猜测值成定值了。

\beforeverb
\begin{verbatim}
>>> x = y
>>> y = (x + a/x) / 2
>>> print y
2.0
>>> x = y
>>> y = (x + a/x) / 2
>>> print y
2.0
\end{verbatim}
\afterverb

当{\tt y == x},我们就可以停止了。下面是一个循环,从一个初始值开始,然后逐步逼近,直到猜测值为定值。

\beforeverb
\begin{verbatim}
while True:
    print x
    y = (x + a/x) / 2
    if y == x:
        break
    x = y
\end{verbatim}
\afterverb
%

对于大多数的{\tt a},这个都适用。但是,一般说来,测试浮点数是否相等是危险地。浮点值
只是近似相等:大多数有理数,像$1/3$,和无理数,像$\sqrt{2}$,不能用一个浮点数
精确的表示。

\index{floating-point 浮点}
\index{epsilon }

与其检查{\tt x}和{\tt y}是否精确相等,不如安全的采用内置的函数{\tt abs}计算
差的绝对值:

\beforeverb
\begin{verbatim}
    if abs(y-x) < epsilon:
        break
\end{verbatim}
\afterverb
%
这里,\verb"epsilon"是一个极小值,像{\tt 0.0000001} ,表示了什么样的接近才是
非常接近了。

\begin{ex}
\label{square_root}
\index{encapsulation 封装}

把这个循环封装在函数\verb"square_root"里,接受{\tt a}为参数,选择一个合适的{\tt x},返回{\tt a}的近似平方根。
\end{ex}

\section{算法}
\index{algorithm 算法}

牛顿方法是算法的一个例子:它是解决一类问题的方法(在这个例子里是计算平方根)。

 定义一个算法可不是件容易的事。从不是算法的方面入手或许会有帮助。当你学习
 单位数相乘时,你背了乘法表。事实上,你记住了100个具体的解法。这种知识不是
 算法。\\

但是,如果你比较“懒”,你可能作些小弊。比如,计算$n$和9的乘积,你可以把十位写成$n-1$,个位写成$10-n$。这个就是解决任何单位数乘以9的一般方法\footnote{译注:比如
$8 * 9 = 72 = (8-1)(10-8)$}。对了,这就是算法。

\index{addition with carrying}
\index{carrying, addition with}
\index{subtraction!with borrowing}
\index{borrowing, subtraction with}

类似地,我们学到的技巧,比如加法进位,减法借位,长除法都是算法。这些算法的共有的
特点就是他们不需要任何的智力来实施。他们是机械的过程,每一步都依靠简单的规则。

从我的观点来看,人们花费大量的时间在学校里学习---不夸张地说,不需要任何智力的算法是非常不值得的。\\

从另一方面来说,设计算法的过程确实有趣的,挑战智力的,也是程序设计的中心内容。

有些事情,人们作起来很自然,没有困难也无须苦思冥想,但却是最难用算法表达的。理解自然语言是一个好的例子。我们都有这种能力,但是至今没有人能解释为什么我们可以,至少我们不能以算法的形式表达出来。

\section{调试}

当我们开始编写大型的程序时,就会发现调试将会花费我们大量的时间。代码越多,意味着
犯错的机会就越大,隐藏bug(s)的地方就越多。

\index{debugging!by bisection}
\index{bisection, debugging by}

减少调试时间的一种方法就是“二分法”。例如,程序有100行代码,每次检查一个,
需要100步。

换一种思路,把问题分成两半。查看程序的中间部分,或者接近中间部分,寻找一个
中间值来检查。添加一个{\tt print}语句(或者其他的能产生验证效果的语句),然后执行程序。

如果中间检查有问题,那么必定是程序的前半部分有问题。反之,则在第二部分。

每次按照这样检查的话,我们缩减了需要检查的代码。至少理论上来说,六步以后,
(这远远小于100),我们就可以缩减到一两行代码了。

实际中,很难界定程序的中间部分是哪里,也不总可能检查它。数代码的行数然后计算精确的中间点是没有任何意义的。相反,仔细想象什么地方最有可能出现错误,什么地方容易插入一个语句来检查。然后,选择一处你认为bug出现在检查点前后几率近似相等的地方。

\section{术语表}

\begin{description}

\item [multiple assignment 多重赋值:] 在程序的执行过程中给同一个变量赋多次值。
\index{mutiple assignement 多重赋值}
\index{assignment!multiple}

\item [update 更新:] 变量的新值依赖原来值的赋值。
\index{update 更新}

\item [initialization 初始化:]给一个将被更新的变量初始值的赋值。
\index{initialization!variable}

\item [increment 增量:] 增加一个变量的更新(通常是1)。
\index{increment}

\item [decrement 减量:]减小一个变量的更新(通常是1)。
\index{decrement}

\item [iteration 迭代:]使用递归或者循环的重复性执行的语句集合。
\index{iteration 迭代}

\item [infinite loop:无限循环]终止条件永远不满足的循环。
\index{infinite loop 无限循环}

\end{description}


\section{练习}

\begin{ex}

\index{algorithm!square root}

测试这章的平方根算法,你可以把结果和{\tt math.sqrt}的结果比较。写一个函数
\verb"test_square_root"打印一个如下的表:

\beforeverb
\begin{verbatim}
1.0 1.0           1.0           0.0
2.0 1.41421356237 1.41421356237 2.22044604925e-16
3.0 1.73205080757 1.73205080757 0.0
4.0 2.0           2.0           0.0
5.0 2.2360679775  2.2360679775  0.0
6.0 2.44948974278 2.44948974278 0.0
7.0 2.64575131106 2.64575131106 0.0
8.0 2.82842712475 2.82842712475 4.4408920985e-16
9.0 3.0           3.0           0.0

\end{verbatim}
\afterverb

第一列是数字$a$,第二列是用\ref{square_root}计算的$a$的平方根,第三列是用
{\tt math.sqrt}计算的平方根,第四列是两个结果的差值。
\end{ex}

\begin{ex}

\index{eval function eval函数}
\index{function!eval}

内置函数{\tt eval}接受一个字符串,然后调用python解释器计算字符串的值,例如:

\beforeverb
\begin{verbatim}
>>> eval('1 + 2 * 3')
7
>>> import math
>>> eval('math.sqrt(5)')
2.2360679774997898
>>> eval('type(math.pi)')
<type 'float'>
\end{verbatim}
\afterverb

编写一个函数\verb"eval_loop"重复的提示用户,接受用户输入,然后调用
{\tt eval}计算值,并打印结果。

程序必须知道用户舒服\verb"'done'"才结束循环,然后返回最后计算的
表达式的值。
\end{ex}

\begin{ex}

\index{Ramanujan, Srinivasa}

杰出的数学家Srinivasa Ramanujan 发现了无穷序列\footnote{参看\url{wikipedia.org/wiki/Pi}.}可以用来产生近似的$pi$值。

\index{pi}

\[\frac{1}{\pi} = \frac{2\sqrt{2}}{9801} 
\sum^\infty_{k=0} \frac{(4k)!(1103+26390k)}{(k!)^4 396^{4k}} \]

编写函数\verb"estimate_pi",函数使用上面的公式计算$pi$值,并返回之,
函数应该使用一个{\tt while}循环计算项的和知道最后一项小鱼{\tt le-15}
(Python里的记法为$10^{-15})$。你可以拿它和{\tt math.pi}比较一下。

可以参看我的解答\url{thinkpython.com/code/pi.py}。
\end{ex}






